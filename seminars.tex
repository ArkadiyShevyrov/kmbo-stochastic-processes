\section{Семинары}

\subsection{Семинар 1}

Задача с книгами с.
Книгами2

\[
	P =
	\begin{array}{c|cccccc}
		  & 1            & 2            & 3            & 4            & 5            & 6            \\ \hline
		1 & \tfrac{1}{6} & \tfrac{1}{3} & 0            & \tfrac{1}{2} & 0            & 0            \\
		2 & \tfrac{1}{6} & \tfrac{1}{3} & 0            & 0            & 0            & \tfrac{1}{2} \\
		3 & 0            & \tfrac{1}{3} & \tfrac{1}{6} & \tfrac{1}{2} & 0            & 0            \\
		4 & 0            & 0            & \tfrac{1}{6} & \tfrac{1}{2} & \tfrac{1}{3} & 0            \\
		5 & \tfrac{1}{6} & 0            & 0            & 0            & \tfrac{1}{3} & \tfrac{1}{2} \\
		6 & 0            & 0            & \tfrac{1}{6} & 0            & \tfrac{1}{3} & \tfrac{1}{2}
	\end{array}
\]

\begin{tikzpicture}[
		>=Stealth, thick,
		state/.style = {draw, circle, minimum size=9mm, font=\small},
		e1/.style   = {->, shorten >=2pt, shorten <=2pt},
		e2/.style   = {->, shorten >=2pt, shorten <=2pt},
		e3/.style   = {->, shorten >=2pt, shorten <=2pt},
		lbl/.style  = {font=\scriptsize, fill=white, inner sep=1pt}
	]

	% --- вероятности ---
	\newcommand{\p}[1]{p_{#1}}
	% если нужны цифры: раскомментировать ↓ и закомментировать ↑
	\renewcommand{\p}[1]{\ifcase#1\relax \or \frac{1}{2}\or \frac{1}{3}\or \frac{1}{6}\fi}

	% --- вершины ---
	\node[state] (Q1) at (90:3)   {$123$};
	\node[state] (Q2) at (30:3)   {$231$};
	\node[state] (Q3) at (-30:3)  {$312$};
	\node[state] (Q4) at (-90:3)  {$321$};
	\node[state] (Q5) at (-150:3) {$213$};
	\node[state] (Q6) at (150:3)  {$132$};

	% --- рёбра ---
	% Q1 = 123
	\path[e1, bend left=12] (Q1) edge node[lbl] {$\p{1}$} (Q2);
	\path[e2, bend left=12] (Q1) edge node[lbl] {$\p{2}$} (Q6);
	\path[e3, loop above]   (Q1) edge node[lbl] {$\p{3}$} (Q1);

	% Q2 = 231
	\path[e1, loop right]   (Q2) edge node[lbl] {$\p{1}$} (Q2);
	\path[e2, bend left=12] (Q2) edge node[lbl] {$\p{2}$} (Q3);
	\path[e3, bend left=12] (Q2) edge node[lbl] {$\p{3}$} (Q5);

	% Q3 = 312
	\path[e1, bend left=12] (Q3) edge node[lbl] {$\p{1}$} (Q4);
	\path[e2, loop right]   (Q3) edge node[lbl] {$\p{2}$} (Q3);
	\path[e3, bend left=12] (Q3) edge node[lbl] {$\p{3}$} (Q1);

	% Q4 = 321
	\path[e1, loop below]   (Q4) edge node[lbl] {$\p{1}$} (Q4);
	\path[e2, bend left=12] (Q4) edge node[lbl] {$\p{2}$} (Q3);
	\path[e3, bend left=12] (Q4) edge node[lbl] {$\p{3}$} (Q5);

	% Q5 = 213
	\path[e1, bend left=12] (Q5) edge node[lbl] {$\p{1}$} (Q2);
	\path[e2, bend left=12] (Q5) edge node[lbl] {$\p{2}$} (Q6);
	\path[e3, loop left]    (Q5) edge node[lbl] {$\p{3}$} (Q5);

	% Q6 = 132
	\path[e1, bend left=12] (Q6) edge node[lbl] {$\p{1}$} (Q4);
	\path[e2, loop left]    (Q6) edge node[lbl] {$\p{2}$} (Q6);
	\path[e3, bend left=12] (Q6) edge node[lbl] {$\p{3}$} (Q1);

\end{tikzpicture}
