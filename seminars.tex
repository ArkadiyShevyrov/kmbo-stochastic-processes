\section{Семинары}

\subsection{Семинар 1}

\subsection{Семинар 2}

\subsubsection*{Задача с книгами}

\begin{tikzpicture}[
		>=Stealth, thick,
		state/.style = {draw, circle, minimum size=9mm, font=\small},
		e1/.style   = {->, shorten >=2pt, shorten <=2pt},
		e2/.style   = {->, shorten >=2pt, shorten <=2pt},
		e3/.style   = {->, shorten >=2pt, shorten <=2pt},
		lbl/.style  = {font=\scriptsize, fill=white, inner sep=1pt}
	]

	% --- вероятности ---
	\newcommand{\p}[1]{p_{#1}}
	\renewcommand{\p}[1]{\ifcase#1\relax \or \frac{1}{2}\or \frac{1}{3}\or \frac{1}{6}\fi}

	% --- вершины ---
	\node[state] (Q1) at (90:3)   {$123$};
	\node[state] (Q2) at (30:3)   {$132$};
	\node[state] (Q3) at (-30:3)  {$213$};
	\node[state] (Q4) at (-90:3)  {$231$};
	\node[state] (Q5) at (-150:3) {$312$};
	\node[state] (Q6) at (150:3)  {$321$};

	% --- рёбра ---
	% Q1 = 123
	\path[e1, bend left=12] (Q1) edge node[lbl] {$\p{1}$} (Q4);
	\path[e2, bend left=12] (Q1) edge node[lbl] {$\p{2}$} (Q2);
	\path[e3, loop above]   (Q1) edge node[lbl] {$\p{3}$} (Q1);

	% Q2 = 132
	\path[e1, bend left=12] (Q2) edge node[lbl] {$\p{1}$} (Q6);
	\path[e2, loop right]   (Q2) edge node[lbl] {$\p{2}$} (Q2);
	\path[e3, bend left=12] (Q2) edge node[lbl] {$\p{3}$} (Q1);

	% Q3 = 213
	\path[e1, bend left=12] (Q3) edge node[lbl] {$\p{1}$} (Q4);
	\path[e2, bend left=12] (Q3) edge node[lbl] {$\p{2}$} (Q2);
	\path[e3, loop right]   (Q3) edge node[lbl] {$\p{3}$} (Q3);

	% Q4 = 231
	\path[e1, loop below]   (Q4) edge node[lbl] {$\p{1}$} (Q4);
	\path[e2, bend left=12] (Q4) edge node[lbl] {$\p{2}$} (Q5);
	\path[e3, bend left=12] (Q4) edge node[lbl] {$\p{3}$} (Q3);

	% Q5 = 312
	\path[e1, bend left=12] (Q5) edge node[lbl] {$\p{1}$} (Q6);
	\path[e2, loop left]    (Q5) edge node[lbl] {$\p{2}$} (Q5);
	\path[e3, bend left=12] (Q5) edge node[lbl] {$\p{3}$} (Q1);

	% Q6 = 321
	\path[e1, loop left]    (Q6) edge node[lbl] {$\p{1}$} (Q6);
	\path[e2, bend left=12] (Q6) edge node[lbl] {$\p{2}$} (Q5);
	\path[e3, bend left=12] (Q6) edge node[lbl] {$\p{3}$} (Q2);

\end{tikzpicture}

\[
	P =
	\begin{array}{c|cccccc}
		    & Q_1          & Q_2          & Q_3          & Q_4          & Q_5          & Q_6          \\
		\hline
		Q_1 & \tfrac{1}{6} & \tfrac{1}{3} & 0            & \tfrac{1}{2} & 0            & 0            \\
		Q_2 & \tfrac{1}{6} & \tfrac{1}{3} & 0            & 0            & 0            & \tfrac{1}{2} \\
		Q_3 & 0            & \tfrac{1}{3} & \tfrac{1}{6} & \tfrac{1}{2} & 0            & 0            \\
		Q_4 & 0            & 0            & \tfrac{1}{6} & \tfrac{1}{2} & \tfrac{1}{3} & 0            \\
		Q_5 & \tfrac{1}{6} & 0            & 0            & 0            & \tfrac{1}{3} & \tfrac{1}{2} \\
		Q_6 & 0            & \tfrac{1}{6} & 0            & 0            & \tfrac{1}{3} & \tfrac{1}{2}
	\end{array}
\]


Стационарное распределение $\overline r=(r_1,r_2,r_3,r_4,r_5,r_6)$
задаётся условиями
\[
	\overline r P = \overline r,
	\qquad \sum_{i=1}^6 r_i = 1,
	\qquad r_i \ge 0.
\]

Из матрицы $P$ получаем систему:
\[
	\begin{array}{rcl}
		\begin{aligned}
			r_1 & = \tfrac{1}{6}r_1 + \tfrac{1}{6}r_2 + \tfrac{1}{6}r_5,                   \\
			r_2 & = \tfrac{1}{3}r_1 + \tfrac{1}{3}r_2 + \tfrac{1}{3}r_3 + \tfrac{1}{6}r_6, \\
			r_3 & = \tfrac{1}{6}r_3 + \tfrac{1}{6}r_4,                                     \\
			r_4 & = \tfrac{1}{2}r_1 + \tfrac{1}{2}r_3 + \tfrac{1}{2}r_4,                   \\
			r_5 & = \tfrac{1}{3}r_4 + \tfrac{1}{3}r_5 + \tfrac{1}{3}r_6,                   \\
			r_6 & = \tfrac{1}{2}r_2 + \tfrac{1}{2}r_5 + \tfrac{1}{2}r_6,
		\end{aligned}
		 & \qquad \Rightarrow \qquad &
		\begin{aligned}
			6r_1 & = r_1 + r_2 + r_5,          \\
			6r_2 & = 2r_1 + 2r_2 + 2r_3 + r_6, \\
			6r_3 & = r_3 + r_4,                \\
			2r_4 & = r_1 + r_3 + r_4,          \\
			3r_5 & = r_4 + r_5 + r_6,          \\
			2r_6 & = r_2 + r_5 + r_6.
		\end{aligned}
	\end{array}
\]

\[
	\begin{array}{rcl}
		\begin{aligned}
			5r_1 - r_2 - r_5          & = 0, \\
			-2r_1 + 4r_2 - 2r_3 - r_6 & = 0, \\
			5r_3 - r_4                & = 0, \\
			-\,r_1 - r_3 + r_4        & = 0, \\
			-\,r_4 + 2r_5 - r_6       & = 0, \\
			r_1+r_2+r_3+r_4+r_5+r_6   & =1,
		\end{aligned}
		 & \qquad \Rightarrow \qquad &
		\underbrace{\!
			\begin{pmatrix}
				5  & -1 & 0  & 0  & -1 & 0  \\
				-2 & 4  & -2 & 0  & 0  & -1 \\
				0  & 0  & 5  & -1 & 0  & 0  \\
				-1 & 0  & -1 & 1  & 0  & 0  \\
				0  & 0  & 0  & -1 & 2  & -1 \\
				1  & 1  & 1  & 1  & 1  & 1
			\end{pmatrix}}_{\displaystyle A}
		\underbrace{\!
			\begin{pmatrix}r_1\\r_2\\r_3\\r_4\\r_5\\r_6\end{pmatrix}}_{\displaystyle x}
		=
		\underbrace{\!
			\begin{pmatrix}0\\0\\0\\0\\0\\1\end{pmatrix}}_{\displaystyle b}.
	\end{array}
\]

\[
	\left[
		\begin{array}{rrrrrr|r}
			5  & -1 & 0  & 0  & -1 & 0  & 0 \\
			-2 & 4  & -2 & 0  & 0  & -1 & 0 \\
			0  & 0  & 5  & -1 & 0  & 0  & 0 \\
			-1 & 0  & -1 & 1  & 0  & 0  & 0 \\
			0  & 0  & 0  & -1 & 2  & -1 & 0 \\
			1  & 1  & 1  & 1  & 1  & 1  & 1
		\end{array}
		\right]
	\overset{}{\longrightarrow}
	\left[
		\begin{array}{rrrrrr|r}
			1 & 0 & 0 & 0 & 0 & 0 & \tfrac{2}{25} \\
			0 & 1 & 0 & 0 & 0 & 0 & \tfrac{3}{20} \\
			0 & 0 & 1 & 0 & 0 & 0 & \tfrac{1}{50} \\
			0 & 0 & 0 & 1 & 0 & 0 & \tfrac{1}{10} \\
			0 & 0 & 0 & 0 & 1 & 0 & \tfrac{1}{4}  \\
			0 & 0 & 0 & 0 & 0 & 1 & \tfrac{2}{5}
		\end{array}
		\right]
\]

\[
	\overline r =
	\left(\tfrac{2}{25},\;\tfrac{3}{20},\;\tfrac{1}{50},\;\tfrac{1}{10},\;\tfrac{1}{4},\;\tfrac{2}{5}\right).
\]


\subsection{Семинар 3}

\subsubsection*{Задача 3.1}

\subsubsection*{Задача 3.2}

\subsubsection*{Задача 3.3.1}

Условие: $p+q=1$, $p,q>0$.

\begin{tikzpicture}[
		>=Stealth, thick,
		state/.style = {draw, circle, minimum size=9mm, font=\small},
		e/.style   = {->, shorten >=2pt, shorten <=2pt},
		lbl/.style  = {font=\scriptsize, fill=white, inner sep=1pt}
	]

	% --- вершины ---
	\node[state] (Q1) at (0,0) {$1$};
	\node[state] (Q2) at (2.5,0) {$2$};
	\node[state] (Q3) at (5,0) {$3$};

	% --- рёбра ---
	% Q1
	\path[e, loop above]   (Q1) edge node[lbl] {$1$} (Q1);

	% Q2
	\path[e, bend left=12] (Q2) edge node[lbl] {$q$} (Q1);
	\path[e, bend left=12] (Q2) edge node[lbl] {$p$} (Q3);

	% Q3
	\path[e, loop above]   (Q3) edge node[lbl] {$1$} (Q3);

\end{tikzpicture}

Найти: $P$, $P^n$, $\overline{r}$ (стационарное распределение), $\lambda_1=1$, $\lambda_2$, $\lambda_3$ (собственные числа).

\paragraph{Решение.}

\[
	P =
	\begin{array}{c|ccc}
		    & Q_1 & Q_2 & Q_3 \\ \hline
		Q_1 & 1   & 0   & 0   \\
		Q_2 & q   & 0   & p   \\
		Q_3 & 0   & 0   & 1
	\end{array}
\]

\[
	P^0=I,\qquad P^2=P \;\Rightarrow\; P^n=P\ (n\ge1).
\]

Стационарное распределение $\overline r=(r_1,r_2,r_3)$ задаётся условиями
\[
	\overline r P = \overline r,\quad r_1+r_2+r_3=1,\quad r_i\ge0.
\]
Из $rP=\bigl(\,r\cdot\text{col}_1,\ r\cdot\text{col}_2,\ r\cdot\text{col}_3\,\bigr)$ и столбцов $P$ получаем систему
\[
	\begin{aligned}
		r_1 & = r_1 + q r_2, \\
		r_2 & = 0,           \\
		r_3 & = p r_2 + r_3,
	\end{aligned}
	\qquad\Rightarrow\qquad
	r_2=0,\ \ r_1+r_3=1.
\]
Отсюда
\[
	\overline r=(\alpha,\,0,\,1-\alpha),\quad \alpha\in[0,1].
\]

Собственные числа: вычислим характеристический многочлен
\[
	\det(P-\lambda I)=\det
	\begin{pmatrix}
		1-\lambda & 0        & 0         \\
		q         & -\lambda & p         \\
		0         & 0        & 1-\lambda
	\end{pmatrix}
	=(1-\lambda)\cdot(-\lambda)\cdot(1-\lambda)
	=-\lambda(1-\lambda)^2.
\]
Следовательно,
\[
	\boxed{\ \lambda_1=1,\ \lambda_2=1,\ \lambda_3=0\ }.
\]

% ─────────────────────────────────────────────────────────────────────────────

\subsubsection*{Задача 3.3.2}

Условие:

\begin{tikzpicture}[
		>=Stealth, thick,
		state/.style = {draw, circle, minimum size=9mm, font=\small},
		e/.style   = {->, shorten >=2pt, shorten <=2pt},
		lbl/.style  = {font=\scriptsize, fill=white, inner sep=1pt}
	]

	% --- вершины ---
	\node[state] (Q1) at (0,0) {$1$};
	\node[state] (Q2) at (2.5,0) {$2$};
	\node[state] (Q3) at (5,0) {$3$};

	% --- рёбра ---
	\path[e, bend left=12] (Q1) edge node[lbl] {$1$} (Q2);

	\path[e, bend left=12] (Q2) edge node[lbl] {$q$} (Q1);
	\path[e, bend left=12] (Q2) edge node[lbl] {$p$} (Q3);

	\path[e, bend left=12] (Q3) edge node[lbl] {$1$} (Q2);

\end{tikzpicture}

Найти: $P$, $P^n$, $\overline{r}$ (стационарное распределение), $\lambda_1=1$, $\lambda_2$, $\lambda_3$ (собственные числа).

\paragraph{Решение.}

\[
	P =
	\begin{array}{c|ccc}
		    & Q_1 & Q_2 & Q_3 \\ \hline
		Q_1 & 0   & 1   & 0   \\
		Q_2 & q   & 0   & p   \\
		Q_3 & 0   & 1   & 0
	\end{array}
\]

\emph{Степени.} Прямым перемножением:
\[
	P^2=
	\begin{pmatrix}
		q & 0 & p \\
		0 & 1 & 0 \\
		q & 0 & p
	\end{pmatrix},
	\qquad
	P^{2k}=P^2,\quad P^{2k+1}=P\ \ (k\ge0),\quad P^0=I.
\]

\emph{Стационарное распределение.} Пусть $\overline r=(r_1,r_2,r_3)$.
По столбцам $P$ имеем
\[
	\begin{aligned}
		r_1 & = q r_2,     \\
		r_2 & = r_1 + r_3, \\
		r_3 & = p r_2,
	\end{aligned}
	\qquad r_1+r_2+r_3=1.
\]
Подставляя $r_1=q r_2$, $r_3=p r_2$ во второе уравнение:
\[
	r_2=(q r_2)+(p r_2)=(p+q)r_2=r_2 \quad(\text{тождество}).
\]
Из нормировки:
\[
	r_1+r_2+r_3=(q r_2)+r_2+(p r_2)=(1+p+q)r_2=2r_2=1 \Rightarrow r_2=\tfrac12,
\]
следовательно
\[
	\overline r=\Bigl(\tfrac{q}{2},\ \tfrac12,\ \tfrac{p}{2}\Bigr).
\]

\emph{Собственные числа.} Находим
\[
	\det(P-\lambda I)=
	\det\begin{pmatrix}
		-\lambda & 1        & 0        \\
		q        & -\lambda & p        \\
		0        & 1        & -\lambda
	\end{pmatrix}
	=(-\lambda)\!\det\!\begin{pmatrix}-\lambda&p\\[2pt]1&-\lambda\end{pmatrix}
	-1\cdot\det\!\begin{pmatrix}q&p\\[2pt]0&-\lambda\end{pmatrix}.
\]
Отсюда
\[
	\det(P-\lambda I)=(-\lambda)(\lambda^2-p) - ( -q\lambda )
	=-\lambda^3 + (p+q)\lambda
	=-\lambda(\lambda^2-1).
\]
Так как $p+q=1$, получаем
\[
	\boxed{\ \lambda_1=1,\ \lambda_2=0,\ \lambda_3=-1\ }.
\]

% ─────────────────────────────────────────────────────────────────────────────

\subsubsection*{Задача 3.3.3}

Условие:

\begin{tikzpicture}[
		>=Stealth, thick,
		state/.style = {draw, circle, minimum size=9mm, font=\small},
		e/.style   = {->, shorten >=2pt, shorten <=2pt},
		lbl/.style  = {font=\scriptsize, fill=white, inner sep=1pt}
	]

	% --- вершины ---
	\node[state] (Q1) at (0,0) {$1$};
	\node[state] (Q2) at (2.5,0) {$2$};
	\node[state] (Q3) at (5,0) {$3$};

	% --- рёбра ---
	\path[e, loop above]   (Q1) edge node[lbl] {$q$} (Q1);
	\path[e, bend left=12] (Q1) edge node[lbl] {$p$} (Q2);

	\path[e, bend left=12] (Q2) edge node[lbl] {$q$} (Q1);
	\path[e, bend left=12] (Q2) edge node[lbl] {$p$} (Q3);

	\path[e, loop above]   (Q3) edge node[lbl] {$p$} (Q3);
	\path[e, bend left=12] (Q3) edge node[lbl] {$q$} (Q2);

\end{tikzpicture}

Найти: $P$, $P^n$, $\overline{r}$ (стационарное распределение), $\lambda_1=1$, $\lambda_2$, $\lambda_3$ (собственные числа).

\paragraph{Решение.}

\[
	P =
	\begin{array}{c|ccc}
		    & Q_1 & Q_2 & Q_3 \\ \hline
		Q_1 & q   & p   & 0   \\
		Q_2 & q   & 0   & p   \\
		Q_3 & 0   & q   & p
	\end{array}
\]

\emph{Стационарное распределение.} Пусть $\overline r=(r_1,r_2,r_3)$.
Из $rP=\overline r$ по столбцам получаем
\[
	\begin{aligned}
		r_1 & = q(r_1+r_2),    \\
		r_2 & = p r_1 + q r_3, \\
		r_3 & = p(r_2+r_3),
	\end{aligned}
	\qquad r_1+r_2+r_3=1.
\]
Из первого и третьего:
\[
	(1-q)r_1=q r_2 \ \Rightarrow\ p r_1=q r_2 \ \Rightarrow\ r_2=\frac{p}{q}r_1,
	\qquad
	(1-p)r_3=p r_2 \ \Rightarrow\ q r_3=p r_2 \ \Rightarrow\ r_3=\frac{p}{q}r_2.
\]
Следовательно
\[
	r_1:r_2:r_3 = 1 : \frac{p}{q} : \frac{p^2}{q^2} = q^2 : pq : p^2.
\]
Нормируя на единицу, получаем
\[
	\overline r=\pi=
	\frac{1}{\,q^2+pq+p^2\,}\,(q^2,\ pq,\ p^2).
\]

\emph{Собственные числа.} Находим характеристический многочлен:
\[
	\det(P-\lambda I)=
	\det\begin{pmatrix}
		q-\lambda & p        & 0         \\
		q         & -\lambda & p         \\
		0         & q        & p-\lambda
	\end{pmatrix}
	=(q-\lambda)\bigl(\lambda^2-\lambda p-pq\bigr)-pq(p-\lambda).
\]
Раскрывая и используя $p+q=1$:
\[
	\det(P-\lambda I)
	= -\lambda^3+\lambda^2+pq\,\lambda-pq
	= -(\lambda-1)(\lambda^2-pq).
\]
Итак,
\[
	\boxed{\ \lambda_1=1,\quad \lambda_2=\sqrt{pq},\quad \lambda_3=-\sqrt{pq}\ }.
\]

\[
	P^2 =
	\begin{pmatrix}
		q   & pq  & p^2 \\
		q^2 & 2pq & p^2 \\
		q^2 & pq  & p
	\end{pmatrix},
	\qquad
	P^3 =
	\begin{pmatrix}
		q^2(1+p) & pq(1+p) & p^2      \\
		q^2(1+p) & pq      & p^2(1+q) \\
		q^2      & pq(1+q) & p^2(1+q)
	\end{pmatrix}.
\]

\[
	P^4 =
	\begin{pmatrix}
		q^2(1+p)  & pq(1+pq)  & p^2(1+pq) \\
		q^2(1+pq) & pq(1+2pq) & p^2(1+pq) \\
		q^2(1+pq) & pq(1+pq)  & p^2(1+q)
	\end{pmatrix}.
\]
% ─────────────────────────────────────────────────────────────────────────────
\subsubsection*{Задача 4}

\begin{tikzpicture}[
		>=Stealth, thick,
		state/.style = {draw, circle, minimum size=9mm, font=\small},
		e/.style   = {->, shorten >=2pt, shorten <=2pt},
		lbl/.style  = {font=\scriptsize, fill=white, inner sep=1pt}
	]
	% --- вершины ---
	\node[state] (Q0) at (0,0)   {$0$};
	\node[state] (Q1) at (2.5,0) {$1$};
	\node[state] (Q2) at (5,0)   {$2$};
	\node[state] (Q3) at (7.5,0) {$3$};
	\node[state] (Q4) at (10,0)  {$4$};

	% --- рёбра ---
	\path[e, loop above]   (Q0) edge node[lbl] {$p_0$} (Q0);
	\path[e, bend left=12] (Q0) edge node[lbl] {$p_1$} (Q1);
	\path[e, bend left=56] (Q0) edge node[lbl] {$p_2$} (Q2);

	\path[e, bend left=12] (Q1) edge node[lbl] {$p_0$} (Q0);
	\path[e, loop above]   (Q1) edge node[lbl] {$p_1$} (Q1);
	\path[e, bend left=12] (Q1) edge node[lbl] {$p_2$} (Q2);

	\path[e, bend left=12] (Q2) edge node[lbl] {$p_0$} (Q1);
	\path[e, loop above]   (Q2) edge node[lbl] {$p_1$} (Q2);
	\path[e, bend left=12] (Q2) edge node[lbl] {$p_2$} (Q3);

	\path[e, bend left=12] (Q3) edge node[lbl] {$p_0$} (Q2);
	\path[e, loop above]   (Q3) edge node[lbl] {$p_1$} (Q3);
	\path[e, bend left=12] (Q3) edge node[lbl] {$p_2$} (Q4);

	\path[e, bend left=12] (Q4) edge node[lbl] {$p_0$} (Q3);
	\path[e, loop above]   (Q4) edge node[lbl] {$1-p_0$} (Q4);
\end{tikzpicture}

Найти: $P$, $\overline{r}$ (стационарное распределение), $que$.

\paragraph{Решение.}

Матрица переходных вероятностей:
\[
	P =
	\begin{array}{c|ccccc}
		    & Q_0 & Q_1 & Q_2 & Q_3 & Q_4   \\
		\hline
		Q_0 & p_0 & p_1 & p_2 & 0   & 0     \\
		Q_1 & p_0 & p_1 & p_2 & 0   & 0     \\
		Q_2 & 0   & p_0 & p_1 & p_2 & 0     \\
		Q_3 & 0   & 0   & p_0 & p_1 & p_2   \\
		Q_4 & 0   & 0   & 0   & p_0 & 1-p_0
	\end{array}
\]

Стационарное распределение $\overline r=(r_0,r_1,r_2,r_3,r_4)$:
\[
	\overline r P = \overline r,\qquad \sum_{i=0}^4 r_i = 1,\qquad r_i\ge 0.
\]

Линейная система из $\overline r P=\overline r$:
\[
	\begin{aligned}
		r_0 & = p_0 r_0 + p_0 r_1,                     \\
		r_1 & = p_1 r_0 + p_1 r_1 + p_0 r_2,           \\
		r_2 & = p_2 r_0 + p_2 r_1 + p_1 r_2 + p_0 r_3, \\
		r_3 & = p_2 r_2 + p_1 r_3 + p_0 r_4,           \\
		r_4 & = p_2 r_3 + (1-p_0) r_4.
	\end{aligned}
\]

Среднее число в системе:
$g(Q_i)=v_i$ — физическое значение в состоянии $Q_i$. Тогда
$$que=\sum_{i=0}^{4} v_i r_i$$

\subsubsection*{Задача 4.1}

При $p_0=\tfrac{1}{2},\; p_1=\tfrac{1}{4},\; p_2=\tfrac{1}{4}$:
\[
	\begin{aligned}
		r_0 & = \tfrac{1}{2}r_0 + \tfrac{1}{2}r_1,                                     \\
		r_1 & = \tfrac{1}{4}r_0 + \tfrac{1}{4}r_1 + \tfrac{1}{2}r_2,                   \\
		r_2 & = \tfrac{1}{4}r_0 + \tfrac{1}{4}r_1 + \tfrac{1}{4}r_2 + \tfrac{1}{2}r_3, \\
		r_3 & = \tfrac{1}{4}r_2 + \tfrac{1}{4}r_3 + \tfrac{1}{2}r_4,                   \\
		r_4 & = \tfrac{1}{4}r_3 + \tfrac{1}{2}r_4,\qquad
		r_0+r_1+r_2+r_3+r_4=1.
	\end{aligned}
\]

\[
	\begin{array}{rcl}
		\begin{aligned}
			2r_0 & = r_0 + r_1,              \\
			4r_1 & = r_0 + r_1 + 2r_2,       \\
			4r_2 & = r_0 + r_1 + r_2 + 2r_3, \\
			4r_3 & = r_2 + r_3 + 2r_4,       \\
			2r_4 & = r_3 + 2r_4,
		\end{aligned}
		 & \qquad \Rightarrow \qquad &
		\begin{aligned}
			r_0 - r_1                  & = 0, \\
			-\,r_0 + 3r_1 - 2r_2       & = 0, \\
			-\,r_0 - r_1 + 3r_2 - 2r_3 & = 0, \\
			-\,r_2 + 3r_3 - 2r_4       & = 0, \\
			r_0+r_1+r_2+r_3+r_4        & =1.
		\end{aligned}
	\end{array}
\]

\[
	\left[
		\begin{array}{rrrrr|r}
			1  & -1 & 0  & 0  & 0  & 0 \\
			-1 & 3  & -2 & 0  & 0  & 0 \\
			-1 & -1 & 3  & -2 & 0  & 0 \\
			0  & 0  & -1 & 3  & -2 & 0 \\
			1  & 1  & 1  & 1  & 1  & 1
		\end{array}
		\right]
	\overset{}{\longrightarrow}
	\left[
		\begin{array}{rrrrr|r}
			1 & 0 & 0 & 0 & 0 & \tfrac{4}{15} \\
			0 & 1 & 0 & 0 & 0 & \tfrac{4}{15} \\
			0 & 0 & 1 & 0 & 0 & \tfrac{4}{15} \\
			0 & 0 & 0 & 1 & 0 & \tfrac{2}{15} \\
			0 & 0 & 0 & 0 & 1 & \tfrac{1}{15}
		\end{array}
		\right]
\]

\[
	\overline r =
	\left(\tfrac{4}{15},\;\tfrac{4}{15},\;\tfrac{4}{15},\;\tfrac{2}{15},\;\tfrac{1}{15}\right).
\]

\[
	que = \sum_{i=0}^{4} v_i r_i= 0\cdot\tfrac{4}{15} + 1\cdot\tfrac{4}{15} + 2\cdot\tfrac{4}{15} + 3\cdot\tfrac{2}{15} + 4\cdot\tfrac{1}{15} = \frac{22}{15} \approx 1.4667.
\]

\[
	\boxed{\ que=\frac{22}{15}\approx1.4667\ }.
\]

\subsubsection*{Задача 4.2}

При вероятностях $p_0=\tfrac{1}{4},\; p_1=\tfrac{1}{2},\; p_2=\tfrac{1}{4}$

\[
	\begin{aligned}
		r_0 & = \tfrac{1}{4}r_0 + \tfrac{1}{4}r_1,                                     \\
		r_1 & = \tfrac{1}{2}r_0 + \tfrac{1}{2}r_1 + \tfrac{1}{4}r_2,                   \\
		r_2 & = \tfrac{1}{4}r_0 + \tfrac{1}{2}r_1 + \tfrac{1}{4}r_2 + \tfrac{1}{4}r_3, \\
		r_3 & = \tfrac{1}{4}r_2 + \tfrac{1}{2}r_3 + \tfrac{1}{4}r_4,                   \\
		r_4 & = \tfrac{1}{4}r_3 + \tfrac{3}{4}r_4,\qquad
		r_0+r_1+r_2+r_3+r_4=1.
	\end{aligned}
\]

\[
	\begin{array}{rcl}
		\begin{aligned}
			4r_0 & = r_0 + r_1,              \\
			4r_1 & = 2r_0 + 2r_1 + r_2,      \\
			4r_2 & = r_0 + 2r_1 + r_2 + r_3, \\
			4r_3 & = r_2 + 2r_3 + r_4,       \\
			4r_4 & = r_3 + 3r_4,
		\end{aligned}
		 & \qquad \Rightarrow \qquad &
		\begin{aligned}
			3r_0 - r_1                 & = 0, \\
			-\,2r_0 + 2r_1 - r_2       & = 0, \\
			-\,r_0 - 2r_1 + 3r_2 - r_3 & = 0, \\
			-\,r_2 + 2r_3 - r_4        & = 0, \\
			r_0+r_1+r_2+r_3+r_4        & = 1.
		\end{aligned}
	\end{array}
\]

\[
	\left[
		\begin{array}{rrrrr|r}
			3  & -1 & 0  & 0  & 0  & 0 \\
			-2 & 2  & -1 & 0  & 0  & 0 \\
			-1 & -2 & 3  & -1 & 0  & 0 \\
			0  & 0  & -1 & 2  & -1 & 0 \\
			1  & 1  & 1  & 1  & 1  & 1
		\end{array}
		\right]
	\longrightarrow
	\left[
		\begin{array}{rrrrr|r}
			1 & 0 & 0 & 0 & 0 & \tfrac{1}{16} \\
			0 & 1 & 0 & 0 & 0 & \tfrac{3}{16} \\
			0 & 0 & 1 & 0 & 0 & \tfrac{1}{4}  \\
			0 & 0 & 0 & 1 & 0 & \tfrac{1}{4}  \\
			0 & 0 & 0 & 0 & 1 & \tfrac{1}{4}
		\end{array}
		\right]
\]

\[
	\overline r=\left(\tfrac{1}{16},\,\tfrac{3}{16},\,\tfrac{1}{4},\,\tfrac{1}{4},\,\tfrac{1}{4}\right).
\]

\[
	que=\sum_{i=0}^{4} v_i r_i
	= 0\cdot\tfrac{1}{16}+1\cdot\tfrac{3}{16}+2\cdot\tfrac{1}{4}+3\cdot\tfrac{1}{4}+4\cdot\tfrac{1}{4}
	= \frac{39}{16}=2.4375.
\]

\[
	\boxed{\,que=\frac{39}{16}\approx 2.4375\, }.
\]

\subsubsection*{Задача 4.3}

При вероятностях $p_0=\tfrac{1}{4},\; p_1=\tfrac{1}{4},\; p_2=\tfrac{1}{2}$

\[
	\begin{aligned}
		r_0 & = \tfrac{1}{4}r_0 + \tfrac{1}{4}r_1,                                     \\
		r_1 & = \tfrac{1}{4}r_0 + \tfrac{1}{4}r_1 + \tfrac{1}{4}r_2,                   \\
		r_2 & = \tfrac{1}{2}r_0 + \tfrac{1}{2}r_1 + \tfrac{1}{4}r_2 + \tfrac{1}{4}r_3, \\
		r_3 & = \tfrac{1}{2}r_2 + \tfrac{1}{4}r_3 + \tfrac{1}{4}r_4,                   \\
		r_4 & = \tfrac{1}{2}r_3 + \tfrac{3}{4}r_4,\qquad
		r_0+r_1+r_2+r_3+r_4=1.
	\end{aligned}
\]

\[
	\begin{array}{rcl}
		\begin{aligned}
			4r_0 & = r_0 + r_1,               \\
			4r_1 & = r_0 + r_1 + r_2,         \\
			4r_2 & = 2r_0 + 2r_1 + r_2 + r_3, \\
			4r_3 & = 2r_2 + r_3 + r_4,        \\
			4r_4 & = 2r_3 + 3r_4,
		\end{aligned}
		 & \Rightarrow &
		\begin{aligned}
			3r_0 - r_1                  & = 0, \\
			-\,r_0 + 3r_1 - r_2         & = 0, \\
			-\,2r_0 - 2r_1 + 3r_2 - r_3 & = 0, \\
			-\,2r_2 + 3r_3 - r_4        & = 0, \\
			-\,2r_3 + r_4               & = 0, \\
			r_0+r_1+r_2+r_3+r_4         & = 1.
		\end{aligned}
	\end{array}
\]

\[
	\left[
		\begin{array}{rrrrr|r}
			3  & -1 & 0  & 0  & 0  & 0 \\
			-1 & 3  & -1 & 0  & 0  & 0 \\
			-2 & -2 & 3  & -1 & 0  & 0 \\
			0  & 0  & -2 & 3  & -1 & 0 \\
			0  & 0  & 0  & -2 & 1  & 0 \\
			1  & 1  & 1  & 1  & 1  & 1
		\end{array}
		\right]
	\longrightarrow
	\left[
		\begin{array}{rrrrr|r}
			1 & 0 & 0 & 0 & 0 & \tfrac{1}{60} \\
			0 & 1 & 0 & 0 & 0 & \tfrac{1}{20} \\
			0 & 0 & 1 & 0 & 0 & \tfrac{2}{15} \\
			0 & 0 & 0 & 1 & 0 & \tfrac{4}{15} \\
			0 & 0 & 0 & 0 & 1 & \tfrac{8}{15}
		\end{array}
		\right]
\]

\[
	\overline r=\left(\tfrac{1}{60},\,\tfrac{1}{20},\,\tfrac{2}{15},\,\tfrac{4}{15},\,\tfrac{8}{15}\right).
\]

\[
	que=\sum_{i=0}^{4} v_i r_i
	= 0\cdot\tfrac{1}{60}+1\cdot\tfrac{1}{20}+2\cdot\tfrac{2}{15}
	+3\cdot\tfrac{4}{15}+4\cdot\tfrac{8}{15}
	= \frac{13}{4}=3.25.
\]
\[
	\boxed{\,que=\frac{13}{4}=3.25\, }.
\]


\subsection{Семинар 5}

\subsubsection*{Задача 5}

\begin{tikzpicture}[
		>=Stealth, thick,
		state/.style = {draw, circle, minimum size=9mm, font=\small},
		e/.style   = {->, shorten >=2pt, shorten <=2pt},
		lbl/.style  = {font=\scriptsize, fill=white, inner sep=1pt}
	]

	% --- вершины ---
	\node[state] (Q0) at (0,0) {$0$};
	\node[state] (Q1) at (2.5,0) {$1$};
	\node[state] (Q2) at (5,0) {$2$};
	\node[state] (Q3) at (7.5,0) {$3$};
	\node[state] (Q4) at (10,0) {$4$};
	\node[state] (Q5) at (12.5,0) {$\ldots$};

	% --- рёбра ---
	\path[e, loop above]   (Q0) edge node[lbl] {$p_0$} (Q0);
	\path[e, bend left=12] (Q0) edge node[lbl] {$p_1$} (Q1);
	\path[e, bend left=56] (Q0) edge node[lbl] {$p_2$} (Q2);

	\path[e, bend left=12] (Q1) edge node[lbl] {$p_0$} (Q0);
	\path[e, loop above]   (Q1) edge node[lbl] {$p_1$} (Q1);
	\path[e, bend left=12] (Q1) edge node[lbl] {$p_2$} (Q2);

	\path[e, bend left=12] (Q2) edge node[lbl] {$p_0$} (Q1);
	\path[e, loop above]   (Q2) edge node[lbl] {$p_1$} (Q2);
	\path[e, bend left=12] (Q2) edge node[lbl] {$p_2$} (Q3);

	\path[e, bend left=12] (Q3) edge node[lbl] {$p_0$} (Q2);
	\path[e, loop above]   (Q3) edge node[lbl] {$p_1$} (Q3);
	\path[e, bend left=12] (Q3) edge node[lbl] {$p_2$} (Q4);

	\path[e, bend left=12] (Q4) edge node[lbl] {$p_0$} (Q3);
	\path[e, loop above]   (Q4) edge node[lbl] {$p_1$} (Q4);
	\path[e, bend left=12] (Q4) edge node[lbl] {$p_2$} (Q5);

\end{tikzpicture}

Найти: $P$, $\overline{r}$ (стационарное распределение), que.
\paragraph{Решение.}

\[
	P =
	\begin{array}{c|cccccc}
		       & Q_0    & Q_1    & Q_2    & Q_3    & Q_4    & \ldots \\ \hline
		Q_0    & p_0    & p_1    & p_2    & 0      & 0      & \ldots \\
		Q_1    & p_0    & p_1    & p_2    & 0      & 0      & \ldots \\
		Q_2    & 0      & p_0    & p_1    & p_2    & 0      & \ldots \\
		Q_3    & 0      & 0      & p_0    & p_1    & p_2    & \ldots \\
		Q_4    & 0      & 0      & 0      & p_0    & p_1    & \ldots \\
		\vdots & \vdots & \vdots & \vdots & \vdots & \vdots & \ddots
	\end{array}
\]

Стационарное распределение $\overline r=(r_0,r_1,r_2,r_3,\ldots)$ задаётся условиями
\[
	\overline r P = \overline r,\quad \sum_{i=0}^\infty r_i = 1,\quad r_i\ge0.
\]
Из матрицы $P$ получаем систему:

\[
	\begin{array}{rcl}
		\begin{aligned}
			r_0 & = p_0 r_0 + p_0 r_1,                                 \\
			r_1 & = p_1 r_0 + p_1 r_1 + p_0 r_2,                       \\
			r_2 & = p_2 r_0 + p_2 r_1 + p_1 r_2 + p_0 r_3,             \\
			r_k & = p_2 r_{k-1} + p_1 r_k + p_0 r_{k+1}, \quad k\ge 2.
		\end{aligned}
	\end{array}
\]


\subsubsection*{Задача 5.1}

При вероятностях $p_0=\tfrac{1}{2},\quad p_1=\tfrac{1}{4},\quad p_2=\tfrac{1}{4}$
\[
	\begin{array}{rcl}
		\begin{aligned}
			r_0 & = \tfrac{1}{2} r_0 + \tfrac{1}{2} r_1,                                           \\
			r_1 & = \tfrac{1}{4} r_0 + \tfrac{1}{4} r_1 + \tfrac{1}{2} r_2,                        \\
			r_2 & = \tfrac{1}{4} r_0 + \tfrac{1}{4} r_1 + \tfrac{1}{4} r_2 + \tfrac{1}{2} r_3,     \\
			r_k & = \tfrac{1}{4} r_{k-1} + \tfrac{1}{4} r_k + \tfrac{1}{2} r_{k+1}, \quad k \ge 3.
		\end{aligned}
		 & \qquad \Rightarrow \qquad &
		\begin{aligned}
			r_0 & = r_0,                                 \\
			r_1 & = r_0,                                 \\
			r_2 & = \tfrac{1}{2} r_0,                    \\
			r_k & = \tfrac{1}{2^{k-2}} r_0,\quad k\ge 3.
		\end{aligned}
	\end{array}
\]

\[
	\sum_{i=0}^\infty r_i
	= r_0 + r_0 + \tfrac{1}{2}r_0 + \tfrac{1}{4}r_0 + \tfrac{1}{8}r_0 + \ldots
	= r_0\!\left(1 + 1 + \tfrac{1}{2} + \sum_{k=3}^\infty 2^{-(k-1)}\right)
	= 4r_0 = 1 \;\Rightarrow\; r_0 = \tfrac{1}{4}.
\]


Следовательно,
\[
	\overline r = \left(\tfrac{1}{4},\ \tfrac{1}{4},\ \tfrac{1}{4},\ \tfrac{1}{8},\ \tfrac{1}{16},\ \tfrac{1}{32},\ \ldots\right).
\]

\[
	que = \sum_{i=0}^\infty v_i r_i =
	= \tfrac{1}{4} + \tfrac{2}{8} + \sum_{k=3}^\infty k \cdot \tfrac{1}{2^{k+1}}
	= \tfrac{1}{4} + \tfrac{1}{4} + \tfrac{1}{2}
	= 1.
\]


\subsubsection*{Задача 5.2}

При вероятностях $p_0=\tfrac{1}{4},\quad p_1=\tfrac{1}{2},\quad p_2=\tfrac{1}{4}$
\[
	\begin{array}{rcl}
		\begin{aligned}
			r_0 & = \tfrac{1}{4} r_0 + \tfrac{1}{4} r_1,                                           \\
			r_1 & = \tfrac{1}{2} r_0 + \tfrac{1}{2} r_1 + \tfrac{1}{4} r_2,                        \\
			r_2 & = \tfrac{1}{4} r_0 + \tfrac{1}{2} r_1 + \tfrac{1}{4} r_2 + \tfrac{1}{4} r_3,     \\
			r_k & = \tfrac{1}{4} r_{k-1} + \tfrac{1}{2} r_k + \tfrac{1}{4} r_{k+1}, \quad k \ge 3.
		\end{aligned}
		 & \qquad \Rightarrow \qquad &
		\begin{aligned}
			r_0 & = r_0,                 \\
			r_1 & = 3r_0,                \\
			r_2 & = 4r_0,                \\
			r_k & = 4r_0, \quad k \ge 3.
		\end{aligned}
	\end{array}
\]

Нормированного решения нет, стационарного распределения не существует.


\subsection{Семинар 6}

\subsubsection*{Задача 6.1}
