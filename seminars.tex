\section{Семинары}

\subsection{Семинар 1}

Задача с книгами с Книгами

\begin{tikzpicture}[
		>=Stealth, thick,
		state/.style = {draw, circle, minimum size=9mm, font=\small},
		e1/.style   = {->, shorten >=2pt, shorten <=2pt},
		e2/.style   = {->, shorten >=2pt, shorten <=2pt},
		e3/.style   = {->, shorten >=2pt, shorten <=2pt},
		lbl/.style  = {font=\scriptsize, fill=white, inner sep=1pt}
	]

	% --- вероятности ---
	\newcommand{\p}[1]{p_{#1}}
	\renewcommand{\p}[1]{\ifcase#1\relax \or \frac{1}{2}\or \frac{1}{3}\or \frac{1}{6}\fi}

	% --- вершины ---
	\node[state] (Q1) at (90:3)   {$123$};
	\node[state] (Q2) at (30:3)   {$132$};
	\node[state] (Q3) at (-30:3)  {$213$};
	\node[state] (Q4) at (-90:3)  {$231$};
	\node[state] (Q5) at (-150:3) {$312$};
	\node[state] (Q6) at (150:3)  {$321$};

	% --- рёбра ---
	% Q1 = 123
	\path[e1, bend left=12] (Q1) edge node[lbl] {$\p{1}$} (Q4);
	\path[e2, bend left=12] (Q1) edge node[lbl] {$\p{2}$} (Q2);
	\path[e3, loop above]   (Q1) edge node[lbl] {$\p{3}$} (Q1);

	% Q2 = 132
	\path[e1, bend left=12] (Q2) edge node[lbl] {$\p{1}$} (Q6);
	\path[e2, loop right]   (Q2) edge node[lbl] {$\p{2}$} (Q2);
	\path[e3, bend left=12] (Q2) edge node[lbl] {$\p{3}$} (Q1);

	% Q3 = 213
	\path[e1, bend left=12] (Q3) edge node[lbl] {$\p{1}$} (Q4);
	\path[e2, bend left=12] (Q3) edge node[lbl] {$\p{2}$} (Q2);
	\path[e3, loop right]   (Q3) edge node[lbl] {$\p{3}$} (Q3);

	% Q4 = 231
	\path[e1, loop below]   (Q4) edge node[lbl] {$\p{1}$} (Q4);
	\path[e2, bend left=12] (Q4) edge node[lbl] {$\p{2}$} (Q5);
	\path[e3, bend left=12] (Q4) edge node[lbl] {$\p{3}$} (Q3);

	% Q5 = 312
	\path[e1, bend left=12] (Q5) edge node[lbl] {$\p{1}$} (Q6);
	\path[e2, loop left]    (Q5) edge node[lbl] {$\p{2}$} (Q5);
	\path[e3, bend left=12] (Q5) edge node[lbl] {$\p{3}$} (Q1);

	% Q6 = 321
	\path[e1, loop left]    (Q6) edge node[lbl] {$\p{1}$} (Q6);
	\path[e2, bend left=12] (Q6) edge node[lbl] {$\p{2}$} (Q5);
	\path[e3, bend left=12] (Q6) edge node[lbl] {$\p{3}$} (Q2);

\end{tikzpicture}

\[
	P =
	\begin{array}{c|cccccc}
		    & Q_1          & Q_2          & Q_3          & Q_4          & Q_5          & Q_6          \\
		\hline
		Q_1 & \tfrac{1}{6} & \tfrac{1}{3} & 0            & \tfrac{1}{2} & 0            & 0            \\
		Q_2 & \tfrac{1}{6} & \tfrac{1}{3} & 0            & 0            & 0            & \tfrac{1}{2} \\
		Q_3 & 0            & \tfrac{1}{3} & \tfrac{1}{6} & \tfrac{1}{2} & 0            & 0            \\
		Q_4 & 0            & 0            & \tfrac{1}{6} & \tfrac{1}{2} & \tfrac{1}{3} & 0            \\
		Q_5 & \tfrac{1}{6} & 0            & 0            & 0            & \tfrac{1}{3} & \tfrac{1}{2} \\
		Q_6 & 0            & \tfrac{1}{6} & 0            & 0            & \tfrac{1}{3} & \tfrac{1}{2}
	\end{array}
\]


Стационарное распределение $\overline r=(r_1,r_2,r_3,r_4,r_5,r_6)$
задаётся условиями
\[
	\overline r P = \overline r,
	\qquad \sum_{i=1}^6 r_i = 1,
	\qquad r_i \ge 0.
\]

Из матрицы $P$ получаем систему:
\[
	\begin{aligned}
		r_1 & = \tfrac{1}{6}r_1 + \tfrac{1}{6}r_2 + \tfrac{1}{6}r_5,                   \\
		r_2 & = \tfrac{1}{3}r_1 + \tfrac{1}{3}r_2 + \tfrac{1}{3}r_3 + \tfrac{1}{6}r_6, \\
		r_3 & = \tfrac{1}{6}r_3 + \tfrac{1}{6}r_4,                                     \\
		r_4 & = \tfrac{1}{2}r_1 + \tfrac{1}{2}r_3 + \tfrac{1}{2}r_4,                   \\
		r_5 & = \tfrac{1}{3}r_4 + \tfrac{1}{3}r_5 + \tfrac{1}{3}r_6,                   \\
		r_6 & = \tfrac{1}{2}r_2 + \tfrac{1}{2}r_5 + \tfrac{1}{2}r_6.
	\end{aligned}
\]

$\Rightarrow$

\[
	\begin{aligned}
		6r_1 & = r_1 + r_2 + r_5,          \\
		6r_2 & = 2r_1 + 2r_2 + 2r_3 + r_6, \\
		6r_3 & = r_3 + r_4,                \\
		2r_4 & = r_1 + r_3 + r_4,          \\
		3r_5 & = r_4 + r_5 + r_6,          \\
		2r_6 & = r_2 + r_5 + r_6.
	\end{aligned}
\]

$\Rightarrow$

\[
	\begin{aligned}
		5r_1 - r_2 - r_5          & = 0,                           \\
		-2r_1 + 4r_2 - 2r_3 - r_6 & = 0,                           \\
		5r_3 - r_4                & = 0,                           \\
		-\,r_1 - r_3 + r_4        & = 0,                           \\
		-\,r_4 + 2r_5 - r_6       & = 0,                           \\
		-\,r_2 - r_5 + r_6        & = 0,                           \\
		r_1+r_2+r_3+r_4+r_5+r_6   & = 1 \quad \text{(нормировка).}
	\end{aligned}
\]


\[
	\underbrace{
		\begin{pmatrix}
			5  & -1 & 0  & 0  & -1 & 0  \\
			-2 & 4  & -2 & 0  & 0  & -1 \\
			0  & 0  & 5  & -1 & 0  & 0  \\
			-1 & 0  & -1 & 1  & 0  & 0  \\
			0  & 0  & 0  & -1 & 2  & -1 \\
			1  & 1  & 1  & 1  & 1  & 1
		\end{pmatrix}}_{\displaystyle A}
	\;
	\underbrace{
		\begin{pmatrix}
			r_1 \\ r_2\\ r_3\\ r_4\\ r_5\\ r_6
		\end{pmatrix}}_{\displaystyle x}
	=
	\underbrace{
		\begin{pmatrix}
			0 \\ 0\\ 0\\ 0\\ 0\\ 1
		\end{pmatrix}}_{\displaystyle b}.
\]

\[
	\left[
		\begin{array}{rrrrrr|r}
			5  & -1 & 0  & 0  & -1 & 0  & 0 \\
			-2 & 4  & -2 & 0  & 0  & -1 & 0 \\
			0  & 0  & 5  & -1 & 0  & 0  & 0 \\
			-1 & 0  & -1 & 1  & 0  & 0  & 0 \\
			0  & 0  & 0  & -1 & 2  & -1 & 0 \\
			1  & 1  & 1  & 1  & 1  & 1  & 1
		\end{array}
		\right]=[\,A\,|\,b\,].
\]

\[
	\overline r =
	\left(\tfrac{4}{43},\;\tfrac{15}{86},\;\tfrac{1}{43},\;\tfrac{5}{43},\;\tfrac{25}{86},\;\tfrac{20}{43}\right).
\]