\section{Цепи Маркова}

\subsection*{Общая теория}

Стол прозрачный, граф круглый.

\subsection*{Задачи}
\begin{problem}

\begin{condition}
	(Nun)

	В цепи Маркова дана матрица $P$ вероятностей перехода за один шаг:
	$p_{13}=0$, $p_{12}=p_{23}=p_{33}=0.25$,
	$p_{22}=0.625$, $p_{31}=0.5$
	и начальное распределение вероятностей состояний $\vec{p}(0)=(0;1;0)$.
	Построить граф цепи Маркова, найти $\vec{p}(2)$ и стационарное распределение.
\end{condition}

\begin{solution}

	Матрица переходных вероятностей:
	$$
		P =
		\begin{pmatrix}
			\frac{3}{4} & \frac{1}{4} & 0           \\
			\frac{1}{8} & \frac{5}{8} & \frac{1}{4} \\
			\frac{1}{2} & \frac{1}{4} & \frac{1}{4}
		\end{pmatrix}.
	$$

	$$
		\vec{p}(2)=\vec{p}(0)P^2
		=(0;1;0)P^2.
	$$

	$$
		\vec{p}(1)=\vec{p}(0)P
		=(0;1;0)
		\begin{pmatrix}
			\frac{3}{4} & \frac{1}{4} & 0           \\
			\frac{1}{8} & \frac{5}{8} & \frac{1}{4} \\
			\frac{1}{2} & \frac{1}{4} & \frac{1}{4}
		\end{pmatrix}
		=\left(\frac{1}{8};\frac{5}{8};\frac{1}{4}\right).
	$$

	$$
		\vec{p}(2)=\vec{p}(1)P
		=\left(\frac{1}{8};\frac{5}{8};\frac{1}{4}\right)
		\begin{pmatrix}
			\frac{3}{4} & \frac{1}{4} & 0           \\
			\frac{1}{8} & \frac{5}{8} & \frac{1}{4} \\
			\frac{1}{2} & \frac{1}{4} & \frac{1}{4}
		\end{pmatrix}
		=\left(\frac{19}{64};\frac{31}{64};\frac{7}{32}\right).
	$$

	Стационарное распределение $\bar r=(r_1,r_2,r_3)$:
	$$
		\bar r=(r_1,r_2,r_3), \qquad \bar r P=\bar r, \qquad r_1+r_2+r_3=1.
	$$

	$$
		\begin{cases}
			r_1=\frac{3}{4}r_1+\frac{1}{8}r_2+\frac{1}{2}r_3, \\
			r_2=\frac{1}{4}r_1+\frac{5}{8}r_2+\frac{1}{4}r_3, \\
			r_3=\frac{1}{4}r_2+\frac{1}{4}r_3,                \\
			r_1+r_2+r_3=1.
		\end{cases}
		\Rightarrow
		\begin{cases}
			8r_1 = 6r_1 + r_2 + 4r_3,  \\
			8r_2 = 2r_1 + 5r_2 + 2r_3, \\
			8r_3 = 2r_2 + 2r_3,        \\
			r_1 + r_2 + r_3 = 1.
		\end{cases}
	$$


	$$
		\bar r=\left(\frac{19}{64};\frac{31}{64};\frac{7}{32}\right).
	$$

\end{solution}

\begin{conclusion}

\end{conclusion}

\end{problem}



\begin{problem}

\begin{condition}
	(16.1)

	В цепи Маркова дана матрица $P$ вероятностей перехода за один шаг:
	$p_{13}=p_{22}=p_{31}=0$,
	$p_{12}=0.8$, $p_{21}=0.5$, $p_{32}=0.6$
	и начальное распределение вероятностей состояний $\vec{p}(0)=(0;1;0)$.
	Построить граф цепи Маркова, найти $\vec{p}(2)$ и стационарное распределение.
\end{condition}

\begin{solution}

\end{solution}

\begin{conclusion}

\end{conclusion}

\end{problem}



\begin{problem}

\begin{condition}
	(15.1)

	В цепи Маркова дана матрица $P$ вероятностей перехода за один шаг:
	$p_{11}=p_{22}=p_{33}=0$,
	$p_{12}=p_{31}=0.75$, $p_{23}=0.5$
	и начальное распределение вероятностей состояний $\vec{p}(0)=(0;0;1)$.
	Построить граф цепи Маркова, найти $\vec{p}(2)$ и стационарное распределение.
\end{condition}

\begin{solution}

\end{solution}

\begin{conclusion}

\end{conclusion}

\end{problem}



\begin{problem}

\begin{condition}
	(31.1)

	В цепи Маркова дана матрица $P$ вероятностей перехода за один шаг:
	$p_{32}=p_{21}=p_{12}=0$,
	$p_{22}=0.2$, $p_{11}=p_{33}=0.5$
	и начальное распределение вероятностей состояний $\vec{p}(0)=(1;0;0)$.
	Построить граф цепи Маркова, найти $\vec{p}(2)$ и стационарное распределение.
\end{condition}

\begin{solution}

\end{solution}

\begin{conclusion}

\end{conclusion}

\end{problem}



\begin{problem}

\begin{condition}
	(11.1)

	В цепи Маркова дана матрица $P$ вероятностей перехода за один шаг:
	$p_{11}=0.25$,$p_{12}=p_{33}=p_{21}=0$,
	$p_{22}=0.4$, $p_{31}=0.5$
	и начальное распределение вероятностей состояний $\vec{p}(0)=(0;1;0)$.
	Построить граф цепи Маркова, найти $\vec{p}(2)$ и стационарное распределение.
\end{condition}

\begin{solution}

\end{solution}

\begin{conclusion}

\end{conclusion}

\end{problem}



\begin{problem}

\begin{condition}
	(12.1)

	В цепи Маркова дана матрица $P$ вероятностей перехода за один шаг:
	$p_{21}=0.75$,$p_{12}=p_{33}=p_{21}=0$,
	$p_{33}=0.375$, $p_{21}=0.5$
	и начальное распределение вероятностей состояний $\vec{p}(0)=(0;0;1)$.
	Построить граф цепи Маркова, найти $\vec{p}(2)$ и стационарное распределение.
\end{condition}

\begin{solution}

\end{solution}

\begin{conclusion}

\end{conclusion}

\end{problem}



\begin{problem}

\begin{condition}
	(33.1)

	В цепи Маркова дана матрица $P$ вероятностей перехода за один шаг:
	$p_{12}=p_{23}=p_{32}=0$,
	$p_{21}=0.8$, $p_{11}=p_{33}=0.5$
	и начальное распределение вероятностей состояний $\vec{p}(0)=(0;0;1)$.
	Построить граф цепи Маркова, найти $\vec{p}(2)$ и стационарное распределение.
\end{condition}

\begin{solution}

\end{solution}

\begin{conclusion}

\end{conclusion}

\end{problem}
