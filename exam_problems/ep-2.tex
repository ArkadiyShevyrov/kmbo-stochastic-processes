\section{Цепи Маркова}

\subsection*{Общая теория}

Стол прозрачный, граф круглый.

\subsection*{Задачи}

\begin{problem}

\begin{condition}
	(Nun)

	В цепи Маркова дана матрица $P$ вероятностей перехода за один шаг:
	$p_{13}=0$, $p_{12}=p_{23}=p_{33}=0.25$,
	$p_{22}=0.625$, $p_{31}=0.5$
	и начальное распределение вероятностей состояний $\vec{p}(0)=(0;1;0)$.
	Построить граф цепи Маркова, найти $\vec{p}(2)$ и стационарное распределение.
\end{condition}

\begin{solution}

	$$
		P =
		\begin{pmatrix}
			\frac{3}{4} & \frac{1}{4} & 0           \\
			\frac{1}{8} & \frac{5}{8} & \frac{1}{4} \\
			\frac{1}{2} & \frac{1}{4} & \frac{1}{4}
		\end{pmatrix}.
	$$

	$$
		\vec{p}(2)=\vec{p}(0)P^2
		=(0;1;0)P^2.
	$$

	$$
		\vec{p}(1)=\vec{p}(0)P
		=(0;1;0)
		\begin{pmatrix}
			\frac{3}{4} & \frac{1}{4} & 0           \\
			\frac{1}{8} & \frac{5}{8} & \frac{1}{4} \\
			\frac{1}{2} & \frac{1}{4} & \frac{1}{4}
		\end{pmatrix}
		=\left(\frac{1}{8};\frac{5}{8};\frac{1}{4}\right).
	$$

	$$
		\vec{p}(2)=\vec{p}(1)P
		=\left(\frac{1}{8};\frac{5}{8};\frac{1}{4}\right)
		\begin{pmatrix}
			\frac{3}{4} & \frac{1}{4} & 0           \\
			\frac{1}{8} & \frac{5}{8} & \frac{1}{4} \\
			\frac{1}{2} & \frac{1}{4} & \frac{1}{4}
		\end{pmatrix}
		=\left(\frac{19}{64};\frac{31}{64};\frac{7}{32}\right).
	$$

	$$
		\bar r=(r_1,r_2,r_3), \qquad \bar r P=\bar r, \qquad r_1+r_2+r_3=1.
	$$

	$$
		\begin{cases}
			r_1=\frac{3}{4}r_1+\frac{1}{8}r_2+\frac{1}{2}r_3, \\
			r_2=\frac{1}{4}r_1+\frac{5}{8}r_2+\frac{1}{4}r_3, \\
			r_3=\frac{1}{4}r_2+\frac{1}{4}r_3,                \\
			r_1+r_2+r_3=1.
		\end{cases}
		\Rightarrow
		\begin{cases}
			8r_1 = 6r_1 + r_2 + 4r_3,  \\
			8r_2 = 2r_1 + 5r_2 + 2r_3, \\
			4r_3 = r_2 + r_3,          \\
			r_1 + r_2 + r_3 = 1.
		\end{cases}
	$$

	$$
		\begin{cases}
			2r_1 =  r_2 + 4r_3, \\
			3r_2 = 2r_1 + 2r_3, \\
			3r_3 = r_2,         \\
			r_1 + r_2 + r_3 = 1.
		\end{cases}
		\Rightarrow
		\begin{cases}
			2r_1 =  7r_3, \\
			7r_3 = 2r_1,  \\
			3r_3 = r_2,   \\
			r_1 + r_2 + r_3 = 1.
		\end{cases}
	$$

	$$
		r_3(\frac{7}{2}+3+1)=1
		\Rightarrow
		r_3=\frac{2}{15}
	$$


	$$
		\bar r=\left(\frac{7}{15};\frac{2}{5};\frac{2}{15}\right).
	$$

\end{solution}

\begin{conclusion}

	$$
		\vec{p}(2)=\left(\frac{19}{64};\frac{31}{64};\frac{7}{32}\right),
		\bar r=\left(\frac{7}{15};\frac{2}{5};\frac{2}{15}\right).
	$$
\end{conclusion}

\end{problem}



\begin{problem}

\begin{condition}
	(16.1)

	В цепи Маркова дана матрица $P$ вероятностей перехода за один шаг:
	$p_{13}=p_{22}=p_{31}=0$,
	$p_{12}=0.8$, $p_{21}=0.5$, $p_{32}=0.6$
	и начальное распределение вероятностей состояний $\vec{p}(0)=(0;1;0)$.
	Построить граф цепи Маркова, найти $\vec{p}(2)$ и стационарное распределение.
\end{condition}

\begin{solution}

	$$
		P =
		\begin{pmatrix}
			\frac{1}{5} & \frac{4}{5} & 0           \\
			\frac{1}{2} & 0           & \frac{1}{2} \\
			0           & \frac{3}{5} & \frac{2}{5}
		\end{pmatrix}.
	$$

	$$
		\vec{p}(2)=\vec{p}(0)P^2
		=(0;1;0)P^2.
	$$

	$$
		\vec{p}(1)=\vec{p}(0)P
		=(0;1;0)
		\begin{pmatrix}
			\frac{1}{5} & \frac{4}{5} & 0           \\
			\frac{1}{2} & 0           & \frac{1}{2} \\
			0           & \frac{3}{5} & \frac{2}{5}
		\end{pmatrix}
		=\left(\frac{1}{2};0;\frac{1}{2}\right).
	$$

	$$
		\vec{p}(2)=\vec{p}(1)P
		=\left(\frac{1}{2};0;\frac{1}{2}\right)
		\begin{pmatrix}
			\frac{1}{5} & \frac{4}{5} & 0           \\
			\frac{1}{2} & 0           & \frac{1}{2} \\
			0           & \frac{3}{5} & \frac{2}{5}
		\end{pmatrix}
		=\left(\frac{1}{10};\frac{7}{10};\frac{1}{5}\right).
	$$

	$$
		\bar r=(r_1,r_2,r_3), \qquad \bar r P=\bar r, \qquad r_1+r_2+r_3=1.
	$$

	$$
		\begin{cases}
			r_1=\frac{1}{5}r_1+\frac{1}{2}r_2, \\
			r_2=\frac{4}{5}r_1+\frac{3}{5}r_3, \\
			r_3=\frac{1}{2}r_2+\frac{2}{5}r_3, \\
			r_1+r_2+r_3=1.
		\end{cases}
		\Rightarrow
		\begin{cases}
			10r_1=2r_1+5r_2, \\
			5r_2=4r_1+3r_3,  \\
			10r_3=5r_2+4r_3, \\
			r_1+r_2+r_3=1.
		\end{cases}
	$$

	$$
		\begin{cases}
			8r_1=5r_2,      \\
			5r_2=4r_1+3r_3, \\
			6r_3=5r_2,      \\
			r_1+r_2+r_3=1.
		\end{cases}
		\Rightarrow
		\begin{cases}
			r_1=\frac{3}{4}r_3, \\
			r_2=\frac{6}{5}r_3, \\
			r_1+r_2+r_3=1.
		\end{cases}
	$$

	$$
		r_3\left(\frac{3}{4}+\frac{6}{5}+1\right)=1
		\Rightarrow
		r_3=\frac{20}{59}.
	$$

	$$
		\bar r=\left(\frac{15}{59};\frac{24}{59};\frac{20}{59}\right).
	$$

\end{solution}

\begin{conclusion}

	$$
		\vec{p}(2)=\left(\frac{1}{10};\frac{7}{10};\frac{1}{5}\right),
		\bar r=\left(\frac{15}{59};\frac{24}{59};\frac{20}{59}\right).
	$$

\end{conclusion}

\end{problem}



\begin{problem}

\begin{condition}
	(15.1)

	В цепи Маркова дана матрица $P$ вероятностей перехода за один шаг:
	$p_{11}=p_{22}=p_{33}=0$,
	$p_{12}=p_{31}=0.75$, $p_{23}=0.5$
	и начальное распределение вероятностей состояний $\vec{p}(0)=(0;0;1)$.
	Построить граф цепи Маркова, найти $\vec{p}(2)$ и стационарное распределение.
\end{condition}

\begin{solution}

	$$
		P =
		\begin{pmatrix}
			0           & \frac{3}{4} & \frac{1}{4} \\
			\frac{1}{2} & 0           & \frac{1}{2} \\
			\frac{3}{4} & \frac{1}{4} & 0
		\end{pmatrix}.
	$$

	$$
		\vec{p}(2)=\vec{p}(0)P^2
		=(0;0;1)P^2.
	$$

	$$
		\vec{p}(1)=\vec{p}(0)P
		=(0;0;1)
		\begin{pmatrix}
			0           & \frac{3}{4} & \frac{1}{4} \\
			\frac{1}{2} & 0           & \frac{1}{2} \\
			\frac{3}{4} & \frac{1}{4} & 0
		\end{pmatrix}
		=\left(\frac{3}{4};\frac{1}{4};0\right).
	$$

	$$
		\vec{p}(2)=\vec{p}(1)P
		=\left(\frac{3}{4};\frac{1}{4};0\right)
		\begin{pmatrix}
			0           & \frac{3}{4} & \frac{1}{4} \\
			\frac{1}{2} & 0           & \frac{1}{2} \\
			\frac{3}{4} & \frac{1}{4} & 0
		\end{pmatrix}
		=\left(\frac{1}{8};\frac{9}{16};\frac{5}{16}\right).
	$$

	$$
		\bar r=(r_1,r_2,r_3), \qquad \bar r P=\bar r, \qquad r_1+r_2+r_3=1.
	$$

	$$
		\begin{cases}
			r_1=\frac{1}{2}r_2+\frac{3}{4}r_3, \\
			r_2=\frac{3}{4}r_1+\frac{1}{4}r_3, \\
			r_3=\frac{1}{4}r_1+\frac{1}{2}r_2, \\
			r_1+r_2+r_3=1.
		\end{cases}
		\Rightarrow
		\begin{cases}
			4r_1=2r_2+3r_3, \\
			4r_2=3r_1+r_3,  \\
			4r_3=r_1+2r_2,  \\
			r_1+r_2+r_3=1.
		\end{cases}
	$$

	$$
		\begin{cases}
			r_2=\frac{13}{14}r_1, \\
			r_3=\frac{5}{7}r_1,   \\
			r_1+r_2+r_3=1.
		\end{cases}
	$$

	$$
		r_1\left(1+\frac{13}{14}+\frac{5}{7}\right)=1
		\Rightarrow
		r_1=\frac{14}{37}.
	$$

	$$
		\bar r=\left(\frac{14}{37};\frac{13}{37};\frac{10}{37}\right).
	$$

\end{solution}

\begin{conclusion}

	$$
		\vec{p}(2)=\left(\frac{1}{8};\frac{9}{16};\frac{5}{16}\right),
		\bar r=\left(\frac{14}{37};\frac{13}{37};\frac{10}{37}\right).
	$$

\end{conclusion}

\end{problem}



\begin{problem}

\begin{condition}
	(31.1)

	В цепи Маркова дана матрица $P$ вероятностей перехода за один шаг:
	$p_{32}=p_{21}=p_{12}=0$,
	$p_{22}=0.2$, $p_{11}=p_{33}=0.5$
	и начальное распределение вероятностей состояний $\vec{p}(0)=(1;0;0)$.
	Построить граф цепи Маркова, найти $\vec{p}(2)$ и стационарное распределение.
\end{condition}

\begin{solution}

	$$
		P =
		\begin{pmatrix}
			\frac{1}{2} & 0           & \frac{1}{2} \\
			0           & \frac{1}{5} & \frac{4}{5} \\
			\frac{1}{2} & 0           & \frac{1}{2}
		\end{pmatrix}.
	$$

	$$
		\vec{p}(2)=\vec{p}(0)P^2
		=(1;0;0)P^2.
	$$

	$$
		\vec{p}(1)=\vec{p}(0)P
		=(1;0;0)
		\begin{pmatrix}
			\frac{1}{2} & 0           & \frac{1}{2} \\
			0           & \frac{1}{5} & \frac{4}{5} \\
			\frac{1}{2} & 0           & \frac{1}{2}
		\end{pmatrix}
		=\left(\frac{1}{2};0;\frac{1}{2}\right).
	$$

	$$
		\vec{p}(2)=\vec{p}(1)P
		=\left(\frac{1}{2};0;\frac{1}{2}\right)
		\begin{pmatrix}
			\frac{1}{2} & 0           & \frac{1}{2} \\
			0           & \frac{1}{5} & \frac{4}{5} \\
			\frac{1}{2} & 0           & \frac{1}{2}
		\end{pmatrix}
		=\left(\frac{1}{2};0;\frac{1}{2}\right).
	$$

	$$
		\bar r=(r_1,r_2,r_3), \qquad \bar r P=\bar r, \qquad r_1+r_2+r_3=1.
	$$

	$$
		\begin{cases}
			r_1=\frac{1}{2}r_1+\frac{1}{2}r_3, \\
			r_2=\frac{1}{5}r_2,                \\
			r_3=\frac{1}{2}r_1+\frac{1}{2}r_3, \\
			r_1+r_2+r_3=1.
		\end{cases}
		\Rightarrow
		\begin{cases}
			2r_1=r_1+r_3, \\
			5r_2=r_2,     \\
			2r_3=r_1+r_3, \\
			r_1+r_2+r_3=1.
		\end{cases}
	$$

	$$
		\begin{cases}
			r_1=r_3, \\
			r_2=0,   \\
			r_1+r_3=1.
		\end{cases}
	$$

	$$
		r_1=r_3=\frac{1}{2}.
	$$

	$$
		\bar r=\left(\frac{1}{2};0;\frac{1}{2}\right).
	$$

\end{solution}

\begin{conclusion}

	$$
		\vec{p}(2)=\left(\frac{1}{2};0;\frac{1}{2}\right),
		\bar r=\left(\frac{1}{2};0;\frac{1}{2}\right).
	$$

\end{conclusion}

\end{problem}



\begin{problem}

\begin{condition}
	(11.1)

	В цепи Маркова дана матрица $P$ вероятностей перехода за один шаг:
	$p_{11}=0.25$,$p_{12}=p_{33}=p_{21}=0$,
	$p_{22}=0.4$, $p_{31}=0.5$
	и начальное распределение вероятностей состояний $\vec{p}(0)=(0;1;0)$.
	Построить граф цепи Маркова, найти $\vec{p}(2)$ и стационарное распределение.
\end{condition}

\begin{solution}

\end{solution}

\begin{conclusion}

\end{conclusion}

\end{problem}



\begin{problem}

\begin{condition}
	(12.1)

	В цепи Маркова дана матрица $P$ вероятностей перехода за один шаг:
	$p_{21}=0.75$,$p_{12}=p_{33}=p_{21}=0$,
	$p_{33}=0.375$, $p_{21}=0.5$
	и начальное распределение вероятностей состояний $\vec{p}(0)=(0;0;1)$.
	Построить граф цепи Маркова, найти $\vec{p}(2)$ и стационарное распределение.
\end{condition}

\begin{solution}

\end{solution}

\begin{conclusion}

\end{conclusion}

\end{problem}



\begin{problem}

\begin{condition}
	(33.1)

	В цепи Маркова дана матрица $P$ вероятностей перехода за один шаг:
	$p_{12}=p_{23}=p_{32}=0$,
	$p_{21}=0.8$, $p_{11}=p_{33}=0.5$
	и начальное распределение вероятностей состояний $\vec{p}(0)=(0;0;1)$.
	Построить граф цепи Маркова, найти $\vec{p}(2)$ и стационарное распределение.
\end{condition}

\begin{solution}

\end{solution}

\begin{conclusion}

\end{conclusion}

\end{problem}
