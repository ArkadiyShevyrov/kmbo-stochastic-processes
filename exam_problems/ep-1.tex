\section{Интенсивности марковского процесса}

\subsection*{Общая теория}

Стол прозрачный, граф квадратный.

\subsection*{Задачи}

\begin{problem}

\begin{condition}
	(Nun)

	Даны интенсивности марковского процесса с тремя состояниями:
	$\lambda_{12}=\lambda_{21}=\lambda_{32}=0$,
	$\lambda_{13}=1$, $\lambda_{23}=2$, $\lambda_{31}=3$.
	Построить граф процесса, составить уравнения Колмогорова для вероятностей состояний, найти стационарное распределение вероятностей.
\end{condition}

\begin{solution}

	$$
		\Lambda =
		\begin{pmatrix}
			-1 & 0  & 1  \\
			0  & -2 & 2  \\
			3  & 0  & -3
		\end{pmatrix}
	$$

	Стационарное распределение $\overline r=(r_1,r_2,r_3)$
	задаётся условиями
	$$
		\overline r \Lambda = 0,
		r_1 + r_2 + r_3 = 1.
	$$

	$$
		\begin{array}{rcl}
			\begin{aligned}
				-1 r_1 + 3 r_3        & = 0,  \\
				-2 r_2                & =  0, \\
				1 r_1 + 2 r_2 - 3 r_3 & = 0.
			\end{aligned}
			 & \qquad \Rightarrow \qquad &
			\begin{aligned}
				r_1 & = 3r_3,          \\
				r_2 & = 0              \\
				r_3 & = \frac{r_1}{3}.
			\end{aligned}
		\end{array}
	$$

	$$
		r_1(1+0+\frac{1}{3})=1
		, \quad
		r_1=\frac{3}{4}
	$$

	$$
		\vec{r}=\left(\frac{3}{4}, 0, \frac{1}{4}\right)
	$$
\end{solution}

\begin{conclusion}

	$$
		\vec{r}=\left(\frac{3}{4}, 0, \frac{1}{4}\right)
	$$
\end{conclusion}

\end{problem}



\begin{problem}

\begin{condition}
	(16.1)

	Даны интенсивности марковского процесса с тремя состояниями:
	$\lambda_{12}=\lambda_{23}=\lambda_{31}=2$,
	$\lambda_{13}=1$
	Построить граф процесса, составить уравнения Колмогорова для вероятностей состояний, найти стационарное распределение вероятностей.
\end{condition}

\begin{solution}

	$$
		\Lambda =
		\begin{pmatrix}
			-3 & 2  & 1  \\
			0  & -2 & 2  \\
			2  & 0  & -2
		\end{pmatrix}
	$$

	Стационарное распределение $\overline r=(r_1,r_2,r_3)$
	задаётся условиями
	$$
		\overline r \Lambda = 0,
		r_1 + r_2 + r_3 = 1.
	$$

	$$
		\begin{array}{rcl}
			\begin{aligned}
				-3 r_1 + 2 r_3        & = 0,  \\
				2 r_1 - 2 r_2         & =  0, \\
				1 r_1 + 2 r_2 - 2 r_3 & = 0.
			\end{aligned}
			 & \qquad \Rightarrow \qquad &
			\begin{aligned}
				r_1 & = \frac{2r_3}{3}, \\
				r_2 & = r_1             \\
				r_3 & = \frac{3r_1}{2}.
			\end{aligned}
		\end{array}
	$$

	$$
		r_1(1+1+\frac{3}{2})=1
		, \quad
		r_1=\frac{2}{7}
	$$

	$$
		\vec{r}=\left(\frac{2}{7}, \frac{2}{7}, \frac{3}{7}\right)
	$$
\end{solution}

\begin{conclusion}

	$$
		\vec{r}=\left(\frac{2}{7}, \frac{2}{7}, \frac{3}{7}\right)
	$$
\end{conclusion}

\end{problem}



\begin{problem}

\begin{condition}
	(15.1)

	Даны интенсивности марковского процесса с тремя состояниями:
	$\lambda_{23}=\lambda_{32}=2$,
	$\lambda_{13}=\lambda_{21}=3$
	Построить граф процесса, составить уравнения Колмогорова для вероятностей состояний, найти стационарное распределение вероятностей.
\end{condition}

\begin{solution}

	$$
		\Lambda =
		\begin{pmatrix}
			-3 & 0  & 3  \\
			3  & -5 & 2  \\
			0  & 2  & -2
		\end{pmatrix}
	$$

	Стационарное распределение $\overline r=(r_1,r_2,r_3)$
	задаётся условиями
	$$
		\overline r \Lambda = 0,
		r_1 + r_2 + r_3 = 1.
	$$

	$$
		\begin{array}{rcl}
			\begin{aligned}
				-3 r_1 + 3 r_2        & = 0,  \\
				-5 r_2 + 2 r_3        & =  0, \\
				3 r_1 + 2 r_2 - 2 r_3 & = 0.
			\end{aligned}
			 & \qquad \Rightarrow \qquad &
			\begin{aligned}
				r_1 & = r_2,            \\
				r_2 & = \frac{2r_3}{5}  \\
				r_3 & = \frac{5r_2}{2}.
			\end{aligned}
		\end{array}
	$$

	$$
		r_2(1+1+\frac{5}{2})=1
		, \quad
		r_2=\frac{2}{9}
	$$

	$$
		\vec{r}=\left(\frac{2}{9}, \frac{2}{9}, \frac{5}{9}\right)
	$$
\end{solution}

\begin{conclusion}

	$$
		\vec{r}=\left(\frac{2}{9}, \frac{2}{9}, \frac{5}{9}\right)
	$$
\end{conclusion}

\end{problem}



\begin{problem}

\begin{condition}
	(31.1)

	Даны интенсивности марковского процесса с тремя состояниями:
	$\lambda_{31}=\lambda_{32}=1$,
	$\lambda_{13}=\lambda_{23}=3$,
	$\lambda_{21}=\lambda_{12}=2$.
	Построить граф процесса, составить уравнения Колмогорова для вероятностей состояний, найти стационарное распределение вероятностей.
\end{condition}

\begin{solution}

	$$
		\Lambda =
		\begin{pmatrix}
			-5 & 2  & 3  \\
			2  & -5 & 3  \\
			1  & 1  & -2
		\end{pmatrix}
	$$

	Стационарное распределение $\overline r=(r_1,r_2,r_3)$
	задаётся условиями
	$$
		\overline r \Lambda = 0,
		r_1 + r_2 + r_3 = 1.
	$$

	$$
		\begin{array}{rclcl}
			\begin{aligned}
				-5 r_1 + 2 r_2 + r_3  & = 0,  \\
				2 r_1 - 5 r_2 + 1 r_3 & =  0, \\
				3 r_1 + 3 r_2 - 2 r_3 & = 0.
			\end{aligned}
			 & \qquad \Rightarrow \qquad &
			\begin{aligned}
				r_1 & = \frac{2r_2+r_3}{5},  \\
				r_2 & = \frac{2r_1+r_3}{5}   \\
				r_3 & = \frac{3r_1+3r_2}{2}.
			\end{aligned}
			 & \qquad \Rightarrow \qquad &
			\begin{aligned}
				r_1 & = \frac{2r_2+r_3}{5}, \\
				r_2 & = r_1                 \\
				r_3 & = 3r_1.
			\end{aligned}
		\end{array}
	$$

	$$
		r_1(1+1+3)=1
		, \quad
		r_1=\frac{1}{5}
	$$

	$$
		\vec{r}=\left(\frac{1}{5}, \frac{1}{5}, \frac{3}{5}\right)
	$$
\end{solution}

\begin{conclusion}

	$$
		\vec{r}=\left(\frac{1}{5}, \frac{1}{5}, \frac{3}{5}\right)
	$$
\end{conclusion}

\end{problem}



\begin{problem}

\begin{condition}
	(11.1)

	Даны интенсивности марковского процесса с тремя состояниями:
	$\lambda_{21}=\lambda_{23}=1$,
	$\lambda_{13}=3$, $\lambda_{32}=2$
	Построить граф процесса, составить уравнения Колмогорова для вероятностей состояний, найти стационарное распределение вероятностей.
\end{condition}

\begin{solution}

	$$
		\Lambda =
		\begin{pmatrix}
			-3 & 0  & 3  \\
			1  & -2 & 1  \\
			0  & 2  & -2
		\end{pmatrix}
	$$

	Стационарное распределение $\overline r=(r_1,r_2,r_3)$
	задаётся условиями
	$$
		\overline r \Lambda = 0,
		r_1 + r_2 + r_3 = 1.
	$$

	$$
		\begin{array}{rcl}
			\begin{aligned}
				-3 r_1 + 1 r_2        & = 0,  \\
				- 2 r_2 + 2 r_3       & =  0, \\
				3 r_1 + 1 r_2 - 2 r_3 & = 0.
			\end{aligned}
			 & \qquad \Rightarrow \qquad &
			\begin{aligned}
				r_1 & = \frac{r_2}{3}, \\
				r_2 & = r_3            \\
				r_3 & = 3r_1.
			\end{aligned}
		\end{array}
	$$

	$$
		r_1(1+3+3)=1
		, \quad
		r_1=\frac{1}{7}
	$$

	$$
		\vec{r}=\left(\frac{1}{7}, \frac{3}{7}, \frac{3}{7}\right)
	$$
\end{solution}

\begin{conclusion}

	$$
		\vec{r}=\left(\frac{1}{7}, \frac{3}{7}, \frac{3}{7}\right)
	$$
\end{conclusion}

\end{problem}



\begin{problem}

\begin{condition}
	(12.1)

	Даны интенсивности марковского процесса с тремя состояниями:
	$\lambda_{13}=8$, $\lambda_{21}=1$, $\lambda_{23}=2$,
	$\lambda_{31}=5$, $\lambda_{32}=3$.
	Построить граф процесса, составить уравнения Колмогорова для вероятностей состояний, найти стационарное распределение вероятностей.
\end{condition}

\begin{solution}

	$$
		\Lambda =
		\begin{pmatrix}
			-8 & 0  & 8  \\
			1  & -3 & 2  \\
			5  & 3  & -8
		\end{pmatrix}
	$$

	Стационарное распределение $\overline r=(r_1,r_2,r_3)$
	задаётся условиями
	$$
		\overline r \Lambda = 0,
		r_1 + r_2 + r_3 = 1.
	$$

	$$
		\begin{array}{rcl}
			\begin{aligned}
				-8 r_1 + 1 r_2 + 5 r_3 & = 0,  \\
				- 3 r_2 + 3 r_3        & =  0, \\
				8 r_1 + 2 r_2 - 8 r_3  & = 0.
			\end{aligned}
			 & \qquad \Rightarrow \qquad &
			\begin{aligned}
				r_1 & = \frac{r_2+5r_3}{8}, \\
				r_2 & = r_3                 \\
				r_3 & = \frac{8r_1}{6}.
			\end{aligned}
		\end{array}
	$$

	$$
		r_1(1+\frac{8}{6}+\frac{8}{6})=1
		, \quad
		r_1=\frac{6}{22}
	$$

	$$
		\vec{r}=\left(\frac{6}{22}, \frac{8}{22}, \frac{8}{22}\right)
	$$
\end{solution}

\begin{conclusion}

	$$
		\vec{r}=\left(\frac{3}{11}, \frac{4}{11}, \frac{4}{11}\right)
	$$
\end{conclusion}

\end{problem}



\begin{problem}

\begin{condition}
	(33.1)

	Даны интенсивности марковского процесса с тремя состояниями:
	$\lambda_{12}=\lambda_{31}=1$, $\lambda_{13}=5$,
	$\lambda_{23}=3$, $\lambda_{21}=\lambda_{32}=2$.
	Построить граф процесса, составить уравнения Колмогорова для вероятностей состояний, найти стационарное распределение вероятностей.
\end{condition}

\begin{solution}

	$$
		\Lambda =
		\begin{pmatrix}
			-6 & 1  & 5  \\
			2  & -5 & 3  \\
			1  & 2  & -3
		\end{pmatrix}
	$$

	Стационарное распределение $\overline r=(r_1,r_2,r_3)$
	задаётся условиями
	$$
		\overline r \Lambda = 0,
		r_1 + r_2 + r_3 = 1.
	$$

	$$
		\begin{array}{rcl}
			\begin{aligned}
				-6 r_1 + 2 r_2 + 1 r_3 & = 0,  \\
				r_1 - 5 r_2 + 2 r_3    & =  0, \\
				5 r_1 + 3 r_2 - 3 r_3  & = 0.
			\end{aligned}
			 & \Rightarrow &
			\begin{aligned}
				-30 r_2 + 12 r_3 + 2 r_2 + 1 r_3 & = 0,           \\
				r_1                              & = 5r_2 - 2r_3, \\
				5 r_1 + 3 r_2 - 3 r_3            & = 0.
			\end{aligned}
		\end{array}
	$$


	$$
		\begin{array}{rclcl}
			\begin{aligned}
				r_2 & = \frac{13r_3}{28}, \\
				r_1 & = 5r_2 - 2r_3.
			\end{aligned}
			 & \Rightarrow &
			\begin{aligned}
				r_2 & = \frac{13r_3}{28},       \\
				r_1 & = \frac{65r_3}{28}- 2r_3.
			\end{aligned}
			 & \Rightarrow &
			\begin{aligned}
				r_2 & = r_3\frac{13}{28}, \\
				r_1 & = r_3\frac{9}{28} .
			\end{aligned}
		\end{array}
	$$

	$$
		r_3(\frac{9}{28}+\frac{13}{28}+1)=1
		, \quad
		r_3=\frac{28}{50}
	$$

	$$
		\vec{r}=\left(\frac{9}{50}, \frac{13}{50}, \frac{28}{50}\right)
	$$
\end{solution}

\begin{conclusion}

	$$
		\vec{r}=\left(\frac{9}{50}, \frac{13}{50}, \frac{28}{50}\right)
	$$
\end{conclusion}

\end{problem}