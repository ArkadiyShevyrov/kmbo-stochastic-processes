\section{Динамическая система (наиболее вероятный тип задачи)}

\subsection*{Общая теория}

$$
	\int_{0}^{\infty} e^{a\tau} \, = \frac{1}{a}
$$

$$
	\int_{0}^{\infty} \tau e^{-a \tau} \, d\tau = \frac{1}{a^2}
$$

$$
	\int_{0}^{\infty} e^{-a\tau} \cos(\omega \tau) \, d\tau = \frac{a}{a^2+\omega^2}
$$

$$
	\int_{0}^{\infty} \tau  e^{-a\tau} \cos(\omega \tau) \, d\tau = \frac{a^2+\omega^2}{(a^2+\omega^2)^2}
$$

\textbf{Формула Выч1:}

$$
	\int_{-\infty}^{+\infty} e^{i\omega\tau} \frac{1}{\omega^2+a^2} \, d\omega = \frac{\pi}{a} e^{-a|\tau|}
$$

$(\omega^2 + a^2)=(\omega-i a)(\omega+i a)$ - полюс первого порядка

По лемме Жордана берём $\tau>0$, то есть только $i a$.

$$
	Rez(f,i a), f=\frac{P(z)}{Q(z)}
	\Rightarrow
	Req(f,i a)=\left[\frac{P(z)}{Q'(z)}\right]_{z_0=i a}
	\Rightarrow
	Req(f,i a)=\left[\frac{e^{i\omega \tau}}{2\omega}\right]_{\omega_0=i a}
	=\frac{e^{-a\tau}}{2\cdot i a}
$$

$$
	\int_{-\infty}^{+\infty} e^{i\omega\tau} \frac{1}{\omega^2+a^2}
	=\left[2\pi i \cdot Rez(f,i a)\right]
	=\left[2\pi i \cdot \frac{e^{-a\tau}}{2\cdot i a}\right]
	=\frac{\pi}{a} e^{-a|\tau|}
$$

\textbf{Формула Выч2:}

$$
	\int_{-\infty}^{+\infty} \frac{1}{\omega^2 +a^2} \, d\omega = \frac{\pi}{a}
$$

$(\omega^2 + a^2)=(\omega-i a)(\omega+i a)$ - полюс первого порядка

По лемме Жордана берём $\tau>0$, то есть только $i a$.

$$
	Rez(f,i a), f=\frac{P(z)}{Q(z)}
	\Rightarrow
	Req(f,i a)=\left[\frac{P(z)}{Q'(z)}\right]_{z_0=i a}
	\Rightarrow
	Req(f,i a)=\left[\frac{1}{2\omega}\right]_{\omega_0=i a}
	=\frac{1}{2\cdot i a}
$$

$$
	\int_{-\infty}^{+\infty} e^{i\omega\tau} \frac{1}{\omega^2+a^2}
	=\left[2\pi i \cdot Rez(f,i a)\right]
	=\left[2\pi i \cdot \frac{1}{2\cdot i a}\right]
	=\frac{\pi}{a}
$$

\textbf{Формула Выч3:}

$$
	\int_{-\infty}^{+\infty} \frac{1}{(\omega^2 +a^2)^2} \, d\omega = \frac{\pi}{2a^3}
$$

$(\omega^2 + a^2)^2=(\omega-i a)^2(\omega+i a)^2$ - полюс второго порядка

По лемме Жордана берём $\tau>0$, то есть только $i a$.

$$
	\begin{aligned}
		Req(f,i a) & =\lim_{z\to i a}\left(\left[(z-ia)^2\frac{1}{(z-ia)^2(z+ia)^2}\right]'\right)
		=\lim_{z\to i a}\left(\left[\frac{1}{(z+ia)^2}\right]'\right)
		\\ & = \lim_{z\to i a}\left(-\frac{2}{(z+ia)^3}\right)
		=-\frac{2}{(2ia)^3}
		=-\frac{2}{8i^3a^3}
		=\frac{1}{4ia^3}
	\end{aligned}
$$

$$
	\int_{-\infty}^{+\infty} e^{i\omega\tau} \frac{1}{\omega^2+a^2}
	=\left[2\pi i \cdot Rez(f,i a)\right]
	=\left[2\pi i \cdot \frac{1}{4ia^3}\right]
	=\frac{pi}{2a^3}
$$

\textbf{Формула Выч4:}

$$
	\int_{-\infty}^{+\infty} \frac{\omega^2}{(\omega^2 +a^2)^2} \, d\omega = \frac{\pi}{2a}
$$


$(\omega^2 + a^2)^2=(\omega-i a)^2(\omega+i a)^2$ - полюс второго порядка

По лемме Жордана берём $\tau>0$, то есть только $i a$.

$$
	\begin{aligned}
		Req(f,i a) & =\lim_{z\to i a}\left(\left[(z-ia)^2\frac{z^2}{(z-ia)^2(z+ia)^2}\right]'\right)
		=\lim_{z\to i a}\left(\left[\frac{z^2}{(z+ia)^2}\right]'\right)
		\\ & = \lim_{z\to i a}\left(\frac{2z(z+ia)^2-2z^2(z+ia)}{(z+ia)^4}\right)
		= \lim_{z\to i a}\left(\frac{2z(z+ia)-2z^2}{(z+ia)^3}\right)
		\\ & = \lim_{z\to i a}\left(\frac{2zia}{(z+ia)^3}\right)
		= \frac{2(ia)(ia)}{(2ia)^3}
		= \frac{-2a^2}{8i^3a^3}
		= \frac{1}{4ia}
	\end{aligned}
$$

$$
	\int_{-\infty}^{+\infty} e^{i\omega\tau} \frac{1}{\omega^2+a^2}
	=\left[2\pi i \cdot Rez(f,i a)\right]
	=\left[2\pi i \cdot \frac{1}{4ia}\right]
	=\frac{\pi}{2a}
$$

Формулы далее

$$
	K(\tau)=\int_{0}^{\infty} S(\omega) \cos(\omega\tau) \, d\omega
$$

$$
	S(\omega)=\frac{2}{\pi} \int_{0}^{\infty} K(\tau) \cos(\omega\tau) \, d\tau
$$

$$
	S(\omega)=2\cdot S^*(\omega) = 2\cdot S^*(-\omega)
$$

------


$$
	K(\tau)=\int_{-\infty}^{+\infty} e^{i\omega \tau} S^*(\omega) \, d\omega
$$


\subsection*{Задачи}

\begin{problem}

\begin{condition}
	(Nun)

	Найти спектральное разложение корреляционной функции стационарного
	случайного процесса
	$K_X(\tau)=2e^{-5|\tau|}$
\end{condition}

\begin{solution}

	На $[0,\infty]$: $K(\tau)=2e^{-5\tau}$.

	$$
		\begin{aligned}
			S(\omega) & =\frac{2}{\pi} \int_{0}^{\infty} K(\tau) \cos(\omega\tau)
			= \frac{2}{\pi} \int_{0}^{\infty} 2e^{-5\tau} \cos(\omega\tau)
			= \frac{4}{\pi} \int_{0}^{\infty} e^{-5\tau} \cos(\omega\tau)
			\\ & = \frac{4}{\pi} \frac{5}{25+\omega^2}
			= \frac{20}{\pi} \frac{1}{25+\omega^2}
		\end{aligned}
	$$

	$$
		S^*(\omega)=\frac{1}{2} S(\omega) = \frac{10}{\pi} \frac{1}{25+\omega^2}
	$$
\end{solution}

\begin{conclusion}

	$$S^*(\omega)=\frac{10}{\pi} \frac{1}{25+\omega^2}$$
\end{conclusion}

\end{problem}



\begin{problem}

\begin{condition}
	(16.1)

	Уравнение динамической системы:
	$3\frac{dX(t)}{dt} - 6 X(t) = \frac{d^2 Y(t)}{d t^2}+\frac{d Y(t)}{dt} - 6 Y(t)$.
	На вход поступает "белый шум" с интенсивностью
	$G=3$.

	Найти корреляционную функцию процесса $Y(t)$ на выходе.
\end{condition}

\begin{solution}

\end{solution}

\begin{conclusion}

\end{conclusion}

\end{problem}



\begin{problem}

\begin{condition}
	(15.1)

	Уравнение динамической системы:
	$2\frac{dX(t)}{dt} + 6 X(t) = \frac{d Y(t)}{dt} + 5 Y(t)$.
	На вход поступает стационарный процесс $X(t)$
	с корреляционной функцией
	$K_X(\tau)=5e^{-3|\tau|}$.

	Найти корреляционную функцию процесса $Y(t)$ на выходе.
\end{condition}

\begin{solution}

\end{solution}

\begin{conclusion}

\end{conclusion}

\end{problem}



\begin{problem}

\begin{condition}
	(31.1)

	Спектральная плотность стационарного процесса
	$S^*(\omega)=\frac{2}{\pi(\omega^2+25)}$.

	Найти корреляционную функцию.
\end{condition}

\begin{solution}

	$$
		\begin{aligned}
			K(\tau) & =\int_{-\infty}^{+\infty} e^{i\omega \tau} S^*(\omega) \, d\omega
			=\int_{-\infty}^{+\infty} e^{i\omega \tau} \frac{2}{\pi(\omega^2+25)} \, d\omega
			\\ & = \frac{2}{\pi} \int_{-\infty}^{+\infty} e^{i\omega \tau} \frac{1}{(\omega^2+5^2)} \, d\omega
			= \frac{2}{\pi} \frac{\pi}{5} e^{-5|\tau|}
			= \frac{2}{5} e^{-5|\tau|}
		\end{aligned}
	$$
\end{solution}

\begin{conclusion}

	$$K(\tau) = \frac{2}{5} e^{-5|\tau|}$$
\end{conclusion}

\end{problem}



\begin{problem}

\begin{condition}
	(11.1)

	Работа динамической системы описывается уравнением
	$3 X(t) = \frac{d Y(t)}{dt} + 5 Y(t)$.
	На вход поступает "белый шум" с интенсивностью
	$G=2$.

	Найти корреляционную функцию процесса $Y(t)$ на выходе.
\end{condition}

\begin{solution}

\end{solution}

\begin{conclusion}

\end{conclusion}

\end{problem}



\begin{problem}

\begin{condition}
	(12.1)

	По спектральной плотности
	$S^*_X(\omega)=\frac{\omega^2+5}{\pi(\omega^2+1)^2}$
	стационарного случайного процесса $X(t)$ найти его дисперсию.
\end{condition}

\begin{solution}

	$$
		\begin{aligned}
			D & = \int_{-\infty}^{+\infty} S^*(\omega) \, d\omega
			= \int_{-\infty}^{+\infty} \frac{\omega^2+5}{\pi(\omega^2+1)^2} \, d\omega
			\\ & = \frac{5}{\pi} \int_{-\infty}^{+\infty} \frac{1}{(\omega^2+1)^2} \, d\omega + \frac{1}{\pi}  \int_{-\infty}^{+\infty} \frac{\omega^2}{(\omega^2+1)^2} \, d\omega
			\\ & = \frac{5}{\pi} \frac{\pi}{2} + \frac{1}{\pi} \frac{\pi}{2}
			= 3
		\end{aligned}
	$$
\end{solution}

\begin{conclusion}

	$$D = 3$$
\end{conclusion}

\end{problem}



\begin{problem}

\begin{condition}
	(33.1)

	Дана спектральная плотность стационарного процесса
	$S^*(\omega)=\frac{3}{\pi(9+\omega^2)}$.

	Найти дисперсию процесса.
\end{condition}

\begin{solution}

	$$
		\begin{aligned}
			D & = \int_{-\infty}^{+\infty} S^*(\omega) \, d\omega
			= \int_{-\infty}^{+\infty} \frac{3}{\pi(9+\omega^2)} \, d\omega
			\\ & = \frac{3}{\pi} \int_{-\infty}^{+\infty} \frac{1}{9+\omega^2} \, d\omega
			\\ & = \frac{3}{\pi} \frac{\pi}{3}
			= 1
		\end{aligned}
	$$
\end{solution}

\begin{conclusion}

	$$D = 1$$
\end{conclusion}

\end{problem}
