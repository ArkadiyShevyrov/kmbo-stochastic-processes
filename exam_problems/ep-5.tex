\section{СМО}


\subsection*{Общая теория}

СМО - система массового обслуживания.

Три типа задач:
$M|M|n|0$, $M|M|n|\infty$, $M|M|n|m$.

Где $M$ - пуассоновский поток заявок,
$M$ - экспоненциальное время обслуживания,
$n$ - число каналов обслуживаниея,
$m$ - ёмкость системы.

Где $m=0$ - нет очереди,
$m=\infty$ - неограниченная очередь,
$m=m$ - ограченная очередь на $m$ мест.

$\lambda$ - интенсивность входного потока заявок,
$\mu$ - интенсивность обслуживания одним прибором,
$\rho=\frac{\lambda}{\mu}$ - приведённая нагрузка системы,
$\nu=\frac{\rho}{n}=\frac{\lambda}{n\mu}$ - коэффициент загрузки системы.

Термины:
$P_{\text{отк}}$ - вероятность отказа,
$Q$ - относительная пропускная способность, \\
$A$ - абсолютная пропускная способность,
$\bar{k}$ - среднее число занятых приборов,
$\bar{r}$ - среднее число заявок в очереди,
$\bar{z}$ - среднее число заявок в системе,
$\bar{t}_{\text{оч}}$ - среднее время ожидания,
$\bar{t}_{\text{СМО}}$-  среднее время в системе.

\renewcommand{\arraystretch}{1.3}
\begin{table}[h]
	\centering
	\begin{tabular}{c|c|c|c}
		                       & $(M|M|n|0)$                                         & $(M|M|n|\infty)$, $\nu<1$                                                              & $(M|M|n|m)$                                                                                                         \\
		\hline
		$r_0$                  & $\left(\sum_{k=0}^{n}\frac{\rho^k}{k!}\right)^{-1}$ & $\left(\sum_{k=0}^{n-1}\frac{\rho^k}{k!}+\frac{\rho^n}{n!}\frac{1}{1-\nu}\right)^{-1}$ & $\left(\sum_{k=0}^{n}\frac{\rho^k}{k!}+\frac{\rho^n}{n!}\sum_{j=1}^{m}\nu^j\right)^{-1}$                            \\
		\hline
		$P_{\text{отк}}$       & $r_n=\frac{\rho^n}{n!}r_0$                          & $0$                                                                                    & $r_{n+m}=\nu^m \frac{\rho^n}{n!}r_0$                                                                                \\
		\hline
		$Q$                    & $P_{\text{обсл}}=1-P_{\text{отк}}$                  & $P_{\text{обсл}}=1-P_{\text{отк}}=1$                                                   & $P_{\text{обсл}}=1-P_{\text{отк}}$                                                                                  \\
		\hline
		$A$                    & $\lambda Q$                                         & $\lambda Q=\lambda$                                                                    & $\lambda Q$                                                                                                         \\
		\hline
		$\bar{k}$              & $\rho Q$                                            & $\rho Q=\rho$                                                                          & $\rho Q$                                                                                                            \\
		\hline
		$\bar{r}$              & $0$                                                 & $\nu \frac{r_n}{(1-\nu)^2}$                                                            & $\frac{\rho^n}{n!}r_0 \sum_{j=1}^{m} j \nu^j=\frac{\rho^n}{n!}r_0 \frac{\nu(1-(m+1)\nu^m + m\nu^{m+1})}{(1-\mu)^2}$ \\
		\hline
		$\bar{z}$              & $\bar{k}$                                           & $\bar{k}+\bar{r}$                                                                      & $\bar{k}+\bar{r}$                                                                                                   \\
		\hline
		$\bar{t}_{\text{оч}}$  & $\frac{\bar{r}}{\lambda}=0$                         & $\frac{\bar{r}}{\lambda}$                                                              & $\frac{\bar{r}}{\lambda}$                                                                                           \\
		\hline
		$\bar{t}_{\text{СМО}}$ & $\frac{\bar{z}}{\lambda}=\frac{\bar{k}}{\lambda}$   & $\frac{\bar{z}}{\lambda}$                                                              & $\frac{\bar{z}}{\lambda}$                                                                                           \\
		\hline
	\end{tabular}
\end{table}

\begin{table}[h]
	\centering
	\begin{tabular}{c|c|c|c}
		                       & $(M|M|1|0)$                                                             & $(M|M|1|\infty)$, $\nu<1$                          & $(M|M|1|m)$                                                                           \\
		\hline
		$r_0$                  & $(1+\rho)^{-1}$                                                         & $1-\rho$                                           & $\frac{1-\rho}{1-\rho^{m+2}}$                                                         \\
		\hline
		$P_{\text{отк}}$       & $\rho r_0= \frac{\rho}{1+\rho}$                                         & $0$                                                & $\rho_{m+1}=\rho^{m+1}r_0=\frac{(1-\rho)\rho^{m+1}}{1-\rho^{m+2}}$                    \\
		\hline
		$Q$                    & $P_{\text{обсл}}=1-P_{\text{отк}}=\frac{1}{1+\rho}$                     & $P_{\text{обсл}}=1-P_{\text{отк}}=1$               & $P_{\text{обсл}}=1-P_{\text{отк}}$                                                    \\
		\hline
		$A$                    & $\lambda Q = \frac{\lambda}{1+\rho}$                                    & $\lambda Q=\lambda$                                & $\lambda Q $                                                                          \\
		\hline
		$\bar{k}$              & $\rho Q = \frac{\rho}{1+\rho}$                                          & $\rho Q=\rho$                                      & $\rho Q$                                                                              \\
		\hline
		$\bar{r}$              & $0$                                                                     & $\frac{\rho^2}{1-\rho}$                            & $r_0 \sum_{j=1}^{m} j \rho^j = r_0\frac{\rho(1-(m+1)\rho^m+m\rho^{m+1})}{(1-\rho)^2}$ \\
		\hline
		$\bar{z}$              & $\bar{k}$                                                               & $\frac{\rho}{1-\rho}$                              & $\bar{k}+\bar{r}$                                                                     \\
		\hline
		$\bar{t}_{\text{оч}}$  & $\frac{\bar{r}}{\lambda}=0$                                             & $\frac{\bar{r}}{\lambda}=\frac{\rho}{\mu(1-\rho)}$ & $\frac{\bar{r}}{\lambda}$                                                             \\
		\hline
		$\bar{t}_{\text{СМО}}$ & $\frac{\bar{z}}{\lambda}=\frac{\bar{k}}{\lambda}=\frac{1}{\mu(1+\rho)}$ & $\frac{\bar{z}}{\lambda}=\frac{1}{\mu(1-\rho)}$    & $\frac{\bar{z}}{\lambda}$                                                             \\
		\hline
	\end{tabular}
\end{table}


\subsection*{Задачи}

\begin{problem}

\begin{condition}
	(01.4)

	В СМО $(M|M|2|2)$ интенсивность прихода заявок $\lambda=2$ в час,
	среднее время обслуживания каждой заявки $T_{\text{обсл}}=15$ мин.
	Найти среднее число занятых приборов.
\end{condition}

\begin{solution}
	$(M|M|2|2)$

	$\lambda=2$,
	$\mu=\frac{1}{T_{\text{обсл}}}=4$ клиента в час,
	$\rho=\frac{\lambda}{\mu}=\frac{1}{2}$,
	$\nu=\frac{\lambda}{n\mu}=\frac{2}{2\cdot 4}=\frac{1}{4}$

	$$
		\begin{aligned}
			r_0 & =\left(
			\sum_{k=0}^{n}\frac{\rho^k}{k!}+\frac{\rho^n}{n!}\sum_{j=1}^{m}\nu^j
			\right)^{-1}
			=\left(
			\sum_{k=0}^{2}\frac{\rho^k}{k!}+\frac{\rho^2}{n!}\sum_{j=1}^{2}\nu^j
			\right)^{-1}
			\\ & = \left(
			1 + \rho + \frac{\rho^2}{2} + \frac{\rho^2}{2}(\nu+\nu^2)
			\right)^{-1}
			=\left(
			1 + \frac{1}{2} + \frac{1}{8} + \frac{1}{8}\left(\frac{1}{4} + \frac{1}{16}\right)
			\right)^{-1}
			\\ & = \frac{128}{213}
		\end{aligned}
	$$

	$$
		r_{n+m}=r_{4}=\nu^m \frac{\rho^n}{n!}r_0
		=\frac{1}{4}^2 \frac{\frac{1}{2}^2}{2!} \frac{128}{213}
		= \frac{1}{213}
	$$

	Относительная  пропускная способность $Q$

	$$
		Q=P_{\text{обсл}}=1-P_{\text{отк}}=1-r_{n+m}
		= 1 - \frac{1}{213}
		= \frac{212}{213}
	$$

	Среднее число занятых приборов $\bar{k}$

	$$
		\bar{k}=\rho Q
		= \frac{1}{2} \frac{212}{213}
		= \frac{106}{213}
	$$

\end{solution}

\begin{conclusion}

	$$
		\bar{k} = \frac{106}{213}
	$$
\end{conclusion}

\end{problem}



\begin{problem}

\begin{condition}
	(27.5)

	В парихмахерской задачи работают 2 мастера,
	в холле имеются 3 кресла для ожидающих обслуживания клиентов.
	В среднем один мастер обслуживает одного клиента 30 минут.
	Среднее число клиентов в час равно 4.
	Найти вероятность того, что в момент прихода клиента все мастера будут заняты.
\end{condition}

\begin{solution}

\end{solution}

\begin{conclusion}

\end{conclusion}

\end{problem}



\begin{problem}

\begin{condition}
	(51.8)

	В парихмахерской работают 5 мастеров.
	В среднем один мастер обслуживает одного клиента 20 минут.
	В среднем приходит 3 клиента в час.
	Если все мастера заняты, то клиент уходит.
	Найти вероятность того, что в момент прихода клиента не менее трёх мастеров будут свободны.
\end{condition}

\begin{solution}

\end{solution}

\begin{conclusion}

\end{conclusion}

\end{problem}



\begin{problem}

\begin{condition}
	(2.5)

	В СМО $(M|M|2|2)$ интенсивность прихода заявок $\lambda=3$ в час,
	среднее время обслуживания одной заявки $T=20$ минут.
	Найти среднее число занятых приборов.
\end{condition}

\begin{solution}

\end{solution}

\begin{conclusion}

\end{conclusion}

\end{problem}



\begin{problem}

\begin{condition}
	(31.5)

	У входа на стадион работает одна касса.
	В среднем подходят 300 болельщиков в час.
	Среднее врямя покупки билета составляет 10 секунд.
	Найти среднюю длину очереди.
\end{condition}

\begin{solution}

\end{solution}

\begin{conclusion}

\end{conclusion}

\end{problem}



\begin{problem}

\begin{condition}
	(45.3)

	Найти среднее время нахождения заявки в СМО $(M|M|2)$,
	если за час поступает в среднем 9 заявок,
	а одна заявка обслуживается в среднем 12 минут.
\end{condition}

\begin{solution}

\end{solution}

\begin{conclusion}

\end{conclusion}

\end{problem}



\begin{problem}

\begin{condition}
	(43.3)

	Клиент использует банкомат в сренем 3 минуты.
	За час к банкомату подходят в среднем 18 клиентов.
	Найти среднее время, которое клиент проводит в очереди,
	при условии, что поток клиентов пуассоновский,
	а время обслуживания распределено по показательному обслуживанию.
\end{condition}

\begin{solution}

\end{solution}

\begin{conclusion}

\end{conclusion}

\end{problem}



\begin{problem}

\begin{condition}
	(22.2)

	Клиент использует банкомат в среднем 3 минуты.
	За час к банкомату подходят в среднем 20 клиентов.
	Если в очереди уже стоят 8 человек, то клиент уходит.
	Найти среднее время пребывания клиента в очереди.
\end{condition}

\begin{solution}

\end{solution}

\begin{conclusion}

\end{conclusion}

\end{problem}



\begin{problem}

\begin{condition}
	(01.7)

	Найти среднее время пребывания заявки в СМО $(M|M|2)$,
	если в СМО в всреднем поступает 60 заявок в час,
	а среднее время обслуживания одним прибором одной заявки равно 1 минуте.
\end{condition}

\begin{solution}

\end{solution}

\begin{conclusion}

\end{conclusion}

\end{problem}



\begin{problem}

\begin{condition}
	(3.1)

	У входа на стадион работает одна касса.
	В среднем подходят 120 болельщиков в час.
	Найти среднее время покупки билета, если вероятность того,
	что перед кассой стоит только один болельших, равна $\frac{2}{9}$.
\end{condition}

\begin{solution}

\end{solution}

\begin{conclusion}

\end{conclusion}

\end{problem}



\begin{problem}

\begin{condition}
	(42.3)

	В СМО $(M|M|3|0)$ поступает за час в среднем 8 заявок,
	одна заявка обслуживается в среднем 15 минут.
	Построить граф СМО, найти среднее число занятых приборов.
\end{condition}

\begin{solution}

\end{solution}

\begin{conclusion}

\end{conclusion}

\end{problem}