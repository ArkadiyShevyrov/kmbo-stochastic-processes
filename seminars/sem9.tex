\subsection{Семинар 9}

\subsubsection*{Задача 9.1}

\begin{tikzpicture}[
		>=Stealth, thick,
		state/.style = {draw, circle, minimum size=9mm, font=\small},
		e/.style   = {->, shorten >=2pt, shorten <=2pt},
		lbl/.style  = {font=\scriptsize, fill=white, inner sep=1pt}
	]

	% --- вершины ---
	\node[state] (Q0) at (0,0) {$E_1$};
	\node[state] (Q1) at (2.5,0) {$E_2$};
	\node[state] (Q2) at (5,0) {$E_3$};

	% --- рёбра ---
	\path[e, bend left=12] (Q1) edge node[lbl] {$0.75$} (Q0);
	\path[e, bend left=12] (Q1) edge node[lbl] {$0.25$} (Q2);
	\path[e, bend left=12] (Q2) edge node[lbl] {$1$} (Q1);

\end{tikzpicture}

\[
	\Lambda =
	\begin{pmatrix}
		0    & 0  & 0    \\
		0.75 & -1 & 0.25 \\
		0    & 1  & -1
	\end{pmatrix}
\]

\[
	\vec{p}(0) = (0,1,0).
\]


\subsection{Семинар 9}

\subsubsection*{Задача 9.1}

\begin{tikzpicture}[
		>=Stealth, thick,
		state/.style = {draw, circle, minimum size=9mm, font=\small},
		e/.style   = {->, shorten >=2pt, shorten <=2pt},
		lbl/.style  = {font=\scriptsize, fill=white, inner sep=1pt}
	]

	% --- вершины ---
	\node[state] (Q0) at (0,0) {$E_1$};
	\node[state] (Q1) at (2.5,0) {$E_2$};
	\node[state] (Q2) at (5,0) {$E_3$};

	% --- рёбра ---
	\path[e, bend left=12] (Q1) edge node[lbl] {$0.75$} (Q0);
	\path[e, bend left=12] (Q1) edge node[lbl] {$0.25$} (Q2);
	\path[e, bend left=12] (Q2) edge node[lbl] {$1$} (Q1);

\end{tikzpicture}

\[
	\Lambda =
	\begin{pmatrix}
		0           & 0  & 0           \\
		\frac{3}{4} & -1 & \frac{1}{4} \\
		0           & 1  & -1
	\end{pmatrix}
\]

\[
	\vec{p}(0) = (0,1,0).
\]

\[
	\vec{p}'(t) = \vec{p}(t)\Lambda,\quad \vec{p}(t)=(p_1(t),p_2(t),p_3(t)).
\]

\[
	\begin{aligned}
		p_1'(t) & = \tfrac{3}{4}p_2(t),        \\
		p_2'(t) & = -p_2(t)+p_3(t),            \\
		p_3'(t) & = \tfrac{1}{4}p_2(t)-p_3(t).
	\end{aligned}
\]

Преобразование Лапласа:

\[
	\mathcal{L}\{\vec p'(t)\}(s)=s \vec \pi(s)-\vec p(0)
	,\quad
	\mathcal{L}\{\vec p(t)\}(s)=\vec \pi(s)
	,\quad
	s \vec \pi(s)-\vec p(0)=\vec \pi(s)\Lambda.
\]

\[
	\pi_1+\pi_2+\pi_3=\frac{1}{s}
\]

\[
	\begin{aligned}
		s \pi_1(s)-0 & = \tfrac{3}{4}\pi_2(s),          \\
		s \pi_2(s)-1 & = -\pi_2(s)+\pi_3(s),            \\
		s \pi_3(s)-0 & = \tfrac{1}{4}\pi_2(s)-\pi_3(s).
	\end{aligned}
\]

\[
	\begin{aligned}
		(s+1)\pi_2(s)-\pi_3(s)              & = 1, \\
		-\tfrac{1}{4}\pi_2(s)+(s+1)\pi_3(s) & = 0,
	\end{aligned}
	\qquad
	\Rightarrow\qquad
	\pi_3(s)=\frac{1}{4}\frac{\pi_2(s)}{s+1}.
\]

\[
	\Bigl((s+1) - \frac{1}{4}\frac{1}{s+1}\Bigr)\pi_2(s)=1
	\;\Rightarrow\;
	\pi_2(s)=\frac{s+1}{(s+1)^2-\tfrac{1}{4}}
	=\frac{s+1}{(s+\tfrac{1}{2})(s+\tfrac{3}{2})}.
\]

\[
	\pi_3(s)=\frac{1}{4}\cdot\frac{1}{(s+\tfrac{1}{2})(s+\tfrac{3}{2})},
	\qquad
	\pi_1(s)=\frac{3}{4}\cdot\frac{s+1}{s\,(s+\tfrac{1}{2})(s+\tfrac{3}{2})}.
\]

Разложения на простые дроби:
\[
	\pi_2(s)=\frac{1}{2}\left(\frac{1}{s+\tfrac{1}{2}}+\frac{1}{s+\tfrac{3}{2}}\right),
	\quad
	\pi_3(s)=\frac{1}{4}\left(\frac{1}{s+\tfrac{1}{2}}-\frac{1}{s+\tfrac{3}{2}}\right),
\]
\[
	\pi_1(s)=\frac{1}{s}-\frac{3}{4}\cdot\frac{1}{s+\tfrac{1}{2}}-\frac{1}{4}\cdot\frac{1}{s+\tfrac{3}{2}}.
\]

Обратное преобразование Лапласа:
\[
	\mathcal{L}^{-1}\!\left\{\frac{1}{s+a}\right\}=e^{-at},\qquad
	\mathcal{L}^{-1}\!\left\{\frac{1}{s}\right\}=1.
\]

\[
	\begin{aligned}
		p_1(t) & = 1 - \frac{3}{4}e^{-\tfrac{1}{2}t} - \frac{1}{4}e^{-\tfrac{3}{2}t}, \\
		p_2(t) & = \frac{1}{2}e^{-\tfrac{1}{2}t} + \frac{1}{2}e^{-\tfrac{3}{2}t},     \\
		p_3(t) & = \frac{1}{4}e^{-\tfrac{1}{2}t} - \frac{1}{4}e^{-\tfrac{3}{2}t}.
	\end{aligned}
\]

\[
	\vec{p}(t)\xrightarrow [t\to\infty]{}\vec{p}(\infty)=\left(1,\,0,\,0\right).
\]

Стационарное распределение $\overline r=(r_1,r_2,r_3,\ldots)$
\[
	\begin{cases}
		r_1 = \tfrac{3}{4} r_2,       \\
		r_2 = - r_2 + r_3,            \\
		r_3 = \tfrac{1}{4} r_2 - r_3, \\
		r_1 + r_2 + r_3 = 1.
	\end{cases}
	,\quad
	\begin{cases}
		\tfrac{3}{4} r_2 - r_1 = 0, \\
		- r_2 + r_3 = 0,            \\
		\tfrac{1}{4} r_2 - r_3 = 0, \\
		r_1 + r_2 + r_3 = 1.
	\end{cases}
\]

\[
	r_2 = 0,\quad r_3=0,\quad r_1=1
	\;\Rightarrow\;
	\vec r = (1,0,0).
\]


\subsection{Семинар 9}

\subsubsection*{Задача 9.1}

\begin{tikzpicture}[
		>=Stealth, thick,
		state/.style = {draw, circle, minimum size=9mm, font=\small},
		e/.style   = {->, shorten >=2pt, shorten <=2pt},
		lbl/.style  = {font=\scriptsize, fill=white, inner sep=1pt}
	]

	% --- вершины ---
	\node[state] (Q0) at (0,0) {$E_1$};
	\node[state] (Q1) at (2.5,0) {$E_2$};
	\node[state] (Q2) at (5,0) {$E_3$};

	% --- рёбра ---
	\path[e, bend left=12] (Q1) edge node[lbl] {$0.75$} (Q0);
	\path[e, bend left=12] (Q1) edge node[lbl] {$0.25$} (Q2);
	\path[e, bend left=12] (Q2) edge node[lbl] {$1$} (Q1);

\end{tikzpicture}

\[
	\Lambda =
	\begin{pmatrix}
		0 & \tfrac{3}{4} & 0  \\
		0 & -1           & 1  \\
		0 & \tfrac{1}{4} & -1
	\end{pmatrix}
\]

\[
	\vec{p}(0) = (0,1,0).
\]

\[
	\vec{p}'(t) = \vec{p}(t)\Lambda,\quad \vec{p}(t)=(p_1(t),p_2(t),p_3(t)).
\]

\[
	\begin{aligned}
		p_1'(t) & = \tfrac{3}{4}p_2(t),        \\
		p_2'(t) & = -p_2(t)+p_3(t),            \\
		p_3'(t) & = \tfrac{1}{4}p_2(t)-p_3(t).
	\end{aligned}
\]

Преобразование Лапласа:

\[
	\mathcal{L}\{\vec p'(t)\}(s)=s \vec \pi(s)-\vec p(0)
	,\quad
	\mathcal{L}\{\vec p(t)\}(s)=\vec \pi(s)
	,\quad
	s \vec \pi(s)-\vec p(0)=\vec \pi(s)\Lambda.
\]

\[
	\pi_1+\pi_2+\pi_3=\frac{1}{s}.
\]

\[
	\begin{aligned}
		s \pi_1(s)-0 & = \tfrac{3}{4}\pi_2(s),          \\
		s \pi_2(s)-1 & = -\pi_2(s)+\pi_3(s),            \\
		s \pi_3(s)-0 & = \tfrac{1}{4}\pi_2(s)-\pi_3(s).
	\end{aligned}
\]

\[
	\begin{aligned}
		(s+1)\pi_2(s)-\pi_3(s)              & = 1, \\
		-\tfrac{1}{4}\pi_2(s)+(s+1)\pi_3(s) & = 0,
	\end{aligned}
	\qquad
	\Rightarrow\qquad
	\pi_3(s)=\frac{1}{4}\frac{\pi_2(s)}{s+1}.
\]

\[
	\Bigl((s+1) - \frac{1}{4}\frac{1}{s+1}\Bigr)\pi_2(s)=1
	\;\Rightarrow\;
	\pi_2(s)=\frac{s+1}{(s+1)^2-\tfrac{1}{4}}
	=\frac{s+1}{\left(s+\tfrac12\right)\left(s+\tfrac32\right)}.
\]

\[
	\pi_3(s)=\frac{1}{4}\cdot\frac{1}{\left(s+\tfrac12\right)\left(s+\tfrac32\right)},
	\qquad
	\pi_1(s)=\frac{3}{4}\cdot\frac{s+1}{s\left(s+\tfrac12\right)\left(s+\tfrac32\right)}.
\]

Разложения на простые дроби и вычисление коэффициентов (метод предельных значений):

\[
	\begin{aligned}
		\pi_2(s)
		  & = \frac{D}{s+\tfrac12}+\frac{E}{s+\tfrac32},                      \\
		D & = \lim_{s\to -\tfrac12}\!\left(s+\tfrac12\right)\pi_2(s)
		= \frac{-\tfrac12+1}{-\tfrac12+\tfrac32}=\frac12,                     \\
		E & = \lim_{s\to -\tfrac32}\!\left(s+\tfrac32\right)\pi_2(s)
		= \frac{-\tfrac32+1}{-\tfrac32+\tfrac12}=\frac12,
		\\[6pt]
		\pi_3(s)
		  & = \frac{G}{s+\tfrac12}+\frac{H}{s+\tfrac32},                      \\
		G & = \lim_{s\to -\tfrac12}\!\left(s+\tfrac12\right)\pi_3(s)
		= \frac{1/4}{-\tfrac12+\tfrac32}=\frac14,                             \\
		H & = \lim_{s\to -\tfrac32}\!\left(s+\tfrac32\right)\pi_3(s)
		= \frac{1/4}{-\tfrac32+\tfrac12}=-\frac14,
		\\[6pt]
		\pi_1(s)
		  & = \frac{A}{s}+\frac{B}{s+\tfrac12}+\frac{C}{s+\tfrac32},          \\
		A & = \lim_{s\to 0} s\,\pi_1(s)
		= \frac{\tfrac34(0+1)}{\left(\tfrac12\right)\left(\tfrac32\right)}=1, \\
		B & = \lim_{s\to -\tfrac12}\!\left(s+\tfrac12\right)\pi_1(s)
		= \frac{\tfrac34(-\tfrac12+1)}{(-\tfrac12)\cdot 1}=-\frac34,          \\
		C & = \lim_{s\to -\tfrac32}\!\left(s+\tfrac32\right)\pi_1(s)
		= \frac{\tfrac34(-\tfrac32+1)}{(-\tfrac32)\cdot(-1)}=-\frac14.
	\end{aligned}
\]

\[
	\begin{aligned}
		\pi_1(s) & = \frac{1}{s} - \frac{3}{4}\frac{1}{s+\tfrac12} - \frac{1}{4}\frac{1}{s+\tfrac32}, \\
		\pi_2(s) & = \frac{1}{2}\frac{1}{s+\tfrac12} + \frac{1}{2}\frac{1}{s+\tfrac32},               \\
		\pi_3(s) & = \frac{1}{4}\frac{1}{s+\tfrac12} - \frac{1}{4}\frac{1}{s+\tfrac32}.
	\end{aligned}
\]

Обратное преобразование Лапласа:
\[
	\mathcal{L}^{-1}\!\left\{\frac{1}{s}\right\}=1,\quad
	\mathcal{L}^{-1}\!\left\{\frac{1}{s+a}\right\}=e^{-at}.
\]

\[
	\begin{aligned}
		p_1(t) & = 1 - \frac{3}{4}e^{-\tfrac{1}{2}t} - \frac{1}{4}e^{-\tfrac{3}{2}t}, \\
		p_2(t) & = \frac{1}{2}e^{-\tfrac{1}{2}t} + \frac{1}{2}e^{-\tfrac{3}{2}t},     \\
		p_3(t) & = \frac{1}{4}e^{-\tfrac{1}{2}t} - \frac{1}{4}e^{-\tfrac{3}{2}t}.
	\end{aligned}
\]

\[
	\vec{p}(t)\xrightarrow [t\to\infty]{}\vec{p}(\infty)=\left(1,\,0,\,0\right).
\]

Стационарное распределение $\overline r=(r_1,r_2,r_3,\ldots)$
\[
	\begin{cases}
		r_1 = \tfrac{3}{4} r_2,       \\
		r_2 = - r_2 + r_3,            \\
		r_3 = \tfrac{1}{4} r_2 - r_3, \\
		r_1 + r_2 + r_3 = 1.
	\end{cases}
	,\quad
	\begin{cases}
		\tfrac{3}{4} r_2 - r_1 = 0, \\
		- r_2 + r_3 = 0,            \\
		\tfrac{1}{4} r_2 - r_3 = 0, \\
		r_1 + r_2 + r_3 = 1.
	\end{cases}
\]

\[
	r_2 = 0,\quad r_3=0,\quad r_1=1
	\;\Rightarrow\;
	\vec r = (1,0,0).
\]
