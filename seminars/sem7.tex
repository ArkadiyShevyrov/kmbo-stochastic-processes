
\subsection{Семинар 7}

\subsubsection*{Задача 7.1}


\begin{tikzpicture}[
		>=Stealth, thick,
		state/.style = {draw, circle, minimum size=9mm, font=\small},
		e/.style   = {->, shorten >=2pt, shorten <=2pt},
		lbl/.style  = {font=\scriptsize, fill=white, inner sep=1pt}
	]

	% --- вершины ---
	\node[state] (Q0) at (0,0) {$1$};
	\node[state] (Q1) at (2.5,0) {$2$};
	\node[state] (Q2) at (5,0) {$3$};

	% --- рёбра ---
	\path[e, bend left=12] (Q1) edge node[lbl] {$2$} (Q0);
	\path[e, bend left=12]   (Q1) edge node[lbl] {$1$} (Q2);

	\path[e, bend left=12]   (Q2) edge node[lbl] {$2$} (Q1);

\end{tikzpicture}


\[
	\Lambda =
	\begin{pmatrix}
		0 & 0  & 0  \\
		2 & -3 & 1  \\
		0 & 2  & -2
	\end{pmatrix}
\]

\[
	\vec{p}(0) = (0,0,1).
\]

\[
	\vec{p}'(t) = \vec{p}(t)\Lambda,\quad \vec{p}(t)=(p_1(t),p_2(t),p_3(t)).
\]

\[
	\begin{aligned}
		p_1'(t) & = p_1 (t)\cdot 0 + p_2(t)\cdot 2 + p_3(t)\cdot 0 = 2p_2(t),            \\
		p_2'(t) & = p_1 (t)\cdot 0 + p_2(t)\cdot(-3) + p_3(t)\cdot 2 = -3p_2(t)+2p_3(t), \\
		p_3'(t) & = p_1 (t)\cdot 0 + p_2(t)\cdot 1 + p_3(t)\cdot(-2) = p_2(t)-2p_3(t).
	\end{aligned}
\]

Преобразование Лапласа:

\[
	\mathcal{L}\{\vec p'(t)\}(s)=s \vec \pi(s)-\vec p(0)
	,\quad
	\mathcal{L}\{\vec p(t)\}(s)=\vec \pi(s)
	,\quad
	s \vec \pi(s)-\vec p(0)=\vec \pi(s)\Lambda.
\]

\[
	\pi_1+\pi_2+\pi_3=\frac{1}{s}
\]

\[
	\begin{aligned}
		s \pi_1(s)-0 & = \pi_1(s)\cdot(0) + \pi_2(s)\cdot 2 + \pi_3(s)\cdot 0 = 2\pi_2(s),             \\
		s \pi_2(s)-0 & = \pi_1(s)\cdot 0 + \pi_2(s)\cdot(-3) + \pi_3(s)\cdot 2 = -3\pi_2(s)+2\pi_3(s), \\
		s \pi_3(s)-1 & = \pi_1(s)\cdot 0 + \pi_2(s)\cdot 1 + \pi_3(s)\cdot(-2) = \pi_2(s)-2\pi_3(s).
	\end{aligned}
\]

\[
	\begin{aligned}
		(s)\pi_1(s)-2\pi_2(s)    & = 0, \\
		-2\pi_3(s)+(s+3)\pi_2(s) & = 0, \\
		-\pi_2(s)+(s+2)\pi_3(s)  & = 1.
	\end{aligned}
\]

\[
	\begin{aligned}
		\pi_1(s) & = \frac{2}{s}\pi_2(s),   \\
		\pi_3(s) & = \frac{s+3}{2}\pi_2(s).
	\end{aligned}
\]

\[
	\pi_2(s)\left(\frac{2}{s}+1+\frac{s+3}{2}\right) = \frac{1}{s}.
\]

\[
	\pi_2(s)\left(\frac{4+2s+s(s+3)}{2s}\right) = \frac{1}{s}.
\]

\[
	\pi_2(s) = \frac{2}{4+2s+s(s+3)}
	= \frac{2}{s^2+5s+4}
	=\frac{2}{(s+1)(s+4)}.
\]

\[
	\pi_1(s) = \frac{2}{s} \pi_2(s) = \frac{4}{s(s+1)(s+4)},
	\quad
	\pi_3(s) = \frac{s+3}{2} \pi_2(s) = \frac{s+3}{(s+1)(s+4)}.
\]

\[
	\begin{aligned}
		\pi_1(s) & = \frac{4}{s(s+1)(s+4)}
		= \frac{A}{s} + \frac{B}{s+1} + \frac{C}{s+4}                         \\
		\pi_2(s) & = \frac{2}{(s+1)(s+4)} = \frac{D}{s+1} + \frac{E}{s+4}     \\
		\pi_3(s) & =	 \frac{s+3}{(s+1)(s+4)} = \frac{F}{s+1} + \frac{G}{s+4}.
	\end{aligned}
\]

\[
	\begin{aligned}
		A & = \lim_{s\rightarrow 0} s\pi_1(s)      = \frac{4}{(0+1)(0+4)} = 1,         \\
		B & = \lim_{s\rightarrow -1} (s+1)\pi_1(s) = \frac{4}{(-1)(3)} = -\frac{4}{3}, \\
		C & = \lim_{s\rightarrow -4} (s+4)\pi_1(s) = \frac{4}{(-4)(-3)} = \frac{1}{3}, \\
		D & = \lim_{s\rightarrow -1} (s+1)\pi_2(s) = \frac{2}{(-1+4)} = \frac{2}{3},   \\
		E & = \lim_{s\rightarrow -4} (s+4)\pi_2(s) = \frac{2}{(-4+1)} = -\frac{2}{3},  \\
		F & = \lim_{s\rightarrow -1} (s+1)\pi_3(s) = \frac{-1+3}{-1+4} = \frac{2}{3},  \\
		G & = \lim_{s\rightarrow -4} (s+4)\pi_3(s) = \frac{-4+3}{-4+1} = \frac{1}{3}.
	\end{aligned}
\]

\[
	\begin{aligned}
		\pi_1(s) & = \frac{1}{s} - \frac{4}{3(s+1)} + \frac{1}{3(s+4)}, \\
		\pi_2(s) & = \frac{2}{3(s+1)} - \frac{2}{3(s+4)},               \\
		\pi_3(s) & = \frac{2}{3(s+1)} + \frac{1}{3(s+4)}.
	\end{aligned}
\]

\[
	\begin{aligned}
		p_1(t) & = \mathcal{L}^{-1}\{\pi_1(s)\}(t) = 1 - \tfrac{4}{3}e^{-t} + \tfrac{1}{3}e^{-4t}, \\
		p_2(t) & = \mathcal{L}^{-1}\{\pi_2(s)\}(t) = \tfrac{2}{3}e^{-t} - \tfrac{2}{3}e^{-4t},     \\
		p_3(t) & = \mathcal{L}^{-1}\{\pi_3(s)\}(t) = \tfrac{2}{3}e^{-t} + \tfrac{1}{3}e^{-4t}.
	\end{aligned}
\]

\[
	\vec{p}(t)\xrightarrow[t\to\infty]{}\vec{p}(\infty)=\left(1,\,0,\,0\right).
\]

\subsubsection*{Задача 7.2}


\begin{tikzpicture}[
		>=Stealth, thick,
		state/.style = {draw, circle, minimum size=9mm, font=\small},
		e/.style   = {->, shorten >=2pt, shorten <=2pt},
		lbl/.style  = {font=\scriptsize, fill=white, inner sep=1pt}
	]

	% --- вершины ---
	\node[state] (Q0) at (0,0) {$1$};
	\node[state] (Q1) at (2.5,0) {$2$};
	\node[state] (Q2) at (5,0) {$3$};

	% --- рёбра ---
	\path[e, bend left=12] (Q0) edge node[lbl] {$9$} (Q1);

	\path[e, bend left=12] (Q1) edge node[lbl] {$2$} (Q2);

	\path[e, bend left=24] (Q2) edge node[lbl] {$2$} (Q0);

\end{tikzpicture}


\[
	\Lambda =
	\begin{pmatrix}
		-9 & 9  & 0  \\
		0  & -2 & 2  \\
		2  & 0  & -2
	\end{pmatrix}
\]

\[
	\vec{p}(0) = (1,0,0).
\]

\[
	\vec{p}'(t) = \vec{p}(t)\Lambda,\quad \vec{p}(t)=(p_1(t),p_2(t),p_3(t)).
\]

\[
	\begin{aligned}
		p_1'(t) & =  p_1 (t)\cdot(-9) + p_2(t)\cdot 0 + p_3(t)\cdot 2 = -9p_1(t)+2p_3(t), \\
		p_2'(t) & =  p_1 (t)\cdot 9 + p_2(t)\cdot(-2) + p_3(t)\cdot 0 = 9p_1(t)-2p_2(t),  \\
		p_3'(t) & =  p_1 (t)\cdot 0 + p_2(t)\cdot 2 + p_3(t)\cdot(-2) = 2p_2(t)-2p_3(t).
	\end{aligned}
\]

Преобразование Лапласа:

\[
	\mathcal{L}\{\vec p'(t)\}(s)=s \vec \pi(s)-\vec p(0)
	,\quad
	\mathcal{L}\{\vec p(t)\}(s)=\vec \pi(s)
	,\quad
	s \vec \pi(s)-\vec p(0)=\vec \pi(s)\Lambda.
\]

\[
	\pi_1+\pi_2+\pi_3=\frac{1}{s}
\]

\[
	\begin{aligned}
		s \pi_1(s)-1 & = \pi_1(s)\cdot(-9) + \pi_2(s)\cdot 0 + \pi_3(s)\cdot 2 = -9\pi_1(s)+2\pi_3(s), \\
		s \pi_2(s)-0 & = \pi_1(s)\cdot 9 + \pi_2(s)\cdot(-2) + \pi_3(s)\cdot 0 = 9\pi_1(s)-2\pi_2(s),  \\
		s \pi_3(s)-0 & = \pi_1(s)\cdot 0 + \pi_2(s)\cdot 2 + \pi_3(s)\cdot(-2) = 2\pi_2(s)-2\pi_3(s).
	\end{aligned}
\]

\[
	\begin{aligned}
		(s+9)\pi_1(s)-2\pi_3(s)  & = 1, \\
		-9\pi_1(s)+(s+2)\pi_2(s) & = 0, \\
		-2\pi_2(s)+(s+2)\pi_3(s) & = 0.
	\end{aligned}
\]

\[
	\begin{aligned}
		\pi_2(s) & = \frac{9}{s+2}\pi_1(s),                              \\
		\pi_3(s) & = \frac{2}{s+2}\pi_2(s) = \frac{18}{(s+2)^2}\pi_1(s).
	\end{aligned}
\]

\[
	\pi_1(1+ \frac{9}{s+2} + \frac{18}{(s+2)^2}) = \frac{1}{s}.
\]

\[
	\pi_1\left(\frac{(s+2)^2+9s(s+2)^2+ 18s}{(s+2)^2}\right) = \frac{1}{s}.
\]

\[
	\pi_1(s) = \frac{(s+2)^2}{s((s+2)^2+9(s+2)^2+ 18)}
	= \frac{(s+2)^2}{s(s^2+13s+40)}
	= \frac{(s+2)^2}{s(s+5)(s+8)}.
\]

\[
	\pi_2(s) = \frac{9}{s+2}\pi_1(s) = \frac{9(s+2)}{s(s+5)(s+8)},
	\quad
	\pi_3(s) = \frac{18}{(s+2)^2}\pi_1(s) = \frac{18}{s(s+5)(s+8)}.
\]

\[
	\begin{aligned}
		\pi_1(s) & = \frac{(s+2)^2}{s(s+5)(s+8)} = \frac{A}{s} + \frac{B}{s+5} + \frac{C}{s+8}, \\
		\pi_2(s) & = \frac{9(s+2)}{s(s+5)(s+8)} = \frac{D}{s} + \frac{E}{s+5} + \frac{F}{s+8},  \\
		\pi_3(s) & = \frac{18}{s(s+5)(s+8)} = \frac{G}{s} + \frac{H}{s+5} + \frac{I}{s+8}.
	\end{aligned}
\]

\[
	\begin{aligned}
		A & = \lim_{s\rightarrow 0} s\pi_1(s)      & = \frac{(0+2)^2}{(0+5)(0+8)} = \frac{4}{40} = \frac{1}{10}, \\
		B & = \lim_{s\rightarrow -5} (s+5)\pi_1(s) & = \frac{(-5+2)^2}{-5(-5+8)} = \frac{9}{-15} = -\frac{3}{5}, \\
		C & = \lim_{s\rightarrow -8} (s+8)\pi_1(s) & = \frac{(-8+2)^2}{-8(-8+5)} = \frac{36}{24} = \frac{3}{2},  \\
		D & = \lim_{s\rightarrow 0} s\pi_2(s)      & = \frac{9(0+2)}{(0+5)(0+8)} = \frac{18}{40} = \frac{9}{20}, \\
		E & = \lim_{s\rightarrow -5} (s+5)\pi_2(s) & = \frac{9(-5+2)}{-5(-5+8)} = \frac{-27}{-15} = \frac{9}{5}, \\
		F & = \lim_{s\rightarrow -8} (s+8)\pi_2(s) & = \frac{9(-8+2)}{-8(-8+5)} = \frac{-54}{24} = -\frac{9}{4}, \\
		G & = \lim_{s\rightarrow 0} s\pi_3(s)      & = \frac{18}{(0+5)(0+8)} = \frac{18}{40} = \frac{9}{20},     \\
		H & = \lim_{s\rightarrow -5} (s+5)\pi_3(s) & = \frac{18}{-5(-5+8)} = \frac{18}{-15} = -\frac{6}{5},      \\
		I & = \lim_{s\rightarrow -8} (s+8)\pi_3(s) & = \frac{18}{-8(-8+5)} = \frac{18}{24} = \frac{3}{4}.
	\end{aligned}
\]

\[
	\begin{aligned}
		\pi_1(s) & = \frac{1}{10s} - \frac{3}{5(s+5)} + \frac{3}{2(s+8)}, \\
		\pi_2(s) & = \frac{9}{20s} + \frac{9}{5(s+5)} - \frac{9}{4(s+8)}, \\
		\pi_3(s) & = \frac{9}{20s} - \frac{6}{5(s+5)} + \frac{3}{4(s+8)}.
	\end{aligned}
\]

\[
	\begin{aligned}
		p_1(t) & = \mathcal{L}^{-1}\{\pi_1(s)\}(t)
		= \frac{1}{10} - \frac{3}{5}e^{-5t} + \frac{3}{2}e^{-8t}, \\
		p_2(t) & = \mathcal{L}^{-1}\{\pi_2(s)\}(t)
		= \frac{9}{20} + \frac{9}{5}e^{-5t} - \frac{9}{4}e^{-8t}, \\
		p_3(t) & = \mathcal{L}^{-1}\{\pi_3(s)\}(t)
		= \frac{9}{20} - \frac{6}{5}e^{-5t} + \frac{3}{4}e^{-8t}.
	\end{aligned}
\]

\[
	\vec{p}(t)\xrightarrow [t\to\infty]{}\vec{p}(\infty)=\left(\tfrac{1}{10},\,\tfrac{9}{20},\,\tfrac{9}{20}\right).
\]

Стационарное распределение $\overline r=(r_1,r_2,r_3,\ldots)$

\[
	\begin{cases}
		r_1 = -9 r_1 + 2 r_3 \\
		r_2 = 9 r_1 -2 r_2   \\
		r_3 = 2 r_2 -2 r_2   \\
		r_1 + r_2 + r_3 = 1.
	\end{cases}
	, \quad
	\begin{cases}
		-9r_1 + 2r_3 = 0, \\
		9r_1 - 2r_2 = 0,  \\
		2r_2 - 2r_3 = 0,  \\
		r_1 + r_2 + r_3 = 1.
	\end{cases}
\]

\[
	r_2=r_3=\frac{9}{2}r_1
	, \quad
	r_1(1+\frac{9}{2}+\frac{9}{2})=1
	, \quad
	r_1=\frac{1}{10}
\]

\[
	\vec r = (\frac{1}{10}, \frac{9}{20}, \frac{9}{20})
\]