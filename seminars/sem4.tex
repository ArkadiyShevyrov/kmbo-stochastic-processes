\subsection{Семинар 4}


% ─────────────────────────────────────────────────────────────────────────────
\subsubsection*{Задача 4}

\begin{tikzpicture}[
		>=Stealth, thick,
		state/.style = {draw, circle, minimum size=9mm, font=\small},
		e/.style   = {->, shorten >=2pt, shorten <=2pt},
		lbl/.style  = {font=\scriptsize, fill=white, inner sep=1pt}
	]
	% --- вершины ---
	\node[state] (Q0) at (0,0)   {$0$};
	\node[state] (Q1) at (2.5,0) {$1$};
	\node[state] (Q2) at (5,0)   {$2$};
	\node[state] (Q3) at (7.5,0) {$3$};
	\node[state] (Q4) at (10,0)  {$4$};

	% --- рёбра ---
	\path[e, loop above]   (Q0) edge node[lbl] {$p_0$} (Q0);
	\path[e, bend left=12] (Q0) edge node[lbl] {$p_1$} (Q1);
	\path[e, bend left=56] (Q0) edge node[lbl] {$p_2$} (Q2);

	\path[e, bend left=12] (Q1) edge node[lbl] {$p_0$} (Q0);
	\path[e, loop above]   (Q1) edge node[lbl] {$p_1$} (Q1);
	\path[e, bend left=12] (Q1) edge node[lbl] {$p_2$} (Q2);

	\path[e, bend left=12] (Q2) edge node[lbl] {$p_0$} (Q1);
	\path[e, loop above]   (Q2) edge node[lbl] {$p_1$} (Q2);
	\path[e, bend left=12] (Q2) edge node[lbl] {$p_2$} (Q3);

	\path[e, bend left=12] (Q3) edge node[lbl] {$p_0$} (Q2);
	\path[e, loop above]   (Q3) edge node[lbl] {$p_1$} (Q3);
	\path[e, bend left=12] (Q3) edge node[lbl] {$p_2$} (Q4);

	\path[e, bend left=12] (Q4) edge node[lbl] {$p_0$} (Q3);
	\path[e, loop above]   (Q4) edge node[lbl] {$1-p_0$} (Q4);
\end{tikzpicture}

Найти: $P$, $\overline{r}$ (стационарное распределение), $que$.

\paragraph{Решение.}

Матрица переходных вероятностей:
\[
	P =
	\begin{array}{c|ccccc}
		    & Q_0 & Q_1 & Q_2 & Q_3 & Q_4   \\
		\hline
		Q_0 & p_0 & p_1 & p_2 & 0   & 0     \\
		Q_1 & p_0 & p_1 & p_2 & 0   & 0     \\
		Q_2 & 0   & p_0 & p_1 & p_2 & 0     \\
		Q_3 & 0   & 0   & p_0 & p_1 & p_2   \\
		Q_4 & 0   & 0   & 0   & p_0 & 1-p_0
	\end{array}
\]

Стационарное распределение $\overline r=(r_0,r_1,r_2,r_3,r_4)$:
\[
	\overline r P = \overline r,\qquad \sum_{i=0}^4 r_i = 1,\qquad r_i\ge 0.
\]

Линейная система из $\overline r P=\overline r$:
\[
	\begin{aligned}
		r_0 & = p_0 r_0 + p_0 r_1,                     \\
		r_1 & = p_1 r_0 + p_1 r_1 + p_0 r_2,           \\
		r_2 & = p_2 r_0 + p_2 r_1 + p_1 r_2 + p_0 r_3, \\
		r_3 & = p_2 r_2 + p_1 r_3 + p_0 r_4,           \\
		r_4 & = p_2 r_3 + (1-p_0) r_4.
	\end{aligned}
\]

Среднее число в системе:
$g(Q_i)=v_i$ — физическое значение в состоянии $Q_i$. Тогда
$$que=\sum_{i=0}^{4} v_i r_i$$

\subsubsection*{Задача 4.1}

При $p_0=\tfrac{1}{2},\; p_1=\tfrac{1}{4},\; p_2=\tfrac{1}{4}$:
\[
	\begin{aligned}
		r_0 & = \tfrac{1}{2}r_0 + \tfrac{1}{2}r_1,                                     \\
		r_1 & = \tfrac{1}{4}r_0 + \tfrac{1}{4}r_1 + \tfrac{1}{2}r_2,                   \\
		r_2 & = \tfrac{1}{4}r_0 + \tfrac{1}{4}r_1 + \tfrac{1}{4}r_2 + \tfrac{1}{2}r_3, \\
		r_3 & = \tfrac{1}{4}r_2 + \tfrac{1}{4}r_3 + \tfrac{1}{2}r_4,                   \\
		r_4 & = \tfrac{1}{4}r_3 + \tfrac{1}{2}r_4,\qquad
		r_0+r_1+r_2+r_3+r_4=1.
	\end{aligned}
\]

\[
	\begin{array}{rcl}
		\begin{aligned}
			2r_0 & = r_0 + r_1,              \\
			4r_1 & = r_0 + r_1 + 2r_2,       \\
			4r_2 & = r_0 + r_1 + r_2 + 2r_3, \\
			4r_3 & = r_2 + r_3 + 2r_4,       \\
			2r_4 & = r_3 + 2r_4,
		\end{aligned}
		 & \qquad \Rightarrow \qquad &
		\begin{aligned}
			r_0 - r_1                  & = 0, \\
			-\,r_0 + 3r_1 - 2r_2       & = 0, \\
			-\,r_0 - r_1 + 3r_2 - 2r_3 & = 0, \\
			-\,r_2 + 3r_3 - 2r_4       & = 0, \\
			r_0+r_1+r_2+r_3+r_4        & =1.
		\end{aligned}
	\end{array}
\]

\[
	\left[
		\begin{array}{rrrrr|r}
			1  & -1 & 0  & 0  & 0  & 0 \\
			-1 & 3  & -2 & 0  & 0  & 0 \\
			-1 & -1 & 3  & -2 & 0  & 0 \\
			0  & 0  & -1 & 3  & -2 & 0 \\
			1  & 1  & 1  & 1  & 1  & 1
		\end{array}
		\right]
	\overset{}{\longrightarrow}
	\left[
		\begin{array}{rrrrr|r}
			1 & 0 & 0 & 0 & 0 & \tfrac{4}{15} \\
			0 & 1 & 0 & 0 & 0 & \tfrac{4}{15} \\
			0 & 0 & 1 & 0 & 0 & \tfrac{4}{15} \\
			0 & 0 & 0 & 1 & 0 & \tfrac{2}{15} \\
			0 & 0 & 0 & 0 & 1 & \tfrac{1}{15}
		\end{array}
		\right]
\]

\[
	\overline r =
	\left(\tfrac{4}{15},\;\tfrac{4}{15},\;\tfrac{4}{15},\;\tfrac{2}{15},\;\tfrac{1}{15}\right).
\]

\[
	que = \sum_{i=0}^{4} v_i r_i= 0\cdot\tfrac{4}{15} + 1\cdot\tfrac{4}{15} + 2\cdot\tfrac{4}{15} + 3\cdot\tfrac{2}{15} + 4\cdot\tfrac{1}{15} = \frac{22}{15} \approx 1.4667.
\]

\[
	\boxed{\ que=\frac{22}{15}\approx1.4667\ }.
\]

\subsubsection*{Задача 4.2}

При вероятностях $p_0=\tfrac{1}{4},\; p_1=\tfrac{1}{2},\; p_2=\tfrac{1}{4}$

\[
	\begin{aligned}
		r_0 & = \tfrac{1}{4}r_0 + \tfrac{1}{4}r_1,                                     \\
		r_1 & = \tfrac{1}{2}r_0 + \tfrac{1}{2}r_1 + \tfrac{1}{4}r_2,                   \\
		r_2 & = \tfrac{1}{4}r_0 + \tfrac{1}{2}r_1 + \tfrac{1}{4}r_2 + \tfrac{1}{4}r_3, \\
		r_3 & = \tfrac{1}{4}r_2 + \tfrac{1}{2}r_3 + \tfrac{1}{4}r_4,                   \\
		r_4 & = \tfrac{1}{4}r_3 + \tfrac{3}{4}r_4,\qquad
		r_0+r_1+r_2+r_3+r_4=1.
	\end{aligned}
\]

\[
	\begin{array}{rcl}
		\begin{aligned}
			4r_0 & = r_0 + r_1,              \\
			4r_1 & = 2r_0 + 2r_1 + r_2,      \\
			4r_2 & = r_0 + 2r_1 + r_2 + r_3, \\
			4r_3 & = r_2 + 2r_3 + r_4,       \\
			4r_4 & = r_3 + 3r_4,
		\end{aligned}
		 & \qquad \Rightarrow \qquad &
		\begin{aligned}
			3r_0 - r_1                 & = 0, \\
			-\,2r_0 + 2r_1 - r_2       & = 0, \\
			-\,r_0 - 2r_1 + 3r_2 - r_3 & = 0, \\
			-\,r_2 + 2r_3 - r_4        & = 0, \\
			r_0+r_1+r_2+r_3+r_4        & = 1.
		\end{aligned}
	\end{array}
\]

\[
	\left[
		\begin{array}{rrrrr|r}
			3  & -1 & 0  & 0  & 0  & 0 \\
			-2 & 2  & -1 & 0  & 0  & 0 \\
			-1 & -2 & 3  & -1 & 0  & 0 \\
			0  & 0  & -1 & 2  & -1 & 0 \\
			1  & 1  & 1  & 1  & 1  & 1
		\end{array}
		\right]
	\longrightarrow
	\left[
		\begin{array}{rrrrr|r}
			1 & 0 & 0 & 0 & 0 & \tfrac{1}{16} \\
			0 & 1 & 0 & 0 & 0 & \tfrac{3}{16} \\
			0 & 0 & 1 & 0 & 0 & \tfrac{1}{4}  \\
			0 & 0 & 0 & 1 & 0 & \tfrac{1}{4}  \\
			0 & 0 & 0 & 0 & 1 & \tfrac{1}{4}
		\end{array}
		\right]
\]

\[
	\overline r=\left(\tfrac{1}{16},\,\tfrac{3}{16},\,\tfrac{1}{4},\,\tfrac{1}{4},\,\tfrac{1}{4}\right).
\]

\[
	que=\sum_{i=0}^{4} v_i r_i
	= 0\cdot\tfrac{1}{16}+1\cdot\tfrac{3}{16}+2\cdot\tfrac{1}{4}+3\cdot\tfrac{1}{4}+4\cdot\tfrac{1}{4}
	= \frac{39}{16}=2.4375.
\]

\[
	\boxed{\,que=\frac{39}{16}\approx 2.4375\, }.
\]

\subsubsection*{Задача 4.3}

При вероятностях $p_0=\tfrac{1}{4},\; p_1=\tfrac{1}{4},\; p_2=\tfrac{1}{2}$

\[
	\begin{aligned}
		r_0 & = \tfrac{1}{4}r_0 + \tfrac{1}{4}r_1,                                     \\
		r_1 & = \tfrac{1}{4}r_0 + \tfrac{1}{4}r_1 + \tfrac{1}{4}r_2,                   \\
		r_2 & = \tfrac{1}{2}r_0 + \tfrac{1}{2}r_1 + \tfrac{1}{4}r_2 + \tfrac{1}{4}r_3, \\
		r_3 & = \tfrac{1}{2}r_2 + \tfrac{1}{4}r_3 + \tfrac{1}{4}r_4,                   \\
		r_4 & = \tfrac{1}{2}r_3 + \tfrac{3}{4}r_4,\qquad
		r_0+r_1+r_2+r_3+r_4=1.
	\end{aligned}
\]

\[
	\begin{array}{rcl}
		\begin{aligned}
			4r_0 & = r_0 + r_1,               \\
			4r_1 & = r_0 + r_1 + r_2,         \\
			4r_2 & = 2r_0 + 2r_1 + r_2 + r_3, \\
			4r_3 & = 2r_2 + r_3 + r_4,        \\
			4r_4 & = 2r_3 + 3r_4,
		\end{aligned}
		 & \Rightarrow &
		\begin{aligned}
			3r_0 - r_1                  & = 0, \\
			-\,r_0 + 3r_1 - r_2         & = 0, \\
			-\,2r_0 - 2r_1 + 3r_2 - r_3 & = 0, \\
			-\,2r_2 + 3r_3 - r_4        & = 0, \\
			-\,2r_3 + r_4               & = 0, \\
			r_0+r_1+r_2+r_3+r_4         & = 1.
		\end{aligned}
	\end{array}
\]

\[
	\left[
		\begin{array}{rrrrr|r}
			3  & -1 & 0  & 0  & 0  & 0 \\
			-1 & 3  & -1 & 0  & 0  & 0 \\
			-2 & -2 & 3  & -1 & 0  & 0 \\
			0  & 0  & -2 & 3  & -1 & 0 \\
			0  & 0  & 0  & -2 & 1  & 0 \\
			1  & 1  & 1  & 1  & 1  & 1
		\end{array}
		\right]
	\longrightarrow
	\left[
		\begin{array}{rrrrr|r}
			1 & 0 & 0 & 0 & 0 & \tfrac{1}{60} \\
			0 & 1 & 0 & 0 & 0 & \tfrac{1}{20} \\
			0 & 0 & 1 & 0 & 0 & \tfrac{2}{15} \\
			0 & 0 & 0 & 1 & 0 & \tfrac{4}{15} \\
			0 & 0 & 0 & 0 & 1 & \tfrac{8}{15}
		\end{array}
		\right]
\]

\[
	\overline r=\left(\tfrac{1}{60},\,\tfrac{1}{20},\,\tfrac{2}{15},\,\tfrac{4}{15},\,\tfrac{8}{15}\right).
\]

\[
	que=\sum_{i=0}^{4} v_i r_i
	= 0\cdot\tfrac{1}{60}+1\cdot\tfrac{1}{20}+2\cdot\tfrac{2}{15}
	+3\cdot\tfrac{4}{15}+4\cdot\tfrac{8}{15}
	= \frac{13}{4}=3.25.
\]
\[
	\boxed{\,que=\frac{13}{4}=3.25\, }.
\]
