\section*{§ 1. Цепи Маркова: основные определения. Матрица переходных вероятностей}

Рассмотрим последовательность случайных величин (с.~в.) $\{X_n\}_{n=0}^{\infty}$,
принимающих значения $E_1, E_2, \ldots$.

\textbf{Определение 1.1.}
Последовательность с.~в. $\{X_n\}_{n=0}^{\infty}$ называется цепью Маркова,
если для произвольного набора
$i_0 < i_1 < \dots < i_{k-1} < i_k$ ($k = 3,4,\ldots$)
и любых $E_{j_1},\ldots,E_{j_k}$ справедливо равенство
\[
	P(X_{i_k}=E_{j_k}\mid X_{i_1}=E_{j_1},\ldots,X_{i_{k-1}}=E_{j_{k-1}})=
	P(X_{i_k}=E_{j_k}\mid X_{i_{k-1}}=E_{j_{k-1}}).
\]

\textbf{Определение 1.2.}
Цепь Маркова $\{X_n\}_{n=0}^{\infty}$ называется однородной,
если для всех $i,j$ вероятности $p_{ij}=P(X_{n+1}=E_j\mid X_n=E_i)$
не зависят от $n$.

В дальнейшем рассматриваются только однородные цепи Маркова.
Вероятности $p_{ij}$ называются переходными,
а матрица $P=\|p_{ij}\|$ называется матрицей переходных вероятностей
цепи Маркова.

Если множество значений (множество состояний) цепи Маркова
$S=\{E_1,E_2,\ldots\}$ конечно, то цепь называется конечной цепью Маркова.
При числе элементов в $S$, равном $N=|S|=\infty$,
цепь называется счётной цепью Маркова.

Матрица переходных вероятностей обладает следующими свойствами:
\begin{enumerate}
	\item $p_{ij} \ge 0$;
	\item $\displaystyle \sum_{j=1}^{N} p_{ij} = 1 \quad \text{для всех } i = 1,2,\ldots,N$.
\end{enumerate}

Матрица, удовлетворяющая свойствам 1) и 2), называется стохастической.

Наряду с матрицей переходных вероятностей будем рассматривать матрицу
вероятностей перехода за $n$ шагов (за время $n$):
\[
	P(n) = \|p_{ij}(n)\|, \qquad
	p_{ij}(n) = P(X_{n+k} = E_j \mid X_k = E_i),
\]
при этом $P(1) = P$ — матрица переходных вероятностей.

Распределение $X_n$ будем представлять вектором
\[
	\vec{\rho}(n) = (\rho_1(n), \rho_2(n), \ldots),
	\qquad
	\rho_i(n) = P(X_n = E_i).
\]
Вектор $\vec{\rho}(0)$ называется начальным распределением цепи Маркова.

\textbf{Теорема 1.1.} $P(n) = P^n$.

\textbf{Теорема 1.2.}
$\displaystyle P^{(k+n)} = P^{(k)} P^{(n)}$.

\textbf{Теорема 1.3.}
Для произвольного набора $i_0<i_1<\dots<i_k$ ($k=2,3,\ldots$) и любых
$E_{j_0},E_{j_1},\ldots,E_{j_k}$ верно соотношение
\[
	P(X_{i_k}=E_{j_k},\ldots,X_{i_1}=E_{j_1}\mid X_{i_0}=E_{j_0})
	=
	\prod_{m=1}^{k} p_{\,j_{m-1} j_m}^{\, (i_m-i_{m-1})}.
\]

\textbf{Задача 1.1.}
Дана матрица переходных вероятностей
\[
	P=
	\begin{pmatrix}
		0.5 & 0.1 & 0.4 \\
		0.1 & 0.7 & 0.2 \\
		0.4 & 0.3 & 0.3
	\end{pmatrix},
\]
и начальное распределение
\[
	\vec\rho(0) = (0.5; 0.1; 0.4).
\]

Найти:
\begin{enumerate}
	\item вероятность того, что в моменты времени $t=0,1,2,3$ состояния цепи Маркова будут соответственно $2,1,1,3$;
	\item $\vec\rho(2)$.
\end{enumerate}

\textbf{Решение.}

1) Согласно теореме 1.3 имеем
\begin{align*}
	P(X_0=2,X_1=1,X_2=1,X_3=3)
	 & = 0.1 \cdot 0.2 \cdot 0.5 \cdot 0.4 \\
	 & = 0.004.
\end{align*}

2) По теореме 1.2 имеем
\[
	P(2) = P^{2},
	\qquad
	\vec\rho(2)=\vec\rho(0) P^{2}.
\]

\[
	P^{2} =
	\begin{pmatrix}
		0.43 & 0.23 & 0.34 \\
		0.30 & 0.43 & 0.26 \\
		0.38 & 0.31 & 0.31
	\end{pmatrix},
	\qquad
	\vec\rho(0)=(0.5;\,0.1;\,0.4).
\]

\[
	\vec\rho(2) = (0.397;\,0.283;\,0.320).
\]

\textbf{Задача 1.2.}

В феврале не бывает подряд двух солнечных дней.
Если погода солнечная, то завтра она изменится: с равной вероятностью будет дождь или снег.
Если сегодня снег или дождь, то завтра погода с вероятностью $1/2$ не изменится,
при изменении погоды — в половине случаев будет солнечно.

Сегодня снег. Найти:
\begin{enumerate}
	\item распределение вероятности погоды на завтра;
	\item вероятность того, что снег будет идти завтра и послезавтра.
\end{enumerate}

Введём обозначения:
$E_1$ — дождь,
$E_2$ — солнце,
$E_3$ — снег,
$X_n$ — погода на $n$-й день.

