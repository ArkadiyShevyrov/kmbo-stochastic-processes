\documentclass[a4paper,12pt]{article}

% Пакеты
\usepackage[utf8]{inputenc}
\usepackage[T2A]{fontenc}
\usepackage[russian]{babel}
\usepackage{graphicx}

\usepackage{amsmath,amssymb,amsthm}
\usepackage{multirow}
\usepackage{diagbox}
\usepackage{tabularx}
\usepackage{hyperref}
\usepackage{tikz} 
\usetikzlibrary{arrows.meta,positioning}
\tikzset{lab/.style={font=\footnotesize, inner sep=1pt}}

\usepackage[a4paper,top=2.54cm,bottom=2.54cm,left=3cm,right=1.5cm]{geometry}

\newtheorem{theorem}{Теорема}[section]
\newtheorem{definition}{Определение}[section]
\newtheorem{example}{Пример}[section]

\newcommand{\term}[1]{\textit{#1}\textnormal{}} 

\usepackage{epigraph}

\usepackage[none]{hyphenat}
\pretolerance=10000
\tolerance=2000
\emergencystretch=3em
\hyphenpenalty=10000
\exhyphenpenalty=10000


\setlength{\epigraphwidth}{0.6\textwidth}
\setlength{\epigraphrule}{0pt}          
\renewcommand{\epigraphflush}{flushleft}

% Заголовок
\title{Задачи по теории случайных процессов}
\date{2025}

\begin{document}

\maketitle

\tableofcontents
\newpage

\section*{§ 1. Цепи Маркова: основные определения. Матрица переходных вероятностей}

Рассмотрим последовательность случайных величин (с.~в.) $\{X_n\}_{n=0}^{\infty}$,
принимающих значения $E_1, E_2, \ldots$.

\textbf{Определение 1.1.}
Последовательность с.~в. $\{X_n\}_{n=0}^{\infty}$ называется цепью Маркова,
если для произвольного набора
$i_0 < i_1 < \dots < i_{k-1} < i_k$ ($k = 3,4,\ldots$)
и любых $E_{j_1},\ldots,E_{j_k}$ справедливо равенство
\[
	P(X_{i_k}=E_{j_k}\mid X_{i_1}=E_{j_1},\ldots,X_{i_{k-1}}=E_{j_{k-1}})=
	P(X_{i_k}=E_{j_k}\mid X_{i_{k-1}}=E_{j_{k-1}}).
\]

\textbf{Определение 1.2.}
Цепь Маркова $\{X_n\}_{n=0}^{\infty}$ называется однородной,
если для всех $i,j$ вероятности $p_{ij}=P(X_{n+1}=E_j\mid X_n=E_i)$
не зависят от $n$.

В дальнейшем рассматриваются только однородные цепи Маркова.
Вероятности $p_{ij}$ называются переходными,
а матрица $P=\|p_{ij}\|$ называется матрицей переходных вероятностей
цепи Маркова.

Если множество значений (множество состояний) цепи Маркова
$S=\{E_1,E_2,\ldots\}$ конечно, то цепь называется конечной цепью Маркова.
При числе элементов в $S$, равном $N=|S|=\infty$,
цепь называется счётной цепью Маркова.

Матрица переходных вероятностей обладает следующими свойствами:
\begin{enumerate}
	\item $p_{ij} \ge 0$;
	\item $\displaystyle \sum_{j=1}^{N} p_{ij} = 1 \quad \text{для всех } i = 1,2,\ldots,N$.
\end{enumerate}

Матрица, удовлетворяющая свойствам 1) и 2), называется стохастической.

Наряду с матрицей переходных вероятностей будем рассматривать матрицу
вероятностей перехода за $n$ шагов (за время $n$):
\[
	P(n) = \|p_{ij}(n)\|, \qquad
	p_{ij}(n) = P(X_{n+k} = E_j \mid X_k = E_i),
\]
при этом $P(1) = P$ — матрица переходных вероятностей.

Распределение $X_n$ будем представлять вектором
\[
	\vec{\rho}(n) = (\rho_1(n), \rho_2(n), \ldots),
	\qquad
	\rho_i(n) = P(X_n = E_i).
\]
Вектор $\vec{\rho}(0)$ называется начальным распределением цепи Маркова.

\textbf{Теорема 1.1.}
\[
	P(n) = P^n.
\]

Найдём
\[
	P(Y_1 = 1 \mid Y_{-1} = 2)
	= \frac{P(Y_{-1} = 2,\, Y_1 = 1)}{P(Y_{-1} = 2)}.
\]

\[
	P(Y_{-1} = 2,\, Y_1 = 1)
	= P(X_{-1} = 2,\, X_1 = 1)
	= \frac{3}{8}\cdot \frac{3}{8} + \frac{1}{4}\cdot \frac{1}{4}
	= \frac{3}{10},
\]
\[
	P(X_{-1} = 2) + P(X_{-1} = 3)
	= \frac{3}{8} + \frac{1}{4}.
\]

То есть процесс не является цепью Маркова.

Заметим, что необходимые здесь значения $P(X_{-1}=2)$ и $P(X_{-1}=3)$
следует найти из распределения вероятностей через один шаг.
Непосредственно видно:
\[
	\vec{\rho}(1)=\vec{\rho}(0)P=\left(\tfrac{3}{8},\,\tfrac{3}{8},\,\tfrac{1}{4}\right).
\]

\textbf{Замечание 1.1.}
Если матрица переходных вероятностей $\rho_X$
цепи Маркова $\{X_n\}_{n=0}^{\infty}$ удовлетворяет условию
$\rho_{21}=\rho_{31}=\alpha$, то последовательность с.~в.
$\{Y_n\}_{n=0}^{\infty}$ будет цепью Маркова.

Действительно, при всех $i_0,i_1,\dots,i_{n-1},i_n$ имеем
\begin{align*}
	P(Y_n = 1 \mid Y_0 = i_0,\ldots,Y_{n-1}= i_{n-1})
	 & = P(X_n = 1 \mid Y_0 = i_0,\ldots,Y_{n-1}= i_{n-1}) \\
	 & = p_{11}                                            \\
	 & = P(Y_n = 1 \mid Y_{n-1} = 1).
\end{align*}


\begin{align*}
	 & P(Y_n = 2 \mid Y_0 = i_0,\ldots,Y_{n-1}=1)   \\
	 & = P(X_n = 2 \mid Y_0 = i_0,\ldots,Y_{n-1}=1)
	+ P(X_n = 3 \mid Y_0 = i_0,\ldots,Y_{n-1}=1)    \\
	 & = P(X_n = 2 \mid X_{n-1}=1)
	+ P(X_n = 3 \mid X_{n-1}=1)                     \\
	 & = p_{12} + p_{13}
	= 1 - p_{11}
	= P(Y_n = 2 \mid Y_{n-1}=1).
\end{align*}


\begin{align*}
	P(Y_n = 1 \mid Y_0 = i_0,\ldots,Y_{n-1}=2)
	 & = \frac{P(Y_0=i_0,\ldots,Y_{n-1}=2,\,Y_n=1)}
	{P(Y_0=i_0,\ldots,Y_{n-1}=2)}.
\end{align*}



\newpage
\input{book/book2}
\newpage
\input{book/book3}
\newpage
\input{book/book4}
\newpage
\input{book/book5}
\newpage
\section*{§ 6. Системы массового обслушивания}

Система массового обслуживания (СМО) - это математическая модель реальных систем



\begin{tikzpicture}[
		>=Stealth, thick,
		state/.style = {draw, rectangle, minimum width=10mm, minimum height=8mm, font=\small},
		e/.style   = {->, shorten >=2pt, shorten <=2pt},
		lbl/.style = {font=\scriptsize, fill=white, inner sep=1pt}
	]

	% --- вершины ---
	\node[state] (Q0)   at (0,0)   {$0$};
	\node[state] (Q1)   at (2,0)   {$1$};
	\node (QT) at (4,0) {$\cdots$};
	\node[state] (Qn)   at (6,0)   {$n$};
	\node[state] (Qnp1) at (8,0)  {$n\!+\!1$};
	\node at (10,0) {$\cdots$};
	\node[state] (Qnm)  at (12,0)  {$n\!+\!m\!-\!1$};
	\node[state] (Qnmm) at (14,0)  {$n\!+\!m$};

	% --- рёбра между первыми ---
	\path[e] (Q0) edge[bend left=15] node[lbl] {$\lambda$} (Q1);
	\path[e] (Q1) edge[bend left=15] node[lbl] {$\mu$} (Q0);

	\path[e] (Q1) edge[bend left=15] node[lbl] {$\lambda$} (QT);
	\path[e] (QT) edge[bend left=15] node[lbl] {$2\mu$} (Q1);

	% --- многоточие ---

	% --- центральные рёбра ---
	\path[e] (QT) edge[bend left=15] node[lbl] {$\lambda$} (Qn);
	\path[e] (Qn) edge[bend left=15] node[lbl] {$n\mu$} (QT);

	\path[e] (Qn) edge[bend left=15] node[lbl] {$\lambda$} (Qnp1);
	\path[e] (Qnp1) edge[bend left=15] node[lbl] {$n\mu$} (Qn);

	% --- многоточие справа ---

	% --- последние рёбра ---
	\path[e] (Qnm) edge[bend left=15] node[lbl] {$\lambda$} (Qnmm);
	\path[e] (Qnmm) edge[bend left=15] node[lbl] {$n\mu$} (Qnm);

\end{tikzpicture}

\newpage
\input{book/book7}
\newpage
\input{book/book8}
\newpage
\input{book/book9}
\newpage
\input{book/book10}
\newpage

\end{document}

