\documentclass[a4paper,12pt]{article}

% Пакеты
\usepackage[utf8]{inputenc}
\usepackage[T2A]{fontenc}
\usepackage[russian]{babel}
\usepackage{graphicx}

\usepackage{amsmath,amssymb,amsthm}
\usepackage{multirow}
\usepackage{diagbox}
\usepackage{tabularx}
\usepackage{hyperref}
\usepackage{tikz} 
\usetikzlibrary{arrows.meta,positioning}
\tikzset{lab/.style={font=\footnotesize, inner sep=1pt}}

\usepackage[a4paper,top=2.54cm,bottom=2.54cm,left=3cm,right=1.5cm]{geometry}

\newtheorem{theorem}{Теорема}[section]
\newtheorem{definition}{Определение}[section]
\newtheorem{example}{Пример}[section]

\newcommand{\term}[1]{\textit{#1}\textnormal{}} 

% Заголовок
\title{Случайные процессы}
\date{Сентябрь 2025}

\begin{document}

\maketitle

\tableofcontents
\newpage

% Подключаем остальные файлы
% Подключаем остальные файлы
\newpage
\section{Лекции}

\subsection{Лекция 1}

\subsubsection*{Основные понятия теории случайных процессов}

Теория случайных процессов является развитием теории вероятностей, в ней изучаются не отдельные случайные величины или векторы, а их семейства
$\{X_t,t\in T\subseteq R\}$, зависящие от параметра времени $t$.
Случайным процессом называют семейство в измеримом пространстве $(S,B)$ и определённых на одном вероятностном пространстве $(\Omega, A, P)$.
Пространство $S$ называют пространством состояний случайного процесса.

В зависимости от вида множества параметров $T$ случайный процесс может быть с дискретным или непрерывным временем.
Если $T=\mathbb{Z}_+=\{0,1,2,...\}$ (множество неотрицательных целых числе), то случайный процесс $X_t,t\in T$ называют цепью.
В зависимости от вида $S$ случайный процесс может быть с дискретным или непрерывным пространством состояний.

Случайные процессы применяются для моделирования эволюции реальных стохастических систем. Примерами могут служить: броуновское движение частиц, процессы рождении и гибели в биологических системах, генетическая эволюция, системы массового обслуживания и так далее.

\subsubsection*{Общая классификация случайных процессов}

\begin{table}[ht]
	\centering
	\begin{tabularx}{\textwidth}{|c|>{\raggedright\arraybackslash}X|>{\raggedright\arraybackslash}X|}
		\hline
		\diagbox{$T$}{$S$}                                                            &
		Дискретное                                                                    &
		Непрерывное                                                                     \\
		\hline
		Дискретное                                                                    &
		Последовательность дискретных случайных величин                               &
		Последовательность непрерывных случайных величин                                \\
		\hline
		Непрерывное                                                                   &
		Случайный процесс с непрерывным временем и дискретным пространством состояний &
		Случайный процесс с непрерывным временем и непрерывным пространством состояний  \\
		\hline
	\end{tabularx}
\end{table}

\subsubsection*{Основные понятия теории случайных процессов}

Для любого набора $t_1, t_2, ..., t_n \in T$ вектор
$(X(t_1), X(t_2),..., X(t_n))$ называется конечномерным сечением или $n$-мерным сечением случайного процесса $\{X_t, t\in T\}$.
При фиксированном $\omega \in \Omega$ отображение $t \rightarrow X_t(\omega)$ называется траекторией или выборочной функцией случайного процесса
$\{X_t,\in T \}$.
Семейство $\sigma$-алгебр \{$F_t, t\in T\}$ называется фильтрацией, если
$F_s \subset F_t \subset A$ для всех $s<t$, $s,t\in T$.
Случайный процесс $\{X_t,t\in T\}$ согласован с фильтрацией $\{F_t,t\in T\}$, если случайная величина $X_t$ измерима относительно $F_t$ для всех $t\in T$.

Пусть $S\subseteq R$, $S\in B(R)$, $B=B(S)$.
Случайный процесс называется регулярным, если его траектории в каждой точке $t\in T$ непрерывны справа и имеют конечные пределы слева.
Случайные процессы $\{X_t,t\in T\}$ и $\{Y_t,t\in T\}$ называются стохастически эквивалентными в широком смысле, если для всех $B_i\in B$, $i=1,..,n$, $t_1<t_2<...<t_{n-1}<t_n\in T$, верно равенство $P(X(t_1)\in B_1,...,X(t_n)\in B_n)=P(Y(t_1)\in b_1,...,Y(t_n)\in B_n)$.
Случайные процессы $\{X_t,t\in T\}$ и $\{Y_t, t\in T\}$ называются стохастически эквивалентными, если $P(X(t)=Y(t))=1$ для всех $t\in T$.

Пусть $t_1,t_2,...,t_n\in T$, $t(n)=(t_1,t_2,...,t_n)$, $X(t)=(X(t_1),X(t_2),...,X(t_n))$ - $n$-мерное сечение, $F_{t(n)}(x_1,x_2,...,x_n)=P(X(t_1)\leq x_1, ...,X(t_n)\leq x_n)$ - его функция распределения.
Для любых $t_1,t_2,...,t_n\in T$ $F_{t(n)}(x_1,x_2,...,x_n)$ удовлетворяют всем свойствам функции распределений случайных векторов.
Кроме того, условиям согласованности:
1) $F_{t(n)}(x_1,x_2,...,x_n)=F_{\sigma (t(n))}(x_{\sigma(1)},x_{\sigma(2)},...,x_{\sigma (n)})$, $\sigma (t(n))=(t_{\sigma (1)}, t_{\sigma (2)}, ..., t_{\sigma (n)})$, $\sigma \in S_n$ - группа перестановок;
2) $F_{t(n)}(x_1,...,x_m,+\infty,...,+\infty)=F_{t(m)}(x_1,x_2,...,x_m)$.
Теорема Колмогорова. Пусть задано семейство функций распределения
$F_{t(n)}(x_1,x_2,..x_n)$, удовлетворяющих условиям 1 и 2.
Тогда существует вероятностное пространство $(\Omega, A, P)$ и случайный процесс $\{X_t,t\in T\}$ с данными функциями распределения.

Полная информация о случайном процессе $\{X_t, t\in T\}$ содержится в функциях распределения всех его конечномерных сечений.
В общем случае найти все эти функции распределения практически невозможно.
Эта задача иногда упрощается при рассмотрении марковских случайных процессов.
Случайный процесс $\{X_t, t\in T\}$ называется марковским, если выполняется равенство условных вероятностей
$P(X(t_n)\in B_n | X(t_1)\in B_1, ..., X(t_{n-1})\in B_{n-1})=P(X(t_n)\in B_n | X(t_{n-1})\in B_{n-1})$ для всех $t_1<t_2<...<t_{n-1}<t_n \in T$, $B_i \in B$, $i=1,...,n$.

Вероятности конечномерных сечений для произвольного процесса \\
$P(X(t_1)\in B_1, ..., X(t_{n-1}) \in B_{n-1}, X(t_n) \in B_n) =\\
	P(X(t_n)\in B_n |X(t_1)\in B_1,...,X(t_{n-1})\in B_{n-1})\cdot P(X(t_{n-1})\in B_{n-1} |X(t_1)\in B_1,...,X(t_{n-2})\in B_{n-2})\cdot ... \cdot P(X(t_2)\in B_2 | X(t_1)\in B_1) \cdot P(X(t_1)\in B_1)$.

Вероятности конечномерных сечений для марковского процесса при
$t_1<t_2<...<t_{n-1}<t_n$,
$P(X(t_1)\in B_1,...,X(t_{n-1})\in B_{n-1},X(t_n)\in B_n) = \\
	P(X(t_n)\in B_n | X(t_{n-1})\in B_{n-1}) \cdot P(X(t_{n-1})\in B_{n-1} | X(t_{n-2})\in B_{n-2}) \cdot ... \cdot P(X(t_2)\in B_2 | X(t_1) \in B_1) \cdot P(X(t_1) \in B_1 )$.

\subsection{Лекция 2}

\subsubsection*{Цепи Маркова}

Последовательность случайных величин $\{X_k\}_{k=0}^\infty$ со значениями в $S=\{E_1,E_2,...\}$ называется цепью Маркова, если выполняется равенство условных вероятностей $P(X_{i_k}=E_{j_k} | X_{i_1}=E_{J_1},...,X_{i_{k-1}}=E_{j_{k-1}}) = P(X_{i_{k}}=E_{j_{k}} | X_{i_{k-1}}=E_{j_{k-1}})$ для произвольных $i_1<i_2<...<i_{k-1}<i_k$ и любых $E_{j_{1}},...,E_{j_{k}}$.

Если вероятности $p_{ij}=P(X_{k+1}=E_j | X_l = E_i)$ не зависят от $k$, то цепь Маркова называется однородной. При этом $p_ij$ называются переходными вероятностями, а матрица $P=(p_ij)$ называется матрицей вероятностей перехода за один шаг или переходной матрицей.

В однородной цепи Маркова вероятности $p_{ij}(m)=P(X_{k+m}=E_j | X_k = E_i)$ тоже не зависят от $k$. Матрица $P(m)=(p_{ij}(m))$ называется матрицей вероятностей перехода за $m$ шагов. При этом $0<=P-{ij}(m)<=1$ и $\sum_j p_{ij}(m)=1$. Матрица с двумя такими свойствами называется стохастической.

Вектор $\vec p (m)=(p_1(m), p_2(m), ...)$, где $p_i(m)=P(X_m=E_i)$, называется вектором распределения вероятностей через $m$ шагов. При этом $0\leq p_i(m)\leq 1$ и $\sum_i p_i(m) = 1$.

Вектор $\vec p (0)=(p_1(0),p_2(0),...)$ называется начальным распределением вероятностей цепи Маркова.

\subsubsection*{Свойства цепей Маркова}

Пусть $\{X_k\}_{k=0}^\infty $ - однородная цепь Маркова. Тогда $P(k)=P^k$.

Пусть $\{X_k\}_{k=0}^\infty$ - однородная цепь Маркова. Тогда $\vec p(m)= \vec p (0) P(m)$.

В качестве следствия получаем, что $\vec p(m)=\vec (0)P^m$.
Зная $\vec p(0)$ и $P$ можно найти все конечномерные распределения цепи Маркова: для произвольных $i_1<i_2<...<i_k$ и любых $E_{j_1},...,E_{j_k}$ получаем $P(X_{i_1}=E_{j_1},...,X_{i_k}=E_{j_k}) = P(X_{i_1}=E_{j_1})P(X_{i_2}=E_{j_2}|X_{i_1}=E_{j_1})...P(X_{i_k}=E_{j_k}|X_{i_{k-1}}=E_{j_{k-1}}) = p_{j_1}(i_1)p_{j_1j_2}(i_2-i_1)...p_{j_{k-1}j_k}(i_k-i_{k-1})$.

\subsubsection*{Стационарные распределения}

Пусть $\{X_k\}_{k=0}^\infty$ - однородная цепь Маркова со значениями в $S=\{E_1,E_2,...\}$, с переходной матрицей $P=(p_{ij})$ и начальным распределением вероятностей $\vec p(0)=(p_1(0),p_2(0),...)$.
Если существует $\lim\limits_{m \to \infty} \vec{p}(m) = \lim\limits_{m \to \infty} \vec p (0) P^m = \vec p (\infty )$, то $\vec p (\infty)$ называется предельным распределением вероятностей с начальным распределением $\vec p (0)$.
Если для распределения вероятностей $\vec r (0) = (r_1(0),r_2(0),...)$ выполняется условие $\vec r (m)=\vec r(0)$ для всех $m \geq 1$, то $\vec r = (r_1,r_2,...)=\vec r(0)$ называется стационарным распределением вероятностей.

Если  предельное распределение вероятностей $\vec p (\infty)$ с начальным распределением $\vec p(0)$ существует, то оно будет стационарным, так как $\vec p(\infty)= \lim\limits_{m\to \infty } \vec p (m)= \lim\limits_{m \to \infty} \vec p (m+1) = [\lim\limits_{m \to \infty } \vec p (m)]P = \vec p (\infty ) P$.
Стационарное распределение вероятностей $\vec r = (r_1,r_2,...)$ удовлетворяет уравнениями $r_i=\sum_j r_j p_{ij}$, $i=1,2,...$ и $\sum_j r_j=1$ (уравнение нормировки).

Для конечной цепи Маркова с $n$ состояниями получается система
\[
	\begin{aligned}
		r_1 & = r_1 p_{11} + r_2 p_{21} + \dots + r_n p_{n1}, \\
		r_2 & = r_1 p_{12} + r_2 p_{22} + \dots + r_n p_{n2}, \\
		    & \ \ \vdots                                      \\
		r_n & = r_1 p_{1n} + r_2 p_{2n} + \dots + r_n p_{nn}, \\
		1   & = r_1 + r_2 + \dots + r_n.
	\end{aligned}
\]

Система из первых $n$ уравнений вырожденная, так как из $\vec{r} = \vec{r}P$
следует
$\vec{r}(P - E) = \vec{0}$, ($E$ - единичная матрица),
а матрица $(P - E)$ вырожденная:
\[
	\det(P - E) =
	\det \begin{pmatrix}
		p_{11} - 1 & \dots  & p_{1n}     \\
		\vdots     & \ddots & \vdots     \\
		p_{n1}     & \dots  & p_{nn} - 1
	\end{pmatrix}
	=
	\det \begin{pmatrix}
		p_{11} - 1 & \dots  & 0      \\
		\vdots     & \ddots & \vdots \\
		p_{n1}     & \dots  & 0
	\end{pmatrix}
	= 0.
\]

\subsubsection*{Эргодические цепи Маркова}

Цепь Маркова с переходной матрицей $P = (p_{ij})$ называется эргодической, если

1) существует предел
$\lim\limits_{k \to \infty} p_{ij}(k) = q_{ij};$

2) $q_{ij}$ не зависят от $i$:
$q_{ij} = q_j;$

3)
$q_j > 0 \quad \text{для всех } j.$

Теорема 1. Цепь Маркова эргодическая тогда и только тогда, когда

1) для любого начального распределения $\vec{p}(0)$ существует предел
$\lim\limits_{k \to \infty} p_{j}(k) = q_j;$

2) $q_j$ не зависят от начального распределения;

3) $q_j > 0$ для всех $j$.

Теорема 2. Если для конечной цепи Маркова с переходной матрицей $P=(p_{ij})$ среди корней характеристического уравнения $|P-\lambda E|=0$ условию $|\lambda|=1$ удовлетворяет только корень $\lambda=1$, то предельное распределение существует.

Теорема 3. Если условиях теоремы 2 корень $\lambda=1$ имеет кратность 1, то предельное распределение не зависит от начального распределения.

Теорема 4. (Теорема Маркова) Если для конечной цепи Маркова с переходной матрицей $P=(p_{ij})$ существует такое $S$, что $p_{ij}(s)>0 (P(s)=\{p_{ij}(s)\}=P^s)$ для всех $i$ и $j$, то цепь Маркова является эргодической.

Лемма. Пусть $Q=(q_{ij})$ - стохастическая матрица размерности $n \times n$,
$
	\vec{a} = \begin{pmatrix}
		a_1    \\
		a_2    \\
		\vdots \\
		a_n
	\end{pmatrix}
$ - произвольный вектор-столбец (размерности $n \times 1$), $\vec{b}=Q\vec{a}$; $m_{\vec{a}}=\min a_i$, $M_{\vec{a}}=\max a_i$ ($m_{\vec{b}}$ и $M_{\vec{b}}$ определяются для $\vec{b}$ аналогично).
Тогда:

1) $m\vec{a}\leq m\vec{b} \leq M\vec{b} \leq M\vec{a}$;

2) Если $q_{ij} \geq \varepsilon$ $\forall i,j$, то $M_{}\vec{b}-m_{\vec{b}} \leq  (1-2\varepsilon) (M_{}\vec{a}-m_{\vec{a})}$

\subsubsection*{Классификация состояний цепи Маркова}

Состояние $j$ достижимо из состояния $i$ [ обозначение: $i \to j$], если $\exists k: p_{ij}(k)>0$.
Очевидно, что из $i \to j$ и $j \to s$ следует $i \to s$.
Состояния $i$ и $j$ - сообщающиеся [обозначение $i \rightleftarrows j$] если $i \to j$ и $j \to i$.
Состояние $i$ - существенное, если из $i \to j$ следует $j \to i$.
Если $i$ - существенное состояние и $i \to j$, то состояние $j$ - существенное
Если для состояния $i$ существует такое состояние $j$, что $j$ достижимо из состояния $i$, но $i$ недостижимо из $j$, то состояние $i$ называется несущественным.

Пусть для существенного состояния $i$ $S(i)={j:i \rightleftarrows j}$. Все состояния $j\in S(i)$ являются существенными. Пространство $S$ всех состояний цепи Маркова можно представить в виде $S=S(i_1)\cup S(i_2)\cup ...\cup E$, где $E$ - множество всех несущественных состояний.

Цепь Маркова - неприводимая, если $S=S(i)$ для всех $i\in S$. Периодом состояния $i\in S$ называется $k_i=НОД(k: p_{ii}(k)>0)$.

Теорема. Если $i \rightleftarrows j$, то $k_i=k_j$.

\subsection{Лекция 3}

Обозначим через $f_{ij}(m)$ вероятность того, что из состояния $i$ первый раз попадаем в состояние $j$ на шаге $m$
$f_{ij}(m)=P(X_m=E_j,X_k\ne E_j (0<j<m) | X_0 =E_i))$, через $g_{ij}$ вероятность того, что из состояния $i$ попадаем в состояние $j$ бесконечное число раз. По формуле $f^*_{ij}=\sum_{m=1}^\infty f_{ij}(m)$ находится вероятность того, что исходя из состояния $i$ попадём в состояние $j$ хотя бы один раз.
Верны следующие утверждения:

1) $g_{ij}=\lim_{m \to \infty } \sum _s p_{is}(m) f^*_{sj}$;

2) $g_{ij}=f^*_{ij}g_{jj}$;

3) $i \to j \;\Leftrightarrow\; f_{ij}^* > 0$;

4) $i \rightleftarrows j \;\Leftrightarrow\; f_{ij}^* f_{ji}^* > 0$.

Состояние $i$ называется возвратным, если $f^*_{ii}=1$, и невозвратным, если $f^*_{ii}<1$.
Верны следующие утверждения:

1) $g_{ij}=f^*_{ij}$, если состояние $j$ возвратно, и $g_{ij}=0$, если $j$ невозвратно;

2) $g_{ii}=1$, если состояние $i$ возвратно, и $g_{ii}=0$, если $i$ невозвратно;

3) Если состояние $i$ несущественное, то $i$ невозвратное;

4) Если $g_{ii}=1$ и $f^*_{ij}>0$, то $g_{jj}=1$;

5) Если состояние $i$ возвратно и $i \to j$, то $j$ возвратно и $g_{ij}=g_{ji}=1$;

6) Состояние $i$ возвратное если ряд $\sum_{m=1}^\infty p_{ii}(m)$ расходится, и $i$ невозвратное, если ряд сходится.


Возвратное состояние $i$ называется положительным, если $\lim_{m \to \infty } p_{ii}(mk_i)>0$, и нулевым, если $\lim_{m \to \infty } p_{ii}(mk_i)=0$.
Обозначим через $\mu_{ii}$ среднее время возвращения в состояние $i$ $\mu_{ii}=\sum_{m=1}^\infty mf_{ii}(m)$.
Верны следующие утверждения:

1) Если состояние $i$ возвратное с периодом $k_i$, то $\lim_{m \to \infty} p_{ii} (mk_i)=\frac{k_i}{\mu _{ii}}$;

2) Возвратное состояние $i$ положительно $\Leftrightarrow$ $\mu _{ii}< \infty$;

3) Если состояние $i$ возвратное положительное и $i \rightleftarrows j$, то состояние $j$ так же положительное;

4) Если состояние $i$ возвратное положительное, то $\lim_{m \to \infty} \frac{1}{m} \sum_{k=1}^m p_{ii}(k)=1\frac{1}{\mu _{ii}}$.

Цепь Маркова - неприводимая, если $S=S(i)$ для всех $i\in S$.

Период состояния $i$: $k_i=НОД(k: p_{ii}(k)>0)$.

Цепь Маркова - апериодическая, если $k_i=1$ для всех $i\in S$.

Цепь Маркова - эргодическая $\Leftrightarrow$ она неприводимая и апериодическая.

\subsubsection*{Эргодические цепи Маркова}

Теорема 1. (эргодическая теорема для конечной цепь Маркова). Любая неприводимая непериодическая цепь Маркова $\{v_n, n\geq 0 \}$ с конечным множеством состояний $g$ является эргодической.

Теорема 2. (эргодическая теорема Фостера для счётной цепи Маркова) $\{v_n, n \geq 0 \}$ была эргодической, необходимо и достаточно существование нетривиального решения $\{p_i, i \geq 1 \}$ СУР (4.4)??? такого, что $\sum_{i=1}^\infty |p_i|<\infty$.
Решение $\{p_i, i \geq 1 \}$ с точностью до нормирующего множителя совпадает с предельным (стационарным) распределением.

Теорема 3. (для счётной цепи Маркова). Для того, чтобы  неприводимая непериодическая цепь Маркова $\{v_n, n\geq 0 \}$ была эргодической, достаточно существование числа $\varepsilon > 0$, целого числа $i_0$ и набора неотрицательных чисел $x_1,x_2,...$ таких что
$\sum_{j=1}^\infty p_{ij} x_j \leq x_i - \varepsilon$, $i \geq i_0$;
$\sum_{j=1}^\infty p_{ij} x_i < \infty$, $i < i_0$;

\subsubsection*{Марковские процессы с непрерывным временем}

Случайный процесс $X_t, t \geq 0$ называется марковским, если для любого целого неотрицательного $m$, любых моментов времени $0\leq s_1<s_2<...<s_m<s$, $t>0$, любого набора состояний $E_{i_1}, E_{i_2},...,E_{i_m},E_i, E_j$ выполнено равенство $P(X_{s+t}=E_j | X_{s_1}=E_{i_1}, ..., X_{s_m}=E_{i_m}, X_s=E_i)= P(X_{s+t}=E_j | X_s=E_i)$

\subsubsection*{Однородный марковский процесс с непрерывным временем}

Процесс $X_t$ называется однородным (по времени), если условная вероятность $P(X_{s+t}=E_j|X_s=E_i)$ перехода из состояния $E_i$ в состояние $E_j$ за время $t$ не зависит от $s$.
Обозначим $p_{ij}(t)=P(X_{s+t}=E_j | X_s=E_i)$.
Свойства:

1) $p_{ij}(0)=0$, если $i\ne j$, а $p_{ii}(0)=1$;

2) $0\leq p_{ij}(t)\leq 1$;

3) $\sum _j p_{ij}(t)=1$.

Матрица вероятностей переходя за время $t$: $P(t)=||p_{ij}(t)||$.
Предполагаем что переходные вероятности $p_{ij}(t)$ дифференцируемы в нуле: $p'_{ij}(0)=\lambda_{ij}$, при $i\ne j$ $p_{ij}(t)=\lambda_{ij}t+o(t)$, $p_{ii}(t)=1+\lambda_{ii}t+o(t)$.
$P'(0)=\Lambda=||\lambda_{ij}||$ - матрица интенсивностей (плотностей вероятностей) перехода.
При $i\ne j$ $\lambda_{ij}$ называется интенсивностью (плотностью вероятности) перехода из $E_i$ в $E_j$.

\subsection{Лекция 4}

\subsubsection*{Марковский процесс с непрерывным временем}

Величина $\lambda_{ji} \cdot p_j(t)$ называется потоком вероятности из состояния $E_j$ в состояние $E_i$ в момент времени $t$.

Из дифференциальных уравнений Колмогорова следует: произвольная вероятности состояния равна
сумме всех потоков вероятностей, приходящих в это состояние, минус сумма всех потоков вероятностей,
выходящих из этого состояния.

\textbf{Свойства матрицы интенсивностей перехода:}

1) $\lambda_{ij}\geq 0$, при $i\ne j$;

2) $\lambda_{ii}<0$;

3) $\sum_j \lambda_{ij}=0$, $\lambda_{ii}=-\sum_{j\ne i} \lambda_{ij}$.

\textbf{Система дифференциальных уравнений Колмогорова:}

$\frac{dp_i(t)}{dt}=\sum_j \lambda_{ji} p_j(t)=\lambda_{ii} p_i(t) + \sum_{j\ne i} \lambda_{ji} p_j(t)
	=\sum_{j\ne i} \lambda_{ji} p_j(t) - p_i(t) \sum_{j\ne i} \lambda_{ji}$

---

$\left\{ {P}(t) \right\}_{t \ge 0}$

1) ${P}(0) = {I}$,

2) ${P}(t+s) = {P}(t)\,{P}(s)$,

3) $\vec{p}(t) = \vec{p}(0)\,{P}(t)$.

---

\[
	\lambda_{ij} = \lim_{\tau \to +0}
	\frac{p_{ij}(\tau) - p_{ij}(0)}{\tau}
\]

\[
	i \ne j \Rightarrow
	\lambda_{ij} = \lim_{\tau \to +0}
	\frac{p_{ij}(\tau)}{\tau} \ge 0,
	\qquad
	i = j \Rightarrow
	\lambda_{ii} = \lim_{\tau \to +0}
	\frac{p_{ii}(\tau) - 1}{\tau} \le 0.
\]

---


\[
	\sum_j p_{ij}(t) = 1,
	\qquad
	\left( \sum_j p_{ij}(t) \right)'_{t=0} = (1)' = 0,
\]

\[
	\sum_j \lambda_{ij} = 0,
	\qquad
	\Rightarrow
	\qquad
	\lambda_{ii} = - \sum_{j \ne i} \lambda_{ij}.
\]


---

\[
	\vec{p}'(t)
	= \lim_{\tau \to 0}
	\frac{\vec{p}(t+\tau) - \vec{p}(t)}{\tau}
	=
	\lim_{\tau \to 0}
	\frac{\vec{p}(t){P}(\tau) - \vec{p}(t)}{\tau}
\]

\[
	= \vec{p}(t)
	\lim_{\tau \to 0}
	\frac{P(\tau) - P(0)}{\tau}
	= \vec{p}(t)\,{\Lambda}.
\]

---

\begin{align*}
	p_i'(t)
	 & = \sum_j p_j(t)\,\lambda_{ji}
	\\[4pt]
	 & = \sum_{j \ne i} p_j(t)\,\lambda_{ji} + p_i(t)\,\lambda_{ii}
	\\[4pt]
	 & = \sum_{j \ne i} p_j(t)\,\lambda_{ji} - \sum_{j \ne i} p_i(t)\,\lambda_{ij}.
\end{align*}

---

\begin{align*}
	{P}'(t)
	 & = \lim_{\tau \to 0}
	\frac{{P}(t+\tau) - {P}(t)}{\tau}
	\\[4pt]
	 & = \lim_{\tau \to 0}
	\frac{{P}(t){P}(\tau) - {P}(t)}{\tau}
	\\[4pt]
	 & = {P}(t)
	\lim_{\tau \to 0}
	\frac{{P}(\tau) - {P}(0)}{\tau}
		= {P}(t)\,{\Lambda}
	\\[4pt]
	 & = \lim_{\tau \to 0}
	\frac{{P}(t){P}(\tau) - {P}(t)}{\tau}
	= \lim_{\tau \to 0}
	\frac{{P}(t+\tau) - {P}(t)}{\tau}
	\\[4pt]
	 & =
	\left(
	\lim_{\tau \to 0}
	\frac{{P}(t+\tau) - {P}(t)}{\tau}
	\right)
	{P}(t)
	= {\Lambda}\,{P}(t).
\end{align*}

---


Дифференциальные уравнения Колмогорова

Векторная форма дифференциальных уравнений Колмогорова для вероятностей состояний:
$\vec{p'}(t)=\vec{p}(t)\Lambda$, где $\vec{p}(t)=(p_0(t),p_1(t),...,p_n(t),...)$.

Прямое уравнение Колмогорова: $P'(t)=P\Lambda$.

Обратное уравнение Колмогорова: $P'(t)=\Lambda P(t)$.

---

Марковский процесс с непрерывным временем ($\vec{p}(t), \{P(t)\}_{t>0}$) - эргодический:
\[
	\exists \lim_{t \to +\infty} {P}(t) = {Q}
	=
	\begin{pmatrix}
		q_{11} & q_{12} & \cdots \\
		q_{21} & q_{22} & \cdots \\
		\vdots & \vdots & \ddots
	\end{pmatrix},
	\qquad
	q_{ij} > 0 \ \forall i,j.
\]

---

\begin{enumerate}
	\item
	      $\displaystyle
		      \exists \lim_{t \to +\infty}
		      \overline{\mathbf{p}}(t)
		      =
		      \overline{\mathbf{q}}
		      = (q_1, q_2, \ldots)
	      $

	\item предел не зависит от $\overline{\mathbf{p}}(0)$

	\item $q_j > 0 \ \forall j$
\end{enumerate}

---

Процесс Маркова$\{ X_t \}_{t \ge 0}$ — эргодический $\Longleftrightarrow$
1) неприводим, 2) существует стационарное распределение.
---

Распределение вероятностей состояний, которое не зависит от времени $p_i(t+\tau ) = p_i(t)=p_i$
для любых $t,\tau\leq 0$ и любых $i=1,2,...$ называется стационарным распределением.

Система линейных алгебраических уравнений для стационарных вероятностей
\[
	\begin{cases}
		\sum_j  \lambda_{ji} r_j = 0, \quad i=1,2,... \\
		\sum_j r_j = 1
	\end{cases}
\]

---

\textbf{Свойства матрицы интенсивностей перехода:}
\begin{enumerate}
	\item $\lambda_{ij} \ge 0 \quad$ при $i \ne j$;
	\item $\lambda_{ii} \le 0$;
	\item $\displaystyle \sum_j \lambda_{ij} = 0,
		      \qquad
		      \lambda_{ii} = - \sum_{j \ne i} \lambda_{ij}.$
\end{enumerate}

\vspace{1em}

\textbf{Система дифференциальных уравнений Колмогорова}
\[
	\frac{dp_i(t)}{dt}
	= \sum_j \lambda_{ji}\,p_j(t)
	= \lambda_{ii}\,p_i(t)
	+ \sum_{j \ne i} \lambda_{ji}\,p_j(t)
\]
\[
	= \sum_{j \ne i} \lambda_{ji}\,p_j(t)
	- p_i(t)\sum_{j \ne i} \lambda_{ij},
	\qquad i = 1, 2, \ldots
\]

\textbf{Стациорнаное распределение}

Из уравнений следует: при стационарном распределении для каждого стояния сумма всех потоков вероятностей, приходящих в это состояние, равна сумме всех потоков вероятностей, выходящих из этого состояния.

\[
	\sum_{j \ne i} \lambda_{ji} r_j = r_i \sum_{j \ne i} \lambda_{ij} r_i
\]

---

\subsubsection*{Время пребывания марковского процесса с непрерывным временем в состоянии}

Пусть $\tau(i)$ — время пребывания марковского процесса $X(t)$ в состоянии $i$.
Получаем
$P\bigl(\tau(i) > t\bigr)= \lim_{n \to \infty} [p_{ii}(\frac{t}{n})]^n.$

Но $p_{ii}\!\left(\tfrac{t}{n}\right)= 1 + \lambda_{ii}\tfrac{t}{n} + o\!\left(\tfrac{1}{n}\right).$
Пусть $\lambda_i = -\lambda_{ii} = \sum_{j \ne i} \lambda_{ij}$.

Тогда
$\lim_{n \to \infty} [p_{ii}(\frac{t}{n})]^n
	= \lim_{n \to \infty}
	\left[1 - \lambda_i \tfrac{t}{n} + o\!\left(\tfrac{1}{n}\right)\right]^n
	= e^{-\lambda_i t}.$

Значит, $F_{\tau(i)}(t) = 1 - e^{-\lambda_i t}$,
т.\,е. $\tau(i)$ имеет показатель­ное распределение с параметром $\lambda_i$.


\subsection{Лекция 5}

Пример. В системе 2 устройства и один мастер, проводящий ремонт.
Работа системы описывается марковским процессом $\{X_t, t\in [0, +\infty) \}$,
значением $X_t$ является число исправных устройств в момент времени $t$.
Каждое устройство выходит из строя с интенсивностью $\lambda=1$,
востановление устройства проходит с интенсивностью $\mu=2$.

Граф системы имеет вид:

\begin{center}
	\begin{tikzpicture}[
			>=Stealth, thick,
			state/.style = {draw, circle, minimum size=9mm, font=\small},
			e/.style   = {->, shorten >=2pt, shorten <=2pt},
			lbl/.style  = {font=\scriptsize, fill=white, inner sep=1pt}
		]

		% --- вершины ---
		\node[state] (Q1) at (0,0) {$2$};
		\node[state] (Q2) at (2.5,0) {$1$};
		\node[state] (Q3) at (5,0) {$0$};

		% --- рёбра ---
		% Q1
		\path[e, bend left=12]   (Q1) edge node[lbl] {$2$} (Q2);

		% Q2
		\path[e, bend left=12] (Q2) edge node[lbl] {$2$} (Q1);
		\path[e, bend left=12] (Q2) edge node[lbl] {$1$} (Q3);

		% Q3
		\path[e, bend left=12]   (Q3) edge node[lbl] {$2$} (Q2);

	\end{tikzpicture}
\end{center}

Найдём стационарные вероятности состояний.

---


Составим систему уравнений Колмогорова:
\[
	\begin{cases}
		p_0'(t) = -2p_0(t) + p_1(t);           \\
		p_1'(t) = 2p_0(t) - 3p_1(t) + 2p_2(t); \\
		p_2'(t) = 2p_1(t) - 2p_2(t).
	\end{cases}
\]

Решая систему уравнений для стационарных вероятностей:
\[
	\begin{cases}
		r_1 = 2r_0;         \\
		3r_1 = 2r_0 + 2r_2; \\
		r_1 = r_2;          \\
		r_0 + r_1 + r_2 = 1;
	\end{cases}
\]

находим
\[
	(r_0, r_1, r_2) = (0.2, 0.4, 0.4).
\]

---

\textbf{Задача 4.6.} \\
\textit{Два устройства и один мастер. Пусть $E_i$~--- работают $i$ устройств.}

\vspace{1em}

\begin{center}
	\begin{tikzpicture}[
			>=Stealth, thick,
			state/.style={draw, circle, minimum size=8mm, font=\small}
		]
		\node[state] (2) at (0,0) {2};
		\node[state] (1) at (2.5,0) {1};
		\node[state] (0) at (5,0) {0};

		\path[->] (2) edge[bend left=20] node[above] {$2\lambda$} (1);
		\path[->] (1) edge[bend left=20] node[below] {$\mu$} (2);

		\path[->] (1) edge[bend left=20] node[above] {$\lambda$} (0);
		\path[->] (0) edge[bend left=20] node[below] {$\mu$} (1);
	\end{tikzpicture}
\end{center}

\vspace{1em}

\[
	\begin{cases}
		2\lambda r_2 = \mu r_1,                     \\[4pt]
		(\lambda+\mu) r_1 = 2\lambda r_2 + \mu r_0, \\[4pt]
		\mu r_0 = \lambda r_1,                      \\[4pt]
		r_0 + r_1 + r_2 = 1.
	\end{cases}
\]

Из первого и третьего уравнений:
\[
	r_1 = \dfrac{2\lambda}{\mu} r_2,
	\qquad
	r_0 = \dfrac{\lambda}{\mu} r_1 = \dfrac{2\lambda^2}{\mu^2} r_2.
\]

Подставляем в нормировочное:
\[
	r_2\!\left(1 + \dfrac{2\lambda}{\mu} + \dfrac{2\lambda^2}{\mu^2}\right) = 1,
	\quad\Rightarrow\quad
	r_2 = \dfrac{\mu^2}{\mu^2 + 2\lambda\mu + 2\lambda^2}.
\]

Средняя доля времени простоя мастера: $r_2$

Средняя доля времени занятости мастера:
\[
	1 - r_2 = \dfrac{2\lambda(1+\mu)}{\mu^2 + 2\lambda\mu + 2\lambda^2}.
\]

\[
	r_1 = \dfrac{2\lambda\mu}{\mu^2 + 2\lambda\mu + 2\lambda^2},
	\qquad
	r_0 = \dfrac{2\lambda^2}{\mu^2 + 2\lambda\mu + 2\lambda^2}.
\]


\textbf{Задача 4.6} \quad [$\lambda = 1$, $\mu = 2$]

\vspace{1em}

\begin{center}
	\begin{tikzpicture}[
			>=Stealth, thick,
			state/.style={draw, circle, minimum size=8mm, font=\small}
		]
		\node[state] (2) at (0,0) {2};
		\node[state] (1) at (2.5,0) {1};
		\node[state] (0) at (5,0) {0};

		\path[->] (2) edge[bend left=20] node[above] {$2\lambda$} (1);
		\path[->] (1) edge[bend left=20] node[below] {$\mu$} (2);

		\path[->] (1) edge[bend left=20] node[above] {$\lambda$} (0);
		\path[->] (0) edge[bend left=20] node[below] {$\mu$} (1);
	\end{tikzpicture}
\end{center}

\vspace{1em}

\[
	\begin{cases}
		p_0'(t) = -2p_0(t) + p_1(t),           \\[4pt]
		p_1'(t) = 2p_0(t) + 2p_2(t) - 3p_1(t), \\[4pt]
		p_2'(t) = 2p_1(t) - 2p_2(t),           \\[4pt]
		p_2(0) = 1,\quad p_0(0) = p_1(0) = 0,  \\[4pt]
		p_0 + p_1 + p_2 = 1.
	\end{cases}
\]

$$p_i(t)=\pi_i (s)$$

\vspace{1em}
Переходим к изображению Лапласа:
$$
	\begin{cases}
		S \pi_0  = -2 \pi_0 + \pi_1,     \\
		S \pi_2 - 1 = 2 \pi_1 - 2 \pi_2. \\
	\end{cases}
$$

Сумма вероятностей:
\[
	\pi_0 + \pi_1 + \pi_2 = \frac{1}{S}.
\]

Из первого и последнего уравнений:
\[
	\pi_0 = \frac{\pi_1}{S + 2}, \qquad
	\pi_2 = \frac{1 + 2\pi_1}{S + 2}.
\]

Подставим:
\[
	\pi_1\!\left(\frac{1}{S + 2} + 1 + \frac{2}{S + 2}\right)
	= \text{[тут у Лобузова какой-то бред с числами]}.
\]

Следовательно:
\[
	p_1(t) = \frac{2}{5}\left(1 - e^{-5t}\right).
\]

Аналогично:
\[
	\pi_0 = \frac{1}{S} \cdot \frac{1}{5} - \frac{1}{3} \cdot \frac{1}{S + 2} + \frac{2}{15} \cdot \frac{1}{S + 5},
\]
\[
	p_0(t) = \frac{1}{5} - \frac{1}{3}e^{-2t} + \frac{2}{15}e^{-5t}.
\]

---

\[
	p_1(t) = \frac{2}{5} - \frac{2}{5} e^{-5t}
	\quad \xrightarrow[t \to +\infty]{} \quad
	\frac{2}{5},
\]

\[
	p_0(t) = \frac{1}{5} - \frac{1}{3} e^{-2t} + \frac{2}{15} e^{-5t}
	\quad \xrightarrow[t \to +\infty]{} \quad
	\frac{1}{5},
\]

\[
	p_2(t) = \frac{2}{5} + \frac{1}{3} e^{-2t} + \frac{4}{15} e^{-5t}
	\quad \xrightarrow[t \to +\infty]{} \quad
	\frac{2}{5}.
\]

\vspace{1em}

\[
	\pi_i(s) \; = \; p_i(t),
	\qquad
	p_i'(t) \; = \; s\,\pi_i(s) - p_i(0).
\]

---

Пусть $\tau$ — время нахождения в состоянии $i$,
а $\tau_{(s)}$ — время нахождения в этом состоянии после времени $s$.

\[
	P(\tau_{(s)} > t)
	= P(\tau > s + t \mid \tau > s)
	= \frac{e^{-\lambda (t + s)}}{e^{-\lambda s}}
	= e^{-\lambda t},
	\qquad [\,\lambda = \lambda_i\,].
\]

Отсюда
\[
	P(\tau_{(s)} \le t) =
	\begin{cases}
		0,                  & t < 0,   \\[6pt]
		1 - e^{-\lambda t}, & t \ge 0,
	\end{cases}
	\qquad
	\text{— показатель­ное распределение с параметром } \lambda.
\]

Следовательно $\tau_{(s)} \sim \tau$.

\subsubsection*{Процессы рождения и гибели. Процесс Пуассона}

Граф процесса рождения и гибели с конечным числом состояний

\begin{center}
	\begin{tikzpicture}[
			>=Stealth, thick,
			state/.style={draw, rectangle, minimum width=10mm, minimum height=8mm, font=\small, align=center}
		]

		% узлы
		\node[state] (0) at (0,0) {0};
		\node[state] (1) at (2,0) {1};
		\node[state] (2) at (4,0) {2};
		\node[state] (k) at (6,0) {$\dots$};
		\node[state] (n) at (8,0) {$n$};

		% стрелки вправо (рождения)
		\path[->] (0) edge[bend left=20] node[above] {$\lambda_0$} (1);
		\path[->] (1) edge[bend left=20] node[above] {$\lambda_1$} (2);
		\path[->] (2) edge[bend left=20] node[above] {$\lambda_2$} (k);
		\path[->] (k) edge[bend left=20] node[above] {$\lambda_{n-1}$} (n);

		% стрелки влево (гибели)
		\path[->] (1) edge[bend left=20] node[below] {$\mu_1$} (0);
		\path[->] (2) edge[bend left=20] node[below] {$\mu_2$} (1);
		\path[->] (k) edge[bend left=20] node[below] {$\mu_3$} (2);
		\path[->] (n) edge[bend left=20] node[below] {$\mu_n$} (k);

	\end{tikzpicture}
\end{center}

Система дифференциальных уравнений Колмогорова:

\[
	\begin{cases}
		p_0'(t) = -\lambda_0 p_0(t) + \mu_1 p_1(t), \\[4pt]
		p_k'(t) = \lambda_{k-1} p_{k-1}(t)
		- (\lambda_k + \mu_k) p_k(t)
		+ \mu_{k+1} p_{k+1}(t),
		 & 1 \le k < n,                             \\[4pt]
		p_n'(t) = \lambda_{n-1} p_{n-1}(t) - \mu_n p_n(t).
	\end{cases}
\]

---

Стационарные вероятности состояний $r_0, r_1, r_2, \ldots, r_n$ процесса рождения и гибели
с конечным числом состояний удовлетворяют системе линейных алгебраических уравнений:

\[
	\begin{cases}
		0 = -\lambda_0 r_0 + \mu_1 r_1, \\[4pt]
		0 = \lambda_{k-1} r_{k-1} - (\lambda_k + \mu_k) r_k + \mu_{k+1} r_{k+1},
		 & 1 \le k < n,                 \\[4pt]
		0 = \lambda_{n-1} r_{n-1} - \mu_n r_n.
	\end{cases}
\]

А также уравнению нормировки:
\[
	\sum_{k=0}^{n} r_k = 1.
\]

\subsection{Лекция 6}

Из уравнений для стационарных вероятностей состояний следуют формулы
\[
	\lambda_{k-1} r_{k-1} = \mu_k r_k, \quad \text{при } k = 1, 2, \ldots, n.
\]
Значит
\[
	r_1 = \frac{\lambda_0}{\mu_1} r_0, \quad
	r_k = \frac{\lambda_{k-1}}{\mu_k} r_{k-1} = \ldots =
	\frac{\lambda_{k-1} \lambda_{k-2} \cdots \lambda_0}{\mu_k \mu_{k-1} \cdots \mu_1} r_0.
\]

Из уравнения нормировки получаем
\[
	r_0 =
	\left\{
	1 +
	\frac{\lambda_0}{\mu_1} +
	\frac{\lambda_0 \lambda_1}{\mu_1 \mu_2} +
	\ldots +
	\frac{\lambda_0 \lambda_1 \cdots \lambda_{n-1}}{\mu_1 \mu_2 \cdots \mu_n}
	\right\}^{-1}.
\]

---

\[
	\lambda_k r_k = \lambda_{k-1} r_{k-1} - \mu_k r_k + \mu_{k+1} r_{k+1}.
\]

---

Граф процесса рождения и гибели с бесконечным числом состояний:

\begin{center}
	\begin{tikzpicture}[
			>=Stealth, thick,
			state/.style={draw, rectangle, minimum width=10mm, minimum height=8mm, font=\small, align=center}
		]
		% узлы
		\node[state] (0) at (0,0) {0};
		\node[state] (1) at (2,0) {1};
		\node[state] (2) at (4,0) {2};
		\node[state] (d1) at (6,0) {$\cdots$};

		% стрелки вправо (рождения)
		\path[->] (0) edge[bend left=20] node[above] {$\lambda_0$} (1);
		\path[->] (1) edge[bend left=20] node[above] {$\lambda_1$} (2);
		\path[->] (2) edge[bend left=20] node[above] {$\lambda_2$} (d1);

		% стрелки влево (гибели)
		\path[->] (1) edge[bend left=20] node[below] {$\mu_1$} (0);
		\path[->] (2) edge[bend left=20] node[below] {$\mu_2$} (1);
		\path[->] (d1) edge[bend left=20] node[below] {$\mu_3$} (2);
	\end{tikzpicture}
\end{center}

Система дифференциальных уравнений Колмогорова для процесса рождения и гибели
с бесконечным числом состояний:

\[
	\begin{cases}
		p_0'(t) = -\lambda_0 p_0(t) + \mu_1 p_1(t), \\[6pt]
		p_k'(t) = \lambda_{k-1} p_{k-1}(t)
		- (\lambda_k + \mu_k) p_k(t)
		+ \mu_{k+1} p_{k+1}(t), \quad k \ge 1.
	\end{cases}
\]

---

Стационарные вероятности состояний $r_0, r_1, r_2, \ldots$ процесса рождения и гибели
с бесконечным числом состояний удовлетворяют системе линейных алгебраических уравнений:
\[
	\begin{cases}
		0 = -\lambda_0 r_0 + \mu_1 r_1, \\[6pt]
		0 = \lambda_{k-1} r_{k-1} - (\lambda_k + \mu_k) r_k + \mu_{k+1} r_{k+1}, \quad k \ge 1,
	\end{cases}
\]
а также уравнению нормировки:
\[
	\sum_{k=0}^{\infty} r_k = 1.
\]

---

Из уравнений для стационарных вероятностей состояний следуют формулы
\[
	\lambda_{k-1} r_{k-1} = \mu_k r_k, \quad \text{при } k = 1, 2, \ldots, n.
\]
Значит
\[
	r_1 = \frac{\lambda_0}{\mu_1} r_0, \quad
	r_k = \frac{\lambda_{k-1}}{\mu_k} r_{k-1}
	= \ldots =
	\frac{\lambda_{k-1} \lambda_{k-2} \cdots \lambda_0}{\mu_k \mu_{k-1} \cdots \mu_1} r_0.
\]

Из уравнения нормировки получаем
\[
	r_0 =
	\left\{
	1 +
	\frac{\lambda_0}{\mu_1} +
	\frac{\lambda_0 \lambda_1}{\mu_1 \mu_2} +
	\ldots +
	\frac{\lambda_0 \lambda_1 \cdots \lambda_{n-1}}{\mu_1 \mu_2 \cdots \mu_n}
	+ \ldots
	\right\}^{-1},
	\quad \text{если }
	\lim_{k \to \infty} \frac{\lambda_{k-1}}{\mu_k} < 1.
\]

---

\subsubsection*{Пуассоновский поток}

Пуассоновский (простейший) поток событий — поток, удовлетворяющий свойствам:

\begin{enumerate}
	\item стационарность;
	\item отсутствие последействия;
	\item ординарность.
\end{enumerate}

Для пуассоновского потока событий случайный процесс
$\{ X(t); \, t \ge 0 \}$
является марковским и называется процессом Пуассона.

---

\subsubsection*{Поток однородных событий}

\begin{enumerate}
	\item Случайная последовательность
	      $t_1 \le t_2 \le \ldots$
	      — \textit{моменты наступления событий};

	\item
	      $\{\tau_k = t_k - t_{k-1}; \, k \ge 1\}$,
	      $(t_0 = 0)$ —
	      \textit{интервалы между событиями};

	\item
	      $\{ X(t); \, t \ge 0 \}$ —
	      \textit{число событий на отрезке} $[0, t]$.
\end{enumerate}

---

Два потока называются эквивалентными, если у них совпадают
для любого $n \ge 1$ конечномерные распределения
$(\tau_1, \ldots, \tau_n) \quad \text{и} \quad (\tau_1', \ldots, \tau_n')$,
либо для любых $s_1, \ldots, s_n$ конечномерные распределения
$(X(s_1), \ldots, X(s_n)) \quad \text{и} \quad (X'(s_1), \ldots, X'(s_n))$.

---

\subsubsection*{Рекуррентный поток с запаздыванием}

Рекуррентный поток с запаздыванием, определяемый функциями распределения $F_1(t)$ и $F(t)$:

\begin{enumerate}
	\item $\{\tau_k; \, k \ge 1\}$ — независимы в совокупности;
	\item $F(t) = P(\tau_k \le t), \quad k \ge 2$;
	\item $F_1(t) = P(\tau_1 \le t)$.
\end{enumerate}

Поток называется рекуррентным потоком, если $F_1(t) = F(t)$.

---

\subsubsection*{Пуассоновский поток с параметром $\lambda$}

Пуассоновский поток с параметром $\lambda$ — рекуррентный поток, и

\[
	F(t) = P(\tau_k \le t) = 1 - e^{-\lambda t},
	\quad t \ge 0, \; k \ge 1.
\]

---

\subsubsection*{Простейший поток}

Поток заявок называется \textbf{стационарным}, если конечномерные распределения
\[
	(X(t + s_1), \ldots, X(t + s_n))
\]
не зависят от $t$.

Стационарность означает, что вероятность поступления того или иного числа заявок
на участке времени длины $\tau$ не зависит от его расположения на оси времени,
а зависит только от его длины.

---

Поток заявок называется потоком с отсутствием последействия, если случайные величины
\[
	X(t_1), \; X(t_2) - X(t_1), \; \ldots, \; X(t_n) - X(t_{n-1})
\]
независимы в совокупности для любых $t_1 < t_2 < \ldots < t_n$.

Отсутствие последействия означает, что вероятность попадания того или иного числа заявок
на заданный участок оси времени не зависит от того, сколько заявок пришло
на любой другой, не пересекающийся с ним участок.

---

Поток заявок называется ординарным, если
\[
	P\big(X(s + t) - X(s) > 1\big) = o(t), \quad t \to +0,
\]
т.е.
\[
	\lim_{t \to +0} \frac{P\big(X(s + t) - X(s) > 1\big)}{t} = 0
	\quad \text{при всех } s \ge 0.
\]

Ординарность означает, что заявки поступают по одному,
а не группами по два, три и т.д.

---

Из стационарности следует:


---

\newpage
\section{Семинары}

\subsection{Семинар 1}

\subsection{Семинар 2}

\subsubsection*{Задача с книгами}

\begin{tikzpicture}[
		>=Stealth, thick,
		state/.style = {draw, circle, minimum size=9mm, font=\small},
		e1/.style   = {->, shorten >=2pt, shorten <=2pt},
		e2/.style   = {->, shorten >=2pt, shorten <=2pt},
		e3/.style   = {->, shorten >=2pt, shorten <=2pt},
		lbl/.style  = {font=\scriptsize, fill=white, inner sep=1pt}
	]

	% --- вероятности ---
	\newcommand{\p}[1]{p_{#1}}
	\renewcommand{\p}[1]{\ifcase#1\relax \or \frac{1}{2}\or \frac{1}{3}\or \frac{1}{6}\fi}

	% --- вершины ---
	\node[state] (Q1) at (90:3)   {$123$};
	\node[state] (Q2) at (30:3)   {$132$};
	\node[state] (Q3) at (-30:3)  {$213$};
	\node[state] (Q4) at (-90:3)  {$231$};
	\node[state] (Q5) at (-150:3) {$312$};
	\node[state] (Q6) at (150:3)  {$321$};

	% --- рёбра ---
	% Q1 = 123
	\path[e1, bend left=12] (Q1) edge node[lbl] {$\p{1}$} (Q4);
	\path[e2, bend left=12] (Q1) edge node[lbl] {$\p{2}$} (Q2);
	\path[e3, loop above]   (Q1) edge node[lbl] {$\p{3}$} (Q1);

	% Q2 = 132
	\path[e1, bend left=12] (Q2) edge node[lbl] {$\p{1}$} (Q6);
	\path[e2, loop right]   (Q2) edge node[lbl] {$\p{2}$} (Q2);
	\path[e3, bend left=12] (Q2) edge node[lbl] {$\p{3}$} (Q1);

	% Q3 = 213
	\path[e1, bend left=12] (Q3) edge node[lbl] {$\p{1}$} (Q4);
	\path[e2, bend left=12] (Q3) edge node[lbl] {$\p{2}$} (Q2);
	\path[e3, loop right]   (Q3) edge node[lbl] {$\p{3}$} (Q3);

	% Q4 = 231
	\path[e1, loop below]   (Q4) edge node[lbl] {$\p{1}$} (Q4);
	\path[e2, bend left=12] (Q4) edge node[lbl] {$\p{2}$} (Q5);
	\path[e3, bend left=12] (Q4) edge node[lbl] {$\p{3}$} (Q3);

	% Q5 = 312
	\path[e1, bend left=12] (Q5) edge node[lbl] {$\p{1}$} (Q6);
	\path[e2, loop left]    (Q5) edge node[lbl] {$\p{2}$} (Q5);
	\path[e3, bend left=12] (Q5) edge node[lbl] {$\p{3}$} (Q1);

	% Q6 = 321
	\path[e1, loop left]    (Q6) edge node[lbl] {$\p{1}$} (Q6);
	\path[e2, bend left=12] (Q6) edge node[lbl] {$\p{2}$} (Q5);
	\path[e3, bend left=12] (Q6) edge node[lbl] {$\p{3}$} (Q2);

\end{tikzpicture}

\[
	P =
	\begin{array}{c|cccccc}
		    & Q_1          & Q_2          & Q_3          & Q_4          & Q_5          & Q_6          \\
		\hline
		Q_1 & \tfrac{1}{6} & \tfrac{1}{3} & 0            & \tfrac{1}{2} & 0            & 0            \\
		Q_2 & \tfrac{1}{6} & \tfrac{1}{3} & 0            & 0            & 0            & \tfrac{1}{2} \\
		Q_3 & 0            & \tfrac{1}{3} & \tfrac{1}{6} & \tfrac{1}{2} & 0            & 0            \\
		Q_4 & 0            & 0            & \tfrac{1}{6} & \tfrac{1}{2} & \tfrac{1}{3} & 0            \\
		Q_5 & \tfrac{1}{6} & 0            & 0            & 0            & \tfrac{1}{3} & \tfrac{1}{2} \\
		Q_6 & 0            & \tfrac{1}{6} & 0            & 0            & \tfrac{1}{3} & \tfrac{1}{2}
	\end{array}
\]


Стационарное распределение $\overline r=(r_1,r_2,r_3,r_4,r_5,r_6)$
задаётся условиями
\[
	\overline r P = \overline r,
	\qquad \sum_{i=1}^6 r_i = 1,
	\qquad r_i \ge 0.
\]

Из матрицы $P$ получаем систему:
\[
	\begin{array}{rcl}
		\begin{aligned}
			r_1 & = \tfrac{1}{6}r_1 + \tfrac{1}{6}r_2 + \tfrac{1}{6}r_5,                   \\
			r_2 & = \tfrac{1}{3}r_1 + \tfrac{1}{3}r_2 + \tfrac{1}{3}r_3 + \tfrac{1}{6}r_6, \\
			r_3 & = \tfrac{1}{6}r_3 + \tfrac{1}{6}r_4,                                     \\
			r_4 & = \tfrac{1}{2}r_1 + \tfrac{1}{2}r_3 + \tfrac{1}{2}r_4,                   \\
			r_5 & = \tfrac{1}{3}r_4 + \tfrac{1}{3}r_5 + \tfrac{1}{3}r_6,                   \\
			r_6 & = \tfrac{1}{2}r_2 + \tfrac{1}{2}r_5 + \tfrac{1}{2}r_6,
		\end{aligned}
		 & \qquad \Rightarrow \qquad &
		\begin{aligned}
			6r_1 & = r_1 + r_2 + r_5,          \\
			6r_2 & = 2r_1 + 2r_2 + 2r_3 + r_6, \\
			6r_3 & = r_3 + r_4,                \\
			2r_4 & = r_1 + r_3 + r_4,          \\
			3r_5 & = r_4 + r_5 + r_6,          \\
			2r_6 & = r_2 + r_5 + r_6.
		\end{aligned}
	\end{array}
\]

\[
	\begin{array}{rcl}
		\begin{aligned}
			5r_1 - r_2 - r_5          & = 0, \\
			-2r_1 + 4r_2 - 2r_3 - r_6 & = 0, \\
			5r_3 - r_4                & = 0, \\
			-\,r_1 - r_3 + r_4        & = 0, \\
			-\,r_4 + 2r_5 - r_6       & = 0, \\
			r_1+r_2+r_3+r_4+r_5+r_6   & =1,
		\end{aligned}
		 & \qquad \Rightarrow \qquad &
		\underbrace{\!
			\begin{pmatrix}
				5  & -1 & 0  & 0  & -1 & 0  \\
				-2 & 4  & -2 & 0  & 0  & -1 \\
				0  & 0  & 5  & -1 & 0  & 0  \\
				-1 & 0  & -1 & 1  & 0  & 0  \\
				0  & 0  & 0  & -1 & 2  & -1 \\
				1  & 1  & 1  & 1  & 1  & 1
			\end{pmatrix}}_{\displaystyle A}
		\underbrace{\!
			\begin{pmatrix}r_1\\r_2\\r_3\\r_4\\r_5\\r_6\end{pmatrix}}_{\displaystyle x}
		=
		\underbrace{\!
			\begin{pmatrix}0\\0\\0\\0\\0\\1\end{pmatrix}}_{\displaystyle b}.
	\end{array}
\]

\[
	\left[
		\begin{array}{rrrrrr|r}
			5  & -1 & 0  & 0  & -1 & 0  & 0 \\
			-2 & 4  & -2 & 0  & 0  & -1 & 0 \\
			0  & 0  & 5  & -1 & 0  & 0  & 0 \\
			-1 & 0  & -1 & 1  & 0  & 0  & 0 \\
			0  & 0  & 0  & -1 & 2  & -1 & 0 \\
			1  & 1  & 1  & 1  & 1  & 1  & 1
		\end{array}
		\right]
	\overset{}{\longrightarrow}
	\left[
		\begin{array}{rrrrrr|r}
			1 & 0 & 0 & 0 & 0 & 0 & \tfrac{2}{25} \\
			0 & 1 & 0 & 0 & 0 & 0 & \tfrac{3}{20} \\
			0 & 0 & 1 & 0 & 0 & 0 & \tfrac{1}{50} \\
			0 & 0 & 0 & 1 & 0 & 0 & \tfrac{1}{10} \\
			0 & 0 & 0 & 0 & 1 & 0 & \tfrac{1}{4}  \\
			0 & 0 & 0 & 0 & 0 & 1 & \tfrac{2}{5}
		\end{array}
		\right]
\]

\[
	\overline r =
	\left(\tfrac{2}{25},\;\tfrac{3}{20},\;\tfrac{1}{50},\;\tfrac{1}{10},\;\tfrac{1}{4},\;\tfrac{2}{5}\right).
\]


\subsection{Семинар 3}

\subsubsection*{Задача 3.1}

\subsubsection*{Задача 3.2}

\subsubsection*{Задача 3.3.1}

Условие: $p+q=1$, $p,q>0$.

\begin{tikzpicture}[
		>=Stealth, thick,
		state/.style = {draw, circle, minimum size=9mm, font=\small},
		e/.style   = {->, shorten >=2pt, shorten <=2pt},
		lbl/.style  = {font=\scriptsize, fill=white, inner sep=1pt}
	]

	% --- вершины ---
	\node[state] (Q1) at (0,0) {$1$};
	\node[state] (Q2) at (2.5,0) {$2$};
	\node[state] (Q3) at (5,0) {$3$};

	% --- рёбра ---
	% Q1
	\path[e, loop above]   (Q1) edge node[lbl] {$1$} (Q1);

	% Q2
	\path[e, bend left=12] (Q2) edge node[lbl] {$q$} (Q1);
	\path[e, bend left=12] (Q2) edge node[lbl] {$p$} (Q3);

	% Q3
	\path[e, loop above]   (Q3) edge node[lbl] {$1$} (Q3);

\end{tikzpicture}

Найти: $P$, $P^n$, $\overline{r}$ (стационарное распределение), $\lambda_1=1$, $\lambda_2$, $\lambda_3$ (собственные числа).

\paragraph{Решение.}

\[
	P =
	\begin{array}{c|ccc}
		    & Q_1 & Q_2 & Q_3 \\ \hline
		Q_1 & 1   & 0   & 0   \\
		Q_2 & q   & 0   & p   \\
		Q_3 & 0   & 0   & 1
	\end{array}
\]

\[
	P^0=I,\qquad P^2=P \;\Rightarrow\; P^n=P\ (n\ge1).
\]

Стационарное распределение $\overline r=(r_1,r_2,r_3)$ задаётся условиями
\[
	\overline r P = \overline r,\quad r_1+r_2+r_3=1,\quad r_i\ge0.
\]
Из $rP=\bigl(\,r\cdot\text{col}_1,\ r\cdot\text{col}_2,\ r\cdot\text{col}_3\,\bigr)$ и столбцов $P$ получаем систему
\[
	\begin{aligned}
		r_1 & = r_1 + q r_2, \\
		r_2 & = 0,           \\
		r_3 & = p r_2 + r_3,
	\end{aligned}
	\qquad\Rightarrow\qquad
	r_2=0,\ \ r_1+r_3=1.
\]
Отсюда
\[
	\overline r=(\alpha,\,0,\,1-\alpha),\quad \alpha\in[0,1].
\]

Собственные числа: вычислим характеристический многочлен
\[
	\det(P-\lambda I)=\det
	\begin{pmatrix}
		1-\lambda & 0        & 0         \\
		q         & -\lambda & p         \\
		0         & 0        & 1-\lambda
	\end{pmatrix}
	=(1-\lambda)\cdot(-\lambda)\cdot(1-\lambda)
	=-\lambda(1-\lambda)^2.
\]
Следовательно,
\[
	\boxed{\ \lambda_1=1,\ \lambda_2=1,\ \lambda_3=0\ }.
\]

% ─────────────────────────────────────────────────────────────────────────────

\subsubsection*{Задача 3.3.2}

Условие:

\begin{tikzpicture}[
		>=Stealth, thick,
		state/.style = {draw, circle, minimum size=9mm, font=\small},
		e/.style   = {->, shorten >=2pt, shorten <=2pt},
		lbl/.style  = {font=\scriptsize, fill=white, inner sep=1pt}
	]

	% --- вершины ---
	\node[state] (Q1) at (0,0) {$1$};
	\node[state] (Q2) at (2.5,0) {$2$};
	\node[state] (Q3) at (5,0) {$3$};

	% --- рёбра ---
	\path[e, bend left=12] (Q1) edge node[lbl] {$1$} (Q2);

	\path[e, bend left=12] (Q2) edge node[lbl] {$q$} (Q1);
	\path[e, bend left=12] (Q2) edge node[lbl] {$p$} (Q3);

	\path[e, bend left=12] (Q3) edge node[lbl] {$1$} (Q2);

\end{tikzpicture}

Найти: $P$, $P^n$, $\overline{r}$ (стационарное распределение), $\lambda_1=1$, $\lambda_2$, $\lambda_3$ (собственные числа).

\paragraph{Решение.}

\[
	P =
	\begin{array}{c|ccc}
		    & Q_1 & Q_2 & Q_3 \\ \hline
		Q_1 & 0   & 1   & 0   \\
		Q_2 & q   & 0   & p   \\
		Q_3 & 0   & 1   & 0
	\end{array}
\]

\emph{Степени.} Прямым перемножением:
\[
	P^2=
	\begin{pmatrix}
		q & 0 & p \\
		0 & 1 & 0 \\
		q & 0 & p
	\end{pmatrix},
	\qquad
	P^{2k}=P^2,\quad P^{2k+1}=P\ \ (k\ge0),\quad P^0=I.
\]

\emph{Стационарное распределение.} Пусть $\overline r=(r_1,r_2,r_3)$.
По столбцам $P$ имеем
\[
	\begin{aligned}
		r_1 & = q r_2,     \\
		r_2 & = r_1 + r_3, \\
		r_3 & = p r_2,
	\end{aligned}
	\qquad r_1+r_2+r_3=1.
\]
Подставляя $r_1=q r_2$, $r_3=p r_2$ во второе уравнение:
\[
	r_2=(q r_2)+(p r_2)=(p+q)r_2=r_2 \quad(\text{тождество}).
\]
Из нормировки:
\[
	r_1+r_2+r_3=(q r_2)+r_2+(p r_2)=(1+p+q)r_2=2r_2=1 \Rightarrow r_2=\tfrac12,
\]
следовательно
\[
	\overline r=\Bigl(\tfrac{q}{2},\ \tfrac12,\ \tfrac{p}{2}\Bigr).
\]

\emph{Собственные числа.} Находим
\[
	\det(P-\lambda I)=
	\det\begin{pmatrix}
		-\lambda & 1        & 0        \\
		q        & -\lambda & p        \\
		0        & 1        & -\lambda
	\end{pmatrix}
	=(-\lambda)\!\det\!\begin{pmatrix}-\lambda&p\\[2pt]1&-\lambda\end{pmatrix}
	-1\cdot\det\!\begin{pmatrix}q&p\\[2pt]0&-\lambda\end{pmatrix}.
\]
Отсюда
\[
	\det(P-\lambda I)=(-\lambda)(\lambda^2-p) - ( -q\lambda )
	=-\lambda^3 + (p+q)\lambda
	=-\lambda(\lambda^2-1).
\]
Так как $p+q=1$, получаем
\[
	\boxed{\ \lambda_1=1,\ \lambda_2=0,\ \lambda_3=-1\ }.
\]

% ─────────────────────────────────────────────────────────────────────────────

\subsubsection*{Задача 3.3.3}

Условие:

\begin{tikzpicture}[
		>=Stealth, thick,
		state/.style = {draw, circle, minimum size=9mm, font=\small},
		e/.style   = {->, shorten >=2pt, shorten <=2pt},
		lbl/.style  = {font=\scriptsize, fill=white, inner sep=1pt}
	]

	% --- вершины ---
	\node[state] (Q1) at (0,0) {$1$};
	\node[state] (Q2) at (2.5,0) {$2$};
	\node[state] (Q3) at (5,0) {$3$};

	% --- рёбра ---
	\path[e, loop above]   (Q1) edge node[lbl] {$q$} (Q1);
	\path[e, bend left=12] (Q1) edge node[lbl] {$p$} (Q2);

	\path[e, bend left=12] (Q2) edge node[lbl] {$q$} (Q1);
	\path[e, bend left=12] (Q2) edge node[lbl] {$p$} (Q3);

	\path[e, loop above]   (Q3) edge node[lbl] {$p$} (Q3);
	\path[e, bend left=12] (Q3) edge node[lbl] {$q$} (Q2);

\end{tikzpicture}

Найти: $P$, $P^n$, $\overline{r}$ (стационарное распределение), $\lambda_1=1$, $\lambda_2$, $\lambda_3$ (собственные числа).

\paragraph{Решение.}

\[
	P =
	\begin{array}{c|ccc}
		    & Q_1 & Q_2 & Q_3 \\ \hline
		Q_1 & q   & p   & 0   \\
		Q_2 & q   & 0   & p   \\
		Q_3 & 0   & q   & p
	\end{array}
\]

\emph{Стационарное распределение.} Пусть $\overline r=(r_1,r_2,r_3)$.
Из $rP=\overline r$ по столбцам получаем
\[
	\begin{aligned}
		r_1 & = q(r_1+r_2),    \\
		r_2 & = p r_1 + q r_3, \\
		r_3 & = p(r_2+r_3),
	\end{aligned}
	\qquad r_1+r_2+r_3=1.
\]
Из первого и третьего:
\[
	(1-q)r_1=q r_2 \ \Rightarrow\ p r_1=q r_2 \ \Rightarrow\ r_2=\frac{p}{q}r_1,
	\qquad
	(1-p)r_3=p r_2 \ \Rightarrow\ q r_3=p r_2 \ \Rightarrow\ r_3=\frac{p}{q}r_2.
\]
Следовательно
\[
	r_1:r_2:r_3 = 1 : \frac{p}{q} : \frac{p^2}{q^2} = q^2 : pq : p^2.
\]
Нормируя на единицу, получаем
\[
	\overline r=\pi=
	\frac{1}{\,q^2+pq+p^2\,}\,(q^2,\ pq,\ p^2).
\]

\emph{Собственные числа.} Находим характеристический многочлен:
\[
	\det(P-\lambda I)=
	\det\begin{pmatrix}
		q-\lambda & p        & 0         \\
		q         & -\lambda & p         \\
		0         & q        & p-\lambda
	\end{pmatrix}
	=(q-\lambda)\bigl(\lambda^2-\lambda p-pq\bigr)-pq(p-\lambda).
\]
Раскрывая и используя $p+q=1$:
\[
	\det(P-\lambda I)
	= -\lambda^3+\lambda^2+pq\,\lambda-pq
	= -(\lambda-1)(\lambda^2-pq).
\]
Итак,
\[
	\boxed{\ \lambda_1=1,\quad \lambda_2=\sqrt{pq},\quad \lambda_3=-\sqrt{pq}\ }.
\]

\[
	P^2 =
	\begin{pmatrix}
		q   & pq  & p^2 \\
		q^2 & 2pq & p^2 \\
		q^2 & pq  & p
	\end{pmatrix},
	\qquad
	P^3 =
	\begin{pmatrix}
		q^2(1+p) & pq(1+p) & p^2      \\
		q^2(1+p) & pq      & p^2(1+q) \\
		q^2      & pq(1+q) & p^2(1+q)
	\end{pmatrix}.
\]

\[
	P^4 =
	\begin{pmatrix}
		q^2(1+p)  & pq(1+pq)  & p^2(1+pq) \\
		q^2(1+pq) & pq(1+2pq) & p^2(1+pq) \\
		q^2(1+pq) & pq(1+pq)  & p^2(1+q)
	\end{pmatrix}.
\]
% ─────────────────────────────────────────────────────────────────────────────
\subsubsection*{Задача 4}

\begin{tikzpicture}[
		>=Stealth, thick,
		state/.style = {draw, circle, minimum size=9mm, font=\small},
		e/.style   = {->, shorten >=2pt, shorten <=2pt},
		lbl/.style  = {font=\scriptsize, fill=white, inner sep=1pt}
	]
	% --- вершины ---
	\node[state] (Q0) at (0,0)   {$0$};
	\node[state] (Q1) at (2.5,0) {$1$};
	\node[state] (Q2) at (5,0)   {$2$};
	\node[state] (Q3) at (7.5,0) {$3$};
	\node[state] (Q4) at (10,0)  {$4$};

	% --- рёбра ---
	\path[e, loop above]   (Q0) edge node[lbl] {$p_0$} (Q0);
	\path[e, bend left=12] (Q0) edge node[lbl] {$p_1$} (Q1);
	\path[e, bend left=56] (Q0) edge node[lbl] {$p_2$} (Q2);

	\path[e, bend left=12] (Q1) edge node[lbl] {$p_0$} (Q0);
	\path[e, loop above]   (Q1) edge node[lbl] {$p_1$} (Q1);
	\path[e, bend left=12] (Q1) edge node[lbl] {$p_2$} (Q2);

	\path[e, bend left=12] (Q2) edge node[lbl] {$p_0$} (Q1);
	\path[e, loop above]   (Q2) edge node[lbl] {$p_1$} (Q2);
	\path[e, bend left=12] (Q2) edge node[lbl] {$p_2$} (Q3);

	\path[e, bend left=12] (Q3) edge node[lbl] {$p_0$} (Q2);
	\path[e, loop above]   (Q3) edge node[lbl] {$p_1$} (Q3);
	\path[e, bend left=12] (Q3) edge node[lbl] {$p_2$} (Q4);

	\path[e, bend left=12] (Q4) edge node[lbl] {$p_0$} (Q3);
	\path[e, loop above]   (Q4) edge node[lbl] {$1-p_0$} (Q4);
\end{tikzpicture}

Найти: $P$, $\overline{r}$ (стационарное распределение), $que$.

\paragraph{Решение.}

Матрица переходных вероятностей:
\[
	P =
	\begin{array}{c|ccccc}
		    & Q_0 & Q_1 & Q_2 & Q_3 & Q_4   \\
		\hline
		Q_0 & p_0 & p_1 & p_2 & 0   & 0     \\
		Q_1 & p_0 & p_1 & p_2 & 0   & 0     \\
		Q_2 & 0   & p_0 & p_1 & p_2 & 0     \\
		Q_3 & 0   & 0   & p_0 & p_1 & p_2   \\
		Q_4 & 0   & 0   & 0   & p_0 & 1-p_0
	\end{array}
\]

Стационарное распределение $\overline r=(r_0,r_1,r_2,r_3,r_4)$:
\[
	\overline r P = \overline r,\qquad \sum_{i=0}^4 r_i = 1,\qquad r_i\ge 0.
\]

Линейная система из $\overline r P=\overline r$:
\[
	\begin{aligned}
		r_0 & = p_0 r_0 + p_0 r_1,                     \\
		r_1 & = p_1 r_0 + p_1 r_1 + p_0 r_2,           \\
		r_2 & = p_2 r_0 + p_2 r_1 + p_1 r_2 + p_0 r_3, \\
		r_3 & = p_2 r_2 + p_1 r_3 + p_0 r_4,           \\
		r_4 & = p_2 r_3 + (1-p_0) r_4.
	\end{aligned}
\]

Среднее число в системе:
$g(Q_i)=v_i$ — физическое значение в состоянии $Q_i$. Тогда
$$que=\sum_{i=0}^{4} v_i r_i$$

\subsubsection*{Задача 4.1}

При $p_0=\tfrac{1}{2},\; p_1=\tfrac{1}{4},\; p_2=\tfrac{1}{4}$:
\[
	\begin{aligned}
		r_0 & = \tfrac{1}{2}r_0 + \tfrac{1}{2}r_1,                                     \\
		r_1 & = \tfrac{1}{4}r_0 + \tfrac{1}{4}r_1 + \tfrac{1}{2}r_2,                   \\
		r_2 & = \tfrac{1}{4}r_0 + \tfrac{1}{4}r_1 + \tfrac{1}{4}r_2 + \tfrac{1}{2}r_3, \\
		r_3 & = \tfrac{1}{4}r_2 + \tfrac{1}{4}r_3 + \tfrac{1}{2}r_4,                   \\
		r_4 & = \tfrac{1}{4}r_3 + \tfrac{1}{2}r_4,\qquad
		r_0+r_1+r_2+r_3+r_4=1.
	\end{aligned}
\]

\[
	\begin{array}{rcl}
		\begin{aligned}
			2r_0 & = r_0 + r_1,              \\
			4r_1 & = r_0 + r_1 + 2r_2,       \\
			4r_2 & = r_0 + r_1 + r_2 + 2r_3, \\
			4r_3 & = r_2 + r_3 + 2r_4,       \\
			2r_4 & = r_3 + 2r_4,
		\end{aligned}
		 & \qquad \Rightarrow \qquad &
		\begin{aligned}
			r_0 - r_1                  & = 0, \\
			-\,r_0 + 3r_1 - 2r_2       & = 0, \\
			-\,r_0 - r_1 + 3r_2 - 2r_3 & = 0, \\
			-\,r_2 + 3r_3 - 2r_4       & = 0, \\
			r_0+r_1+r_2+r_3+r_4        & =1.
		\end{aligned}
	\end{array}
\]

\[
	\left[
		\begin{array}{rrrrr|r}
			1  & -1 & 0  & 0  & 0  & 0 \\
			-1 & 3  & -2 & 0  & 0  & 0 \\
			-1 & -1 & 3  & -2 & 0  & 0 \\
			0  & 0  & -1 & 3  & -2 & 0 \\
			1  & 1  & 1  & 1  & 1  & 1
		\end{array}
		\right]
	\overset{}{\longrightarrow}
	\left[
		\begin{array}{rrrrr|r}
			1 & 0 & 0 & 0 & 0 & \tfrac{4}{15} \\
			0 & 1 & 0 & 0 & 0 & \tfrac{4}{15} \\
			0 & 0 & 1 & 0 & 0 & \tfrac{4}{15} \\
			0 & 0 & 0 & 1 & 0 & \tfrac{2}{15} \\
			0 & 0 & 0 & 0 & 1 & \tfrac{1}{15}
		\end{array}
		\right]
\]

\[
	\overline r =
	\left(\tfrac{4}{15},\;\tfrac{4}{15},\;\tfrac{4}{15},\;\tfrac{2}{15},\;\tfrac{1}{15}\right).
\]

\[
	que = \sum_{i=0}^{4} v_i r_i= 0\cdot\tfrac{4}{15} + 1\cdot\tfrac{4}{15} + 2\cdot\tfrac{4}{15} + 3\cdot\tfrac{2}{15} + 4\cdot\tfrac{1}{15} = \frac{22}{15} \approx 1.4667.
\]

\[
	\boxed{\ que=\frac{22}{15}\approx1.4667\ }.
\]

\subsubsection*{Задача 4.2}

При вероятностях $p_0=\tfrac{1}{4},\; p_1=\tfrac{1}{2},\; p_2=\tfrac{1}{4}$

\[
	\begin{aligned}
		r_0 & = \tfrac{1}{4}r_0 + \tfrac{1}{4}r_1,                                     \\
		r_1 & = \tfrac{1}{2}r_0 + \tfrac{1}{2}r_1 + \tfrac{1}{4}r_2,                   \\
		r_2 & = \tfrac{1}{4}r_0 + \tfrac{1}{2}r_1 + \tfrac{1}{4}r_2 + \tfrac{1}{4}r_3, \\
		r_3 & = \tfrac{1}{4}r_2 + \tfrac{1}{2}r_3 + \tfrac{1}{4}r_4,                   \\
		r_4 & = \tfrac{1}{4}r_3 + \tfrac{3}{4}r_4,\qquad
		r_0+r_1+r_2+r_3+r_4=1.
	\end{aligned}
\]

\[
	\begin{array}{rcl}
		\begin{aligned}
			4r_0 & = r_0 + r_1,              \\
			4r_1 & = 2r_0 + 2r_1 + r_2,      \\
			4r_2 & = r_0 + 2r_1 + r_2 + r_3, \\
			4r_3 & = r_2 + 2r_3 + r_4,       \\
			4r_4 & = r_3 + 3r_4,
		\end{aligned}
		 & \qquad \Rightarrow \qquad &
		\begin{aligned}
			3r_0 - r_1                 & = 0, \\
			-\,2r_0 + 2r_1 - r_2       & = 0, \\
			-\,r_0 - 2r_1 + 3r_2 - r_3 & = 0, \\
			-\,r_2 + 2r_3 - r_4        & = 0, \\
			r_0+r_1+r_2+r_3+r_4        & = 1.
		\end{aligned}
	\end{array}
\]

\[
	\left[
		\begin{array}{rrrrr|r}
			3  & -1 & 0  & 0  & 0  & 0 \\
			-2 & 2  & -1 & 0  & 0  & 0 \\
			-1 & -2 & 3  & -1 & 0  & 0 \\
			0  & 0  & -1 & 2  & -1 & 0 \\
			1  & 1  & 1  & 1  & 1  & 1
		\end{array}
		\right]
	\longrightarrow
	\left[
		\begin{array}{rrrrr|r}
			1 & 0 & 0 & 0 & 0 & \tfrac{1}{16} \\
			0 & 1 & 0 & 0 & 0 & \tfrac{3}{16} \\
			0 & 0 & 1 & 0 & 0 & \tfrac{1}{4}  \\
			0 & 0 & 0 & 1 & 0 & \tfrac{1}{4}  \\
			0 & 0 & 0 & 0 & 1 & \tfrac{1}{4}
		\end{array}
		\right]
\]

\[
	\overline r=\left(\tfrac{1}{16},\,\tfrac{3}{16},\,\tfrac{1}{4},\,\tfrac{1}{4},\,\tfrac{1}{4}\right).
\]

\[
	que=\sum_{i=0}^{4} v_i r_i
	= 0\cdot\tfrac{1}{16}+1\cdot\tfrac{3}{16}+2\cdot\tfrac{1}{4}+3\cdot\tfrac{1}{4}+4\cdot\tfrac{1}{4}
	= \frac{39}{16}=2.4375.
\]

\[
	\boxed{\,que=\frac{39}{16}\approx 2.4375\, }.
\]

\subsubsection*{Задача 4.3}

При вероятностях $p_0=\tfrac{1}{4},\; p_1=\tfrac{1}{4},\; p_2=\tfrac{1}{2}$

\[
	\begin{aligned}
		r_0 & = \tfrac{1}{4}r_0 + \tfrac{1}{4}r_1,                                     \\
		r_1 & = \tfrac{1}{4}r_0 + \tfrac{1}{4}r_1 + \tfrac{1}{4}r_2,                   \\
		r_2 & = \tfrac{1}{2}r_0 + \tfrac{1}{2}r_1 + \tfrac{1}{4}r_2 + \tfrac{1}{4}r_3, \\
		r_3 & = \tfrac{1}{2}r_2 + \tfrac{1}{4}r_3 + \tfrac{1}{4}r_4,                   \\
		r_4 & = \tfrac{1}{2}r_3 + \tfrac{3}{4}r_4,\qquad
		r_0+r_1+r_2+r_3+r_4=1.
	\end{aligned}
\]

\[
	\begin{array}{rcl}
		\begin{aligned}
			4r_0 & = r_0 + r_1,               \\
			4r_1 & = r_0 + r_1 + r_2,         \\
			4r_2 & = 2r_0 + 2r_1 + r_2 + r_3, \\
			4r_3 & = 2r_2 + r_3 + r_4,        \\
			4r_4 & = 2r_3 + 3r_4,
		\end{aligned}
		 & \Rightarrow &
		\begin{aligned}
			3r_0 - r_1                  & = 0, \\
			-\,r_0 + 3r_1 - r_2         & = 0, \\
			-\,2r_0 - 2r_1 + 3r_2 - r_3 & = 0, \\
			-\,2r_2 + 3r_3 - r_4        & = 0, \\
			-\,2r_3 + r_4               & = 0, \\
			r_0+r_1+r_2+r_3+r_4         & = 1.
		\end{aligned}
	\end{array}
\]

\[
	\left[
		\begin{array}{rrrrr|r}
			3  & -1 & 0  & 0  & 0  & 0 \\
			-1 & 3  & -1 & 0  & 0  & 0 \\
			-2 & -2 & 3  & -1 & 0  & 0 \\
			0  & 0  & -2 & 3  & -1 & 0 \\
			0  & 0  & 0  & -2 & 1  & 0 \\
			1  & 1  & 1  & 1  & 1  & 1
		\end{array}
		\right]
	\longrightarrow
	\left[
		\begin{array}{rrrrr|r}
			1 & 0 & 0 & 0 & 0 & \tfrac{1}{60} \\
			0 & 1 & 0 & 0 & 0 & \tfrac{1}{20} \\
			0 & 0 & 1 & 0 & 0 & \tfrac{2}{15} \\
			0 & 0 & 0 & 1 & 0 & \tfrac{4}{15} \\
			0 & 0 & 0 & 0 & 1 & \tfrac{8}{15}
		\end{array}
		\right]
\]

\[
	\overline r=\left(\tfrac{1}{60},\,\tfrac{1}{20},\,\tfrac{2}{15},\,\tfrac{4}{15},\,\tfrac{8}{15}\right).
\]

\[
	que=\sum_{i=0}^{4} v_i r_i
	= 0\cdot\tfrac{1}{60}+1\cdot\tfrac{1}{20}+2\cdot\tfrac{2}{15}
	+3\cdot\tfrac{4}{15}+4\cdot\tfrac{8}{15}
	= \frac{13}{4}=3.25.
\]
\[
	\boxed{\,que=\frac{13}{4}=3.25\, }.
\]


\subsection{Семинар 5}

\subsubsection*{Задача 5}

\begin{tikzpicture}[
		>=Stealth, thick,
		state/.style = {draw, circle, minimum size=9mm, font=\small},
		e/.style   = {->, shorten >=2pt, shorten <=2pt},
		lbl/.style  = {font=\scriptsize, fill=white, inner sep=1pt}
	]

	% --- вершины ---
	\node[state] (Q0) at (0,0) {$0$};
	\node[state] (Q1) at (2.5,0) {$1$};
	\node[state] (Q2) at (5,0) {$2$};
	\node[state] (Q3) at (7.5,0) {$3$};
	\node[state] (Q4) at (10,0) {$4$};
	\node[state] (Q5) at (12.5,0) {$\ldots$};

	% --- рёбра ---
	\path[e, loop above]   (Q0) edge node[lbl] {$p_0$} (Q0);
	\path[e, bend left=12] (Q0) edge node[lbl] {$p_1$} (Q1);
	\path[e, bend left=56] (Q0) edge node[lbl] {$p_2$} (Q2);

	\path[e, bend left=12] (Q1) edge node[lbl] {$p_0$} (Q0);
	\path[e, loop above]   (Q1) edge node[lbl] {$p_1$} (Q1);
	\path[e, bend left=12] (Q1) edge node[lbl] {$p_2$} (Q2);

	\path[e, bend left=12] (Q2) edge node[lbl] {$p_0$} (Q1);
	\path[e, loop above]   (Q2) edge node[lbl] {$p_1$} (Q2);
	\path[e, bend left=12] (Q2) edge node[lbl] {$p_2$} (Q3);

	\path[e, bend left=12] (Q3) edge node[lbl] {$p_0$} (Q2);
	\path[e, loop above]   (Q3) edge node[lbl] {$p_1$} (Q3);
	\path[e, bend left=12] (Q3) edge node[lbl] {$p_2$} (Q4);

	\path[e, bend left=12] (Q4) edge node[lbl] {$p_0$} (Q3);
	\path[e, loop above]   (Q4) edge node[lbl] {$p_1$} (Q4);
	\path[e, bend left=12] (Q4) edge node[lbl] {$p_2$} (Q5);

\end{tikzpicture}

Найти: $P$, $\overline{r}$ (стационарное распределение), $\mathrm{que} := \mathbb E[K]$.

\paragraph{Решение.}

\[
	P =
	\begin{array}{c|cccccc}
		       & Q_0    & Q_1    & Q_2    & Q_3    & Q_4    & \ldots \\ \hline
		Q_0    & p_0    & p_1    & p_2    & 0      & 0      & \ldots \\
		Q_1    & p_0    & p_1    & p_2    & 0      & 0      & \ldots \\
		Q_2    & 0      & p_0    & p_1    & p_2    & 0      & \ldots \\
		Q_3    & 0      & 0      & p_0    & p_1    & p_2    & \ldots \\
		Q_4    & 0      & 0      & 0      & p_0    & p_1    & \ldots \\
		\vdots & \vdots & \vdots & \vdots & \vdots & \vdots & \ddots
	\end{array}
\]

Стационарное распределение $\overline r=(r_0,r_1,r_2,r_3,\ldots)$ задаётся условиями
\[
	\overline r P = \overline r,\quad \sum_{i=0}^\infty r_i = 1,\quad r_i\ge0.
\]
Из матрицы $P$ получаем систему:
\[
	\begin{array}{rcl}
		\begin{aligned}
			r_0 & = p_0 r_0 + p_0 r_1,                                 \\
			r_1 & = p_1 r_0 + p_1 r_1 + p_0 r_2,                       \\
			r_2 & = p_2 r_0 + p_2 r_1 + p_1 r_2 + p_0 r_3,             \\
			r_k & = p_2 r_{k-1} + p_1 r_k + p_0 r_{k+1}, \quad k\ge 2.
		\end{aligned}
	\end{array}
\]

\subsubsection*{Задача 5.1}

Пусть $p_0=\tfrac{1}{2},\ p_1=\tfrac{1}{4},\ p_2=\tfrac{1}{4}$. Тогда
\[
	\begin{array}{rcl}
		\begin{aligned}
			r_0 & = \tfrac{1}{2} r_0 + \tfrac{1}{2} r_1,                                           \\
			r_1 & = \tfrac{1}{4} r_0 + \tfrac{1}{4} r_1 + \tfrac{1}{2} r_2,                        \\
			r_2 & = \tfrac{1}{4} r_0 + \tfrac{1}{4} r_1 + \tfrac{1}{4} r_2 + \tfrac{1}{2} r_3,     \\
			r_k & = \tfrac{1}{4} r_{k-1} + \tfrac{1}{4} r_k + \tfrac{1}{2} r_{k+1}, \quad k \ge 2.
		\end{aligned}
		 & \qquad \Rightarrow \qquad &
		\begin{aligned}
			r_1 & = r_0,                         \\
			r_2 & = r_0,                         \\
			r_k & = r_0\,2^{\,2-k},\quad k\ge 2.
		\end{aligned}
	\end{array}
\]

\[
	\sum_{i=0}^\infty r_i = r_0+r_1+r_2+\sum_{k=3}^\infty r_0\,2^{\,2-k}
	=3r_0+r_0\!\sum_{m=1}^\infty 2^{-m}=4r_0=1
	\ \Rightarrow\ r_0=\tfrac14.
\]


\[
	\overline r = \left(\tfrac{1}{4},\ \tfrac{1}{4},\ \tfrac{1}{4},\ \tfrac{1}{8},\ \tfrac{1}{16},\ \tfrac{1}{32},\ \ldots\right).
\]

\[
	\mathrm{que}=\mathbb E[K]=\sum_{k=0}^\infty k\,r_k
	=\tfrac14+\tfrac12+\sum_{k=3}^\infty k\,2^{-k}
	=\tfrac34+\Bigl(\sum_{k=1}^\infty k\,2^{-k}-\tfrac12-\tfrac12\Bigr)
	=\tfrac34+1=\tfrac{7}{4}.
\]

\subsubsection*{Задача 5.2}

При вероятностях $p_0=\tfrac{1}{4},\quad p_1=\tfrac{1}{2},\quad p_2=\tfrac{1}{4}$
\[
	\begin{array}{rcl}
		\begin{aligned}
			r_0 & = \tfrac{1}{4} r_0 + \tfrac{1}{4} r_1,                                           \\
			r_1 & = \tfrac{1}{2} r_0 + \tfrac{1}{2} r_1 + \tfrac{1}{4} r_2,                        \\
			r_2 & = \tfrac{1}{4} r_0 + \tfrac{1}{2} r_1 + \tfrac{1}{4} r_2 + \tfrac{1}{4} r_3,     \\
			r_k & = \tfrac{1}{4} r_{k-1} + \tfrac{1}{2} r_k + \tfrac{1}{4} r_{k+1}, \quad k \ge 2,
		\end{aligned}
		 & \qquad \Rightarrow \qquad &
		\begin{aligned}
			r_1 & = 3r_0,                    \\
			r_2 & = 4r_0,                    \\
			r_k & = r_0\,(k+2),\quad k\ge 0.
		\end{aligned}
	\end{array}
\]
Серия \(\sum_{k\ge0} r_k\) расходится, нормированного решения нет; стационарное распределение не существует.


\subsection{Семинар 6}

\subsubsection*{Задача 6.1}

\begin{tikzpicture}[
		>=Stealth, thick,
		state/.style = {draw, circle, minimum size=9mm, font=\small},
		e/.style   = {->, shorten >=2pt, shorten <=2pt},
		lbl/.style  = {font=\scriptsize, fill=white, inner sep=1pt}
	]

	% --- вершины ---
	\node[state] (Q0) at (0,0) {$1$};
	\node[state] (Q1) at (2.5,0) {$2$};
	\node[state] (Q2) at (5,0) {$3$};

	% --- рёбра ---
	\path[e, bend left=12] (Q1) edge node[lbl] {$1$} (Q0);
	\path[e, bend left=12]   (Q1) edge node[lbl] {$1$} (Q2);

\end{tikzpicture}

\[
	\Lambda =
	\begin{pmatrix}
		0 & 0  & 0 \\
		1 & -2 & 1 \\
		0 & 0  & 0
	\end{pmatrix}
\]

\[
	\vec{p}(0) = (0,1,0).
\]

\[
	\vec{p}'(t) = \vec{p}(t)\Lambda,\quad \vec{p}(t)=(p_1(t),p_2(t),p_3(t)).
\]

\[
	\begin{aligned}
		p_1'(t) & = p_1 (t)\cdot 0 + p_2(t)\cdot 1 + p_3(t)\cdot 0 = p_2(t),     \\
		p_2'(t) & = p_1 (t)\cdot 0 + p_2(t)\cdot(-2) + p_3(t)\cdot 0 = -2p_2(t), \\
		p_3'(t) & = p_1 (t)\cdot 0 + p_2(t)\cdot 1 + p_3(t)\cdot 0 = p_2(t).
	\end{aligned}
\]

\[
	\begin{aligned}
		p_2(t) & = p_2(0)e^{-2t} = e^{-2t},                                       \\
		p_1(t) & = \int p_2(t)\,dt = \int e^{-2t}\,dt = -\tfrac{1}{2}e^{-2t}+C_1, \\
		p_3(t) & = \int p_2(t)\,dt = \int e^{-2t}\,dt = -\tfrac{1}{2}e^{-2t}+C_2.
	\end{aligned}
\]

\[
	\begin{aligned}
		p_1(0) & = -\tfrac{1}{2}+C_1=0 \ \Rightarrow\ C_1=\tfrac{1}{2}, \\
		p_3(0) & = -\tfrac{1}{2}+C_2=0 \ \Rightarrow\ C_2=\tfrac{1}{2}.
	\end{aligned}
\]

\[
	\begin{aligned}
		p_1(t) & = \tfrac{1}{2}-\tfrac{1}{2}e^{-2t}, \\
		p_2(t) & = e^{-2t},                          \\
		p_3(t) & = \tfrac{1}{2}-\tfrac{1}{2}e^{-2t}.
	\end{aligned}
\]

\[
	\vec{p}(t)\xrightarrow[t\to\infty]{}\vec{p}(\infty)=\left(\tfrac{1}{2},\,0,\,\tfrac{1}{2}\right).
\]

\subsubsection*{Задача 6.2}


\begin{tikzpicture}[
		>=Stealth, thick,
		state/.style = {draw, circle, minimum size=9mm, font=\small},
		e/.style   = {->, shorten >=2pt, shorten <=2pt},
		lbl/.style  = {font=\scriptsize, fill=white, inner sep=1pt}
	]

	% --- вершины ---
	\node[state] (Q0) at (0,0) {$1$};
	\node[state] (Q1) at (2.5,0) {$2$};
	\node[state] (Q2) at (5,0) {$3$};

	% --- рёбра ---
	\path[e, bend left=12] (Q1) edge node[lbl] {$1$} (Q0);
	\path[e, bend left=12]   (Q1) edge node[lbl] {$1$} (Q2);

\end{tikzpicture}

\[
	\Lambda =
	\begin{pmatrix}
		0 & 0  & 0 \\
		1 & -2 & 1 \\
		0 & 0  & 0
	\end{pmatrix}
\]


\[
	\vec{p}(0) = (\beta_1,\beta_2,\beta_3).
\]

\[
	\vec{p}'(t) = \vec{p}(t)\Lambda,\quad \vec{p}(t)=(p_1(t),p_2(t),p_3(t)).
\]

\[
	\begin{aligned}
		p_1'(t) & = p_1 (t)\cdot 0 + p_2(t)\cdot 1 + p_3(t)\cdot 0 = p_2(t),     \\
		p_2'(t) & = p_1 (t)\cdot 0 + p_2(t)\cdot(-2) + p_3(t)\cdot 0 = -2p_2(t), \\
		p_3'(t) & = p_1 (t)\cdot 0 + p_2(t)\cdot 1 + p_3(t)\cdot 0 = p_2(t).
	\end{aligned}
\]

\[
	\begin{aligned}
		p_2(t) & = p_2(0)e^{-2t} = \beta_2 e^{-2t},                                             \\
		p_1(t) & = \int p_2(t)\,dt = \int \beta_2 e^{-2t}\,dt = -\tfrac{\beta_2}{2}e^{-2t}+C_1, \\
		p_3(t) & = \int p_2(t)\,dt = \int \beta_2 e^{-2t}\,dt = -\tfrac{\beta_2}{2}e^{-2t}+C_2.
	\end{aligned}
\]

\[
	\begin{aligned}
		p_1(0) & = -\tfrac{\beta_2}{2}+C_1=\beta_1 \ \Rightarrow\ C_1=\beta_1+\tfrac{\beta_2}{2}, \\
		p_3(0) & = -\tfrac{\beta_2}{2}+C_2=\beta_3 \ \Rightarrow\ C_2=\beta_3+\tfrac{\beta_2}{2}.
	\end{aligned}
\]

\[
	\begin{aligned}
		p_1(t) & = \beta_1+\tfrac{\beta_2}{2}-\tfrac{\beta_2}{2}e^{-2t}, \\
		p_2(t) & = \beta_2 e^{-2t},                                      \\
		p_3(t) & = \beta_3+\tfrac{\beta_2}{2}-\tfrac{\beta_2}{2}e^{-2t}.
	\end{aligned}
\]

\[
	\vec{p}(t)\xrightarrow[t\to\infty]{}\vec{p}(\infty)=\left(\beta_1+\tfrac{\beta_2}{2},\,0,\,\beta_3+\tfrac{\beta_2}{2}\right).
\]

\subsection{Семинар 7}

\subsubsection*{Задача 7.1}


\begin{tikzpicture}[
		>=Stealth, thick,
		state/.style = {draw, circle, minimum size=9mm, font=\small},
		e/.style   = {->, shorten >=2pt, shorten <=2pt},
		lbl/.style  = {font=\scriptsize, fill=white, inner sep=1pt}
	]

	% --- вершины ---
	\node[state] (Q0) at (0,0) {$1$};
	\node[state] (Q1) at (2.5,0) {$2$};
	\node[state] (Q2) at (5,0) {$3$};

	% --- рёбра ---
	\path[e, bend left=12] (Q1) edge node[lbl] {$2$} (Q0);
	\path[e, bend left=12]   (Q1) edge node[lbl] {$1$} (Q2);

	\path[e, bend left=12]   (Q2) edge node[lbl] {$2$} (Q1);

\end{tikzpicture}


\[
	\Lambda =
	\begin{pmatrix}
		0 & 0  & 0  \\
		2 & -3 & 1  \\
		0 & 2  & -2
	\end{pmatrix}
\]

\[
	\vec{p}(0) = (0,0,1).
\]

\[
	\vec{p}'(t) = \vec{p}(t)\Lambda,\quad \vec{p}(t)=(p_1(t),p_2(t),p_3(t)).
\]

\[
	\begin{aligned}
		p_1'(t) & = p_1 (t)\cdot 0 + p_2(t)\cdot 2 + p_3(t)\cdot 0 = 2p_2(t),            \\
		p_2'(t) & = p_1 (t)\cdot 0 + p_2(t)\cdot(-3) + p_3(t)\cdot 2 = -3p_2(t)+2p_3(t), \\
		p_3'(t) & = p_1 (t)\cdot 0 + p_2(t)\cdot 1 + p_3(t)\cdot(-2) = p_2(t)-2p_3(t).
	\end{aligned}
\]

Преобразование Лапласа:

\[
	\mathcal{L}\{\vec p'(t)\}(s)=s \vec \pi(s)-\vec p(0)
	,\quad
	\mathcal{L}\{\vec p(t)\}(s)=\vec \pi(s)
	,\quad
	s \vec \pi(s)-\vec p(0)=\vec \pi(s)\Lambda.
\]

\[
	\pi_1+\pi_2+\pi_3=\frac{1}{s}
\]

\[
	\begin{aligned}
		s \pi_1(s)-0 & = \pi_1(s)\cdot(0) + \pi_2(s)\cdot 2 + \pi_3(s)\cdot 0 = 2\pi_2(s),             \\
		s \pi_2(s)-0 & = \pi_1(s)\cdot 0 + \pi_2(s)\cdot(-3) + \pi_3(s)\cdot 2 = -3\pi_2(s)+2\pi_3(s), \\
		s \pi_3(s)-1 & = \pi_1(s)\cdot 0 + \pi_2(s)\cdot 1 + \pi_3(s)\cdot(-2) = \pi_2(s)-2\pi_3(s).
	\end{aligned}
\]

\[
	\begin{aligned}
		(s)\pi_1(s)-2\pi_2(s)    & = 0, \\
		-2\pi_3(s)+(s+3)\pi_2(s) & = 0, \\
		-\pi_2(s)+(s+2)\pi_3(s)  & = 1.
	\end{aligned}
\]

\[
	\begin{aligned}
		\pi_1(s) & = \frac{2}{s}\pi_2(s),   \\
		\pi_3(s) & = \frac{s+3}{2}\pi_2(s).
	\end{aligned}
\]

\[
	\pi_2(s)\left(\frac{2}{s}+1+\frac{s+3}{2}\right) = \frac{1}{s}.
\]

\[
	\pi_2(s)\left(\frac{4+2s+s(s+3)}{2s}\right) = \frac{1}{s}.
\]

\[
	\pi_2(s) = \frac{2}{4+2s+s(s+3)}
	= \frac{2}{s^2+5s+4}
	=\frac{2}{(s+1)(s+4)}.
\]

\[
	\pi_1(s) = \frac{2}{s} \pi_2(s) = \frac{4}{s(s+1)(s+4)},
	\quad
	\pi_3(s) = \frac{s+3}{2} \pi_2(s) = \frac{s+3}{(s+1)(s+4)}.
\]

\[
	\begin{aligned}
		\pi_1(s) & = \frac{4}{s(s+1)(s+4)}
		= \frac{A}{s} + \frac{B}{s+1} + \frac{C}{s+4}                         \\
		\pi_2(s) & = \frac{2}{(s+1)(s+4)} = \frac{D}{s+1} + \frac{E}{s+4}     \\
		\pi_3(s) & =	 \frac{s+3}{(s+1)(s+4)} = \frac{F}{s+1} + \frac{G}{s+4}.
	\end{aligned}
\]

\[
	\begin{aligned}
		A & = \lim_{s\rightarrow 0} s\pi_1(s)      = \frac{4}{(0+1)(0+4)} = 1,         \\
		B & = \lim_{s\rightarrow -1} (s+1)\pi_1(s) = \frac{4}{(-1)(3)} = -\frac{4}{3}, \\
		C & = \lim_{s\rightarrow -4} (s+4)\pi_1(s) = \frac{4}{(-4)(-3)} = \frac{1}{3}, \\
		D & = \lim_{s\rightarrow -1} (s+1)\pi_2(s) = \frac{2}{(-1+4)} = \frac{2}{3},   \\
		E & = \lim_{s\rightarrow -4} (s+4)\pi_2(s) = \frac{2}{(-4+1)} = -\frac{2}{3},  \\
		F & = \lim_{s\rightarrow -1} (s+1)\pi_3(s) = \frac{-1+3}{-1+4} = \frac{2}{3},  \\
		G & = \lim_{s\rightarrow -4} (s+4)\pi_3(s) = \frac{-4+3}{-4+1} = \frac{1}{3}.
	\end{aligned}
\]

\[
	\begin{aligned}
		\pi_1(s) & = \frac{1}{s} - \frac{4}{3(s+1)} + \frac{1}{3(s+4)}, \\
		\pi_2(s) & = \frac{2}{3(s+1)} - \frac{2}{3(s+4)},               \\
		\pi_3(s) & = \frac{2}{3(s+1)} + \frac{1}{3(s+4)}.
	\end{aligned}
\]

\[
	\begin{aligned}
		p_1(t) & = \mathcal{L}^{-1}\{\pi_1(s)\}(t) = 1 - \tfrac{4}{3}e^{-t} + \tfrac{1}{3}e^{-4t}, \\
		p_2(t) & = \mathcal{L}^{-1}\{\pi_2(s)\}(t) = \tfrac{2}{3}e^{-t} - \tfrac{2}{3}e^{-4t},     \\
		p_3(t) & = \mathcal{L}^{-1}\{\pi_3(s)\}(t) = \tfrac{2}{3}e^{-t} + \tfrac{1}{3}e^{-4t}.
	\end{aligned}
\]

\[
	\vec{p}(t)\xrightarrow[t\to\infty]{}\vec{p}(\infty)=\left(1,\,0,\,0\right).
\]

\subsubsection*{Задача 7.2}


\begin{tikzpicture}[
		>=Stealth, thick,
		state/.style = {draw, circle, minimum size=9mm, font=\small},
		e/.style   = {->, shorten >=2pt, shorten <=2pt},
		lbl/.style  = {font=\scriptsize, fill=white, inner sep=1pt}
	]

	% --- вершины ---
	\node[state] (Q0) at (0,0) {$1$};
	\node[state] (Q1) at (2.5,0) {$2$};
	\node[state] (Q2) at (5,0) {$3$};

	% --- рёбра ---
	\path[e, bend left=12] (Q0) edge node[lbl] {$9$} (Q1);

	\path[e, bend left=12] (Q1) edge node[lbl] {$2$} (Q2);

	\path[e, bend left=24] (Q2) edge node[lbl] {$2$} (Q0);

\end{tikzpicture}


\[
	\Lambda =
	\begin{pmatrix}
		-9 & 9  & 0  \\
		0  & -2 & 2  \\
		2  & 0  & -2
	\end{pmatrix}
\]

\[
	\vec{p}(0) = (1,0,0).
\]

\[
	\vec{p}'(t) = \vec{p}(t)\Lambda,\quad \vec{p}(t)=(p_1(t),p_2(t),p_3(t)).
\]

\[
	\begin{aligned}
		p_1'(t) & =  p_1 (t)\cdot(-9) + p_2(t)\cdot 0 + p_3(t)\cdot 2 = -9p_1(t)+2p_3(t), \\
		p_2'(t) & =  p_1 (t)\cdot 9 + p_2(t)\cdot(-2) + p_3(t)\cdot 0 = 9p_1(t)-2p_2(t),  \\
		p_3'(t) & =  p_1 (t)\cdot 0 + p_2(t)\cdot 2 + p_3(t)\cdot(-2) = 2p_2(t)-2p_3(t).
	\end{aligned}
\]

Преобразование Лапласа:

\[
	\mathcal{L}\{\vec p'(t)\}(s)=s \vec \pi(s)-\vec p(0)
	,\quad
	\mathcal{L}\{\vec p(t)\}(s)=\vec \pi(s)
	,\quad
	s \vec \pi(s)-\vec p(0)=\vec \pi(s)\Lambda.
\]

\[
	\pi_1+\pi_2+\pi_3=\frac{1}{s}
\]

\[
	\begin{aligned}
		s \pi_1(s)-1 & = \pi_1(s)\cdot(-9) + \pi_2(s)\cdot 0 + \pi_3(s)\cdot 2 = -9\pi_1(s)+2\pi_3(s), \\
		s \pi_2(s)-0 & = \pi_1(s)\cdot 9 + \pi_2(s)\cdot(-2) + \pi_3(s)\cdot 0 = 9\pi_1(s)-2\pi_2(s),  \\
		s \pi_3(s)-0 & = \pi_1(s)\cdot 0 + \pi_2(s)\cdot 2 + \pi_3(s)\cdot(-2) = 2\pi_2(s)-2\pi_3(s).
	\end{aligned}
\]

\[
	\begin{aligned}
		(s+9)\pi_1(s)-2\pi_3(s)  & = 1, \\
		-9\pi_1(s)+(s+2)\pi_2(s) & = 0, \\
		-2\pi_2(s)+(s+2)\pi_3(s) & = 0.
	\end{aligned}
\]

\[
	\begin{aligned}
		\pi_2(s) & = \frac{9}{s+2}\pi_1(s),                              \\
		\pi_3(s) & = \frac{2}{s+2}\pi_2(s) = \frac{18}{(s+2)^2}\pi_1(s).
	\end{aligned}
\]

\[
	\pi_1(1+ \frac{9}{s+2} + \frac{18}{(s+2)^2}) = \frac{1}{s}.
\]

\[
	\pi_1\left(\frac{(s+2)^2+9s(s+2)^2+ 18s}{(s+2)^2}\right) = \frac{1}{s}.
\]

\[
	\pi_1(s) = \frac{(s+2)^2}{s((s+2)^2+9(s+2)^2+ 18)}
	= \frac{(s+2)^2}{s(s^2+13s+40)}
	= \frac{(s+2)^2}{s(s+5)(s+8)}.
\]

\[
	\pi_2(s) = \frac{9}{s+2}\pi_1(s) = \frac{9(s+2)}{s(s+5)(s+8)},
	\quad
	\pi_3(s) = \frac{18}{(s+2)^2}\pi_1(s) = \frac{18}{s(s+5)(s+8)}.
\]

\[
	\begin{aligned}
		\pi_1(s) & = \frac{(s+2)^2}{s(s+5)(s+8)} = \frac{A}{s} + \frac{B}{s+5} + \frac{C}{s+8}, \\
		\pi_2(s) & = \frac{9(s+2)}{s(s+5)(s+8)} = \frac{D}{s} + \frac{E}{s+5} + \frac{F}{s+8},  \\
		\pi_3(s) & = \frac{18}{s(s+5)(s+8)} = \frac{G}{s} + \frac{H}{s+5} + \frac{I}{s+8}.
	\end{aligned}
\]

\[
	\begin{aligned}
		A & = \lim_{s\rightarrow 0} s\pi_1(s)      & = \frac{(0+2)^2}{(0+5)(0+8)} = \frac{4}{40} = \frac{1}{10}, \\
		B & = \lim_{s\rightarrow -5} (s+5)\pi_1(s) & = \frac{(-5+2)^2}{-5(-5+8)} = \frac{9}{-15} = -\frac{3}{5}, \\
		C & = \lim_{s\rightarrow -8} (s+8)\pi_1(s) & = \frac{(-8+2)^2}{-8(-8+5)} = \frac{36}{24} = \frac{3}{2},  \\
		D & = \lim_{s\rightarrow 0} s\pi_2(s)      & = \frac{9(0+2)}{(0+5)(0+8)} = \frac{18}{40} = \frac{9}{20}, \\
		E & = \lim_{s\rightarrow -5} (s+5)\pi_2(s) & = \frac{9(-5+2)}{-5(-5+8)} = \frac{-27}{-15} = \frac{9}{5}, \\
		F & = \lim_{s\rightarrow -8} (s+8)\pi_2(s) & = \frac{9(-8+2)}{-8(-8+5)} = \frac{-54}{24} = -\frac{9}{4}, \\
		G & = \lim_{s\rightarrow 0} s\pi_3(s)      & = \frac{18}{(0+5)(0+8)} = \frac{18}{40} = \frac{9}{20},     \\
		H & = \lim_{s\rightarrow -5} (s+5)\pi_3(s) & = \frac{18}{-5(-5+8)} = \frac{18}{-15} = -\frac{6}{5},      \\
		I & = \lim_{s\rightarrow -8} (s+8)\pi_3(s) & = \frac{18}{-8(-8+5)} = \frac{18}{24} = \frac{3}{4}.
	\end{aligned}
\]

\[
	\begin{aligned}
		\pi_1(s) & = \frac{1}{10s} - \frac{3}{5(s+5)} + \frac{3}{2(s+8)}, \\
		\pi_2(s) & = \frac{9}{20s} + \frac{9}{5(s+5)} - \frac{9}{4(s+8)}, \\
		\pi_3(s) & = \frac{9}{20s} - \frac{6}{5(s+5)} + \frac{3}{4(s+8)}.
	\end{aligned}
\]

\[
	\begin{aligned}
		p_1(t) & = \mathcal{L}^{-1}\{\pi_1(s)\}(t)
		= \frac{1}{10} - \frac{3}{5}e^{-5t} + \frac{3}{2}e^{-8t}, \\
		p_2(t) & = \mathcal{L}^{-1}\{\pi_2(s)\}(t)
		= \frac{9}{20} + \frac{9}{5}e^{-5t} - \frac{9}{4}e^{-8t}, \\
		p_3(t) & = \mathcal{L}^{-1}\{\pi_3(s)\}(t)
		= \frac{9}{20} - \frac{6}{5}e^{-5t} + \frac{3}{4}e^{-8t}.
	\end{aligned}
\]

\[
	\vec{p}(t)\xrightarrow [t\to\infty]{}\vec{p}(\infty)=\left(\tfrac{1}{10},\,\tfrac{9}{20},\,\tfrac{9}{20}\right).
\]

Стационарное распределение $\overline r=(r_1,r_2,r_3,\ldots)$

\[
	\begin{cases}
		r_1 = -9 r_1 + 2 r_3 \\
		r_2 = 9 r_1 -2 r_2   \\
		r_3 = 2 r_2 -2 r_2   \\
		r_1 + r_2 + r_3 = 1.
	\end{cases}
	, \quad
	\begin{cases}
		-9r_1 + 2r_3 = 0, \\
		9r_1 - 2r_2 = 0,  \\
		2r_2 - 2r_3 = 0,  \\
		r_1 + r_2 + r_3 = 1.
	\end{cases}
\]

\[
	r_2=r_3=\frac{9}{2}r_1
	, \quad
	r_1(1+\frac{9}{2}+\frac{9}{2})=1
	, \quad
	r_1=\frac{1}{10}
\]

\[
	\vec r = (\frac{1}{10}, \frac{9}{20}, \frac{9}{20})
\]

\newpage
\section{Граф знаний}

\subsection{Определения}

\begin{definition}
	\term{Случайный процесс} — семейство в измеримом пространстве $(S,B)$, определённое на одном вероятностном пространстве $(\Omega, A, P)$.
	Где $S$ - пространство состояний случайного процесса.
\end{definition}



\subsection{Теоремы}
\begin{theorem}[пример теоремы]
	Формулировка теоремы.
\end{theorem}



\newpage
\section{Вопросы по экзамену}


\newpage
\section{Задачи по экзамену}



\end{document}
