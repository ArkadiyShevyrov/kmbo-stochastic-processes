\documentclass[a4paper,12pt]{article}

% Пакеты
\usepackage[utf8]{inputenc}
\usepackage[T2A]{fontenc}
\usepackage[russian]{babel}
\usepackage{graphicx}

\usepackage{amsmath,amssymb,amsthm}
\usepackage{multirow}
\usepackage{diagbox}
\usepackage{tabularx}
\usepackage{hyperref}
\usepackage{tikz} 
\usetikzlibrary{arrows.meta,positioning}
\tikzset{lab/.style={font=\footnotesize, inner sep=1pt}}

\usepackage[a4paper,top=2.54cm,bottom=2.54cm,left=3cm,right=1.5cm]{geometry}

\newtheorem{theorem}{Теорема}[section]
\newtheorem{definition}{Определение}[section]
\newtheorem{example}{Пример}[section]

\newcommand{\term}[1]{\textit{#1}\textnormal{}} 

\usepackage{epigraph}

\setlength{\epigraphwidth}{0.6\textwidth} % ширина цитаты
\setlength{\epigraphrule}{0pt}            % убрать линию над цитатой
\renewcommand{\epigraphflush}{flushleft}  % выравнивание по левому краю

% Заголовок
\title{Случайные процессы}
\date{Сентябрь 2025}

\begin{document}

\maketitle

\tableofcontents
\newpage

% Подключаем остальные файлы
% Подключаем остальные файлы
\newpage
\section{Лекции}

\subsection{Лекция 1}

\subsubsection*{Основные понятия теории случайных процессов}

Теория случайных процессов является развитием теории вероятностей, в ней изучаются не отдельные случайные величины или векторы, а их семейства
$\{X_t,t\in T\subseteq R\}$, зависящие от параметра времени $t$.
Случайным процессом называют семейство в измеримом пространстве $(S,B)$ и определённых на одном вероятностном пространстве $(\Omega, A, P)$.
Пространство $S$ называют пространством состояний случайного процесса.

В зависимости от вида множества параметров $T$ случайный процесс может быть с дискретным или непрерывным временем.
Если $T=\mathbb{Z}_+=\{0,1,2,...\}$ (множество неотрицательных целых числе), то случайный процесс $X_t,t\in T$ называют цепью.
В зависимости от вида $S$ случайный процесс может быть с дискретным или непрерывным пространством состояний.

Случайные процессы применяются для моделирования эволюции реальных стохастических систем. Примерами могут служить: броуновское движение частиц, процессы рождении и гибели в биологических системах, генетическая эволюция, системы массового обслуживания и так далее.

\subsubsection*{Общая классификация случайных процессов}

\begin{table}[ht]
	\centering
	\begin{tabularx}{\textwidth}{|c|>{\raggedright\arraybackslash}X|>{\raggedright\arraybackslash}X|}
		\hline
		\diagbox{$T$}{$S$}                                                            &
		Дискретное                                                                    &
		Непрерывное                                                                     \\
		\hline
		Дискретное                                                                    &
		Последовательность дискретных случайных величин                               &
		Последовательность непрерывных случайных величин                                \\
		\hline
		Непрерывное                                                                   &
		Случайный процесс с непрерывным временем и дискретным пространством состояний &
		Случайный процесс с непрерывным временем и непрерывным пространством состояний  \\
		\hline
	\end{tabularx}
\end{table}

\subsubsection*{Основные понятия теории случайных процессов}

Для любого набора $t_1, t_2, ..., t_n \in T$ вектор
$(X(t_1), X(t_2),..., X(t_n))$ называется конечномерным сечением или $n$-мерным сечением случайного процесса $\{X_t, t\in T\}$.
При фиксированном $\omega \in \Omega$ отображение $t \rightarrow X_t(\omega)$ называется траекторией или выборочной функцией случайного процесса
$\{X_t,\in T \}$.
Семейство $\sigma$-алгебр \{$F_t, t\in T\}$ называется фильтрацией, если
$F_s \subset F_t \subset A$ для всех $s<t$, $s,t\in T$.
Случайный процесс $\{X_t,t\in T\}$ согласован с фильтрацией $\{F_t,t\in T\}$, если случайная величина $X_t$ измерима относительно $F_t$ для всех $t\in T$.

Пусть $S\subseteq R$, $S\in B(R)$, $B=B(S)$.
Случайный процесс называется регулярным, если его траектории в каждой точке $t\in T$ непрерывны справа и имеют конечные пределы слева.
Случайные процессы $\{X_t,t\in T\}$ и $\{Y_t,t\in T\}$ называются стохастически эквивалентными в широком смысле, если для всех $B_i\in B$, $i=1,..,n$, $t_1<t_2<...<t_{n-1}<t_n\in T$, верно равенство $P(X(t_1)\in B_1,...,X(t_n)\in B_n)=P(Y(t_1)\in b_1,...,Y(t_n)\in B_n)$.
Случайные процессы $\{X_t,t\in T\}$ и $\{Y_t, t\in T\}$ называются стохастически эквивалентными, если $P(X(t)=Y(t))=1$ для всех $t\in T$.

Пусть $t_1,t_2,...,t_n\in T$, $t(n)=(t_1,t_2,...,t_n)$, $X(t)=(X(t_1),X(t_2),...,X(t_n))$ - $n$-мерное сечение, $F_{t(n)}(x_1,x_2,...,x_n)=P(X(t_1)\leq x_1, ...,X(t_n)\leq x_n)$ - его функция распределения.
Для любых $t_1,t_2,...,t_n\in T$ $F_{t(n)}(x_1,x_2,...,x_n)$ удовлетворяют всем свойствам функции распределений случайных векторов.
Кроме того, условиям согласованности:
1) $F_{t(n)}(x_1,x_2,...,x_n)=F_{\sigma (t(n))}(x_{\sigma(1)},x_{\sigma(2)},...,x_{\sigma (n)})$, $\sigma (t(n))=(t_{\sigma (1)}, t_{\sigma (2)}, ..., t_{\sigma (n)})$, $\sigma \in S_n$ - группа перестановок;
2) $F_{t(n)}(x_1,...,x_m,+\infty,...,+\infty)=F_{t(m)}(x_1,x_2,...,x_m)$.
Теорема Колмогорова. Пусть задано семейство функций распределения
$F_{t(n)}(x_1,x_2,..x_n)$, удовлетворяющих условиям 1 и 2.
Тогда существует вероятностное пространство $(\Omega, A, P)$ и случайный процесс $\{X_t,t\in T\}$ с данными функциями распределения.

Полная информация о случайном процессе $\{X_t, t\in T\}$ содержится в функциях распределения всех его конечномерных сечений.
В общем случае найти все эти функции распределения практически невозможно.
Эта задача иногда упрощается при рассмотрении марковских случайных процессов.
Случайный процесс $\{X_t, t\in T\}$ называется марковским, если выполняется равенство условных вероятностей
$P(X(t_n)\in B_n | X(t_1)\in B_1, ..., X(t_{n-1})\in B_{n-1})=P(X(t_n)\in B_n | X(t_{n-1})\in B_{n-1})$ для всех $t_1<t_2<...<t_{n-1}<t_n \in T$, $B_i \in B$, $i=1,...,n$.

Вероятности конечномерных сечений для произвольного процесса \\
$P(X(t_1)\in B_1, ..., X(t_{n-1}) \in B_{n-1}, X(t_n) \in B_n) =\\
	P(X(t_n)\in B_n |X(t_1)\in B_1,...,X(t_{n-1})\in B_{n-1})\cdot P(X(t_{n-1})\in B_{n-1} |X(t_1)\in B_1,...,X(t_{n-2})\in B_{n-2})\cdot ... \cdot P(X(t_2)\in B_2 | X(t_1)\in B_1) \cdot P(X(t_1)\in B_1)$.

Вероятности конечномерных сечений для марковского процесса при
$t_1<t_2<...<t_{n-1}<t_n$,
$P(X(t_1)\in B_1,...,X(t_{n-1})\in B_{n-1},X(t_n)\in B_n) = \\
	P(X(t_n)\in B_n | X(t_{n-1})\in B_{n-1}) \cdot P(X(t_{n-1})\in B_{n-1} | X(t_{n-2})\in B_{n-2}) \cdot ... \cdot P(X(t_2)\in B_2 | X(t_1) \in B_1) \cdot P(X(t_1) \in B_1 )$.

\subsection{Лекция 2}

\subsubsection*{Цепи Маркова}

Последовательность случайных величин $\{X_k\}_{k=0}^\infty$ со значениями в $S=\{E_1,E_2,...\}$ называется цепью Маркова, если выполняется равенство условных вероятностей $P(X_{i_k}=E_{j_k} | X_{i_1}=E_{J_1},...,X_{i_{k-1}}=E_{j_{k-1}}) = P(X_{i_{k}}=E_{j_{k}} | X_{i_{k-1}}=E_{j_{k-1}})$ для произвольных $i_1<i_2<...<i_{k-1}<i_k$ и любых $E_{j_{1}},...,E_{j_{k}}$.

Если вероятности $p_{ij}=P(X_{k+1}=E_j | X_l = E_i)$ не зависят от $k$, то цепь Маркова называется однородной. При этом $p_ij$ называются переходными вероятностями, а матрица $P=(p_ij)$ называется матрицей вероятностей перехода за один шаг или переходной матрицей.

В однородной цепи Маркова вероятности $p_{ij}(m)=P(X_{k+m}=E_j | X_k = E_i)$ тоже не зависят от $k$. Матрица $P(m)=(p_{ij}(m))$ называется матрицей вероятностей перехода за $m$ шагов. При этом $0<=P-{ij}(m)<=1$ и $\sum_j p_{ij}(m)=1$. Матрица с двумя такими свойствами называется стохастической.

Вектор $\vec p (m)=(p_1(m), p_2(m), ...)$, где $p_i(m)=P(X_m=E_i)$, называется вектором распределения вероятностей через $m$ шагов. При этом $0\leq p_i(m)\leq 1$ и $\sum_i p_i(m) = 1$.

Вектор $\vec p (0)=(p_1(0),p_2(0),...)$ называется начальным распределением вероятностей цепи Маркова.

\subsubsection*{Свойства цепей Маркова}

Пусть $\{X_k\}_{k=0}^\infty $ - однородная цепь Маркова. Тогда $P(k)=P^k$.

Пусть $\{X_k\}_{k=0}^\infty$ - однородная цепь Маркова. Тогда $\vec p(m)= \vec p (0) P(m)$.

В качестве следствия получаем, что $\vec p(m)=\vec (0)P^m$.
Зная $\vec p(0)$ и $P$ можно найти все конечномерные распределения цепи Маркова: для произвольных $i_1<i_2<...<i_k$ и любых $E_{j_1},...,E_{j_k}$ получаем $P(X_{i_1}=E_{j_1},...,X_{i_k}=E_{j_k}) = P(X_{i_1}=E_{j_1})P(X_{i_2}=E_{j_2}|X_{i_1}=E_{j_1})...P(X_{i_k}=E_{j_k}|X_{i_{k-1}}=E_{j_{k-1}}) = p_{j_1}(i_1)p_{j_1j_2}(i_2-i_1)...p_{j_{k-1}j_k}(i_k-i_{k-1})$.

\subsubsection*{Стационарные распределения}

Пусть $\{X_k\}_{k=0}^\infty$ - однородная цепь Маркова со значениями в $S=\{E_1,E_2,...\}$, с переходной матрицей $P=(p_{ij})$ и начальным распределением вероятностей $\vec p(0)=(p_1(0),p_2(0),...)$.
Если существует $\lim\limits_{m \to \infty} \vec{p}(m) = \lim\limits_{m \to \infty} \vec p (0) P^m = \vec p (\infty )$, то $\vec p (\infty)$ называется предельным распределением вероятностей с начальным распределением $\vec p (0)$.
Если для распределения вероятностей $\vec r (0) = (r_1(0),r_2(0),...)$ выполняется условие $\vec r (m)=\vec r(0)$ для всех $m \geq 1$, то $\vec r = (r_1,r_2,...)=\vec r(0)$ называется стационарным распределением вероятностей.

Если  предельное распределение вероятностей $\vec p (\infty)$ с начальным распределением $\vec p(0)$ существует, то оно будет стационарным, так как $\vec p(\infty)= \lim\limits_{m\to \infty } \vec p (m)= \lim\limits_{m \to \infty} \vec p (m+1) = [\lim\limits_{m \to \infty } \vec p (m)]P = \vec p (\infty ) P$.
Стационарное распределение вероятностей $\vec r = (r_1,r_2,...)$ удовлетворяет уравнениями $r_i=\sum_j r_j p_{ij}$, $i=1,2,...$ и $\sum_j r_j=1$ (уравнение нормировки).

Для конечной цепи Маркова с $n$ состояниями получается система
\[
	\begin{aligned}
		r_1 & = r_1 p_{11} + r_2 p_{21} + \dots + r_n p_{n1}, \\
		r_2 & = r_1 p_{12} + r_2 p_{22} + \dots + r_n p_{n2}, \\
		    & \ \ \vdots                                      \\
		r_n & = r_1 p_{1n} + r_2 p_{2n} + \dots + r_n p_{nn}, \\
		1   & = r_1 + r_2 + \dots + r_n.
	\end{aligned}
\]

Система из первых $n$ уравнений вырожденная, так как из $\vec{r} = \vec{r}P$
следует
$\vec{r}(P - E) = \vec{0}$, ($E$ - единичная матрица),
а матрица $(P - E)$ вырожденная:
\[
	\det(P - E) =
	\det \begin{pmatrix}
		p_{11} - 1 & \dots  & p_{1n}     \\
		\vdots     & \ddots & \vdots     \\
		p_{n1}     & \dots  & p_{nn} - 1
	\end{pmatrix}
	=
	\det \begin{pmatrix}
		p_{11} - 1 & \dots  & 0      \\
		\vdots     & \ddots & \vdots \\
		p_{n1}     & \dots  & 0
	\end{pmatrix}
	= 0.
\]

\subsubsection*{Эргодические цепи Маркова}

Цепь Маркова с переходной матрицей $P = (p_{ij})$ называется эргодической, если

1) существует предел
$\lim\limits_{k \to \infty} p_{ij}(k) = q_{ij};$

2) $q_{ij}$ не зависят от $i$:
$q_{ij} = q_j;$

3)
$q_j > 0 \quad \text{для всех } j.$

Теорема 1. Цепь Маркова эргодическая тогда и только тогда, когда

1) для любого начального распределения $\vec{p}(0)$ существует предел
$\lim\limits_{k \to \infty} p_{j}(k) = q_j;$

2) $q_j$ не зависят от начального распределения;

3) $q_j > 0$ для всех $j$.

Теорема 2. Если для конечной цепи Маркова с переходной матрицей $P=(p_{ij})$ среди корней характеристического уравнения $|P-\lambda E|=0$ условию $|\lambda|=1$ удовлетворяет только корень $\lambda=1$, то предельное распределение существует.

Теорема 3. Если условиях теоремы 2 корень $\lambda=1$ имеет кратность 1, то предельное распределение не зависит от начального распределения.

Теорема 4. (Теорема Маркова) Если для конечной цепи Маркова с переходной матрицей $P=(p_{ij})$ существует такое $S$, что $p_{ij}(s)>0 (P(s)=\{p_{ij}(s)\}=P^s)$ для всех $i$ и $j$, то цепь Маркова является эргодической.

Лемма. Пусть $Q=(q_{ij})$ - стохастическая матрица размерности $n \times n$,
$
	\vec{a} = \begin{pmatrix}
		a_1    \\
		a_2    \\
		\vdots \\
		a_n
	\end{pmatrix}
$ - произвольный вектор-столбец (размерности $n \times 1$), $\vec{b}=Q\vec{a}$; $m_{\vec{a}}=\min a_i$, $M_{\vec{a}}=\max a_i$ ($m_{\vec{b}}$ и $M_{\vec{b}}$ определяются для $\vec{b}$ аналогично).
Тогда:

1) $m\vec{a}\leq m\vec{b} \leq M\vec{b} \leq M\vec{a}$;

2) Если $q_{ij} \geq \varepsilon$ $\forall i,j$, то $M_{}\vec{b}-m_{\vec{b}} \leq  (1-2\varepsilon) (M_{}\vec{a}-m_{\vec{a})}$

\subsubsection*{Классификация состояний цепи Маркова}

Состояние $j$ достижимо из состояния $i$ [ обозначение: $i \to j$], если $\exists k: p_{ij}(k)>0$.
Очевидно, что из $i \to j$ и $j \to s$ следует $i \to s$.
Состояния $i$ и $j$ - сообщающиеся [обозначение $i \rightleftarrows j$] если $i \to j$ и $j \to i$.
Состояние $i$ - существенное, если из $i \to j$ следует $j \to i$.
Если $i$ - существенное состояние и $i \to j$, то состояние $j$ - существенное
Если для состояния $i$ существует такое состояние $j$, что $j$ достижимо из состояния $i$, но $i$ недостижимо из $j$, то состояние $i$ называется несущественным.

Пусть для существенного состояния $i$ $S(i)={j:i \rightleftarrows j}$. Все состояния $j\in S(i)$ являются существенными. Пространство $S$ всех состояний цепи Маркова можно представить в виде $S=S(i_1)\cup S(i_2)\cup ...\cup E$, где $E$ - множество всех несущественных состояний.

Цепь Маркова - неприводимая, если $S=S(i)$ для всех $i\in S$. Периодом состояния $i\in S$ называется $k_i=НОД(k: p_{ii}(k)>0)$.

Теорема. Если $i \rightleftarrows j$, то $k_i=k_j$.

\subsection{Лекция 3}

Обозначим через $f_{ij}(m)$ вероятность того, что из состояния $i$ первый раз попадаем в состояние $j$ на шаге $m$
$f_{ij}(m)=P(X_m=E_j,X_k\ne E_j (0<j<m) | X_0 =E_i))$, через $g_{ij}$ вероятность того, что из состояния $i$ попадаем в состояние $j$ бесконечное число раз. По формуле $f^*_{ij}=\sum_{m=1}^\infty f_{ij}(m)$ находится вероятность того, что исходя из состояния $i$ попадём в состояние $j$ хотя бы один раз.
Верны следующие утверждения:

1) $g_{ij}=\lim_{m \to \infty } \sum _s p_{is}(m) f^*_{sj}$;

2) $g_{ij}=f^*_{ij}g_{jj}$;

3) $i \to j \;\Leftrightarrow\; f_{ij}^* > 0$;

4) $i \rightleftarrows j \;\Leftrightarrow\; f_{ij}^* f_{ji}^* > 0$.

Состояние $i$ называется возвратным, если $f^*_{ii}=1$, и невозвратным, если $f^*_{ii}<1$.
Верны следующие утверждения:

1) $g_{ij}=f^*_{ij}$, если состояние $j$ возвратно, и $g_{ij}=0$, если $j$ невозвратно;

2) $g_{ii}=1$, если состояние $i$ возвратно, и $g_{ii}=0$, если $i$ невозвратно;

3) Если состояние $i$ несущественное, то $i$ невозвратное;

4) Если $g_{ii}=1$ и $f^*_{ij}>0$, то $g_{jj}=1$;

5) Если состояние $i$ возвратно и $i \to j$, то $j$ возвратно и $g_{ij}=g_{ji}=1$;

6) Состояние $i$ возвратное если ряд $\sum_{m=1}^\infty p_{ii}(m)$ расходится, и $i$ невозвратное, если ряд сходится.


Возвратное состояние $i$ называется положительным, если $\lim_{m \to \infty } p_{ii}(mk_i)>0$, и нулевым, если $\lim_{m \to \infty } p_{ii}(mk_i)=0$.
Обозначим через $\mu_{ii}$ среднее время возвращения в состояние $i$ $\mu_{ii}=\sum_{m=1}^\infty mf_{ii}(m)$.
Верны следующие утверждения:

1) Если состояние $i$ возвратное с периодом $k_i$, то $\lim_{m \to \infty} p_{ii} (mk_i)=\frac{k_i}{\mu _{ii}}$;

2) Возвратное состояние $i$ положительно $\Leftrightarrow$ $\mu _{ii}< \infty$;

3) Если состояние $i$ возвратное положительное и $i \rightleftarrows j$, то состояние $j$ так же положительное;

4) Если состояние $i$ возвратное положительное, то $\lim_{m \to \infty} \frac{1}{m} \sum_{k=1}^m p_{ii}(k)=1\frac{1}{\mu _{ii}}$.

Цепь Маркова - неприводимая, если $S=S(i)$ для всех $i\in S$.

Период состояния $i$: $k_i=НОД(k: p_{ii}(k)>0)$.

Цепь Маркова - апериодическая, если $k_i=1$ для всех $i\in S$.

Цепь Маркова - эргодическая $\Leftrightarrow$ она неприводимая и апериодическая.

\subsubsection*{Эргодические цепи Маркова}

Теорема 1. (эргодическая теорема для конечной цепь Маркова). Любая неприводимая непериодическая цепь Маркова $\{v_n, n\geq 0 \}$ с конечным множеством состояний $g$ является эргодической.

Теорема 2. (эргодическая теорема Фостера для счётной цепи Маркова) $\{v_n, n \geq 0 \}$ была эргодической, необходимо и достаточно существование нетривиального решения $\{p_i, i \geq 1 \}$ СУР (4.4)??? такого, что $\sum_{i=1}^\infty |p_i|<\infty$.
Решение $\{p_i, i \geq 1 \}$ с точностью до нормирующего множителя совпадает с предельным (стационарным) распределением.

Теорема 3. (для счётной цепи Маркова). Для того, чтобы  неприводимая непериодическая цепь Маркова $\{v_n, n\geq 0 \}$ была эргодической, достаточно существование числа $\varepsilon > 0$, целого числа $i_0$ и набора неотрицательных чисел $x_1,x_2,...$ таких что
$\sum_{j=1}^\infty p_{ij} x_j \leq x_i - \varepsilon$, $i \geq i_0$;
$\sum_{j=1}^\infty p_{ij} x_i < \infty$, $i < i_0$;

\subsubsection*{Марковские процессы с непрерывным временем}

Случайный процесс $X_t, t \geq 0$ называется марковским, если для любого целого неотрицательного $m$, любых моментов времени $0\leq s_1<s_2<...<s_m<s$, $t>0$, любого набора состояний $E_{i_1}, E_{i_2},...,E_{i_m},E_i, E_j$ выполнено равенство $P(X_{s+t}=E_j | X_{s_1}=E_{i_1}, ..., X_{s_m}=E_{i_m}, X_s=E_i)= P(X_{s+t}=E_j | X_s=E_i)$

\subsubsection*{Однородный марковский процесс с непрерывным временем}

Процесс $X_t$ называется однородным (по времени), если условная вероятность $P(X_{s+t}=E_j|X_s=E_i)$ перехода из состояния $E_i$ в состояние $E_j$ за время $t$ не зависит от $s$.
Обозначим $p_{ij}(t)=P(X_{s+t}=E_j | X_s=E_i)$.
Свойства:

1) $p_{ij}(0)=0$, если $i\ne j$, а $p_{ii}(0)=1$;

2) $0\leq p_{ij}(t)\leq 1$;

3) $\sum _j p_{ij}(t)=1$.

Матрица вероятностей переходя за время $t$: $P(t)=||p_{ij}(t)||$.
Предполагаем что переходные вероятности $p_{ij}(t)$ дифференцируемы в нуле: $p'_{ij}(0)=\lambda_{ij}$, при $i\ne j$ $p_{ij}(t)=\lambda_{ij}t+o(t)$, $p_{ii}(t)=1+\lambda_{ii}t+o(t)$.
$P'(0)=\Lambda=||\lambda_{ij}||$ - матрица интенсивностей (плотностей вероятностей) перехода.
При $i\ne j$ $\lambda_{ij}$ называется интенсивностью (плотностью вероятности) перехода из $E_i$ в $E_j$.

\subsection{Лекция 4}

\subsubsection*{Марковский процесс с непрерывным временем}

Величина $\lambda_{ji} \cdot p_j(t)$ называется потоком вероятности из состояния $E_j$ в состояние $E_i$ в момент времени $t$.

Из дифференциальных уравнений Колмогорова следует: произвольная вероятности состояния равна
сумме всех потоков вероятностей, приходящих в это состояние, минус сумма всех потоков вероятностей,
выходящих из этого состояния.

\textbf{Свойства матрицы интенсивностей перехода:}

1) $\lambda_{ij}\geq 0$, при $i\ne j$;

2) $\lambda_{ii}<0$;

3) $\sum_j \lambda_{ij}=0$, $\lambda_{ii}=-\sum_{j\ne i} \lambda_{ij}$.

\textbf{Система дифференциальных уравнений Колмогорова:}

$\frac{dp_i(t)}{dt}=\sum_j \lambda_{ji} p_j(t)=\lambda_{ii} p_i(t) + \sum_{j\ne i} \lambda_{ji} p_j(t)
	=\sum_{j\ne i} \lambda_{ji} p_j(t) - p_i(t) \sum_{j\ne i} \lambda_{ji}$

---

$\left\{ {P}(t) \right\}_{t \ge 0}$

1) ${P}(0) = {I}$,

2) ${P}(t+s) = {P}(t)\,{P}(s)$,

3) $\vec{p}(t) = \vec{p}(0)\,{P}(t)$.

---

\[
	\lambda_{ij} = \lim_{\tau \to +0}
	\frac{p_{ij}(\tau) - p_{ij}(0)}{\tau}
\]

\[
	i \ne j \Rightarrow
	\lambda_{ij} = \lim_{\tau \to +0}
	\frac{p_{ij}(\tau)}{\tau} \ge 0,
	\qquad
	i = j \Rightarrow
	\lambda_{ii} = \lim_{\tau \to +0}
	\frac{p_{ii}(\tau) - 1}{\tau} \le 0.
\]

---


\[
	\sum_j p_{ij}(t) = 1,
	\qquad
	\left( \sum_j p_{ij}(t) \right)'_{t=0} = (1)' = 0,
\]

\[
	\sum_j \lambda_{ij} = 0,
	\qquad
	\Rightarrow
	\qquad
	\lambda_{ii} = - \sum_{j \ne i} \lambda_{ij}.
\]


---

\[
	\vec{p}'(t)
	= \lim_{\tau \to 0}
	\frac{\vec{p}(t+\tau) - \vec{p}(t)}{\tau}
	=
	\lim_{\tau \to 0}
	\frac{\vec{p}(t){P}(\tau) - \vec{p}(t)}{\tau}
\]

\[
	= \vec{p}(t)
	\lim_{\tau \to 0}
	\frac{P(\tau) - P(0)}{\tau}
	= \vec{p}(t)\,{\Lambda}.
\]

---

\begin{align*}
	p_i'(t)
	 & = \sum_j p_j(t)\,\lambda_{ji}
	\\[4pt]
	 & = \sum_{j \ne i} p_j(t)\,\lambda_{ji} + p_i(t)\,\lambda_{ii}
	\\[4pt]
	 & = \sum_{j \ne i} p_j(t)\,\lambda_{ji} - \sum_{j \ne i} p_i(t)\,\lambda_{ij}.
\end{align*}

---

\begin{align*}
	{P}'(t)
	 & = \lim_{\tau \to 0}
	\frac{{P}(t+\tau) - {P}(t)}{\tau}
	\\[4pt]
	 & = \lim_{\tau \to 0}
	\frac{{P}(t){P}(\tau) - {P}(t)}{\tau}
	\\[4pt]
	 & = {P}(t)
	\lim_{\tau \to 0}
	\frac{{P}(\tau) - {P}(0)}{\tau}
		= {P}(t)\,{\Lambda}
	\\[4pt]
	 & = \lim_{\tau \to 0}
	\frac{{P}(t){P}(\tau) - {P}(t)}{\tau}
	= \lim_{\tau \to 0}
	\frac{{P}(t+\tau) - {P}(t)}{\tau}
	\\[4pt]
	 & =
	\left(
	\lim_{\tau \to 0}
	\frac{{P}(t+\tau) - {P}(t)}{\tau}
	\right)
	{P}(t)
	= {\Lambda}\,{P}(t).
\end{align*}

---


Дифференциальные уравнения Колмогорова

Векторная форма дифференциальных уравнений Колмогорова для вероятностей состояний:
$\vec{p'}(t)=\vec{p}(t)\Lambda$, где $\vec{p}(t)=(p_0(t),p_1(t),...,p_n(t),...)$.

Прямое уравнение Колмогорова: $P'(t)=P\Lambda$.

Обратное уравнение Колмогорова: $P'(t)=\Lambda P(t)$.

---

Марковский процесс с непрерывным временем ($\vec{p}(t), \{P(t)\}_{t>0}$) - эргодический:
\[
	\exists \lim_{t \to +\infty} {P}(t) = {Q}
	=
	\begin{pmatrix}
		q_{11} & q_{12} & \cdots \\
		q_{21} & q_{22} & \cdots \\
		\vdots & \vdots & \ddots
	\end{pmatrix},
	\qquad
	q_{ij} > 0 \ \forall i,j.
\]

---

\begin{enumerate}
	\item
	      $\displaystyle
		      \exists \lim_{t \to +\infty}
		      \overline{\mathbf{p}}(t)
		      =
		      \overline{\mathbf{q}}
		      = (q_1, q_2, \ldots)
	      $

	\item предел не зависит от $\overline{\mathbf{p}}(0)$

	\item $q_j > 0 \ \forall j$
\end{enumerate}

---

Процесс Маркова$\{ X_t \}_{t \ge 0}$ — эргодический $\Longleftrightarrow$
1) неприводим, 2) существует стационарное распределение.
---

Распределение вероятностей состояний, которое не зависит от времени $p_i(t+\tau ) = p_i(t)=p_i$
для любых $t,\tau\leq 0$ и любых $i=1,2,...$ называется стационарным распределением.

Система линейных алгебраических уравнений для стационарных вероятностей
\[
	\begin{cases}
		\sum_j  \lambda_{ji} r_j = 0, \quad i=1,2,... \\
		\sum_j r_j = 1
	\end{cases}
\]

---

\textbf{Свойства матрицы интенсивностей перехода:}
\begin{enumerate}
	\item $\lambda_{ij} \ge 0 \quad$ при $i \ne j$;
	\item $\lambda_{ii} \le 0$;
	\item $\displaystyle \sum_j \lambda_{ij} = 0,
		      \qquad
		      \lambda_{ii} = - \sum_{j \ne i} \lambda_{ij}.$
\end{enumerate}

\vspace{1em}

\textbf{Система дифференциальных уравнений Колмогорова}
\[
	\frac{dp_i(t)}{dt}
	= \sum_j \lambda_{ji}\,p_j(t)
	= \lambda_{ii}\,p_i(t)
	+ \sum_{j \ne i} \lambda_{ji}\,p_j(t)
\]
\[
	= \sum_{j \ne i} \lambda_{ji}\,p_j(t)
	- p_i(t)\sum_{j \ne i} \lambda_{ij},
	\qquad i = 1, 2, \ldots
\]

\textbf{Стациорнаное распределение}

Из уравнений следует: при стационарном распределении для каждого стояния сумма всех потоков вероятностей, приходящих в это состояние, равна сумме всех потоков вероятностей, выходящих из этого состояния.

\[
	\sum_{j \ne i} \lambda_{ji} r_j = r_i \sum_{j \ne i} \lambda_{ij} r_i
\]

---

\subsubsection*{Время пребывания марковского процесса с непрерывным временем в состоянии}

Пусть $\tau(i)$ — время пребывания марковского процесса $X(t)$ в состоянии $i$.
Получаем
$P\bigl(\tau(i) > t\bigr)= \lim_{n \to \infty} [p_{ii}(\frac{t}{n})]^n.$

Но $p_{ii}\!\left(\tfrac{t}{n}\right)= 1 + \lambda_{ii}\tfrac{t}{n} + o\!\left(\tfrac{1}{n}\right).$
Пусть $\lambda_i = -\lambda_{ii} = \sum_{j \ne i} \lambda_{ij}$.

Тогда
$\lim_{n \to \infty} [p_{ii}(\frac{t}{n})]^n
	= \lim_{n \to \infty}
	\left[1 - \lambda_i \tfrac{t}{n} + o\!\left(\tfrac{1}{n}\right)\right]^n
	= e^{-\lambda_i t}.$

Значит, $F_{\tau(i)}(t) = 1 - e^{-\lambda_i t}$,
т.\,е. $\tau(i)$ имеет показатель­ное распределение с параметром $\lambda_i$.


\subsection{Лекция 5}

Пример. В системе 2 устройства и один мастер, проводящий ремонт.
Работа системы описывается марковским процессом $\{X_t, t\in [0, +\infty) \}$,
значением $X_t$ является число исправных устройств в момент времени $t$.
Каждое устройство выходит из строя с интенсивностью $\lambda=1$,
востановление устройства проходит с интенсивностью $\mu=2$.

Граф системы имеет вид:

\begin{center}
	\begin{tikzpicture}[
			>=Stealth, thick,
			state/.style = {draw, circle, minimum size=9mm, font=\small},
			e/.style   = {->, shorten >=2pt, shorten <=2pt},
			lbl/.style  = {font=\scriptsize, fill=white, inner sep=1pt}
		]

		% --- вершины ---
		\node[state] (Q1) at (0,0) {$2$};
		\node[state] (Q2) at (2.5,0) {$1$};
		\node[state] (Q3) at (5,0) {$0$};

		% --- рёбра ---
		% Q1
		\path[e, bend left=12]   (Q1) edge node[lbl] {$2$} (Q2);

		% Q2
		\path[e, bend left=12] (Q2) edge node[lbl] {$2$} (Q1);
		\path[e, bend left=12] (Q2) edge node[lbl] {$1$} (Q3);

		% Q3
		\path[e, bend left=12]   (Q3) edge node[lbl] {$2$} (Q2);

	\end{tikzpicture}
\end{center}

Найдём стационарные вероятности состояний.

---


Составим систему уравнений Колмогорова:
\[
	\begin{cases}
		p_0'(t) = -2p_0(t) + p_1(t);           \\
		p_1'(t) = 2p_0(t) - 3p_1(t) + 2p_2(t); \\
		p_2'(t) = 2p_1(t) - 2p_2(t).
	\end{cases}
\]

Решая систему уравнений для стационарных вероятностей:
\[
	\begin{cases}
		r_1 = 2r_0;         \\
		3r_1 = 2r_0 + 2r_2; \\
		r_1 = r_2;          \\
		r_0 + r_1 + r_2 = 1;
	\end{cases}
\]

находим
\[
	(r_0, r_1, r_2) = (0.2, 0.4, 0.4).
\]

---

\textbf{Задача 4.6.} \\
\textit{Два устройства и один мастер. Пусть $E_i$~--- работают $i$ устройств.}

\vspace{1em}

\begin{center}
	\begin{tikzpicture}[
			>=Stealth, thick,
			state/.style={draw, circle, minimum size=8mm, font=\small}
		]
		\node[state] (2) at (0,0) {2};
		\node[state] (1) at (2.5,0) {1};
		\node[state] (0) at (5,0) {0};

		\path[->] (2) edge[bend left=20] node[above] {$2\lambda$} (1);
		\path[->] (1) edge[bend left=20] node[below] {$\mu$} (2);

		\path[->] (1) edge[bend left=20] node[above] {$\lambda$} (0);
		\path[->] (0) edge[bend left=20] node[below] {$\mu$} (1);
	\end{tikzpicture}
\end{center}

\vspace{1em}

\[
	\begin{cases}
		2\lambda r_2 = \mu r_1,                     \\[4pt]
		(\lambda+\mu) r_1 = 2\lambda r_2 + \mu r_0, \\[4pt]
		\mu r_0 = \lambda r_1,                      \\[4pt]
		r_0 + r_1 + r_2 = 1.
	\end{cases}
\]

Из первого и третьего уравнений:
\[
	r_1 = \dfrac{2\lambda}{\mu} r_2,
	\qquad
	r_0 = \dfrac{\lambda}{\mu} r_1 = \dfrac{2\lambda^2}{\mu^2} r_2.
\]

Подставляем в нормировочное:
\[
	r_2\!\left(1 + \dfrac{2\lambda}{\mu} + \dfrac{2\lambda^2}{\mu^2}\right) = 1,
	\quad\Rightarrow\quad
	r_2 = \dfrac{\mu^2}{\mu^2 + 2\lambda\mu + 2\lambda^2}.
\]

Средняя доля времени простоя мастера: $r_2$

Средняя доля времени занятости мастера:
\[
	1 - r_2 = \dfrac{2\lambda(1+\mu)}{\mu^2 + 2\lambda\mu + 2\lambda^2}.
\]

\[
	r_1 = \dfrac{2\lambda\mu}{\mu^2 + 2\lambda\mu + 2\lambda^2},
	\qquad
	r_0 = \dfrac{2\lambda^2}{\mu^2 + 2\lambda\mu + 2\lambda^2}.
\]


\textbf{Задача 4.6} \quad [$\lambda = 1$, $\mu = 2$]

\vspace{1em}

\begin{center}
	\begin{tikzpicture}[
			>=Stealth, thick,
			state/.style={draw, circle, minimum size=8mm, font=\small}
		]
		\node[state] (2) at (0,0) {2};
		\node[state] (1) at (2.5,0) {1};
		\node[state] (0) at (5,0) {0};

		\path[->] (2) edge[bend left=20] node[above] {$2\lambda$} (1);
		\path[->] (1) edge[bend left=20] node[below] {$\mu$} (2);

		\path[->] (1) edge[bend left=20] node[above] {$\lambda$} (0);
		\path[->] (0) edge[bend left=20] node[below] {$\mu$} (1);
	\end{tikzpicture}
\end{center}

\vspace{1em}

\[
	\begin{cases}
		p_0'(t) = -2p_0(t) + p_1(t),           \\[4pt]
		p_1'(t) = 2p_0(t) + 2p_2(t) - 3p_1(t), \\[4pt]
		p_2'(t) = 2p_1(t) - 2p_2(t),           \\[4pt]
		p_2(0) = 1,\quad p_0(0) = p_1(0) = 0,  \\[4pt]
		p_0 + p_1 + p_2 = 1.
	\end{cases}
\]

$$p_i(t)=\pi_i (s)$$

\vspace{1em}
Переходим к изображению Лапласа:
$$
	\begin{cases}
		S \pi_0  = -2 \pi_0 + \pi_1,     \\
		S \pi_2 - 1 = 2 \pi_1 - 2 \pi_2. \\
	\end{cases}
$$

Сумма вероятностей:
\[
	\pi_0 + \pi_1 + \pi_2 = \frac{1}{S}.
\]

Из первого и последнего уравнений:
\[
	\pi_0 = \frac{\pi_1}{S + 2}, \qquad
	\pi_2 = \frac{1 + 2\pi_1}{S + 2}.
\]

Подставим:
\[
	\pi_1\!\left(\frac{1}{S + 2} + 1 + \frac{2}{S + 2}\right)
	= \text{[тут у Лобузова какой-то бред с числами]}.
\]

Следовательно:
\[
	p_1(t) = \frac{2}{5}\left(1 - e^{-5t}\right).
\]

Аналогично:
\[
	\pi_0 = \frac{1}{S} \cdot \frac{1}{5} - \frac{1}{3} \cdot \frac{1}{S + 2} + \frac{2}{15} \cdot \frac{1}{S + 5},
\]
\[
	p_0(t) = \frac{1}{5} - \frac{1}{3}e^{-2t} + \frac{2}{15}e^{-5t}.
\]

---

\[
	p_1(t) = \frac{2}{5} - \frac{2}{5} e^{-5t}
	\quad \xrightarrow[t \to +\infty]{} \quad
	\frac{2}{5},
\]

\[
	p_0(t) = \frac{1}{5} - \frac{1}{3} e^{-2t} + \frac{2}{15} e^{-5t}
	\quad \xrightarrow[t \to +\infty]{} \quad
	\frac{1}{5},
\]

\[
	p_2(t) = \frac{2}{5} + \frac{1}{3} e^{-2t} + \frac{4}{15} e^{-5t}
	\quad \xrightarrow[t \to +\infty]{} \quad
	\frac{2}{5}.
\]

\vspace{1em}

\[
	\pi_i(s) \; = \; p_i(t),
	\qquad
	p_i'(t) \; = \; s\,\pi_i(s) - p_i(0).
\]

---

Пусть $\tau$ — время нахождения в состоянии $i$,
а $\tau_{(s)}$ — время нахождения в этом состоянии после времени $s$.

\[
	P(\tau_{(s)} > t)
	= P(\tau > s + t \mid \tau > s)
	= \frac{e^{-\lambda (t + s)}}{e^{-\lambda s}}
	= e^{-\lambda t},
	\qquad [\,\lambda = \lambda_i\,].
\]

Отсюда
\[
	P(\tau_{(s)} \le t) =
	\begin{cases}
		0,                  & t < 0,   \\[6pt]
		1 - e^{-\lambda t}, & t \ge 0,
	\end{cases}
	\qquad
	\text{— показатель­ное распределение с параметром } \lambda.
\]

Следовательно $\tau_{(s)} \sim \tau$.

\subsubsection*{Процессы рождения и гибели. Процесс Пуассона}

Граф процесса рождения и гибели с конечным числом состояний

\begin{center}
	\begin{tikzpicture}[
			>=Stealth, thick,
			state/.style={draw, rectangle, minimum width=10mm, minimum height=8mm, font=\small, align=center}
		]

		% узлы
		\node[state] (0) at (0,0) {0};
		\node[state] (1) at (2,0) {1};
		\node[state] (2) at (4,0) {2};
		\node[state] (k) at (6,0) {$\dots$};
		\node[state] (n) at (8,0) {$n$};

		% стрелки вправо (рождения)
		\path[->] (0) edge[bend left=20] node[above] {$\lambda_0$} (1);
		\path[->] (1) edge[bend left=20] node[above] {$\lambda_1$} (2);
		\path[->] (2) edge[bend left=20] node[above] {$\lambda_2$} (k);
		\path[->] (k) edge[bend left=20] node[above] {$\lambda_{n-1}$} (n);

		% стрелки влево (гибели)
		\path[->] (1) edge[bend left=20] node[below] {$\mu_1$} (0);
		\path[->] (2) edge[bend left=20] node[below] {$\mu_2$} (1);
		\path[->] (k) edge[bend left=20] node[below] {$\mu_3$} (2);
		\path[->] (n) edge[bend left=20] node[below] {$\mu_n$} (k);

	\end{tikzpicture}
\end{center}

Система дифференциальных уравнений Колмогорова:

\[
	\begin{cases}
		p_0'(t) = -\lambda_0 p_0(t) + \mu_1 p_1(t), \\[4pt]
		p_k'(t) = \lambda_{k-1} p_{k-1}(t)
		- (\lambda_k + \mu_k) p_k(t)
		+ \mu_{k+1} p_{k+1}(t),
		 & 1 \le k < n,                             \\[4pt]
		p_n'(t) = \lambda_{n-1} p_{n-1}(t) - \mu_n p_n(t).
	\end{cases}
\]

---

Стационарные вероятности состояний $r_0, r_1, r_2, \ldots, r_n$ процесса рождения и гибели
с конечным числом состояний удовлетворяют системе линейных алгебраических уравнений:

\[
	\begin{cases}
		0 = -\lambda_0 r_0 + \mu_1 r_1, \\[4pt]
		0 = \lambda_{k-1} r_{k-1} - (\lambda_k + \mu_k) r_k + \mu_{k+1} r_{k+1},
		 & 1 \le k < n,                 \\[4pt]
		0 = \lambda_{n-1} r_{n-1} - \mu_n r_n.
	\end{cases}
\]

А также уравнению нормировки:
\[
	\sum_{k=0}^{n} r_k = 1.
\]

\newpage
\section{Семинары}

\subsection{Семинар 2}

\subsubsection*{Задача с книгами}

\begin{tikzpicture}[
		>=Stealth, thick,
		state/.style = {draw, circle, minimum size=9mm, font=\small},
		e1/.style   = {->, shorten >=2pt, shorten <=2pt},
		e2/.style   = {->, shorten >=2pt, shorten <=2pt},
		e3/.style   = {->, shorten >=2pt, shorten <=2pt},
		lbl/.style  = {font=\scriptsize, fill=white, inner sep=1pt}
	]

	% --- вероятности ---
	\newcommand{\p}[1]{p_{#1}}
	\renewcommand{\p}[1]{\ifcase#1\relax \or \frac{1}{2}\or \frac{1}{3}\or \frac{1}{6}\fi}

	% --- вершины ---
	\node[state] (Q1) at (90:3)   {$123$};
	\node[state] (Q2) at (30:3)   {$132$};
	\node[state] (Q3) at (-30:3)  {$213$};
	\node[state] (Q4) at (-90:3)  {$231$};
	\node[state] (Q5) at (-150:3) {$312$};
	\node[state] (Q6) at (150:3)  {$321$};

	% --- рёбра ---
	% Q1 = 123
	\path[e1, bend left=12] (Q1) edge node[lbl] {$\p{1}$} (Q4);
	\path[e2, bend left=12] (Q1) edge node[lbl] {$\p{2}$} (Q2);
	\path[e3, loop above]   (Q1) edge node[lbl] {$\p{3}$} (Q1);

	% Q2 = 132
	\path[e1, bend left=12] (Q2) edge node[lbl] {$\p{1}$} (Q6);
	\path[e2, loop right]   (Q2) edge node[lbl] {$\p{2}$} (Q2);
	\path[e3, bend left=12] (Q2) edge node[lbl] {$\p{3}$} (Q1);

	% Q3 = 213
	\path[e1, bend left=12] (Q3) edge node[lbl] {$\p{1}$} (Q4);
	\path[e2, bend left=12] (Q3) edge node[lbl] {$\p{2}$} (Q2);
	\path[e3, loop right]   (Q3) edge node[lbl] {$\p{3}$} (Q3);

	% Q4 = 231
	\path[e1, loop below]   (Q4) edge node[lbl] {$\p{1}$} (Q4);
	\path[e2, bend left=12] (Q4) edge node[lbl] {$\p{2}$} (Q5);
	\path[e3, bend left=12] (Q4) edge node[lbl] {$\p{3}$} (Q3);

	% Q5 = 312
	\path[e1, bend left=12] (Q5) edge node[lbl] {$\p{1}$} (Q6);
	\path[e2, loop left]    (Q5) edge node[lbl] {$\p{2}$} (Q5);
	\path[e3, bend left=12] (Q5) edge node[lbl] {$\p{3}$} (Q1);

	% Q6 = 321
	\path[e1, loop left]    (Q6) edge node[lbl] {$\p{1}$} (Q6);
	\path[e2, bend left=12] (Q6) edge node[lbl] {$\p{2}$} (Q5);
	\path[e3, bend left=12] (Q6) edge node[lbl] {$\p{3}$} (Q2);

\end{tikzpicture}

\[
	P =
	\begin{array}{c|cccccc}
		    & Q_1          & Q_2          & Q_3          & Q_4          & Q_5          & Q_6          \\
		\hline
		Q_1 & \tfrac{1}{6} & \tfrac{1}{3} & 0            & \tfrac{1}{2} & 0            & 0            \\
		Q_2 & \tfrac{1}{6} & \tfrac{1}{3} & 0            & 0            & 0            & \tfrac{1}{2} \\
		Q_3 & 0            & \tfrac{1}{3} & \tfrac{1}{6} & \tfrac{1}{2} & 0            & 0            \\
		Q_4 & 0            & 0            & \tfrac{1}{6} & \tfrac{1}{2} & \tfrac{1}{3} & 0            \\
		Q_5 & \tfrac{1}{6} & 0            & 0            & 0            & \tfrac{1}{3} & \tfrac{1}{2} \\
		Q_6 & 0            & \tfrac{1}{6} & 0            & 0            & \tfrac{1}{3} & \tfrac{1}{2}
	\end{array}
\]


Стационарное распределение $\overline r=(r_1,r_2,r_3,r_4,r_5,r_6)$
задаётся условиями
\[
	\overline r P = \overline r,
	\qquad \sum_{i=1}^6 r_i = 1,
	\qquad r_i \ge 0.
\]

Из матрицы $P$ получаем систему:
\[
	\begin{array}{rcl}
		\begin{aligned}
			r_1 & = \tfrac{1}{6}r_1 + \tfrac{1}{6}r_2 + \tfrac{1}{6}r_5,                   \\
			r_2 & = \tfrac{1}{3}r_1 + \tfrac{1}{3}r_2 + \tfrac{1}{3}r_3 + \tfrac{1}{6}r_6, \\
			r_3 & = \tfrac{1}{6}r_3 + \tfrac{1}{6}r_4,                                     \\
			r_4 & = \tfrac{1}{2}r_1 + \tfrac{1}{2}r_3 + \tfrac{1}{2}r_4,                   \\
			r_5 & = \tfrac{1}{3}r_4 + \tfrac{1}{3}r_5 + \tfrac{1}{3}r_6,                   \\
			r_6 & = \tfrac{1}{2}r_2 + \tfrac{1}{2}r_5 + \tfrac{1}{2}r_6,
		\end{aligned}
		 & \qquad \Rightarrow \qquad &
		\begin{aligned}
			6r_1 & = r_1 + r_2 + r_5,          \\
			6r_2 & = 2r_1 + 2r_2 + 2r_3 + r_6, \\
			6r_3 & = r_3 + r_4,                \\
			2r_4 & = r_1 + r_3 + r_4,          \\
			3r_5 & = r_4 + r_5 + r_6,          \\
			2r_6 & = r_2 + r_5 + r_6.
		\end{aligned}
	\end{array}
\]

\[
	\begin{array}{rcl}
		\begin{aligned}
			5r_1 - r_2 - r_5          & = 0, \\
			-2r_1 + 4r_2 - 2r_3 - r_6 & = 0, \\
			5r_3 - r_4                & = 0, \\
			-\,r_1 - r_3 + r_4        & = 0, \\
			-\,r_4 + 2r_5 - r_6       & = 0, \\
			r_1+r_2+r_3+r_4+r_5+r_6   & =1,
		\end{aligned}
		 & \qquad \Rightarrow \qquad &
		\underbrace{\!
			\begin{pmatrix}
				5  & -1 & 0  & 0  & -1 & 0  \\
				-2 & 4  & -2 & 0  & 0  & -1 \\
				0  & 0  & 5  & -1 & 0  & 0  \\
				-1 & 0  & -1 & 1  & 0  & 0  \\
				0  & 0  & 0  & -1 & 2  & -1 \\
				1  & 1  & 1  & 1  & 1  & 1
			\end{pmatrix}}_{\displaystyle A}
		\underbrace{\!
			\begin{pmatrix}r_1\\r_2\\r_3\\r_4\\r_5\\r_6\end{pmatrix}}_{\displaystyle x}
		=
		\underbrace{\!
			\begin{pmatrix}0\\0\\0\\0\\0\\1\end{pmatrix}}_{\displaystyle b}.
	\end{array}
\]

\[
	\left[
		\begin{array}{rrrrrr|r}
			5  & -1 & 0  & 0  & -1 & 0  & 0 \\
			-2 & 4  & -2 & 0  & 0  & -1 & 0 \\
			0  & 0  & 5  & -1 & 0  & 0  & 0 \\
			-1 & 0  & -1 & 1  & 0  & 0  & 0 \\
			0  & 0  & 0  & -1 & 2  & -1 & 0 \\
			1  & 1  & 1  & 1  & 1  & 1  & 1
		\end{array}
		\right]
	\overset{}{\longrightarrow}
	\left[
		\begin{array}{rrrrrr|r}
			1 & 0 & 0 & 0 & 0 & 0 & \tfrac{2}{25} \\
			0 & 1 & 0 & 0 & 0 & 0 & \tfrac{3}{20} \\
			0 & 0 & 1 & 0 & 0 & 0 & \tfrac{1}{50} \\
			0 & 0 & 0 & 1 & 0 & 0 & \tfrac{1}{10} \\
			0 & 0 & 0 & 0 & 1 & 0 & \tfrac{1}{4}  \\
			0 & 0 & 0 & 0 & 0 & 1 & \tfrac{2}{5}
		\end{array}
		\right]
\]

\[
	\overline r =
	\left(\tfrac{2}{25},\;\tfrac{3}{20},\;\tfrac{1}{50},\;\tfrac{1}{10},\;\tfrac{1}{4},\;\tfrac{2}{5}\right).
\]


\subsection{Семинар 3}

\subsubsection*{Задача}

\newpage
\section{Граф знаний}

\subsection{Определения}

\begin{definition}
	\term{Случайный процесс} — семейство в измеримом пространстве $(S,B)$, определённое на одном вероятностном пространстве $(\Omega, A, P)$.
	Где $S$ - пространство состояний случайного процесса.
\end{definition}



\subsection{Теоремы}
\begin{theorem}[пример теоремы]
	Формулировка теоремы.
\end{theorem}



\newpage
\documentclass[a4paper,12pt]{article}

% Пакеты
\usepackage[utf8]{inputenc}
\usepackage[T2A]{fontenc}
\usepackage[russian]{babel}
\usepackage{graphicx}

\usepackage{amsmath,amssymb,amsthm}
\usepackage{multirow}
\usepackage{diagbox}
\usepackage{tabularx}
\usepackage{hyperref}
\usepackage{tikz} 
\usetikzlibrary{arrows.meta,positioning}
\tikzset{lab/.style={font=\footnotesize, inner sep=1pt}}

\usepackage[a4paper,top=2.54cm,bottom=2.54cm,left=2.5cm,right=1.5cm]{geometry}

\newtheorem{theorem}{Теорема}[section]
\newtheorem{definition}{Определение}[section]
\newtheorem{example}{Пример}[section]

\newcommand{\term}[1]{\textit{#1}\textnormal{}} 

\usepackage{epigraph}

\usepackage[none]{hyphenat}
\pretolerance=10000
\tolerance=2000
\emergencystretch=3em
\hyphenpenalty=10000
\exhyphenpenalty=10000


\setlength{\epigraphwidth}{0.6\textwidth}
\setlength{\epigraphrule}{0pt}          
\renewcommand{\epigraphflush}{flushleft}

\newenvironment{condition}{\par\textbf{Условие.}}{\par}

% Заголовок
\title{Вопросы к экзамену. Лобузов}
\date{2025}

\begin{document}

\maketitle

\tableofcontents
\newpage

\subsection*{Вопрос 1}

\begin{condition}
	Траектории, сечения, конечномерные распределения случайных процессов. Классификация случайных процессов. Определение и свойства марковского процесса.
\end{condition}

% \newpage
\subsection*{Вопрос 2}

\begin{condition}
	Цепи Маркова. Переходные вероятности. Матрица переходных вероятностей однородной цепи Маркова, её свойства. Примеры.
\end{condition}

Последовательность случайных величин $\{X_k\}_{k=0}^{\infty}$ со значениями в
$S=\{E_1,E_2,\ldots\}$ называется цепью Маркова, если выполняется равенство
условных вероятностей
$
	P\!\left(X_{i_k}=E_{j_k}\mid X_{i_1}=E_{j_1},\ldots,X_{i_{k-1}}=E_{j_{k-1}}\right)
	=
	P\!\left(X_{i_k}=E_{j_k}\mid X_{i_{k-1}}=E_{j_{k-1}}\right)
$
для произвольных $i_1<i_2<\ldots<i_{k-1}<i_k$, $(k=3,4,\ldots)$ и любых
$E_{j_1},\ldots,E_{j_k}$.

Если вероятности
$
	p_{ij}=P(X_{k+1}=E_j\mid X_k=E_i)
$
не зависят от $k$, то цепь Маркова называется однородной. При этом $p_{ij}$
называются переходными вероятностями, а матрица
$
	P=(p_{ij})
$
называется матрицей вероятностей перехода за один шаг или переходной матрицей.

В однородной цепи Маркова вероятности
$
	p_{ij}(m)=P(X_{k+m}=E_j\mid X_k=E_i)
$
тоже не зависят от $k$. Матрица
$
	P(m)=(p_{ij}(m))
$
называется матрицей вероятностей перехода за $m$ шагов. При этом
$
	0\le p_{ij}(m)\le 1,\qquad \sum_j p_{ij}(m)=1.
$
Матрицы с такими свойствами называются стохастическими.

Вектор
$
	\vec p(m)=(p_1(m),p_2(m),\ldots),
	\qquad
	p_i(m)=P(X_m=E_i),
$
называется вектором распределения вероятностей через $m$ шагов. При этом
$
	0\le p_i(m)\le 1,\qquad \sum_i p_i(m)=1.
$

Вектор
$
	\vec p(0)=(p_1(0),p_2(0),\ldots)
$
называется начальным распределением вероятностей цепи Маркова.

Пусть $\{X_k\}_{k=0}^{\infty}$ — однородная цепь Маркова. Тогда
$
	P(m)=P^m.
$

Пусть $\{X_k\}_{k=0}^{\infty}$ — однородная цепь Маркова. Тогда
$
	\vec p(m)=\vec p(0)\,P(m).
$

В качестве следствия получаем
$
	\vec p(m)=\vec p(0)\,P^m.
$

Зная $\vec p(0)$ и $P$, можно найти все конечномерные распределения цепи Маркова:
для произвольных $i_1<i_2<\ldots<i_k$ и любых $E_{j_1},\ldots,E_{j_k}$ имеем
$
	\begin{aligned}
		P(X_{i_1}=E_{j_1},\ldots,X_{i_k}=E_{j_k})
		 & =
		P(X_{i_1}=E_{j_1})
		P(X_{i_2}=E_{j_2}\mid X_{i_1}=E_{j_1})
		\cdots
		P(X_{i_k}=E_{j_k}\mid X_{i_{k-1}}=E_{j_{k-1}}) \\
		 & =
		p_{j_1}(i_1)\,
		p_{j_1j_2}(i_2-i_1)\cdots
		p_{j_{k-1}j_k}(i_k-i_{k-1}).
	\end{aligned}
$

% \newpage
\subsection*{Вопрос 3}

\begin{condition}
	Конечные однородные цепи Маркова. Матрица переходных вероятностей за n шагов. Примеры.
\end{condition}

Если вероятности
$
	p_{ij} = P(X_{k+1}=E_j \mid X_k=E_i)
$
не зависят от $k$, то цепь Маркова называется \textit{однородной}.

Матрица
$
	P(m) = \bigl(p_{ij}(m)\bigr)
$
называется матрицей переходных вероятностей за $m$ шагов, где
$
	p_{ij}(m) = P(X_{k+m}=E_j \mid X_k=E_i).
$

При этом выполняются условия
$
	0 \le p_{ij}(m) \le 1, \qquad \sum_{j} p_{ij}(m) = 1,
$
то есть сумма элементов каждой строки равна единице.

% \newpage
\subsection*{Вопрос 4}

\begin{condition}
Стационарное распределение вероятностей конечной однородной цепи Маркова. Его вид для цепи Маркова с двумя состояниями.
\end{condition}

% \newpage
\subsection*{Вопрос 5}

\begin{condition}
Предельные вероятности. Эргодические цепи Маркова. Теорема Маркова.
\end{condition}

% \newpage
\subsection*{Вопрос 6}

\begin{condition}
Классификация состояний цепи Маркова. Существенные и несущественные состояния. Период состояния. Критерий эргодичности цепи Маркова.
\end{condition}

% \newpage
\subsection*{Вопрос 7}

\begin{condition}
	Марковский процесс с дискретным множеством состояний и непрерывным временем. Матрица интенсивностей перехода, её свойства.
\end{condition}

Случайный процесс $X_t$, $t\ge 0$, называется \emph{марковским}, если для любого
целого неотрицательного $m$, любых моментов времени
$
	0 \le s_1 < s_2 < \ldots < s_m \le s,\qquad t>0,
$
и любого набора состояний
$
	E_{i_1},E_{i_2},\ldots,E_{i_m},E_i,E_j
$
выполняется равенство
$
	P\!\left(X_{s+t}=E_j \mid X_{s_1}=E_{i_1},\ldots,X_{s_m}=E_{i_m},X_s=E_i\right)
	=
	P\!\left(X_{s+t}=E_j \mid X_s=E_i\right).
$

\medskip

Матрица вероятностей перехода за время $t$ определяется как
$
	P(t)=\|p_{ij}(t)\|.
$

Предполагается, что переходные вероятности $p_{ij}(t)$ дифференцируемы в нуле.
При этом
$
	p'_{ij}(0)=\lambda_{ij}, \qquad i\ne j,
$
и для малых $t$ справедливы разложения
$
	p_{ij}(t)=\lambda_{ij}t+o(t), \qquad i\ne j,
$
$
	p_{ii}(t)=1+\lambda_{ii}t+o(t).
$

Матрица
$
	P'(0)=\Lambda=\|\lambda_{ij}\|
$
называется \emph{матрицей интенсивностей} (или плотностей вероятностей) перехода.

\medskip

\textbf{Свойства матрицы интенсивностей перехода:}
\begin{enumerate}
	\item
	      $
		      \lambda_{ij}\ge 0, \qquad i\ne j;
	      $
	\item
	      $
		      \lambda_{ii}\le 0;
	      $
	\item
	      $
		      \sum_j \lambda_{ij}=0, \qquad
		      \lambda_{ii}=-\sum_{j\ne i}\lambda_{ij}.
	      $
\end{enumerate}

% \newpage
\subsection*{Вопрос 8}

\begin{condition}
Марковский процесс с дискретным множеством состояний и непрерывным временем. Уравнения Колмогорова для вероятностей состояний.
\end{condition}

% \newpage
\subsection*{Вопрос 9}

\begin{condition}
	Марковский процесс с дискретным множеством состояний и непрерывным временем. Уравнения для стационарных вероятностей. Эргодичность.
\end{condition}

Случайный процесс $X_t$, $t\ge 0$, называется \emph{марковским}, если для любого
целого неотрицательного $m$, любых моментов времени
$
	0 \le s_1 < s_2 < \ldots < s_m \le s,\qquad t>0,
$
и любого набора состояний
$
	E_{i_1},E_{i_2},\ldots,E_{i_m},E_i,E_j
$
выполняется равенство
$
	P\!\left(
	X_{s+t}=E_j
	\mid
	X_{s_1}=E_{i_1},\ldots,X_{s_m}=E_{i_m},X_s=E_i
	\right)
	=
	P\!\left(X_{s+t}=E_j \mid X_s=E_i\right).
$

\medskip

\textbf{Система линейных алгебраических уравнений для стационарных вероятностей:}
$
	\begin{cases}
		\displaystyle
		\sum_j \lambda_{ji}\,z_i = 0, \qquad i=1,2,\ldots, \\
		\displaystyle
		\sum_j z_j = 1.
	\end{cases}
$

\medskip

Цепь Маркова называется \emph{неприводимой}, если
$
	S=S(i)\qquad \forall\, i\in S.
$

\medskip

\textbf{Период состояния.}
Период состояния $i$ определяется как
$
	k_i=\gcd\{\,k:\; p_{ii}(k)>0\,\}.
$

\medskip

Цепь Маркова называется \emph{апериодической}, если
$
	k_i=1 \qquad \forall\, i\in S.
$

\medskip

Цепь Маркова называется \emph{эргодической}, если она является
неприводимой и апериодической.

\medskip

Марковский процесс с непрерывным временем
$
	\bigl(\vec p(t),\; P(t)\bigr)_{t\ge 0}
$
является эргодическим.

% \newpage
\subsection*{Вопрос 10}

\begin{condition}
Исследование марковского процесса с двумя состояниями и непрерывным временем.
\end{condition}

% \newpage
\subsection*{Вопрос 11}

\begin{condition}
	Распределение времени пребывания в одном состоянии в цепи Маркова с непрерывным временем. Марковское свойство показательного распределения.
\end{condition}

Пусть $T(i)$ — время пребывания марковского процесса $X(t)$
в состоянии $i$. Тогда
$
	P(T(i)>t)
	=
	\lim_{n\to\infty}\bigl[p_{ii}(t/n)\bigr]^n.
$

Но
$
	p_{ii}(t/n)=1+\lambda_{ii}\frac{t}{n}+o\!\left(\frac{t}{n}\right).
$
Положим
$
	\lambda_i=-\lambda_{ii}=\sum_{j\ne i}\lambda_{ij}.
$
Тогда
$
	\lim_{n\to\infty}\bigl[p_{ii}(t/n)\bigr]^n
	=
	\lim_{n\to\infty}
	\left(1-\lambda_i\frac{t}{n}+o\!\left(\frac{t}{n}\right)\right)^n
	=
	e^{-\lambda_i t}.
$

Следовательно,
$
	F_{T(i)}(t)=1-e^{-\lambda_i t},
$
то есть случайная величина $T(i)$ имеет показательное распределение
с параметром $\lambda_i$.

% \newpage
\subsection*{Вопрос 12}

\begin{condition}
	Процессы рождения и гибели. Составление для них уравнений Колмогорова. Формулы для стационарных вероятностей.
\end{condition}

\textbf{Система дифференциальных уравнений Колмогорова.}

$
	\begin{cases}
		p_0'(t) = -\lambda_0\,p_0(t) + \mu_1\,p_1(t), \\[6pt]
		p_k'(t) = \lambda_{k-1}\,p_{k-1}(t)
		-(\lambda_k+\mu_k)\,p_k(t)
		+\mu_{k+1}\,p_{k+1}(t),
		\qquad 1\le k\le n,                           \\[6pt]
		p_n'(t) = \lambda_{n-1}\,p_{n-1}(t) - \mu_n\,p_n(t).
	\end{cases}
$

\medskip

\textbf{Стационарные вероятности.}

Стационарные вероятности состояний
$z_0,z_1,z_2,\ldots,z_n$
процесса рождения и гибели с конечным числом состояний
удовлетворяют системе линейных алгебраических уравнений:
$
	\begin{cases}
		0 = -\lambda_0 z_0 + \mu_1 z_1, \\[6pt]
		0 = \lambda_{k-1} z_{k-1}
		-(\lambda_k+\mu_k) z_k
		+\mu_{k+1} z_{k+1},
		\qquad 1\le k\le n,             \\[6pt]
		0 = \lambda_{n-1} z_{n-1} - \mu_n z_n,
	\end{cases}
$
а также условию нормировки
$
	\sum_{k=0}^{n} z_k = 1.
$

% \newpage
\subsection*{Вопрос 13}

\begin{condition}
Поток событий. Определение простейшего потока. Распределение времени между событиями в простейшем потоке, свойства этого распределения.
\end{condition}

% \newpage
\subsection*{Вопрос 14}

\begin{condition}
Процесс Пуассона. Составление уравнений Колмогорова и их решение.
\end{condition}

% \newpage
\subsection*{Вопрос 15}

\begin{condition}
	Комплексные случайные процессы, их характеристики. Примеры.
\end{condition}

\textbf{Математическое ожидание комплексного случайного процесса.}

Пусть
$
	X_t = X_t^{(1)} + i X_t^{(2)}.
$
Математическое ожидание комплексного случайного процесса $X_t$ определяется как
$
	m_X(t)=\mathbb{E}X_t
	=
	\mathbb{E}X_t^{(1)} + i\,\mathbb{E}X_t^{(2)},
	\qquad t\in T.
$

\medskip

\textbf{Дисперсия комплексного случайного процесса.}

Дисперсией комплексного случайного процесса $X_t$ называется
$
	D_X(t)=\mathbb{E}\,|X_t-m_X(t)|^2,
	\qquad t\in T.
$

Обозначая
$
	X_t^{0}=X_t-m_X(t)
$
(центрированный случайный процесс), получаем
$
	D_X(t)=\mathbb{E}|X_t^{0}|^2
	=
	\mathbb{E}|X_t|^2-|m_X(t)|^2.
$

Действительно,
$
	\begin{aligned}
		D_X(t)
		 & =
		\mathbb{E}\bigl[(X_t-m_X(t))(\overline{X_t-m_X(t)})\bigr] \\
		 & =
		\mathbb{E}|X_t|^2
		-
		m_X(t)\,\overline{m_X(t)}
		-
		\overline{m_X(t)}\,m_X(t)
		+
		|m_X(t)|^2                                                \\
		 & =
		\mathbb{E}|X_t|^2-|m_X(t)|^2.
	\end{aligned}
$

\medskip

\textbf{Корреляционная функция комплексного случайного процесса.}

Корреляционная функция комплексного случайного процесса $X_t$ определяется как
$
	K_X(t,s)
	=
	\mathbb{E}\bigl[(X_t-m_X(t))(\overline{X_s-m_X(s)})\bigr],
	\qquad t,s\in T.
$

Эквивалентно,
$
	K_X(t,s)
	=
	\mathbb{E}[X_t\overline{X_s}]
	-
	m_X(t)\,\overline{m_X(s)}.
$

% \newpage
\subsection*{Вопрос 16}

\begin{condition}
Корреляционная функция случайного процесса и её свойства. Примеры.
\end{condition}

% \newpage
\subsection*{Вопрос 17}

\begin{condition}
	Каноническое разложение случайного процесса и его характеристики.
\end{condition}

\textbf{Каноническое разложение случайного процесса.}

Случайный процесс $X_t$ допускает каноническое разложение, если
$
	X_t=\sum_{i=1}^{n} V_i\,\psi_i(t)+d(t),
$
где случайные величины $V_i$ удовлетворяют условиям
$
	\mathbb{E}V_i=0,\qquad D V_i<\infty,
$
$
	\operatorname{cov}(V_i,V_j)=0,\quad i\ne j,
$
а функции $\psi_i(t)$ и $d(t)$ являются неслучайными.

\medskip

\textbf{Свойства канонического разложения.}

\begin{enumerate}
	\item
	      Математическое ожидание процесса:
	      $
		      m_X(t)=d(t).
	      $

	\item
	      Корреляционная функция:
	      $
		      K_X(t,s)=\sum_{i=1}^{n}\psi_i(t)\,\overline{\psi_i(s)}\,D V_i.
	      $

	\item
	      Дисперсия процесса:
	      $
		      D_X(t)=\sum_{i=1}^{n}\lvert\psi_i(t)\rvert^2\,D V_i.
	      $
\end{enumerate}

% \newpage
\subsection*{Вопрос 18}

\begin{condition}
Непрерывность случайного процесса в среднем квадратическом. Связь между непрерывностью математического ожидания и корреляционной функции.
\end{condition}

% \newpage
\subsection*{Вопрос 19}

\begin{condition}
	Дифференцируемость случайного процесса в среднем квадратическом. Математическое ожидание и корреляционная функция производной.
\end{condition}

\textbf{Дифференцируемость в среднем квадратическом.}

Случайный процесс $X_t$ называется \emph{дифференцируемым в среднем квадратическом}
на отрезке $[a,b]$, если для всех $t\in[a,b]$ существует предел
$
	\lim_{\Delta t\to 0}
	\frac{X_{t+\Delta t}-X_t}{\Delta t}
	=
	\frac{dX_t}{dt}
$
в среднем квадратическом.

Случайный процесс
$
	Y_t=\frac{dX_t}{dt}
$
называется \emph{производной в среднем квадратическом}
случайного процесса $X_t$.

\medskip

\textbf{Критерий дифференцируемости в среднем квадратическом.}

Случайный процесс $X_t$ дифференцируем в среднем квадратическом
на отрезке $[a,b]$ тогда и только тогда, когда выполняются условия:
\begin{enumerate}
	\item
	      функция математического ожидания $m_X(t)$ дифференцируема на $[a,b]$;
	\item
	      существует смешанная производная
	      $
		      \frac{\partial^2 K_X(t,s)}{\partial t\,\partial s}
	      $
	      при $t=s$ для всех $t\in[a,b]$.
\end{enumerate}

\medskip

При этом для процесса
$
	Y_t=\frac{dX_t}{dt}
$
выполняются соотношения:
$
	m_Y(t)=m_X'(t), \qquad \forall\,t\in[a,b],
$
$
	K_Y(t,s)
	=
	\frac{\partial^2 K_X(t,s)}{\partial t\,\partial s},
	\qquad \forall\, t,s\in(a,b).
$

% \newpage
\subsection*{Вопрос 20}

\begin{condition}
	Интегрируемость случайного процесса на отрезке в среднем квадратическом. Математическое ожидание и корреляционная функция интеграла.
\end{condition}

\textbf{Интегрируемость в среднем квадратическом.}

Случайный процесс $X_t$ называется \emph{интегрируемым в среднем квадратическом}
на отрезке $[a,b]$, если существует предел
$
	\lim_{\max |t_i-t_{i-1}|\to 0}
	\sum_{i=1}^{n}(t_i-t_{i-1})\,X_{t_i}
	=
	\int_a^b X_t\,dt,
$
где
$
	a=t_0<t_1<t_2<\ldots<t_n=b.
$

Случайный процесс
$
	\int_a^b X_t\,dt
$
называется \emph{интегралом в среднем квадратическом}
случайного процесса $X_t$.

\medskip

Если случайный процесс $X_t$ интегрируем в среднем квадратическом
на $[a,b]$ и
$
	Y_t=\int_a^t X_\tau\,d\tau,
$
то выполняются соотношения:
\begin{enumerate}
	\item
	      Математическое ожидание:
	      $
		      m_Y(t)=\int_a^t m_X(\tau)\,d\tau,
		      \qquad \forall\, t\in[a,b].
	      $

	\item
	      Корреляционная функция:
	      $
		      K_Y(t,s)
		      =
		      \int_a^t d\tau
		      \int_a^s K_X(\tau,\sigma)\,d\sigma,
		      \qquad \forall\, t,s\in[a,b].
	      $
\end{enumerate}

% \newpage
\subsection*{Вопрос 21}

\begin{condition}
Линейные преобразования случайных процессов. Нахождение корреляционной функции линейного преобразования.
\end{condition}

% \newpage
\subsection*{Вопрос 22}

\begin{condition}
Стационарные процессы в широком и узком смыслах. Свойства корреляционной функции.
\end{condition}

% \newpage
\subsection*{Вопрос 23}

\begin{condition}
Стационарный процесс с дискретным спектром, спектральное разложение корреляционной функции на отрезке.
\end{condition}

% \newpage
\subsection*{Вопрос 24}

\begin{condition}
Стационарный процесс с непрерывным спектром, спектральное разложение корреляционной функции на оси.
\end{condition}

% \newpage
\subsection*{Вопрос 25}

\begin{condition}
Спектральная плотность стационарного процесса, её свойства. Процесс «белый шум».
\end{condition}

% \newpage
\subsection*{Вопрос 26}

\begin{condition}
	Преобразование стационарного процесса с дискретным спектром линейной стационарной динамической системой.
\end{condition}

\textbf{Стационарные процессы с дискретным спектром}

Если спектральная функция стационарного процесса $\xi_t$ кусочно-постоянна,
то $\xi_t$ называется стационарным процессом с дискретным спектром.

Спектральное разложение корреляционной функции стационарного процесса
с дискретным спектром имеет вид
$
	K_\xi(\tau)=\sum_{k=0}^{\infty} D_k^*(\xi)\, e^{i\omega_k \tau}.
$

\medskip

\textbf{Преобразования стационарных процессов}

Стационарный процесс $\eta_t$ является результатом преобразования
стационарного процесса $\xi_t$ стационарной линейной динамической системой,
если
$
	\sum_{k=0}^{n} a_k \frac{d^k \xi_t}{dt^k}
	=
	\sum_{k=0}^{m} b_k \frac{d^k \eta_t}{dt^k},
$
где $a_k$, $b_k$ — постоянные коэффициенты.

При этом $\xi_t$ называется входным процессом,
а $\eta_t$ — выходным процессом.

Для математических ожиданий процессов получаем
$
	b_0 m_\eta = a_0 m_\xi,
	\qquad
	m_\eta = \frac{a_0}{b_0} m_\xi .
$

Обозначения:
$
	A(p)=\sum_{k=0}^{n} a_k p^k,
	\qquad
	B(p)=\sum_{k=0}^{m} b_k p^k .
$

Многочлены $A(p)$ и $B(p)$ являются характеристическими
для дифференциальных операторов в левой и правой частях
уравнения динамической системы.

\medskip

\textbf{Преобразование дискретного спектра}

Если стационарный процесс $\xi_t$ имеет дискретный спектр
с корреляционной функцией
$
	K_\xi(\tau)=\sum_{k=-\infty}^{+\infty} D_k^*(\xi)\, e^{i\omega_k \tau},
$
то
$
	D_k^*(\eta)=\lvert T(i\omega_k)\rvert^2 D_k^*(\xi).
$

Корреляционная функция выходного стационарного процесса $\eta_t$
имеет вид
$
	K_\eta(\tau)
	=\sum_{k=-\infty}^{+\infty}
	\lvert T(i\omega_k)\rvert^2
	D_k^*(\xi)\, e^{i\omega_k \tau}.
$

Следовательно, процесс $\eta_t$ также имеет дискретный спектр.

% \newpage
\subsection*{Вопрос 27}

\begin{condition}
	Преобразование стационарного процесса с непрерывным спектром линейной стационарной динамической системой.
\end{condition}

\textbf{Стационарные процессы с непрерывным спектром}

Стационарный в широком смысле процесс $\xi_t$ называется процессом
с непрерывным спектром, если его спектральная функция $F_\xi(\omega)$
дифференцируема:
$
	F'_\xi(\omega)=S_\xi^*(\omega).
$

В этом случае
$
	K_\xi(\tau)=\int_{-\infty}^{+\infty} e^{i\omega\tau}\, S_\xi^*(\omega)\, d\omega.
$

Функция $S_\xi^*(\omega)$ является неотрицательной и называется
спектральной плотностью стационарного процесса.

\medskip

\textbf{Преобразования стационарных процессов}

Стационарный процесс $\eta_t$ является результатом преобразования
стационарного процесса $\xi_t$
стационарной линейной динамической системой, если
$
	\sum_{k=0}^{n} a_k \frac{d^k \xi_t}{dt^k}
	=
	\sum_{k=0}^{m} b_k \frac{d^k \eta_t}{dt^k},
$
где $a_k$, $b_k$ — постоянные коэффициенты.

При этом $\xi_t$ называется входным процессом,
а $\eta_t$ — выходным процессом.

Для математических ожиданий процессов получаем
$
	b_0 m_\eta = a_0 m_\xi,
	\qquad
	m_\eta=\frac{a_0}{b_0} m_\xi .
$

Обозначения:
$
	A(p)=\sum_{k=0}^{n} a_k p^k,
	\qquad
	B(p)=\sum_{k=0}^{m} b_k p^k .
$

Многочлены $A(p)$ и $B(p)$ являются характеристическими
для дифференциальных операторов
в левой и правой частях уравнения динамической системы.

\medskip

\textbf{Преобразование непрерывного спектра}

Если входной процесс $\xi_t$ имеет непрерывный спектр
и $S_\xi^*(\omega)$ — его спектральная плотность,
то $\eta_t$ также является стационарным процессом
с непрерывным спектром и спектральной плотностью
$
	S_\eta^*(\omega)=\lvert T(i\omega)\rvert^2 S_\xi^*(\omega).
$

% \newpage
\subsection*{Вопрос 28}

\begin{condition}
Преобразование случайного сигнала четырехполюсником.
\end{condition}

% \newpage
\subsection*{Вопрос 29}

\begin{condition}
Винеровский процесс, его свойства.
\end{condition}

% \newpage
\subsection*{Вопрос 30}

\begin{condition}
Стохастический интеграл Ито, его свойства.
\end{condition}


\end{document}



\newpage
\documentclass[a4paper,12pt]{article}

% Пакеты
\usepackage[utf8]{inputenc}
\usepackage[T2A]{fontenc}
\usepackage[russian]{babel}
\usepackage{graphicx}

\usepackage{amsmath,amssymb,amsthm}
\usepackage{multirow}
\usepackage{diagbox}
\usepackage{tabularx}
\usepackage{hyperref}
\usepackage{tikz} 
\usetikzlibrary{arrows.meta,positioning}
\tikzset{lab/.style={font=\footnotesize, inner sep=1pt}}

\usepackage[a4paper,top=2.54cm,bottom=2.54cm,left=2.5cm,right=1.5cm]{geometry}

\newtheorem{theorem}{Теорема}[section]
\newtheorem{definition}{Определение}[section]
\newtheorem{example}{Пример}[section]

\newcommand{\term}[1]{\textit{#1}\textnormal{}} 

\usepackage{epigraph}

\usepackage[none]{hyphenat}
\pretolerance=10000
\tolerance=2000
\emergencystretch=3em
\hyphenpenalty=10000
\exhyphenpenalty=10000


\setlength{\epigraphwidth}{0.6\textwidth}
\setlength{\epigraphrule}{0pt}          
\renewcommand{\epigraphflush}{flushleft}

%-----
\newcounter{variant}    % номер prac-3-N
\newcounter{problem}[variant]

\renewcommand{\theproblem}{\thevariant.\arabic{problem}}

\newenvironment{problem}{
    \refstepcounter{problem}
    \par\medskip
    \textbf{Задача №\theproblem.}
    \par\medskip
}{
    \par\medskip
}

\newenvironment{condition}{\par\textbf{Условие.}}{\par}
\newenvironment{theory}{\par\textbf{Теория.}}{\par}
\newenvironment{solution}{\par\textbf{Решение.}}{\par}
\newenvironment{conclusion}{\par\textbf{Вывод.}}{\par}
%-----


% Заголовок
\title{Задачи к экзамену. Лобузов}
\date{2025}

\begin{document}

\maketitle

\tableofcontents
\newpage


\setcounter{variant}{1}
\setcounter{problem}{0}
\section{Интенсивности марковского процесса}

\begin{problem}

\begin{condition}
	(Nun)

	Даны интенсивности марковского процесса с тремя состояниями:
	$\lambda_{12}=\lambda_{21}=\lambda{32}=0$,
	$\lambda_{13}=1$, $\lambda_{23}=2$, $\lambda_{31}=3$.
	Найти стационарное распределение вероятностей.
\end{condition}

\end{problem}



\begin{problem}

\begin{condition}
	(16.1)

	Даны интенсивности марковского процесса с тремя состояниями:
	$\lambda_{12}=\lambda_{23}=\lambda_{31}=2$,
	$\lambda_{13}=1$
	Найти стационарное распределение вероятностей.
\end{condition}

\end{problem}



\begin{problem}

\begin{condition}
	(15.1)

	Даны интенсивности марковского процесса с тремя состояниями:
	$\lambda_{23}=\lambda_{32}=2$,
	$\lambda_{13}=\lambda_{21}=3$
	Найти стационарное распределение вероятностей.
\end{condition}

\end{problem}



\begin{problem}

\begin{condition}
	(31.1)

	Даны интенсивности марковского процесса с тремя состояниями:
	$\lambda_{31}=\lambda_{32}=1$,
	$\lambda_{13}=\lambda_{23}=3$,
	$\lambda_{21}=\lambda_{12}=2$.
	Найти стационарное распределение вероятностей.
\end{condition}

\end{problem}



\begin{problem}

\begin{condition}
	(11.1)

	Даны интенсивности марковского процесса с тремя состояниями:
	$\lambda_{21}=\lambda_{23}=1$,
	$\lambda_{13}=3$, $\lambda_{32}=2$
	Найти стационарное распределение вероятностей.
\end{condition}

\end{problem}



\begin{problem}

\begin{condition}
	(12.1)

	Даны интенсивности марковского процесса с тремя состояниями:
	$\lambda_{13}=8$, $\lambda_{21}=1$, $\lambda_{23}=2$,
	$\lambda_{31}=5$, $\lambda_{32}=3$.
	Найти стационарное распределение вероятностей.
\end{condition}

\end{problem}



\begin{problem}

\begin{condition}
	(33.1)

	Даны интенсивности марковского процесса с тремя состояниями:
	$\lambda_{12}=\lambda_{31}=1$, $\lambda_{13}=5$,
	$\lambda_{23}=3$, $\lambda_{21}=\lambda_{32}=2$./
	Найти стационарное распределение вероятностей.
\end{condition}

\end{problem}
\newpage

\setcounter{variant}{2}
\setcounter{problem}{0}
\section{Цепи Маркова}

\subsection*{Общая теория}

Стол прозрачный, граф круглый.

\subsection*{Задачи}

\begin{problem}

\begin{condition}
	(Nun)

	В цепи Маркова дана матрица $P$ вероятностей перехода за один шаг:
	$p_{13}=0$, $p_{12}=p_{23}=p_{33}=0.25$,
	$p_{22}=0.625$, $p_{31}=0.5$
	и начальное распределение вероятностей состояний $\vec{p}(0)=(0;1;0)$.
	Построить граф цепи Маркова, найти $\vec{p}(2)$ и стационарное распределение.
\end{condition}

\begin{solution}

	$$
		P =
		\begin{pmatrix}
			\frac{3}{4} & \frac{1}{4} & 0           \\
			\frac{1}{8} & \frac{5}{8} & \frac{1}{4} \\
			\frac{1}{2} & \frac{1}{4} & \frac{1}{4}
		\end{pmatrix}.
	$$

	$$
		\vec{p}(2)=\vec{p}(0)P^2
		=(0;1;0)P^2.
	$$

	$$
		\vec{p}(1)=\vec{p}(0)P
		=(0;1;0)
		\begin{pmatrix}
			\frac{3}{4} & \frac{1}{4} & 0           \\
			\frac{1}{8} & \frac{5}{8} & \frac{1}{4} \\
			\frac{1}{2} & \frac{1}{4} & \frac{1}{4}
		\end{pmatrix}
		=\left(\frac{1}{8};\frac{5}{8};\frac{1}{4}\right).
	$$

	$$
		\vec{p}(2)=\vec{p}(1)P
		=\left(\frac{1}{8};\frac{5}{8};\frac{1}{4}\right)
		\begin{pmatrix}
			\frac{3}{4} & \frac{1}{4} & 0           \\
			\frac{1}{8} & \frac{5}{8} & \frac{1}{4} \\
			\frac{1}{2} & \frac{1}{4} & \frac{1}{4}
		\end{pmatrix}
		=\left(\frac{19}{64};\frac{31}{64};\frac{7}{32}\right).
	$$

	$$
		\bar r=(r_1,r_2,r_3), \qquad \bar r P=\bar r, \qquad r_1+r_2+r_3=1.
	$$

	$$
		\begin{cases}
			r_1=\frac{3}{4}r_1+\frac{1}{8}r_2+\frac{1}{2}r_3, \\
			r_2=\frac{1}{4}r_1+\frac{5}{8}r_2+\frac{1}{4}r_3, \\
			r_3=\frac{1}{4}r_2+\frac{1}{4}r_3,                \\
			r_1+r_2+r_3=1.
		\end{cases}
		\Rightarrow
		\begin{cases}
			8r_1 = 6r_1 + r_2 + 4r_3,  \\
			8r_2 = 2r_1 + 5r_2 + 2r_3, \\
			4r_3 = r_2 + r_3,          \\
			r_1 + r_2 + r_3 = 1.
		\end{cases}
	$$

	$$
		\begin{cases}
			2r_1 =  r_2 + 4r_3, \\
			3r_2 = 2r_1 + 2r_3, \\
			3r_3 = r_2,         \\
			r_1 + r_2 + r_3 = 1.
		\end{cases}
		\Rightarrow
		\begin{cases}
			2r_1 =  7r_3, \\
			7r_3 = 2r_1,  \\
			3r_3 = r_2,   \\
			r_1 + r_2 + r_3 = 1.
		\end{cases}
	$$

	$$
		r_3(\frac{7}{2}+3+1)=1
		\Rightarrow
		r_3=\frac{2}{15}
	$$


	$$
		\bar r=\left(\frac{7}{15};\frac{2}{5};\frac{2}{15}\right).
	$$

\end{solution}

\begin{conclusion}

	$$
		\vec{p}(2)=\left(\frac{19}{64};\frac{31}{64};\frac{7}{32}\right),
		\bar r=\left(\frac{7}{15};\frac{2}{5};\frac{2}{15}\right).
	$$
\end{conclusion}

\end{problem}



\begin{problem}

\begin{condition}
	(16.1)

	В цепи Маркова дана матрица $P$ вероятностей перехода за один шаг:
	$p_{13}=p_{22}=p_{31}=0$,
	$p_{12}=0.8$, $p_{21}=0.5$, $p_{32}=0.6$
	и начальное распределение вероятностей состояний $\vec{p}(0)=(0;1;0)$.
	Построить граф цепи Маркова, найти $\vec{p}(2)$ и стационарное распределение.
\end{condition}

\begin{solution}

	$$
		P =
		\begin{pmatrix}
			\frac{1}{5} & \frac{4}{5} & 0           \\
			\frac{1}{2} & 0           & \frac{1}{2} \\
			0           & \frac{3}{5} & \frac{2}{5}
		\end{pmatrix}.
	$$

	$$
		\vec{p}(2)=\vec{p}(0)P^2
		=(0;1;0)P^2.
	$$

	$$
		\vec{p}(1)=\vec{p}(0)P
		=(0;1;0)
		\begin{pmatrix}
			\frac{1}{5} & \frac{4}{5} & 0           \\
			\frac{1}{2} & 0           & \frac{1}{2} \\
			0           & \frac{3}{5} & \frac{2}{5}
		\end{pmatrix}
		=\left(\frac{1}{2};0;\frac{1}{2}\right).
	$$

	$$
		\vec{p}(2)=\vec{p}(1)P
		=\left(\frac{1}{2};0;\frac{1}{2}\right)
		\begin{pmatrix}
			\frac{1}{5} & \frac{4}{5} & 0           \\
			\frac{1}{2} & 0           & \frac{1}{2} \\
			0           & \frac{3}{5} & \frac{2}{5}
		\end{pmatrix}
		=\left(\frac{1}{10};\frac{7}{10};\frac{1}{5}\right).
	$$

	$$
		\bar r=(r_1,r_2,r_3), \qquad \bar r P=\bar r, \qquad r_1+r_2+r_3=1.
	$$

	$$
		\begin{cases}
			r_1=\frac{1}{5}r_1+\frac{1}{2}r_2, \\
			r_2=\frac{4}{5}r_1+\frac{3}{5}r_3, \\
			r_3=\frac{1}{2}r_2+\frac{2}{5}r_3, \\
			r_1+r_2+r_3=1.
		\end{cases}
		\Rightarrow
		\begin{cases}
			10r_1=2r_1+5r_2, \\
			5r_2=4r_1+3r_3,  \\
			10r_3=5r_2+4r_3, \\
			r_1+r_2+r_3=1.
		\end{cases}
	$$

	$$
		\begin{cases}
			8r_1=5r_2,      \\
			5r_2=4r_1+3r_3, \\
			6r_3=5r_2,      \\
			r_1+r_2+r_3=1.
		\end{cases}
		\Rightarrow
		\begin{cases}
			r_1=\frac{3}{4}r_3, \\
			r_2=\frac{6}{5}r_3, \\
			r_1+r_2+r_3=1.
		\end{cases}
	$$

	$$
		r_3\left(\frac{3}{4}+\frac{6}{5}+1\right)=1
		\Rightarrow
		r_3=\frac{20}{59}.
	$$

	$$
		\bar r=\left(\frac{15}{59};\frac{24}{59};\frac{20}{59}\right).
	$$

\end{solution}

\begin{conclusion}

	$$
		\vec{p}(2)=\left(\frac{1}{10};\frac{7}{10};\frac{1}{5}\right),
		\bar r=\left(\frac{15}{59};\frac{24}{59};\frac{20}{59}\right).
	$$

\end{conclusion}

\end{problem}



\begin{problem}

\begin{condition}
	(15.1)

	В цепи Маркова дана матрица $P$ вероятностей перехода за один шаг:
	$p_{11}=p_{22}=p_{33}=0$,
	$p_{12}=p_{31}=0.75$, $p_{23}=0.5$
	и начальное распределение вероятностей состояний $\vec{p}(0)=(0;0;1)$.
	Построить граф цепи Маркова, найти $\vec{p}(2)$ и стационарное распределение.
\end{condition}

\begin{solution}

	$$
		P =
		\begin{pmatrix}
			0           & \frac{3}{4} & \frac{1}{4} \\
			\frac{1}{2} & 0           & \frac{1}{2} \\
			\frac{3}{4} & \frac{1}{4} & 0
		\end{pmatrix}.
	$$

	$$
		\vec{p}(2)=\vec{p}(0)P^2
		=(0;0;1)P^2.
	$$

	$$
		\vec{p}(1)=\vec{p}(0)P
		=(0;0;1)
		\begin{pmatrix}
			0           & \frac{3}{4} & \frac{1}{4} \\
			\frac{1}{2} & 0           & \frac{1}{2} \\
			\frac{3}{4} & \frac{1}{4} & 0
		\end{pmatrix}
		=\left(\frac{3}{4};\frac{1}{4};0\right).
	$$

	$$
		\vec{p}(2)=\vec{p}(1)P
		=\left(\frac{3}{4};\frac{1}{4};0\right)
		\begin{pmatrix}
			0           & \frac{3}{4} & \frac{1}{4} \\
			\frac{1}{2} & 0           & \frac{1}{2} \\
			\frac{3}{4} & \frac{1}{4} & 0
		\end{pmatrix}
		=\left(\frac{1}{8};\frac{9}{16};\frac{5}{16}\right).
	$$

	$$
		\bar r=(r_1,r_2,r_3), \qquad \bar r P=\bar r, \qquad r_1+r_2+r_3=1.
	$$

	$$
		\begin{cases}
			r_1=\frac{1}{2}r_2+\frac{3}{4}r_3, \\
			r_2=\frac{3}{4}r_1+\frac{1}{4}r_3, \\
			r_3=\frac{1}{4}r_1+\frac{1}{2}r_2, \\
			r_1+r_2+r_3=1.
		\end{cases}
		\Rightarrow
		\begin{cases}
			4r_1=2r_2+3r_3, \\
			4r_2=3r_1+r_3,  \\
			4r_3=r_1+2r_2,  \\
			r_1+r_2+r_3=1.
		\end{cases}
	$$

	$$
		\begin{cases}
			r_2=\frac{13}{14}r_1, \\
			r_3=\frac{5}{7}r_1,   \\
			r_1+r_2+r_3=1.
		\end{cases}
	$$

	$$
		r_1\left(1+\frac{13}{14}+\frac{5}{7}\right)=1
		\Rightarrow
		r_1=\frac{14}{37}.
	$$

	$$
		\bar r=\left(\frac{14}{37};\frac{13}{37};\frac{10}{37}\right).
	$$

\end{solution}

\begin{conclusion}

	$$
		\vec{p}(2)=\left(\frac{1}{8};\frac{9}{16};\frac{5}{16}\right),
		\bar r=\left(\frac{14}{37};\frac{13}{37};\frac{10}{37}\right).
	$$

\end{conclusion}

\end{problem}



\begin{problem}

\begin{condition}
	(31.1)

	В цепи Маркова дана матрица $P$ вероятностей перехода за один шаг:
	$p_{32}=p_{21}=p_{12}=0$,
	$p_{22}=0.2$, $p_{11}=p_{33}=0.5$
	и начальное распределение вероятностей состояний $\vec{p}(0)=(1;0;0)$.
	Построить граф цепи Маркова, найти $\vec{p}(2)$ и стационарное распределение.
\end{condition}

\begin{solution}

	$$
		P =
		\begin{pmatrix}
			\frac{1}{2} & 0           & \frac{1}{2} \\
			0           & \frac{1}{5} & \frac{4}{5} \\
			\frac{1}{2} & 0           & \frac{1}{2}
		\end{pmatrix}.
	$$

	$$
		\vec{p}(2)=\vec{p}(0)P^2
		=(1;0;0)P^2.
	$$

	$$
		\vec{p}(1)=\vec{p}(0)P
		=(1;0;0)
		\begin{pmatrix}
			\frac{1}{2} & 0           & \frac{1}{2} \\
			0           & \frac{1}{5} & \frac{4}{5} \\
			\frac{1}{2} & 0           & \frac{1}{2}
		\end{pmatrix}
		=\left(\frac{1}{2};0;\frac{1}{2}\right).
	$$

	$$
		\vec{p}(2)=\vec{p}(1)P
		=\left(\frac{1}{2};0;\frac{1}{2}\right)
		\begin{pmatrix}
			\frac{1}{2} & 0           & \frac{1}{2} \\
			0           & \frac{1}{5} & \frac{4}{5} \\
			\frac{1}{2} & 0           & \frac{1}{2}
		\end{pmatrix}
		=\left(\frac{1}{2};0;\frac{1}{2}\right).
	$$

	$$
		\bar r=(r_1,r_2,r_3), \qquad \bar r P=\bar r, \qquad r_1+r_2+r_3=1.
	$$

	$$
		\begin{cases}
			r_1=\frac{1}{2}r_1+\frac{1}{2}r_3, \\
			r_2=\frac{1}{5}r_2,                \\
			r_3=\frac{1}{2}r_1+\frac{1}{2}r_3, \\
			r_1+r_2+r_3=1.
		\end{cases}
		\Rightarrow
		\begin{cases}
			2r_1=r_1+r_3, \\
			5r_2=r_2,     \\
			2r_3=r_1+r_3, \\
			r_1+r_2+r_3=1.
		\end{cases}
	$$

	$$
		\begin{cases}
			r_1=r_3, \\
			r_2=0,   \\
			r_1+r_3=1.
		\end{cases}
	$$

	$$
		r_1=r_3=\frac{1}{2}.
	$$

	$$
		\bar r=\left(\frac{1}{2};0;\frac{1}{2}\right).
	$$

\end{solution}

\begin{conclusion}

	$$
		\vec{p}(2)=\left(\frac{1}{2};0;\frac{1}{2}\right),
		\bar r=\left(\frac{1}{2};0;\frac{1}{2}\right).
	$$

\end{conclusion}

\end{problem}



\begin{problem}

\begin{condition}
	(11.1)

	В цепи Маркова дана матрица $P$ вероятностей перехода за один шаг:
	$p_{11}=0.25$,$p_{12}=p_{33}=p_{21}=0$,
	$p_{22}=0.4$, $p_{31}=0.5$
	и начальное распределение вероятностей состояний $\vec{p}(0)=(0;1;0)$.
	Построить граф цепи Маркова, найти $\vec{p}(2)$ и стационарное распределение.
\end{condition}

\begin{solution}

\end{solution}

\begin{conclusion}

\end{conclusion}

\end{problem}



\begin{problem}

\begin{condition}
	(12.1)

	В цепи Маркова дана матрица $P$ вероятностей перехода за один шаг:
	$p_{21}=0.75$,$p_{12}=p_{33}=p_{21}=0$,
	$p_{33}=0.375$, $p_{21}=0.5$
	и начальное распределение вероятностей состояний $\vec{p}(0)=(0;0;1)$.
	Построить граф цепи Маркова, найти $\vec{p}(2)$ и стационарное распределение.
\end{condition}

\begin{solution}

\end{solution}

\begin{conclusion}

\end{conclusion}

\end{problem}



\begin{problem}

\begin{condition}
	(33.1)

	В цепи Маркова дана матрица $P$ вероятностей перехода за один шаг:
	$p_{12}=p_{23}=p_{32}=0$,
	$p_{21}=0.8$, $p_{11}=p_{33}=0.5$
	и начальное распределение вероятностей состояний $\vec{p}(0)=(0;0;1)$.
	Построить граф цепи Маркова, найти $\vec{p}(2)$ и стационарное распределение.
\end{condition}

\begin{solution}

\end{solution}

\begin{conclusion}

\end{conclusion}

\end{problem}

\newpage

\setcounter{variant}{3}
\setcounter{problem}{0}
\section*{Кореляционная функция}

\subsubsection*{Формулы}

Корреляционная функция $K_x(t,s)$

$$
	Y_t=L_t[X_t]
$$

$$
	K_Y(t,s)=L_t[\overline{L}_s[K_x(t,s)]]=\overline{L}_s[L_t[K_x(t,s)]]
$$

\begin{problem}

\begin{condition}
	(Nun)

	Случайный процесс $X_t$ имеет корреляционную функцию
	$K_X(t,s)=e^{2i(t-s)}$.
	Найти корреляционную функцию случайного процесса
	$Y_t=\int_{0}^{t} i X_\tau \, d\tau$
\end{condition}

\begin{solution}

	$K_x(t,s)=e^{2i(t-s)}=e^{2it-2is}$, $Y_t = \int_{0}^{t} i \tau X_\tau d\tau$

	$$
		L_t[\cdot] = \int_{0}^{t} i [\cdot] d\tau
	$$

	$$
		\overline{L}_s[\cdot] = \int_{0}^{s} -i [\cdot] d\sigma
	$$

	$$
		K_x(t,s)=K_x(\tau,\sigma)=e^{2i\tau-2i\sigma}
	$$

	$$
		\begin{aligned}
			L_t[K_x(\tau,\sigma)] & = \int_{0}^{t} i\,K_x(\tau,\sigma)\,d\tau
			= i e^{-2i\sigma}\int_{0}^{t} e^{2i\tau}\,d\tau
			\\ & = i e^{-2i\sigma}
			\left[
				\frac{1}{2i} e^{2i\tau}
				\right]_0^t
			= i e^{-2i\sigma} \frac{1}{2i} (e^{2it}-1)
			\\ & = e^{-2i\sigma} \frac{1}{2} (e^{2it}-1)
		\end{aligned}
	$$

	$$
		\begin{aligned}
			\overline{L}_s[L_t[K_x(\tau,\sigma)]] & =
			\int_{0}^{s} -i \, L_t[K_x(\tau,\sigma)] \, d\sigma
			= \int_{0}^{s} -i \, i e^{-2i\sigma} \frac{1}{2} (e^{2it}-1) \, d\sigma
			\\ & = \frac{1}{2} (e^{2it}-1) \int_{0}^{s} e^{-2i\sigma}  \, d\sigma
			= \frac{1}{2} (e^{2it}-1)
			\left[
				\frac{1}{-2i} e^{-2i\sigma}
				\right]_0^s
			\\ & = \frac{1}{2} (e^{2it}-1) \frac{1}{-2i} (e^{-2is} -1)
			= \frac{1}{4} (e^{2it}-1) (e^{-2is} -1)
			\\ & = \frac{1}{4} (e^{2it-2is}-e^{2it}-e^{-2is}+1)
		\end{aligned}
	$$

\end{solution}

\begin{conclusion}

	$$
		K_Y(t,s) = \frac{1}{4} (e^{2it-2is}-e^{2it}-e^{-2is}+1)
	$$
\end{conclusion}

\end{problem}



\begin{problem}

\begin{condition}
	(16.1)

	Случайный процесс $X(t)$ имеет корреляционную функцию
	$K_X(t,s)=2e^{i(t-s)}$.
	Найти корреляционную функцию случайного процесса
	$Y(t)=3it \frac{dX(t)}{dt} + e^{it} X(t)$
\end{condition}

\begin{solution}

	$K_X(t,s)=2e^{i(t-s)}=2e^{it-is}$,
	$Y_t=3it \frac{dX_t}{dt} + e^{it} X_t$


	$$
		L_t[\cdot] = 3it\frac{\partial}{\partial t}[\cdot] + e^{it}[\cdot]
	$$

	$$
		\overline{L}_s[\cdot] = -3is\frac{\partial}{\partial s}[\cdot] + e^{-is}[\cdot]
	$$

	$$
		\begin{aligned}
			\overline{L}_s[K_x(t,s)] & =-3is\frac{\partial}{\partial s}K_x(t,s) + e^{-is}K_x(t,s)
			= -3is\frac{\partial}{\partial s}2e^{it-is} + e^{-is}2e^{it-is}
			\\ & = -3is\cdot -2i e^{it-is} + 2 e^{it-2is}
			= -6s e^{it-is} + 2 e^{it-2is}
		\end{aligned}
	$$

	$$
		\begin{aligned}
			L_t[\overline{L}_s[K_x(t,s)]] & = 3it\frac{\partial}{\partial t}\overline{L}_s[K_x(t,s)] + e^{it}\overline{L}_s[K_x(t,s)]
			\\ & = 3it\frac{\partial}{\partial t}(-6s e^{it-is} + 2 e^{it-2is}) + e^{it}(-6s e^{it-is} + 2 e^{it-2is})
			\\ & =3it(-6si e^{it-is}+2ie^{it-2is}) - 6s e^{2it-is} + 2 e^{2it-2is}
			\\ & =18ts e^{it-is} - 6te^{it-2is} - 6s e^{2it-is} + 2 e^{2it-2is}
		\end{aligned}
	$$

\end{solution}

\begin{conclusion}
	$$
		K_Y(t,s) =18ts e^{it-is} - 6te^{it-2is} - 6s e^{2it-is} + 2 e^{2it-2is}
	$$
\end{conclusion}

\end{problem}


\begin{problem}

\begin{condition}
	(15.1)

	Случайный процесс $X(t)$ имеет корреляционную функцию
	$K_X(t,s)=3\cdot \cos(2t) \cdot \cos(2s)$.
	Найти корреляционную функцию случайного процесса
	$Y(t)=e^{it} \frac{dX(t)}{dt} - 3it X(t)$
\end{condition}

\begin{solution}

\end{solution}

\begin{conclusion}

\end{conclusion}

\end{problem}



\begin{problem}

\begin{condition}
	(31.1)

	Случайный процесс $X(t)$ имеет корреляционную функцию
	$K_X(t,s)=\cos(5(t-s))$.
	Найти корреляционную функцию случайного процесса
	$Y(t)=10\frac{dX(t)}{dt}$.
\end{condition}

\begin{solution}

\end{solution}

\begin{conclusion}

\end{conclusion}

\end{problem}



\begin{problem}

\begin{condition}
	(11.1)

	Слуйчайный процесс $X(t)$ имеет корреляционную функцию
	$K_X(t,s)=5e^{i(t-s)}$.
	Найти корреляционную функцию случайного процесса
	$Y(t)=e^{-2it} \frac{dX(t)}{dt} +it X(t)$
\end{condition}

\begin{solution}

\end{solution}

\begin{conclusion}

\end{conclusion}

\end{problem}



\begin{problem}

\begin{condition}
	(12.1)

	Слуйчайный процесс $X(t)$ имеет корреляционную функцию
	$K_X(t,s)=5 \cdot \sin(3t) \cdot \sin(3s)$.
	Найти корреляционную функцию случайного процесса
	$Y(t)=\frac{1}{10} e^{-2it} \frac{dX(t)}{dt}$
\end{condition}

\begin{solution}

\end{solution}

\begin{conclusion}

\end{conclusion}

\end{problem}



\begin{problem}

\begin{condition}
	(33.1)

	Слуйчайный процесс $X(t)$ имеет корреляционную функцию
	$K_X(t,s)=7 e^{3i(t-s)}$.
	Найти корреляционную функцию случайного процесса
	$Y(t)=3 \frac{dX(t)}{dt}$
\end{condition}

\begin{solution}

\end{solution}

\begin{conclusion}

\end{conclusion}

\end{problem}
\newpage

\setcounter{variant}{4}
\setcounter{problem}{0}



\begin{problem}

\begin{condition}
	Найти спектральное разложение корреляционной функции стационарного случайного прочесса


\end{condition}

\end{problem}



\begin{problem}

\begin{condition}


\end{condition}

\end{problem}



\begin{problem}

\begin{condition}


\end{condition}

\end{problem}



\begin{problem}

\begin{condition}


\end{condition}

\end{problem}



\begin{problem}

\begin{condition}


\end{condition}

\end{problem}



\begin{problem}

\begin{condition}


\end{condition}

\end{problem}



\begin{problem}

\begin{condition}


\end{condition}

\end{problem}


\end{document}



\newpage
\section{Консультация по экзамену}

Приходить нужно во время, к своему времени.

Без верхней одежды.

Телефоны в рюкзаках

3 листа бумаги формата А4 (не из тетрадей) и 2 ручики.



Он в том году не сказал им результаты предэкзамена, который был в декабре.

И на консультации, которая была 16 числа, он тоже отказался говорить им результаты.
Он им рассказал только на экзамене

\epigraph{
	Ну я сразу скажу, что многие отнеслись к этим досрочным мероприятиям не очень ответственно, то есть не посмотрели материалы семинаров, лекций.
	И делали грубые ошибки, которые свидетельствуют о том, что люди не знают какие формулы нужно использовать для решения задач.
	Хотя многие формулы были в теоретических сведениях, которые вы вставляли в свои типовые расчёты.
	Нужно было просто понять все эти вещи и применить.
} {Лобузов}


Случайные процессы - это семейства случайных велечин.

\end{document}
