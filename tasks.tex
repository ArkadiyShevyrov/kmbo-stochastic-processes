\documentclass[a4paper,12pt]{article}

% Пакеты
\usepackage[utf8]{inputenc}
\usepackage[T2A]{fontenc}
\usepackage[russian]{babel}
\usepackage{graphicx}

\usepackage{amsmath,amssymb,amsthm}
\usepackage{multirow}
\usepackage{diagbox}
\usepackage{tabularx}
\usepackage{hyperref}
\usepackage{tikz} 
\usetikzlibrary{arrows.meta,positioning}
\tikzset{lab/.style={font=\footnotesize, inner sep=1pt}}

\usepackage[a4paper,top=2.54cm,bottom=2.54cm,left=2.5cm,right=1.5cm]{geometry}

\newtheorem{theorem}{Теорема}[section]
\newtheorem{definition}{Определение}[section]
\newtheorem{example}{Пример}[section]

\newcommand{\term}[1]{\textit{#1}\textnormal{}} 

\usepackage{epigraph}

\usepackage[none]{hyphenat}
\pretolerance=10000
\tolerance=2000
\emergencystretch=3em
\hyphenpenalty=10000
\exhyphenpenalty=10000


\setlength{\epigraphwidth}{0.6\textwidth}
\setlength{\epigraphrule}{0pt}          
\renewcommand{\epigraphflush}{flushleft}

% Заголовок
\title{Задачи Лобузова}
\date{2025}

\begin{document}

\maketitle

\tableofcontents
\newpage

\subsection*{Спектральное разложение}

\subsubsection*{Общее решение для чётных функций}.

Спектральное разложение корреляционной функции $K(\tau)$, заданной на $[-T,T]$.

Разложение ищем в виде:
$$
	K(\tau)=D_0+\sum_{k=1}^{\infty} D_k \cos\left(\frac{k \pi \tau}{T}\right)
$$

Где:

$$
	D_0=\frac{1}{T} \left[\int_{0}^{T} K(\tau) d\tau\right]
$$

$$
	D_k=\frac{2}{T} \left[\int_{0}^{T} K(\tau)\cos(\frac{k \pi \tau}{T}) d\tau\right]
$$

Спектральное разложение корреляционной функции $K(\tau)$, $\tau \in \mathbb{R}$.

Разложение ищем в виде:
$$
	K(\tau)=\frac{1}{2\pi}\int_{-\infty}^{\infty} S(\omega)e^{i\omega \tau} \, d\omega
$$

Где:

$$
	S(\omega)=2\int_{0}^{\infty} K(\tau) \cos(\omega\tau) \, d\tau
$$

\subsubsection*{Задача 9.1}
$K(\tau)=1-|\tau|$, $T=1$

На $\left[0,1\right]$: $K(\tau)=1-\tau$.

$$
	D_0=\frac{1}{1} \left[\int_{0}^{1} (1-\tau) d\tau\right]=\left[\tau-\frac{\tau^2}{2}\right]_0^1=\frac{1}{2}
$$

$$
	D_k=\frac{2}{1} \left[\int_{0}^{1} (1-\tau)\cos(\frac{k \pi \tau}{1}) d\tau\right]=
	2\left(\left[\int_{0}^{1}\cos(k \pi \tau) d\tau\right]-\left[\int_{0}^{1} \tau\cos(k \pi \tau) d\tau\right]\right)
$$

$$
	I_1=\int_{0}^{1}\cos(k \pi \tau) d\tau=\left[\frac{sin(k \pi \tau)}{k\pi}\right]_0^1=0
$$

$$
	\begin{aligned}
		I_2 & = \int_{0}^{1} \tau \cos(k \pi \tau) \, d\tau
		= \left[ \tau \frac{\sin(k \pi \tau)}{k \pi} \right]_0^1 - \int_{0}^{1} \frac{\sin(k \pi \tau)}{k \pi} \, d\tau
		= 0 - \frac{1}{(k\pi)^2}[\cos(k \pi \tau)]_0^1
		\\ & = -\frac{1}{(k\pi)^2}(-\cos(k \pi) + 1)
		= -\frac{1 - \cos(k \pi)}{k^2 \pi^2}
		= -\frac{1-(-1)^k}{k^2\pi^2}
	\end{aligned}
$$

$$
	D_k =2(I_1-I_2)
	=2(0-(-\frac{1-(-1)^k}{k^2\pi^2}))
	=\frac{2(1-(-1)^k)}{k^2\pi^2}
	=
	\begin{cases}
		\frac{4}{k^2 \pi^2}, & k = 2m+1, \\
		0,                   & k = 2m.
	\end{cases}
$$

$$
	\begin{aligned}
		K(\tau) & =D_0+\sum_{k=1}^{\infty} D_k \cos\left(\frac{k \pi \tau}{1}\right)
		=\frac{1}{2} + \sum_{k=1}^{\infty} \frac{2(1-(-1)^k)}{k^2\pi^2} \cos(k \pi \tau)
		\\ & = \frac{1}{2} + \sum_{m=0}^{\infty} \frac{4}{(2m+1)^2\pi^2} \cos((2m+1) \pi \tau)
		= \frac{1}{2} + \frac{4}{\pi^2} \sum_{m=0}^{\infty} \frac{\cos((2m+1) \pi \tau)}{(2m+1)^2}
	\end{aligned}
$$

\subsubsection*{Задача 9.2}
$K(\tau)=1-\tau^2$, $T=1$

На $\left[0,1\right]$: $K(\tau)=1-\tau^2$.

$$
	D_0=\frac{1}{1} \left[\int_{0}^{1} (1-\tau^2) d\tau\right]=\left[\tau-\frac{\tau^3}{3}\right]_0^1=\frac{1}{3}
$$

$$
	D_k=\frac{2}{1} \left[\int_{0}^{1} (1-\tau^2)\cos(\frac{k \pi \tau}{1}) d\tau\right]=
	2\left(\left[\int_{0}^{1}\cos(k \pi \tau) d\tau\right]-\left[\int_{0}^{1} \tau^2\cos(k \pi \tau) d\tau\right]\right)
$$

$$
	I_1=\int_{0}^{1}\cos(k \pi \tau) d\tau=\left[\frac{sin(k \pi \tau)}{k\pi}\right]_0^1=0
$$

$$
	\begin{aligned}
		I_2 & = \int_{0}^{1} \tau^2 \cos(k \pi \tau) \, d\tau
		= \left[ \tau^2 \frac{\sin(k \pi \tau)}{k \pi} \right]_0^1 - \int_{0}^{1} \frac{\sin(k \pi \tau)}{k \pi} 2\tau \, d\tau
		\\ &= 0 - \left(\left[(2\tau)\left(-\frac{\cos(k \pi \tau)}{(k \pi)^2}\right)\right]_0^1 - \int_{0}^{1} 2 \frac{\cos(k \pi \tau)}{(k \pi)^2} d\tau \right)
		\\ &= \frac{2 \cos(k \pi )}{(k \pi)^2} + \left(-\left[\frac{2 \sin(k \pi \tau)}{(k \pi)^3}\right]_0^1\right)
		= \frac{2 \cos(k \pi )}{(k \pi)^2} - 0
		= \frac{2 \cos(k \pi )}{(k \pi)^2}
		= \frac{2 (-1)^k}{(k \pi)^2}
	\end{aligned}
$$

$$
	D_k =2(I_1-I_2)
	=2(0-(-\frac{2 (-1)^k}{(k \pi)^2}))
	=-\frac{4 (-1)^k}{(k \pi)^2}
$$

$$
	\begin{aligned}
		K(\tau) & =D_0+\sum_{k=1}^{\infty} D_k \cos\left(\frac{k \pi \tau}{1}\right)
		=\frac{1}{2} + \sum_{k=1}^{\infty} \frac{2(1-(-1)^k)}{k^2\pi^2} \cos(k \pi \tau)
		\\ & = \frac{1}{2} + \sum_{m=0}^{\infty} \frac{4}{(2m+1)^2\pi^2} \cos((2m+1) \pi \tau)
		= \frac{1}{2} + \frac{4}{\pi^2} \sum_{m=0}^{\infty} \frac{\cos((2m+1) \pi \tau)}{(2m+1)^2}
	\end{aligned}
$$

\subsubsection*{Задача 9.3}
$K(\tau)=2-3|\tau|$, $T=1$

На $\left[0,1\right]$: $K(\tau)=2-3\tau$.

$$
	D_0=\frac{1}{1} \left[\int_{0}^{1} (2-3\tau) d\tau\right]=\left[2\tau-\frac{3\tau^2}{2}\right]_0^1=\frac{1}{2}
$$

$$
	D_k=\frac{2}{1} \left[\int_{0}^{1} (2-3\tau)\cos(\frac{k \pi \tau}{1}) d\tau\right]=
	2\left(2\left[\int_{0}^{1}\cos(k \pi \tau) d\tau\right]-3\left[\int_{0}^{1} \tau\cos(k \pi \tau) d\tau\right]\right)
$$

$$
	I_1=\int_{0}^{1}\cos(k \pi \tau) d\tau=\left[\frac{sin(k \pi \tau)}{k\pi}\right]_0^1=0
$$

$$
	\begin{aligned}
		I_2 & = \int_{0}^{1} \tau \cos(k \pi \tau) \, d\tau
		= \left[ \tau \frac{\sin(k \pi \tau)}{k \pi} \right]_0^1 - \int_{0}^{1} \frac{\sin(k \pi \tau)}{k \pi} \, d\tau
		= 0 - \frac{1}{(k\pi)^2}[\cos(k \pi \tau)]_0^1
		\\ & = -\frac{1}{(k\pi)^2}(-\cos(k \pi) + 1)
		= -\frac{1 - \cos(k \pi)}{k^2 \pi^2}
		= -\frac{1-(-1)^k}{k^2\pi^2}
	\end{aligned}
$$

$$
	D_k =2(2I_1-3I_2)
	=2\left(2\cdot0-3\left(-\frac{1-(-1)^k}{k^2\pi^2}\right)\right)
	=\frac{6(1-(-1)^k)}{k^2\pi^2}
	=
	\begin{cases}
		\frac{12}{k^2 \pi^2}, & k = 2m+1, \\
		0,                    & k = 2m.
	\end{cases}
$$

$$
	\begin{aligned}
		K(\tau) & =D_0+\sum_{k=1}^{\infty} D_k \cos\left(\frac{k \pi \tau}{1}\right)
		=\frac{1}{2} + \sum_{k=1}^{\infty} \frac{6(1-(-1)^k)}{k^2\pi^2} \cos(k \pi \tau)
		\\ & = \frac{1}{2} + \sum_{m=0}^{\infty} \frac{12}{(2m+1)^2\pi^2} \cos((2m+1) \pi \tau)
		= \frac{1}{2} + \frac{12}{\pi^2} \sum_{m=0}^{\infty} \frac{\cos((2m+1) \pi \tau)}{(2m+1)^2}
	\end{aligned}
$$

\subsubsection*{Задача 9.4}
$K(\tau)=2-3|\tau|$, $T=2$

На $\left[0,2\right]$: $K(\tau)=2-3\tau$.

$$
	D_0=\frac{1}{2} \left[\int_{0}^{2} (2-3\tau) d\tau\right]=
	\frac{1}{2} \left[2\tau-\frac{3\tau^2}{2}\right]_0^2=-1
$$

$$
	D_k=\frac{2}{2} \left[\int_{0}^{2} (2-3\tau)\cos(\frac{k \pi \tau}{2}) d\tau\right]=
	\left(2\left[\int_{0}^{2}\cos(\frac{k \pi \tau}{2}) d\tau\right]-3\left[\int_{0}^{2} \tau\cos(\frac{k \pi \tau}{2}) d\tau\right]\right)
$$

$$
	I_1=\int_{0}^{2}\cos(\frac{k \pi \tau}{2}) d\tau=\left[\frac{2\sin(\frac{k \pi \tau}{2})}{k\pi}\right]_0^2=0
$$

$$
	\begin{aligned}
		I_2 & = \int_{0}^{2} \tau \cos(\frac{k \pi \tau}{2}) \, d\tau
		= \left[ \tau \frac{2\sin(\frac{k \pi \tau}{2})}{k \pi} \right]_0^2 - \int_{0}^{2} \frac{2\sin(\frac{k \pi \tau}{2})}{k \pi} \, d\tau
		= 0 - \frac{4}{(k\pi)^2}[\cos(\frac{k \pi \tau}{2})]_0^2
		\\ & = -\frac{4}{(k\pi)^2}(-\cos(k \pi) + 1)
		= -\frac{4(1 - \cos(k \pi))}{k^2 \pi^2}
		= -\frac{4(1-(-1)^k)}{k^2\pi^2}
	\end{aligned}
$$

$$
	D_k =(2I_1-3I_2)
	=\left(2\cdot0-3\left(-\frac{4(1-(-1)^k)}{k^2\pi^2}\right)\right)
	=\frac{12(1-(-1)^k)}{k^2\pi^2}
	=
	\begin{cases}
		\frac{24}{k^2 \pi^2}, & k = 2m+1, \\
		0,                    & k = 2m.
	\end{cases}
$$

$$
	\begin{aligned}
		K(\tau) & =D_0+\sum_{k=1}^{\infty} D_k \cos\left(\frac{k \pi \tau}{2}\right)
		=-1 + \sum_{k=1}^{\infty} \frac{12(1-(-1)^k)}{k^2\pi^2} \cos(\frac{k \pi \tau}{2})
		\\ & = -1 + \sum_{m=0}^{\infty} \frac{24}{(2m+1)^2\pi^2} \cos(\frac{(2m+1) \pi \tau}{2})
		= -1 + \frac{24}{\pi^2} \sum_{m=0}^{\infty} \frac{\cos(\frac{(2m+1) \pi \tau}{2})}{(2m+1)^2}
	\end{aligned}
$$



\subsubsection*{Задача 9.5}
$K_X(\tau)=\cos(\tau) e^{i|\tau|}$, $\tau \in \mathbb{R}$

На $[0,\infty)$: $K(\tau)=\cos(\tau) e^{i\tau}$


$$
	\cos(a\tau)=\frac{e^{i a\tau}+e^{-i a\tau}}{2}
$$

$$
	\int_{0}^{\infty} e^{-a\tau} \, d\tau = \frac{1}{a}
$$

$$
	\int_{0}^{\infty} \tau e^{-a\tau} \, d\tau = \frac{1}{a^2}
$$

$$
	\begin{aligned}
		S(\omega) & = 2\int_{0}^{\infty} (\cos(\tau) e^{i\tau}) \cos(\omega\tau) \, d\tau
		= 2 \int_{0}^{\infty} \frac{e^{i \tau}+e^{-i \tau}}{2} e^{i\tau} \frac{e^{i \omega\tau}+e^{-i \omega\tau}}{2} \, d\tau
		\\ & = \frac{1}{2}
		\left(
		\int_{0}^{\infty} (e^{2i\tau}+1)(e^{i\omega \tau} + e^{-i\omega \tau}) \, d\tau
		\right)
		\\ & = \frac{1}{2}
		\left(
		\int_{0}^{\infty} e^{i(2+\omega)\tau} \, d\tau
		+\int_{0}^{\infty} e^{i(2-\omega)\tau}  \, d\tau
		+\int_{0}^{\infty} e^{i\omega\tau}  \, d\tau
		+\int_{0}^{\infty} e^{-i(\omega)\tau}  \, d\tau
		\right)
	\end{aligned}
$$

$$
	K(\tau)=\frac{1}{2\pi}\int_{-\infty}^{\infty} S(\omega)e^{i\omega \tau} \, d\omega
$$



\subsubsection*{Задача 9.6}
$K_X(\tau)=(2+|\tau|) e^{-|\tau|}$, $\tau \in \mathbb{R}$

На $[0,\infty)$: $K(\tau)=(2+\tau) e^{-\tau}$

Вариант №1


$$
	\cos(a\tau)=\frac{e^{i a\tau}+e^{-i a\tau}}{2}
$$

$$
	\int_{0}^{\infty} e^{-a\tau} \, d\tau = \frac{1}{a}
$$

$$
	\int_{0}^{\infty} \tau e^{-a\tau} \, d\tau = \frac{1}{a^2}
$$


$$
	\begin{aligned}
		S(\omega) & =2\int_{0}^{\infty} ((2+\tau) e^{-\tau}) \cos(\omega\tau) \, d\tau
		\\ & = 2
		\left(
		2\int_{0}^{\infty} e^{-\tau} \cos(\omega \tau) \, d\tau
		+
		\int_{0}^{\infty} \tau e^{-\tau} \cos(\omega \tau) \, d\tau
		\right)
		\\ & = 2
		\left(
		2\int_{0}^{\infty} e^{-\tau} (\frac{e^{i\omega \tau}+e^{-i\omega \tau}}{2}) \, d\tau
		+
		\int_{0}^{\infty} \tau e^{-\tau} (\frac{e^{i\omega \tau}+e^{-i\omega \tau}}{2}) \, d\tau
		\right)
		\\ & = 2
		\left(
		\int_{0}^{\infty} e^{-\tau} (e^{i\omega \tau}+e^{-i\omega \tau}) \, d\tau
		+
		\frac{1}{2}\int_{0}^{\infty} \tau e^{-\tau} (e^{i\omega \tau}+e^{-i\omega \tau}) \, d\tau
		\right)
		\\ & = 2
		\left(
		\int_{0}^{\infty} e^{-\tau + i\omega \tau}+e^{-\tau -i\omega \tau} \, d\tau
		+
		\frac{1}{2}\int_{0}^{\infty} \tau e^{-\tau + i\omega \tau}+\tau e^{-\tau -i\omega \tau}  \, d\tau
		\right)
		\\ & = 2 \int_{0}^{\infty} e^{-(1-i\omega)\tau} \, d\tau
		+ 2 \int_{0}^{\infty} e^{-(1+i\omega)\tau} \, d\tau
		+ \int_{0}^{\infty} \tau e^{-(1-i\omega)\tau} \, d\tau
		+ \int_{0}^{\infty} \tau e^{-(1+i\omega)\tau} \, d\tau
		\\ & =
		2\frac{1}{1-i\omega}
		+ 2\frac{1}{1+i\omega}
		+ \frac{1}{(1-i\omega)^2}
		+ \frac{1}{(1+i\omega)^2}
		\\ & = \frac{4}{1+\omega^2} +\frac{2(1-\omega^2)}{(1+\omega^2)^2}
		= 2 \left(\frac{3+\omega^2}{(1+\omega^2)^2}\right)
	\end{aligned}
$$

Вариант №2

$$
	\int_{0}^{\infty}e^{-a\tau} cos(\omega \tau) d\tau = \frac{a}{a^2+\omega^2}
$$

$$
	\int_{0}^{\infty} \tau e^{-a\tau} cos(\omega \tau) d\tau = \frac{a-\omega^2}{(a^2+\omega^2)^2}
$$


$$
	\begin{aligned}
		S(\omega) & =2\int_{0}^{\infty} ((2+\tau) e^{-\tau}) \cos(\omega\tau) \, d\tau
		\\ & = 2
		\left(
		2\int_{0}^{\infty} e^{-\tau} \cos(\omega \tau) \, d\tau
		+
		\int_{0}^{\infty} \tau e^{-\tau} \cos(\omega \tau) \, d\tau
		\right)
		\\ & = 2
		\left(
		\frac{2}{1+\omega^2} +\frac{1-\omega^2}{(1+\omega^2)^2}
		\right)
		= 2 \left(\frac{3+\omega^2}{(1+\omega^2)^2}\right)
	\end{aligned}
$$

$$
	K(\tau) =\frac{1}{2\pi}\int_{-\infty}^{\infty} S(\omega)e^{i\omega \tau} \, d\omega
	= \frac{1}{2\pi}\int_{-\infty}^{\infty} 2 \left(\frac{3+\omega^2}{(1+\omega^2)^2}\right)e^{i\omega \tau} \, d\omega
$$
\newpage
\subsection*{Корреляционная функция}

\subsubsection*{Общее решение}

Корреляционная функция $K_x(t,s)$

$$
	Y_t=L_t[X_t]
$$

$$
	K_Y(t,s)=L_t[\overline{L}_s[K_x(t,s)]]
$$

\subsubsection*{Задача 8.1}

$K_x(t,s)=e^{-2i(t-s)}$, $Y_t=it \frac{dX_t}{dt}+e^{it}X_t$

$$
	L_t=it\frac{\partial}{\partial t}[\cdot] + e^{it}[\cdot]
$$

$$
	\overline{L}_s=-is\frac{\partial}{\partial s}[\cdot] + e^{-is}[\cdot]
$$

$$
	\begin{aligned}
		\overline{L}_s[K_x(t,s)] & =-is\frac{\partial}{\partial s}K_x(t,s) + e^{-is}K_x(t,s)
		=-is\frac{\partial}{\partial s}(e^{-2i(t-s)}) + e^{-is}(e^{-2i(t-s)})
		\\ & =-is \cdot 2i e^{-2i(t-s)} + e^{-is+(-2it+2is)}
		=2s e^{-2i(t-s)} +e^{-2it+is}
	\end{aligned}
$$

$$
	\begin{aligned}
		L_t[\overline{L}_s[K_x(t,s)]] & = it\frac{\partial}{\partial t}\overline{L}_s[K_x(t,s)] + e^{it}\overline{L}_s[K_x(t,s)]
		\\ & = it\frac{\partial}{\partial t}(2s e^{-2i(t-s)} +e^{-2it+is}) + e^{it}(2s e^{-2i(t-s)} +e^{-2it+is})
		\\ & = it(2s \cdot -2i \cdot e^{-2i(t-s)} - 2i\cdot e^{-2it +is}) + 2s\cdot e^{it-2it+2is} + e ^{it -2it +is}
		\\ & = 4st \cdot e^{-2it+2is} + 2t \cdot e^{-2it+is} + 2s \cdot e^{-it+2is} + e^{-it+is}.
	\end{aligned}
$$
\newpage



$$
	\frac{dX_t}{dt}+X_t=\frac{d^2Y_t}{dt}+2\frac{dY_t}{dt}+Y_t
$$
\newpage
\input{tasks/task4}
\newpage
\input{tasks/task5}
\newpage
\input{tasks/task6}
\newpage

\end{document}

