
\subsection{Лекция 6}

Из уравнений для стационарных вероятностей состояний следуют формулы
\[
	\lambda_{k-1} r_{k-1} = \mu_k r_k, \quad \text{при } k = 1, 2, \ldots, n.
\]
Значит
\[
	r_1 = \frac{\lambda_0}{\mu_1} r_0, \quad
	r_k = \frac{\lambda_{k-1}}{\mu_k} r_{k-1} = \ldots =
	\frac{\lambda_{k-1} \lambda_{k-2} \cdots \lambda_0}{\mu_k \mu_{k-1} \cdots \mu_1} r_0.
\]

Из уравнения нормировки получаем
\[
	r_0 =
	\left\{
	1 +
	\frac{\lambda_0}{\mu_1} +
	\frac{\lambda_0 \lambda_1}{\mu_1 \mu_2} +
	\ldots +
	\frac{\lambda_0 \lambda_1 \cdots \lambda_{n-1}}{\mu_1 \mu_2 \cdots \mu_n}
	\right\}^{-1}.
\]

---

\[
	\lambda_k r_k = \lambda_{k-1} r_{k-1} - \mu_k r_k + \mu_{k+1} r_{k+1}.
\]

---

Граф процесса рождения и гибели с бесконечным числом состояний:

\begin{center}
	\begin{tikzpicture}[
			>=Stealth, thick,
			state/.style={draw, rectangle, minimum width=10mm, minimum height=8mm, font=\small, align=center}
		]
		% узлы
		\node[state] (0) at (0,0) {0};
		\node[state] (1) at (2,0) {1};
		\node[state] (2) at (4,0) {2};
		\node[state] (d1) at (6,0) {$\cdots$};

		% стрелки вправо (рождения)
		\path[->] (0) edge[bend left=20] node[above] {$\lambda_0$} (1);
		\path[->] (1) edge[bend left=20] node[above] {$\lambda_1$} (2);
		\path[->] (2) edge[bend left=20] node[above] {$\lambda_2$} (d1);

		% стрелки влево (гибели)
		\path[->] (1) edge[bend left=20] node[below] {$\mu_1$} (0);
		\path[->] (2) edge[bend left=20] node[below] {$\mu_2$} (1);
		\path[->] (d1) edge[bend left=20] node[below] {$\mu_3$} (2);
	\end{tikzpicture}
\end{center}

Система дифференциальных уравнений Колмогорова для процесса рождения и гибели
с бесконечным числом состояний:

\[
	\begin{cases}
		p_0'(t) = -\lambda_0 p_0(t) + \mu_1 p_1(t), \\[6pt]
		p_k'(t) = \lambda_{k-1} p_{k-1}(t)
		- (\lambda_k + \mu_k) p_k(t)
		+ \mu_{k+1} p_{k+1}(t), \quad k \ge 1.
	\end{cases}
\]

---

Стационарные вероятности состояний $r_0, r_1, r_2, \ldots$ процесса рождения и гибели
с бесконечным числом состояний удовлетворяют системе линейных алгебраических уравнений:
\[
	\begin{cases}
		0 = -\lambda_0 r_0 + \mu_1 r_1, \\[6pt]
		0 = \lambda_{k-1} r_{k-1} - (\lambda_k + \mu_k) r_k + \mu_{k+1} r_{k+1}, \quad k \ge 1,
	\end{cases}
\]
а также уравнению нормировки:
\[
	\sum_{k=0}^{\infty} r_k = 1.
\]

---

Из уравнений для стационарных вероятностей состояний следуют формулы
\[
	\lambda_{k-1} r_{k-1} = \mu_k r_k, \quad \text{при } k = 1, 2, \ldots, n.
\]
Значит
\[
	r_1 = \frac{\lambda_0}{\mu_1} r_0, \quad
	r_k = \frac{\lambda_{k-1}}{\mu_k} r_{k-1}
	= \ldots =
	\frac{\lambda_{k-1} \lambda_{k-2} \cdots \lambda_0}{\mu_k \mu_{k-1} \cdots \mu_1} r_0.
\]

Из уравнения нормировки получаем
\[
	r_0 =
	\left\{
	1 +
	\frac{\lambda_0}{\mu_1} +
	\frac{\lambda_0 \lambda_1}{\mu_1 \mu_2} +
	\ldots +
	\frac{\lambda_0 \lambda_1 \cdots \lambda_{n-1}}{\mu_1 \mu_2 \cdots \mu_n}
	+ \ldots
	\right\}^{-1},
	\quad \text{если }
	\lim_{k \to \infty} \frac{\lambda_{k-1}}{\mu_k} < 1.
\]

---

\subsubsection*{Пуассоновский поток}

Пуассоновский (простейший) поток событий — поток, удовлетворяющий свойствам:

\begin{enumerate}
	\item стационарность;
	\item отсутствие последействия;
	\item ординарность.
\end{enumerate}

Для пуассоновского потока событий случайный процесс
$\{ X(t); \, t \ge 0 \}$
является марковским и называется процессом Пуассона.

---

\subsubsection*{Поток однородных событий}

\begin{enumerate}
	\item Случайная последовательность
	      $t_1 \le t_2 \le \ldots$
	      — \textit{моменты наступления событий};

	\item
	      $\{\tau_k = t_k - t_{k-1}; \, k \ge 1\}$,
	      $(t_0 = 0)$ —
	      \textit{интервалы между событиями};

	\item
	      $\{ X(t); \, t \ge 0 \}$ —
	      \textit{число событий на отрезке} $[0, t]$.
\end{enumerate}

---

Два потока называются эквивалентными, если у них совпадают
для любого $n \ge 1$ конечномерные распределения
$(\tau_1, \ldots, \tau_n) \quad \text{и} \quad (\tau_1', \ldots, \tau_n')$,
либо для любых $s_1, \ldots, s_n$ конечномерные распределения
$(X(s_1), \ldots, X(s_n)) \quad \text{и} \quad (X'(s_1), \ldots, X'(s_n))$.

---

\subsubsection*{Рекуррентный поток с запаздыванием}

Рекуррентный поток с запаздыванием, определяемый функциями распределения $F_1(t)$ и $F(t)$:

\begin{enumerate}
	\item $\{\tau_k; \, k \ge 1\}$ — независимы в совокупности;
	\item $F(t) = P(\tau_k \le t), \quad k \ge 2$;
	\item $F_1(t) = P(\tau_1 \le t)$.
\end{enumerate}

Поток называется рекуррентным потоком, если $F_1(t) = F(t)$.

---

\subsubsection*{Пуассоновский поток с параметром $\lambda$}

Пуассоновский поток с параметром $\lambda$ — рекуррентный поток, и

\[
	F(t) = P(\tau_k \le t) = 1 - e^{-\lambda t},
	\quad t \ge 0, \; k \ge 1.
\]

---

\subsubsection*{Простейший поток}

Поток заявок называется \textbf{стационарным}, если конечномерные распределения
\[
	(X(t + s_1), \ldots, X(t + s_n))
\]
не зависят от $t$.

Стационарность означает, что вероятность поступления того или иного числа заявок
на участке времени длины $\tau$ не зависит от его расположения на оси времени,
а зависит только от его длины.

---

Поток заявок называется потоком с отсутствием последействия, если случайные величины
\[
	X(t_1), \; X(t_2) - X(t_1), \; \ldots, \; X(t_n) - X(t_{n-1})
\]
независимы в совокупности для любых $t_1 < t_2 < \ldots < t_n$.

Отсутствие последействия означает, что вероятность попадания того или иного числа заявок
на заданный участок оси времени не зависит от того, сколько заявок пришло
на любой другой, не пересекающийся с ним участок.

---

Поток заявок называется ординарным, если
\[
	P\big(X(s + t) - X(s) > 1\big) = o(t), \quad t \to +0,
\]
т.е.
\[
	\lim_{t \to +0} \frac{P\big(X(s + t) - X(s) > 1\big)}{t} = 0
	\quad \text{при всех } s \ge 0.
\]

Ординарность означает, что заявки поступают по одному,
а не группами по два, три и т.д.

---

Из стационарности следует:


---