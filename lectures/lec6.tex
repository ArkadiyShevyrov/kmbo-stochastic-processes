
\subsection{Лекция 6}

Из уравнений для стационарных вероятностей состояний следуют формулы
\[
	\lambda_{k-1} r_{k-1} = \mu_k r_k, \quad \text{при } k = 1, 2, \ldots, n.
\]
Значит
\[
	r_1 = \frac{\lambda_0}{\mu_1} r_0, \quad
	r_k = \frac{\lambda_{k-1}}{\mu_k} r_{k-1} = \ldots =
	\frac{\lambda_{k-1} \lambda_{k-2} \cdots \lambda_0}{\mu_k \mu_{k-1} \cdots \mu_1} r_0.
\]

Из уравнения нормировки получаем
\[
	r_0 =
	\left\{
	1 +
	\frac{\lambda_0}{\mu_1} +
	\frac{\lambda_0 \lambda_1}{\mu_1 \mu_2} +
	\ldots +
	\frac{\lambda_0 \lambda_1 \cdots \lambda_{n-1}}{\mu_1 \mu_2 \cdots \mu_n}
	\right\}^{-1}.
\]

---

\[
	\lambda_k r_k = \lambda_{k-1} r_{k-1} - \mu_k r_k + \mu_{k+1} r_{k+1}.
\]

---

Граф процесса рождения и гибели с бесконечным числом состояний:

\begin{center}
	\begin{tikzpicture}[
			>=Stealth, thick,
			state/.style={draw, rectangle, minimum width=10mm, minimum height=8mm, font=\small, align=center}
		]
		% узлы
		\node[state] (0) at (0,0) {0};
		\node[state] (1) at (2,0) {1};
		\node[state] (2) at (4,0) {2};
		\node[state] (d1) at (6,0) {$\cdots$};

		% стрелки вправо (рождения)
		\path[->] (0) edge[bend left=20] node[above] {$\lambda_0$} (1);
		\path[->] (1) edge[bend left=20] node[above] {$\lambda_1$} (2);
		\path[->] (2) edge[bend left=20] node[above] {$\lambda_2$} (d1);

		% стрелки влево (гибели)
		\path[->] (1) edge[bend left=20] node[below] {$\mu_1$} (0);
		\path[->] (2) edge[bend left=20] node[below] {$\mu_2$} (1);
		\path[->] (d1) edge[bend left=20] node[below] {$\mu_3$} (2);
	\end{tikzpicture}
\end{center}

Система дифференциальных уравнений Колмогорова для процесса рождения и гибели
с бесконечным числом состояний:

\[
	\begin{cases}
		p_0'(t) = -\lambda_0 p_0(t) + \mu_1 p_1(t), \\[6pt]
		p_k'(t) = \lambda_{k-1} p_{k-1}(t)
		- (\lambda_k + \mu_k) p_k(t)
		+ \mu_{k+1} p_{k+1}(t), \quad k \ge 1.
	\end{cases}
\]

---

Стационарные вероятности состояний $r_0, r_1, r_2, \ldots$ процесса рождения и гибели
с бесконечным числом состояний удовлетворяют системе линейных алгебраических уравнений:
\[
	\begin{cases}
		0 = -\lambda_0 r_0 + \mu_1 r_1, \\[6pt]
		0 = \lambda_{k-1} r_{k-1} - (\lambda_k + \mu_k) r_k + \mu_{k+1} r_{k+1}, \quad k \ge 1,
	\end{cases}
\]
а также уравнению нормировки:
\[
	\sum_{k=0}^{\infty} r_k = 1.
\]

---

Из уравнений для стационарных вероятностей состояний следуют формулы
\[
	\lambda_{k-1} r_{k-1} = \mu_k r_k, \quad \text{при } k = 1, 2, \ldots, n.
\]
Значит
\[
	r_1 = \frac{\lambda_0}{\mu_1} r_0, \quad
	r_k = \frac{\lambda_{k-1}}{\mu_k} r_{k-1}
	= \ldots =
	\frac{\lambda_{k-1} \lambda_{k-2} \cdots \lambda_0}{\mu_k \mu_{k-1} \cdots \mu_1} r_0.
\]

Из уравнения нормировки получаем
\[
	r_0 =
	\left\{
	1 +
	\frac{\lambda_0}{\mu_1} +
	\frac{\lambda_0 \lambda_1}{\mu_1 \mu_2} +
	\ldots +
	\frac{\lambda_0 \lambda_1 \cdots \lambda_{n-1}}{\mu_1 \mu_2 \cdots \mu_n}
	+ \ldots
	\right\}^{-1},
	\quad \text{если }
	\lim_{k \to \infty} \frac{\lambda_{k-1}}{\mu_k} < 1.
\]

---

\subsubsection*{Пуассоновский поток}

Пуассоновский (простейший) поток событий — поток, удовлетворяющий свойствам:

\begin{enumerate}
	\item стационарность;
	\item отсутствие последействия;
	\item ординарность.
\end{enumerate}

Для пуассоновского потока событий случайный процесс
$\{ X(t); \, t \ge 0 \}$
является марковским и называется процессом Пуассона.

---

\subsubsection*{Поток однородных событий}

\begin{enumerate}
	\item Случайная последовательность
	      $t_1 \le t_2 \le \ldots$
	      — \textit{моменты наступления событий};

	\item
	      $\{\tau_k = t_k - t_{k-1}; \, k \ge 1\}$,
	      $(t_0 = 0)$ —
	      \textit{интервалы между событиями};

	\item
	      $\{ X(t); \, t \ge 0 \}$ —
	      \textit{число событий на отрезке} $[0, t]$.
\end{enumerate}

---

Два потока называются эквивалентными, если у них совпадают
для любого $n \ge 1$ конечномерные распределения
$(\tau_1, \ldots, \tau_n) \quad \text{и} \quad (\tau_1', \ldots, \tau_n')$,
либо для любых $s_1, \ldots, s_n$ конечномерные распределения
$(X(s_1), \ldots, X(s_n)) \quad \text{и} \quad (X'(s_1), \ldots, X'(s_n))$.

---

\subsubsection*{Рекуррентный поток с запаздыванием}

Рекуррентный поток с запаздыванием, определяемый функциями распределения $F_1(t)$ и $F(t)$:

\begin{enumerate}
	\item $\{\tau_k; \, k \ge 1\}$ — независимы в совокупности;
	\item $F(t) = P(\tau_k \le t), \quad k \ge 2$;
	\item $F_1(t) = P(\tau_1 \le t)$.
\end{enumerate}

Поток называется рекуррентным потоком, если $F_1(t) = F(t)$.

---

\subsubsection*{Пуассоновский поток с параметром $\lambda$}

Пуассоновский поток с параметром $\lambda$ — рекуррентный поток, и

\[
	F(t) = P(\tau_k \le t) = 1 - e^{-\lambda t},
	\quad t \ge 0, \; k \ge 1.
\]

---

\subsubsection*{Простейший поток}

Поток заявок называется \textbf{стационарным}, если конечномерные распределения
\[
	(X(t + s_1), \ldots, X(t + s_n))
\]
не зависят от $t$.

Стационарность означает, что вероятность поступления того или иного числа заявок
на участке времени длины $\tau$ не зависит от его расположения на оси времени,
а зависит только от его длины.

---

Поток заявок называется потоком с отсутствием последействия, если случайные величины
\[
	X(t_1), \; X(t_2) - X(t_1), \; \ldots, \; X(t_n) - X(t_{n-1})
\]
независимы в совокупности для любых $t_1 < t_2 < \ldots < t_n$.

Отсутствие последействия означает, что вероятность попадания того или иного числа заявок
на заданный участок оси времени не зависит от того, сколько заявок пришло
на любой другой, не пересекающийся с ним участок.

---

Поток заявок называется ординарным, если
\[
	P\big(X(s + t) - X(s) > 1\big) = o(t), \quad t \to +0,
\]
т.е.
\[
	\lim_{t \to +0} \frac{P\big(X(s + t) - X(s) > 1\big)}{t} = 0
	\quad \text{при всех } s \ge 0.
\]

Ординарность означает, что заявки поступают по одному,
а не группами по два, три и т.д.

---

Из стационарности следует:

\begin{itemize}
	\item число заявок$X((s, s + t]) = X(s + t) - X(s)$
	      на интервале $(s, s + t]$ зависит только от $t$;

	\item $p_k(t) = P(X(t) = k) = P(X(s + t) - X(s) = k)$.
\end{itemize}

Из отсутствия последействия следует, что
\[
	X\!\left(\left(0, \tfrac{t}{n}\right]\right),\,
	X\!\left(\left(\tfrac{t}{n}, \tfrac{2t}{n}\right]\right),\,
	\ldots,\,
	X\!\left(\left(\tfrac{(n - 1)t}{n}, t\right]\right)
\]
независимы, и
\[
	p_0(t) = [p_0\!\left(\tfrac{t}{n}\right)]^n.
\]

Если $p_0(1) = \gamma$, то
\[
	p_0\!\left(\tfrac{1}{n}\right) = \gamma^{\tfrac{1}{n}}
	\quad \text{и} \quad
	p_0\!\left(\tfrac{k}{n}\right) = [\gamma]^{\tfrac{k}{n}}.
\]

---

Если $t \in \left( \frac{k-1}{n}, \frac{k}{n} \right)$, то
$\gamma^n = p_0\!\left(\frac{k-1}{n}\right) \ge p_0(t) \ge p_0\!\left(\frac{k}{n}\right) = \gamma^n.$

Значит, $p_0(t) = \gamma^t$, и так как $\gamma \in (0,1)$, то можно положить
$\gamma = e^{-\lambda}, \ \lambda \in (0, +\infty).$
Получаем $p_0(t) = e^{-\lambda t}.$

Пусть
\[
	(0, \tfrac{t}{n}) \cup (\tfrac{t}{n}, \tfrac{2t}{n}] \cup \ldots \cup (\tfrac{(n-1)t}{n}, t].
\]
Рассмотрим полную группу событий $\{H_0, H_1\}$:
$H_0$ — нет интервалов, содержащих более чем одно событие; \\
$H_1$ — имеется хотя бы один интервал, содержащий более одного события.

---

\[
	p_k(t) = P(X(t) = k) = P(X(t) = k, H_0) + P(X(t) = k, H_1)
\]

Полагая \(\tau = \frac{t}{n}\), \(p_{>1}(\tau) = 1 - p_0(\tau) - p_1(\tau)\), получаем

\[
	P(X(t) = k, H_0) = C_n^k [p_{>1}(\tau)]^k [p_0(\tau)]^{n-k}
\]

\[
	= e^{-\lambda t} \frac{(\lambda t)^k}{k!}
	\frac{n (n-1) \ldots (n-k+1)}{n^k}
	[1 + o(1)] = \frac{(\lambda t)^k}{k!} e^{-\lambda t},
	\quad \text{при } n \to \infty
\]

и

\[
	P(X(t) = k, H_1) < np_1(\tau) + t p_{>1}(\tau) \to 0
\]

---

\subsection*{Процесс Пуассона}

Граф процесса Пуассона

\begin{center}
	\begin{tikzpicture}[
			>=Stealth, thick,
			state/.style={draw, rectangle, minimum width=10mm, minimum height=8mm, font=\small, align=center}
		]
		% узлы
		\node[state] (0) at (0,0) {0};
		\node[state] (1) at (2,0) {1};
		\node[state] (2) at (4,0) {2};
		\node[state] (d1) at (8,0) {$\cdots$};

		% стрелки вправо (интенсивность λ)
		\path[->] (0) edge node[above] {$\lambda$} (1);
		\path[->] (1) edge node[above] {$\lambda$} (2);
		\path[->] (2) edge node[above] {$\lambda$} (d1);
	\end{tikzpicture}
\end{center}


Рассмотрим систему уравнений Колмогорова:

\[
	\begin{cases}
		p_0'(t) = -\lambda p_0(t), \\
		p_k'(t) = \lambda p_{k-1}(t) - \lambda p_k(t), \quad k \ge 1.
	\end{cases}
\]

Решением этой системы с начальными условиями
\[
	(p_0(0), p_1(0), p_2(0), \ldots) = (1, 0, 0, \ldots)
\]
имеет вид:
\[
	p_k(t) = P(X_t = k) = \frac{(\lambda t)^k}{k!} e^{-\lambda t}, \quad k = 0, 1, 2, \ldots
\]
т.е. \( X_t \) имеет распределение Пуассона с параметром \( \lambda t \).

При этом:
\[
	M X_t = D X_t = \lambda t.
\]

--- [слайд 10]

\[
	\begin{tikzpicture}[
			>=Stealth, thick,
			state/.style={draw, rectangle, minimum width=10mm, minimum height=8mm, font=\small, align=center}
		]
		% узлы
		\node[state] (0) at (0,0) {0};
		\node[state] (1) at (2,0) {1};
		\node[state] (2) at (4,0) {2};
		\node[state] (k1) at (6,0) {$k-1$};
		\node[state] (k) at (8,0) {$k$};
		\node[state] (kp1) at (10,0) {$k+1$};
		\node[state] (d1) at (12,0) {$\cdots$};

		% стрелки
		\path[->] (0) edge node[above] {$\lambda$} (1);
		\path[->] (1) edge node[above] {$\lambda$} (2);
		\path[->] (2) edge node[above] {$\lambda$} (k1);
		\path[->] (k1) edge node[above] {$\lambda$} (k);
		\path[->] (k) edge node[above] {$\lambda$} (kp1);
		\path[->] (kp1) edge node[above] {$\lambda$} (d1);
	\end{tikzpicture}
\]

\[
	\begin{cases}
		p_0'(t) = -\lambda p_0(t), \\[4pt]
		p_k'(t) = \lambda p_{k-1}(t) - \lambda p_k(t), \quad k = 1, 2, \ldots
	\end{cases}
\]

\[
	\vec{p}(0) = (1, 0, 0, \ldots)
\]

\[
	p_k(t) = \overline{T}_k(s), \quad
	p_0' = s\,\overline{T}_0(s) - 1, \quad
	p_k' = s\,\overline{T}_k(s), \; k \ge 1
\]

--- [слайд 11]

\[
	\begin{cases}
		s \overline{p}_0 - 1 = -\lambda \overline{p}_0, \\[4pt]
		s \overline{p}_k = \lambda \overline{p}_{k-1} - \lambda \overline{p}_k, \quad k = 1, 2, \ldots
	\end{cases}
\]

\[
	\overline{p}_0 = \frac{1}{s + \lambda}, \qquad
	\overline{p}_k = \frac{\lambda}{s + \lambda} \, \overline{p}_{k-1}
	= \frac{\lambda^k}{(s + \lambda)^{k+1}}
\]

\[
	p_0(t) = e^{-\lambda t}, \qquad
	p_k(t) = \frac{(\lambda t)^k}{k!} e^{-\lambda t}
\]


--- [слайд 12]

\[
	N_t \text{ — пуассоновский процесс}
\]

\[
	\begin{array}{c|cccccc}
		N_t & 0      & 1      & 2      & \cdots & k      & \cdots \\ \hline
		P   & p_0(t) & p_1(t) & p_2(t) & \cdots & p_k(t) & \cdots
	\end{array}
\]

\[
	P(N_t = k) = p_k(t) = \frac{(\lambda t)^k}{k!} e^{-\lambda t}
\]

--- [слайд 13]


\subsection*{Траектории процесса Пуассона}

Процесс Пуассона

\[
	N(t) \text{ — число событий на интервале } [0, t].
\]

\begin{center}
	\begin{tikzpicture}[>=Stealth, thick, scale=1.1]
		% оси
		\draw[->] (0,0) -- (6,0) node[below] {$t$};
		\draw[->] (0,0) -- (0,4.5) node[left] {$N(t)$};

		% метки по времени
		\foreach \x/\label in {1/$t_1$,2/$t_2$,3/$t_3$,4/$t_4$}
		\draw (\x,0.1) -- (\x,-0.1) node[below] {\label};

		% ступенчатая функция Пуассона
		\draw[thick, ->] (0,0) -- (1,0);
		\draw[thick] (1,0) -- (1,1);
		\draw[thick, ->] (1,1) -- (2,1);
		\draw[thick] (2,1) -- (2,2);
		\draw[thick, ->] (2,2) -- (3,2);
		\draw[thick] (3,2) -- (3,3);
		\draw[thick, ->] (3,3) -- (4,3);
		\draw[thick] (4,3) -- (4,4);
		\draw[thick, ->] (4,4) -- (5.5,4);

		% подписи
		\node at (1,1.3) {$1$};
		\node at (2,2.3) {$2$};
		\node at (3,3.3) {$3$};
		\node at (4,4.3) {$4$};
	\end{tikzpicture}
\end{center}

--- [слайд 14]

\subsection*{Интервалы между заявками в пуассоновском потоке}

Пусть
\[
	\Pi = \{ t_1 \le t_2 \le \ldots \}
\]
— пуассоновский поток событий с интервалами
\[
	\{ \tau_k = t_k - t_{k-1}, \; k \ge 1 \}, \quad (t_0 = 0).
\]
В силу отсутствия последействия и ординарности пуассоновского потока все \(\tau_k\) независимы в совокупности.

Рассмотрим функцию распределения
\[
	F_{\tau_k}(t) = P(\tau_k \le t),
\]
получаем:
\[
	P(\tau_k > t) = P(t_k - t_{k-1} > t) = P(X((t_{k-1}, t_{k-1}+t]) = 0) = P(X(t) = 0) = e^{-\lambda t}.
\]

Следовательно,
\[
	F_{\tau_k}(t) = 1 - e^{-\lambda t}.
\]

Значит, все \(\tau_k\) имеют одинаковое показательное распределение с параметром \(\lambda\).

---