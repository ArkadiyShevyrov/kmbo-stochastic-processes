
\subsection{Лекция 3}

Обозначим через $f_{ij}(m)$ вероятность того, что из состояния $i$ первый раз попадаем в состояние $j$ на шаге $m$
$f_{ij}(m)=P(X_m=E_j,X_k\ne E_j (0<j<m) | X_0 =E_i))$, через $g_{ij}$ вероятность того, что из состояния $i$ попадаем в состояние $j$ бесконечное число раз. По формуле $f^*_{ij}=\sum_{m=1}^\infty f_{ij}(m)$ находится вероятность того, что исходя из состояния $i$ попадём в состояние $j$ хотя бы один раз.
Верны следующие утверждения:

1) $g_{ij}=\lim_{m \to \infty } \sum _s p_{is}(m) f^*_{sj}$;

2) $g_{ij}=f^*_{ij}g_{jj}$;

3) $i \to j \;\Leftrightarrow\; f_{ij}^* > 0$;

4) $i \rightleftarrows j \;\Leftrightarrow\; f_{ij}^* f_{ji}^* > 0$.

Состояние $i$ называется возвратным, если $f^*_{ii}=1$, и невозвратным, если $f^*_{ii}<1$.
Верны следующие утверждения:

1) $g_{ij}=f^*_{ij}$, если состояние $j$ возвратно, и $g_{ij}=0$, если $j$ невозвратно;

2) $g_{ii}=1$, если состояние $i$ возвратно, и $g_{ii}=0$, если $i$ невозвратно;

3) Если состояние $i$ несущественное, то $i$ невозвратное;

4) Если $g_{ii}=1$ и $f^*_{ij}>0$, то $g_{jj}=1$;

5) Если состояние $i$ возвратно и $i \to j$, то $j$ возвратно и $g_{ij}=g_{ji}=1$;

6) Состояние $i$ возвратное если ряд $\sum_{m=1}^\infty p_{ii}(m)$ расходится, и $i$ невозвратное, если ряд сходится.


Возвратное состояние $i$ называется положительным, если $\lim_{m \to \infty } p_{ii}(mk_i)>0$, и нулевым, если $\lim_{m \to \infty } p_{ii}(mk_i)=0$.
Обозначим через $\mu_{ii}$ среднее время возвращения в состояние $i$ $\mu_{ii}=\sum_{m=1}^\infty mf_{ii}(m)$.
Верны следующие утверждения:

1) Если состояние $i$ возвратное с периодом $k_i$, то $\lim_{m \to \infty} p_{ii} (mk_i)=\frac{k_i}{\mu _{ii}}$;

2) Возвратное состояние $i$ положительно $\Leftrightarrow$ $\mu _{ii}< \infty$;

3) Если состояние $i$ возвратное положительное и $i \rightleftarrows j$, то состояние $j$ так же положительное;

4) Если состояние $i$ возвратное положительное, то $\lim_{m \to \infty} \frac{1}{m} \sum_{k=1}^m p_{ii}(k)=1\frac{1}{\mu _{ii}}$.

Цепь Маркова - неприводимая, если $S=S(i)$ для всех $i\in S$.

Период состояния $i$: $k_i=НОД(k: p_{ii}(k)>0)$.

Цепь Маркова - апериодическая, если $k_i=1$ для всех $i\in S$.

Цепь Маркова - эргодическая $\Leftrightarrow$ она неприводимая и апериодическая.

\subsubsection*{Эргодические цепи Маркова}

Теорема 1. (эргодическая теорема для конечной цепь Маркова). Любая неприводимая непериодическая цепь Маркова $\{v_n, n\geq 0 \}$ с конечным множеством состояний $g$ является эргодической.

Теорема 2. (эргодическая теорема Фостера для счётной цепи Маркова) $\{v_n, n \geq 0 \}$ была эргодической, необходимо и достаточно существование нетривиального решения $\{p_i, i \geq 1 \}$ СУР (4.4)??? такого, что $\sum_{i=1}^\infty |p_i|<\infty$.
Решение $\{p_i, i \geq 1 \}$ с точностью до нормирующего множителя совпадает с предельным (стационарным) распределением.

Теорема 3. (для счётной цепи Маркова). Для того, чтобы  неприводимая непериодическая цепь Маркова $\{v_n, n\geq 0 \}$ была эргодической, достаточно существование числа $\varepsilon > 0$, целого числа $i_0$ и набора неотрицательных чисел $x_1,x_2,...$ таких что
$\sum_{j=1}^\infty p_{ij} x_j \leq x_i - \varepsilon$, $i \geq i_0$;
$\sum_{j=1}^\infty p_{ij} x_i < \infty$, $i < i_0$;

\subsubsection*{Марковские процессы с непрерывным временем}

Случайный процесс $X_t, t \geq 0$ называется марковским, если для любого целого неотрицательного $m$, любых моментов времени $0\leq s_1<s_2<...<s_m<s$, $t>0$, любого набора состояний $E_{i_1}, E_{i_2},...,E_{i_m},E_i, E_j$ выполнено равенство $P(X_{s+t}=E_j | X_{s_1}=E_{i_1}, ..., X_{s_m}=E_{i_m}, X_s=E_i)= P(X_{s+t}=E_j | X_s=E_i)$

\subsubsection*{Однородный марковский процесс с непрерывным временем}

Процесс $X_t$ называется однородным (по времени), если условная вероятность $P(X_{s+t}=E_j|X_s=E_i)$ перехода из состояния $E_i$ в состояние $E_j$ за время $t$ не зависит от $s$.
Обозначим $p_{ij}(t)=P(X_{s+t}=E_j | X_s=E_i)$.
Свойства:

1) $p_{ij}(0)=0$, если $i\ne j$, а $p_{ii}(0)=1$;

2) $0\leq p_{ij}(t)\leq 1$;

3) $\sum _j p_{ij}(t)=1$.

Матрица вероятностей переходя за время $t$: $P(t)=||p_{ij}(t)||$.
Предполагаем что переходные вероятности $p_{ij}(t)$ дифференцируемы в нуле: $p'_{ij}(0)=\lambda_{ij}$, при $i\ne j$ $p_{ij}(t)=\lambda_{ij}t+o(t)$, $p_{ii}(t)=1+\lambda_{ii}t+o(t)$.
$P'(0)=\Lambda=||\lambda_{ij}||$ - матрица интенсивностей (плотностей вероятностей) перехода.
При $i\ne j$ $\lambda_{ij}$ называется интенсивностью (плотностью вероятности) перехода из $E_i$ в $E_j$.
