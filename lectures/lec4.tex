
\subsection{Лекция 4}

\subsubsection*{Марковский процесс с непрерывным временем}

Величина $\lambda_{ji} \cdot p_j(t)$ называется потоком вероятности из состояния $E_j$ в состояние $E_i$ в момент времени $t$.

Из дифференциальных уравнений Колмогорова следует: произвольная вероятности состояния равна
сумме всех потоков вероятностей, приходящих в это состояние, минус сумма всех потоков вероятностей,
выходящих из этого состояния.

\textbf{Свойства матрицы интенсивностей перехода:}

1) $\lambda_{ij}\geq 0$, при $i\ne j$;

2) $\lambda_{ii}<0$;

3) $\sum_j \lambda_{ij}=0$, $\lambda_{ii}=-\sum_{j\ne i} \lambda_{ij}$.

\textbf{Система дифференциальных уравнений Колмогорова:}

$\frac{dp_i(t)}{dt}=\sum_j \lambda_{ji} p_j(t)=\lambda_{ii} p_i(t) + \sum_{j\ne i} \lambda_{ji} p_j(t)
	=\sum_{j\ne i} \lambda_{ji} p_j(t) - p_i(t) \sum_{j\ne i} \lambda_{ji}$

---

$\left\{ {P}(t) \right\}_{t \ge 0}$

1) ${P}(0) = {I}$,

2) ${P}(t+s) = {P}(t)\,{P}(s)$,

3) $\vec{p}(t) = \vec{p}(0)\,{P}(t)$.

---

\[
	\lambda_{ij} = \lim_{\tau \to +0}
	\frac{p_{ij}(\tau) - p_{ij}(0)}{\tau}
\]

\[
	i \ne j \Rightarrow
	\lambda_{ij} = \lim_{\tau \to +0}
	\frac{p_{ij}(\tau)}{\tau} \ge 0,
	\qquad
	i = j \Rightarrow
	\lambda_{ii} = \lim_{\tau \to +0}
	\frac{p_{ii}(\tau) - 1}{\tau} \le 0.
\]

---


\[
	\sum_j p_{ij}(t) = 1,
	\qquad
	\left( \sum_j p_{ij}(t) \right)'_{t=0} = (1)' = 0,
\]

\[
	\sum_j \lambda_{ij} = 0,
	\qquad
	\Rightarrow
	\qquad
	\lambda_{ii} = - \sum_{j \ne i} \lambda_{ij}.
\]


---

\[
	\vec{p}'(t)
	= \lim_{\tau \to 0}
	\frac{\vec{p}(t+\tau) - \vec{p}(t)}{\tau}
	=
	\lim_{\tau \to 0}
	\frac{\vec{p}(t){P}(\tau) - \vec{p}(t)}{\tau}
\]

\[
	= \vec{p}(t)
	\lim_{\tau \to 0}
	\frac{P(\tau) - P(0)}{\tau}
	= \vec{p}(t)\,{\Lambda}.
\]

---

\begin{align*}
	p_i'(t)
	 & = \sum_j p_j(t)\,\lambda_{ji}
	\\[4pt]
	 & = \sum_{j \ne i} p_j(t)\,\lambda_{ji} + p_i(t)\,\lambda_{ii}
	\\[4pt]
	 & = \sum_{j \ne i} p_j(t)\,\lambda_{ji} - \sum_{j \ne i} p_i(t)\,\lambda_{ij}.
\end{align*}

---

\begin{align*}
	{P}'(t)
	 & = \lim_{\tau \to 0}
	\frac{{P}(t+\tau) - {P}(t)}{\tau}
	\\[4pt]
	 & = \lim_{\tau \to 0}
	\frac{{P}(t){P}(\tau) - {P}(t)}{\tau}
	\\[4pt]
	 & = {P}(t)
	\lim_{\tau \to 0}
	\frac{{P}(\tau) - {P}(0)}{\tau}
		= {P}(t)\,{\Lambda}
	\\[4pt]
	 & = \lim_{\tau \to 0}
	\frac{{P}(t){P}(\tau) - {P}(t)}{\tau}
	= \lim_{\tau \to 0}
	\frac{{P}(t+\tau) - {P}(t)}{\tau}
	\\[4pt]
	 & =
	\left(
	\lim_{\tau \to 0}
	\frac{{P}(t+\tau) - {P}(t)}{\tau}
	\right)
	{P}(t)
	= {\Lambda}\,{P}(t).
\end{align*}

---


Дифференциальные уравнения Колмогорова

Векторная форма дифференциальных уравнений Колмогорова для вероятностей состояний:
$\vec{p'}(t)=\vec{p}(t)\Lambda$, где $\vec{p}(t)=(p_0(t),p_1(t),...,p_n(t),...)$.

Прямое уравнение Колмогорова: $P'(t)=P\Lambda$.

Обратное уравнение Колмогорова: $P'(t)=\Lambda P(t)$.

---

Марковский процесс с непрерывным временем ($\vec{p}(t), \{P(t)\}_{t>0}$) - эргодический:
\[
	\exists \lim_{t \to +\infty} {P}(t) = {Q}
	=
	\begin{pmatrix}
		q_{11} & q_{12} & \cdots \\
		q_{21} & q_{22} & \cdots \\
		\vdots & \vdots & \ddots
	\end{pmatrix},
	\qquad
	q_{ij} > 0 \ \forall i,j.
\]

---

\begin{enumerate}
	\item
	      $\displaystyle
		      \exists \lim_{t \to +\infty}
		      \overline{\mathbf{p}}(t)
		      =
		      \overline{\mathbf{q}}
		      = (q_1, q_2, \ldots)
	      $

	\item предел не зависит от $\overline{\mathbf{p}}(0)$

	\item $q_j > 0 \ \forall j$
\end{enumerate}

---

Процесс Маркова$\{ X_t \}_{t \ge 0}$ — эргодический $\Longleftrightarrow$
1) неприводим, 2) существует стационарное распределение.
---

Распределение вероятностей состояний, которое не зависит от времени $p_i(t+\tau ) = p_i(t)=p_i$
для любых $t,\tau\leq 0$ и любых $i=1,2,...$ называется стационарным распределением.

Система линейных алгебраических уравнений для стационарных вероятностей
\[
	\begin{cases}
		\sum_j  \lambda_{ji} r_j = 0, \quad i=1,2,... \\
		\sum_j r_j = 1
	\end{cases}
\]

---

\textbf{Свойства матрицы интенсивностей перехода:}
\begin{enumerate}
	\item $\lambda_{ij} \ge 0 \quad$ при $i \ne j$;
	\item $\lambda_{ii} \le 0$;
	\item $\displaystyle \sum_j \lambda_{ij} = 0,
		      \qquad
		      \lambda_{ii} = - \sum_{j \ne i} \lambda_{ij}.$
\end{enumerate}

\vspace{1em}

\textbf{Система дифференциальных уравнений Колмогорова}
\[
	\frac{dp_i(t)}{dt}
	= \sum_j \lambda_{ji}\,p_j(t)
	= \lambda_{ii}\,p_i(t)
	+ \sum_{j \ne i} \lambda_{ji}\,p_j(t)
\]
\[
	= \sum_{j \ne i} \lambda_{ji}\,p_j(t)
	- p_i(t)\sum_{j \ne i} \lambda_{ij},
	\qquad i = 1, 2, \ldots
\]

\textbf{Стациорнаное распределение}

Из уравнений следует: при стационарном распределении для каждого стояния сумма всех потоков вероятностей, приходящих в это состояние, равна сумме всех потоков вероятностей, выходящих из этого состояния.

\[
	\sum_{j \ne i} \lambda_{ji} r_j = r_i \sum_{j \ne i} \lambda_{ij} r_i
\]

---

\subsubsection*{Время пребывания марковского процесса с непрерывным временем в состоянии}

Пусть $\tau(i)$ — время пребывания марковского процесса $X(t)$ в состоянии $i$.
Получаем
$P\bigl(\tau(i) > t\bigr)= \lim_{n \to \infty} [p_{ii}(\frac{t}{n})]^n.$

Но $p_{ii}\!\left(\tfrac{t}{n}\right)= 1 + \lambda_{ii}\tfrac{t}{n} + o\!\left(\tfrac{1}{n}\right).$
Пусть $\lambda_i = -\lambda_{ii} = \sum_{j \ne i} \lambda_{ij}$.

Тогда
$\lim_{n \to \infty} [p_{ii}(\frac{t}{n})]^n
	= \lim_{n \to \infty}
	\left[1 - \lambda_i \tfrac{t}{n} + o\!\left(\tfrac{1}{n}\right)\right]^n
	= e^{-\lambda_i t}.$

Значит, $F_{\tau(i)}(t) = 1 - e^{-\lambda_i t}$,
т.\,е. $\tau(i)$ имеет показатель­ное распределение с параметром $\lambda_i$.
