\subsection{Лекция 5}

Пример. В системе 2 устройства и один мастер, проводящий ремонт.
Работа системы описывается марковским процессом $\{X_t, t\in [0, +\infty) \}$,
значением $X_t$ является число исправных устройств в момент времени $t$.
Каждое устройство выходит из строя с интенсивностью $\lambda=1$,
востановление устройства проходит с интенсивностью $\mu=2$.

Граф системы имеет вид:

\begin{center}
	\begin{tikzpicture}[
			>=Stealth, thick,
			state/.style = {draw, circle, minimum size=9mm, font=\small},
			e/.style   = {->, shorten >=2pt, shorten <=2pt},
			lbl/.style  = {font=\scriptsize, fill=white, inner sep=1pt}
		]

		% --- вершины ---
		\node[state] (Q1) at (0,0) {$2$};
		\node[state] (Q2) at (2.5,0) {$1$};
		\node[state] (Q3) at (5,0) {$0$};

		% --- рёбра ---
		% Q1
		\path[e, bend left=12]   (Q1) edge node[lbl] {$2$} (Q2);

		% Q2
		\path[e, bend left=12] (Q2) edge node[lbl] {$2$} (Q1);
		\path[e, bend left=12] (Q2) edge node[lbl] {$1$} (Q3);

		% Q3
		\path[e, bend left=12]   (Q3) edge node[lbl] {$2$} (Q2);

	\end{tikzpicture}
\end{center}

Найдём стационарные вероятности состояний.

---


Составим систему уравнений Колмогорова:
\[
	\begin{cases}
		p_0'(t) = -2p_0(t) + p_1(t);           \\
		p_1'(t) = 2p_0(t) - 3p_1(t) + 2p_2(t); \\
		p_2'(t) = 2p_1(t) - 2p_2(t).
	\end{cases}
\]

Решая систему уравнений для стационарных вероятностей:
\[
	\begin{cases}
		r_1 = 2r_0;         \\
		3r_1 = 2r_0 + 2r_2; \\
		r_1 = r_2;          \\
		r_0 + r_1 + r_2 = 1;
	\end{cases}
\]

находим
\[
	(r_0, r_1, r_2) = (0.2, 0.4, 0.4).
\]

---

\textbf{Задача 4.6.} \\
\textit{Два устройства и один мастер. Пусть $E_i$~--- работают $i$ устройств.}

\vspace{1em}

\begin{center}
	\begin{tikzpicture}[
			>=Stealth, thick,
			state/.style={draw, circle, minimum size=8mm, font=\small}
		]
		\node[state] (2) at (0,0) {2};
		\node[state] (1) at (2.5,0) {1};
		\node[state] (0) at (5,0) {0};

		\path[->] (2) edge[bend left=20] node[above] {$2\lambda$} (1);
		\path[->] (1) edge[bend left=20] node[below] {$\mu$} (2);

		\path[->] (1) edge[bend left=20] node[above] {$\lambda$} (0);
		\path[->] (0) edge[bend left=20] node[below] {$\mu$} (1);
	\end{tikzpicture}
\end{center}

\vspace{1em}

\[
	\begin{cases}
		2\lambda r_2 = \mu r_1,                     \\[4pt]
		(\lambda+\mu) r_1 = 2\lambda r_2 + \mu r_0, \\[4pt]
		\mu r_0 = \lambda r_1,                      \\[4pt]
		r_0 + r_1 + r_2 = 1.
	\end{cases}
\]

Из первого и третьего уравнений:
\[
	r_1 = \dfrac{2\lambda}{\mu} r_2,
	\qquad
	r_0 = \dfrac{\lambda}{\mu} r_1 = \dfrac{2\lambda^2}{\mu^2} r_2.
\]

Подставляем в нормировочное:
\[
	r_2\!\left(1 + \dfrac{2\lambda}{\mu} + \dfrac{2\lambda^2}{\mu^2}\right) = 1,
	\quad\Rightarrow\quad
	r_2 = \dfrac{\mu^2}{\mu^2 + 2\lambda\mu + 2\lambda^2}.
\]

Средняя доля времени простоя мастера: $r_2$

Средняя доля времени занятости мастера:
\[
	1 - r_2 = \dfrac{2\lambda(1+\mu)}{\mu^2 + 2\lambda\mu + 2\lambda^2}.
\]

\[
	r_1 = \dfrac{2\lambda\mu}{\mu^2 + 2\lambda\mu + 2\lambda^2},
	\qquad
	r_0 = \dfrac{2\lambda^2}{\mu^2 + 2\lambda\mu + 2\lambda^2}.
\]


\textbf{Задача 4.6} \quad [$\lambda = 1$, $\mu = 2$]

\vspace{1em}

\begin{center}
	\begin{tikzpicture}[
			>=Stealth, thick,
			state/.style={draw, circle, minimum size=8mm, font=\small}
		]
		\node[state] (2) at (0,0) {2};
		\node[state] (1) at (2.5,0) {1};
		\node[state] (0) at (5,0) {0};

		\path[->] (2) edge[bend left=20] node[above] {$2\lambda$} (1);
		\path[->] (1) edge[bend left=20] node[below] {$\mu$} (2);

		\path[->] (1) edge[bend left=20] node[above] {$\lambda$} (0);
		\path[->] (0) edge[bend left=20] node[below] {$\mu$} (1);
	\end{tikzpicture}
\end{center}

\vspace{1em}

\[
	\begin{cases}
		p_0'(t) = -2p_0(t) + p_1(t),           \\[4pt]
		p_1'(t) = 2p_0(t) + 2p_2(t) - 3p_1(t), \\[4pt]
		p_2'(t) = 2p_1(t) - 2p_2(t),           \\[4pt]
		p_2(0) = 1,\quad p_0(0) = p_1(0) = 0,  \\[4pt]
		p_0 + p_1 + p_2 = 1.
	\end{cases}
\]

$$p_i(t)=\pi_i (s)$$

\vspace{1em}
Переходим к изображению Лапласа:
$$
	\begin{cases}
		S \pi_0  = -2 \pi_0 + \pi_1,     \\
		S \pi_2 - 1 = 2 \pi_1 - 2 \pi_2. \\
	\end{cases}
$$

Сумма вероятностей:
\[
	\pi_0 + \pi_1 + \pi_2 = \frac{1}{S}.
\]

Из первого и последнего уравнений:
\[
	\pi_0 = \frac{\pi_1}{S + 2}, \qquad
	\pi_2 = \frac{1 + 2\pi_1}{S + 2}.
\]

Подставим:
\[
	\pi_1\!\left(\frac{1}{S + 2} + 1 + \frac{2}{S + 2}\right)
	= \text{[тут у Лобузова какой-то бред с числами]}.
\]

Следовательно:
\[
	p_1(t) = \frac{2}{5}\left(1 - e^{-5t}\right).
\]

Аналогично:
\[
	\pi_0 = \frac{1}{S} \cdot \frac{1}{5} - \frac{1}{3} \cdot \frac{1}{S + 2} + \frac{2}{15} \cdot \frac{1}{S + 5},
\]
\[
	p_0(t) = \frac{1}{5} - \frac{1}{3}e^{-2t} + \frac{2}{15}e^{-5t}.
\]

---

\[
	p_1(t) = \frac{2}{5} - \frac{2}{5} e^{-5t}
	\quad \xrightarrow[t \to +\infty]{} \quad
	\frac{2}{5},
\]

\[
	p_0(t) = \frac{1}{5} - \frac{1}{3} e^{-2t} + \frac{2}{15} e^{-5t}
	\quad \xrightarrow[t \to +\infty]{} \quad
	\frac{1}{5},
\]

\[
	p_2(t) = \frac{2}{5} + \frac{1}{3} e^{-2t} + \frac{4}{15} e^{-5t}
	\quad \xrightarrow[t \to +\infty]{} \quad
	\frac{2}{5}.
\]

\vspace{1em}

\[
	\pi_i(s) \; = \; p_i(t),
	\qquad
	p_i'(t) \; = \; s\,\pi_i(s) - p_i(0).
\]

---

Пусть $\tau$ — время нахождения в состоянии $i$,
а $\tau_{(s)}$ — время нахождения в этом состоянии после времени $s$.

\[
	P(\tau_{(s)} > t)
	= P(\tau > s + t \mid \tau > s)
	= \frac{e^{-\lambda (t + s)}}{e^{-\lambda s}}
	= e^{-\lambda t},
	\qquad [\,\lambda = \lambda_i\,].
\]

Отсюда
\[
	P(\tau_{(s)} \le t) =
	\begin{cases}
		0,                  & t < 0,   \\[6pt]
		1 - e^{-\lambda t}, & t \ge 0,
	\end{cases}
	\qquad
	\text{— показатель­ное распределение с параметром } \lambda.
\]

Следовательно $\tau_{(s)} \sim \tau$.

\subsubsection*{Процессы рождения и гибели. Процесс Пуассона}

Граф процесса рождения и гибели с конечным числом состояний

\begin{center}
	\begin{tikzpicture}[
			>=Stealth, thick,
			state/.style={draw, rectangle, minimum width=10mm, minimum height=8mm, font=\small, align=center}
		]

		% узлы
		\node[state] (0) at (0,0) {0};
		\node[state] (1) at (2,0) {1};
		\node[state] (2) at (4,0) {2};
		\node[state] (k) at (6,0) {$\dots$};
		\node[state] (n) at (8,0) {$n$};

		% стрелки вправо (рождения)
		\path[->] (0) edge[bend left=20] node[above] {$\lambda_0$} (1);
		\path[->] (1) edge[bend left=20] node[above] {$\lambda_1$} (2);
		\path[->] (2) edge[bend left=20] node[above] {$\lambda_2$} (k);
		\path[->] (k) edge[bend left=20] node[above] {$\lambda_{n-1}$} (n);

		% стрелки влево (гибели)
		\path[->] (1) edge[bend left=20] node[below] {$\mu_1$} (0);
		\path[->] (2) edge[bend left=20] node[below] {$\mu_2$} (1);
		\path[->] (k) edge[bend left=20] node[below] {$\mu_3$} (2);
		\path[->] (n) edge[bend left=20] node[below] {$\mu_n$} (k);

	\end{tikzpicture}
\end{center}

Система дифференциальных уравнений Колмогорова:

\[
	\begin{cases}
		p_0'(t) = -\lambda_0 p_0(t) + \mu_1 p_1(t), \\[4pt]
		p_k'(t) = \lambda_{k-1} p_{k-1}(t)
		- (\lambda_k + \mu_k) p_k(t)
		+ \mu_{k+1} p_{k+1}(t),
		 & 1 \le k < n,                             \\[4pt]
		p_n'(t) = \lambda_{n-1} p_{n-1}(t) - \mu_n p_n(t).
	\end{cases}
\]

---

Стационарные вероятности состояний $r_0, r_1, r_2, \ldots, r_n$ процесса рождения и гибели
с конечным числом состояний удовлетворяют системе линейных алгебраических уравнений:

\[
	\begin{cases}
		0 = -\lambda_0 r_0 + \mu_1 r_1, \\[4pt]
		0 = \lambda_{k-1} r_{k-1} - (\lambda_k + \mu_k) r_k + \mu_{k+1} r_{k+1},
		 & 1 \le k < n,                 \\[4pt]
		0 = \lambda_{n-1} r_{n-1} - \mu_n r_n.
	\end{cases}
\]

А также уравнению нормировки:
\[
	\sum_{k=0}^{n} r_k = 1.
\]
