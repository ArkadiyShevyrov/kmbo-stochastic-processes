\subsection{Лекция 1}

\subsubsection*{Основные понятия теории случайных процессов}

Теория случайных процессов является развитием теории вероятностей, в ней изучаются не отдельные случайные величины или векторы, а их семейства
$\{X_t,t\in T\subseteq R\}$, зависящие от параметра времени $t$.
Случайным процессом называют семейство в измеримом пространстве $(S,B)$ и определённых на одном вероятностном пространстве $(\Omega, A, P)$.
Пространство $S$ называют пространством состояний случайного процесса.

В зависимости от вида множества параметров $T$ случайный процесс может быть с дискретным или непрерывным временем.
Если $T=\mathbb{Z}_+=\{0,1,2,...\}$ (множество неотрицательных целых числе), то случайный процесс $X_t,t\in T$ называют цепью.
В зависимости от вида $S$ случайный процесс может быть с дискретным или непрерывным пространством состояний.

Случайные процессы применяются для моделирования эволюции реальных стохастических систем. Примерами могут служить: броуновское движение частиц, процессы рождении и гибели в биологических системах, генетическая эволюция, системы массового обслуживания и так далее.

\subsubsection*{Общая классификация случайных процессов}

\begin{table}[ht]
	\centering
	\begin{tabularx}{\textwidth}{|c|>{\raggedright\arraybackslash}X|>{\raggedright\arraybackslash}X|}
		\hline
		\diagbox{$T$}{$S$}                                                            &
		Дискретное                                                                    &
		Непрерывное                                                                     \\
		\hline
		Дискретное                                                                    &
		Последовательность дискретных случайных величин                               &
		Последовательность непрерывных случайных величин                                \\
		\hline
		Непрерывное                                                                   &
		Случайный процесс с непрерывным временем и дискретным пространством состояний &
		Случайный процесс с непрерывным временем и непрерывным пространством состояний  \\
		\hline
	\end{tabularx}
\end{table}

\subsubsection*{Основные понятия теории случайных процессов}

Для любого набора $t_1, t_2, ..., t_n \in T$ вектор
$(X(t_1), X(t_2),..., X(t_n))$ называется конечномерным сечением или $n$-мерным сечением случайного процесса $\{X_t, t\in T\}$.
При фиксированном $\omega \in \Omega$ отображение $t \rightarrow X_t(\omega)$ называется траекторией или выборочной функцией случайного процесса
$\{X_t,\in T \}$.
Семейство $\sigma$-алгебр \{$F_t, t\in T\}$ называется фильтрацией, если
$F_s \subset F_t \subset A$ для всех $s<t$, $s,t\in T$.
Случайный процесс $\{X_t,t\in T\}$ согласован с фильтрацией $\{F_t,t\in T\}$, если случайная величина $X_t$ измерима относительно $F_t$ для всех $t\in T$.

Пусть $S\subseteq R$, $S\in B(R)$, $B=B(S)$.
Случайный процесс называется регулярным, если его траектории в каждой точке $t\in T$ непрерывны справа и имеют конечные пределы слева.
Случайные процессы $\{X_t,t\in T\}$ и $\{Y_t,t\in T\}$ называются стохастически эквивалентными в широком смысле, если для всех $B_i\in B$, $i=1,..,n$, $t_1<t_2<...<t_{n-1}<t_n\in T$, верно равенство $P(X(t_1)\in B_1,...,X(t_n)\in B_n)=P(Y(t_1)\in b_1,...,Y(t_n)\in B_n)$.
Случайные процессы $\{X_t,t\in T\}$ и $\{Y_t, t\in T\}$ называются стохастически эквивалентными, если $P(X(t)=Y(t))=1$ для всех $t\in T$.

Пусть $t_1,t_2,...,t_n\in T$, $t(n)=(t_1,t_2,...,t_n)$, $X(t)=(X(t_1),X(t_2),...,X(t_n))$ - $n$-мерное сечение, $F_{t(n)}(x_1,x_2,...,x_n)=P(X(t_1)\leq x_1, ...,X(t_n)\leq x_n)$ - его функция распределения.
Для любых $t_1,t_2,...,t_n\in T$ $F_{t(n)}(x_1,x_2,...,x_n)$ удовлетворяют всем свойствам функции распределений случайных векторов.
Кроме того, условиям согласованности:
1) $F_{t(n)}(x_1,x_2,...,x_n)=F_{\sigma (t(n))}(x_{\sigma(1)},x_{\sigma(2)},...,x_{\sigma (n)})$, $\sigma (t(n))=(t_{\sigma (1)}, t_{\sigma (2)}, ..., t_{\sigma (n)})$, $\sigma \in S_n$ - группа перестановок;
2) $F_{t(n)}(x_1,...,x_m,+\infty,...,+\infty)=F_{t(m)}(x_1,x_2,...,x_m)$.
Теорема Колмогорова. Пусть задано семейство функций распределения
$F_{t(n)}(x_1,x_2,..x_n)$, удовлетворяющих условиям 1 и 2.
Тогда существует вероятностное пространство $(\Omega, A, P)$ и случайный процесс $\{X_t,t\in T\}$ с данными функциями распределения.

Полная информация о случайном процессе $\{X_t, t\in T\}$ содержится в функциях распределения всех его конечномерных сечений.
В общем случае найти все эти функции распределения практически невозможно.
Эта задача иногда упрощается при рассмотрении марковских случайных процессов.
Случайный процесс $\{X_t, t\in T\}$ называется марковским, если выполняется равенство условных вероятностей
$P(X(t_n)\in B_n | X(t_1)\in B_1, ..., X(t_{n-1})\in B_{n-1})=P(X(t_n)\in B_n | X(t_{n-1})\in B_{n-1})$ для всех $t_1<t_2<...<t_{n-1}<t_n \in T$, $B_i \in B$, $i=1,...,n$.

Вероятности конечномерных сечений для произвольного процесса \\
$P(X(t_1)\in B_1, ..., X(t_{n-1}) \in B_{n-1}, X(t_n) \in B_n) =\\
	P(X(t_n)\in B_n |X(t_1)\in B_1,...,X(t_{n-1})\in B_{n-1})\cdot P(X(t_{n-1})\in B_{n-1} |X(t_1)\in B_1,...,X(t_{n-2})\in B_{n-2})\cdot ... \cdot P(X(t_2)\in B_2 | X(t_1)\in B_1) \cdot P(X(t_1)\in B_1)$.

Вероятности конечномерных сечений для марковского процесса при
$t_1<t_2<...<t_{n-1}<t_n$,
$P(X(t_1)\in B_1,...,X(t_{n-1})\in B_{n-1},X(t_n)\in B_n) = \\
	P(X(t_n)\in B_n | X(t_{n-1})\in B_{n-1}) \cdot P(X(t_{n-1})\in B_{n-1} | X(t_{n-2})\in B_{n-2}) \cdot ... \cdot P(X(t_2)\in B_2 | X(t_1) \in B_1) \cdot P(X(t_1) \in B_1 )$.
