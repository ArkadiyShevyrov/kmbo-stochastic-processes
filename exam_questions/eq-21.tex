\subsection*{Вопрос 21}

\begin{condition}
	Линейные преобразования случайных процессов. Нахождение корреляционной функции линейного преобразования.
\end{condition}

\textbf{Линейные преобразования случайных процессов.}

Отображение $L_t$ пространства случайных процессов на $[a,b]$ в себя
называется \emph{линейным преобразованием}, если для любых процессов
$X_t$, $Y_t$ и любых чисел $\alpha,\beta$ выполняется равенство
$
	L_t[\alpha X_t+\beta Y_t]
	=
	\alpha\,L_t[X_t]+\beta\,L_t[Y_t].
$

\medskip

Пусть
$
	\mathbb{E}|X_t|^2<\infty,
	\qquad
	Y_t=L_t[X_t]+g(t),
$
где $g(t)$ — неслучайная функция.
Тогда справедливы соотношения:

\begin{enumerate}
	\item
	      Математическое ожидание:
	      $
		      m_Y(t)=L_t[m_X(t)]+g(t).
	      $

	\item
	      Корреляционная функция:
	      $
		      K_Y(t,s)
		      =
		      L_t\,\overline{L_s}\!\left[K_X(t,s)\right]
		      =
		      \overline{L_s}\,L_t\!\left[K_X(t,s)\right].
	      $
\end{enumerate}
