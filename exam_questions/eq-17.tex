\subsection*{Вопрос 17}

\begin{condition}
	Каноническое разложение случайного процесса и его характеристики.
\end{condition}

\textbf{Каноническое разложение случайного процесса.}

Случайный процесс $X_t$ допускает каноническое разложение, если
$
	X_t=\sum_{i=1}^{n} V_i\,\psi_i(t)+d(t),
$
где случайные величины $V_i$ удовлетворяют условиям
$
	\mathbb{E}V_i=0,\qquad D V_i<\infty,
$
$
	\operatorname{cov}(V_i,V_j)=0,\quad i\ne j,
$
а функции $\psi_i(t)$ и $d(t)$ являются неслучайными.

\medskip

\textbf{Свойства канонического разложения.}

\begin{enumerate}
	\item
	      Математическое ожидание процесса:
	      $
		      m_X(t)=d(t).
	      $

	\item
	      Корреляционная функция:
	      $
		      K_X(t,s)=\sum_{i=1}^{n}\psi_i(t)\,\overline{\psi_i(s)}\,D V_i.
	      $

	\item
	      Дисперсия процесса:
	      $
		      D_X(t)=\sum_{i=1}^{n}\lvert\psi_i(t)\rvert^2\,D V_i.
	      $
\end{enumerate}
