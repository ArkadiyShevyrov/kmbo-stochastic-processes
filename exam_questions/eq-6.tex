\subsection*{Вопрос 6}

\begin{condition}
	Классификация состояний цепи Маркова. Существенные и несущественные состояния. Период состояния. Критерий эргодичности цепи Маркова.
\end{condition}


\begin{itemize}
	\item
	      Состояние $j$ называется \emph{достижимым} из состояния $i$ $(i\to j)$,
	      если существует такое $k$, что
	      $
		      p_{ij}(k)>0.
	      $

	\item
	      Если $i\to j$ и $j\to s$, то $i\to s$.

	\item
	      Состояния $i$ и $j$ называются \emph{сообщающимися} $(i\leftrightarrow j)$,
	      если $i\to j$ и $j\to i$.

	\item
	      Состояние $i$ называется \emph{существенным}, если из $i\to j$ следует
	      $j\to i$.

	\item
	      Если $i$ — существенное состояние и $i\to j$, то $j$ также является
	      существенным.
	      Действительно, если $i\to j$ и $j\to s$, то $i\to s$.
	      Так как $i$ существенно, то $s\to i$.
	      Из $s\to i$ и $i\to j$ следует $s\to j$.

	\item
	      Если для состояния $i$ существует такое $j$, что $j$ достижимо из $i$,
	      но $i$ недостижимо из $j$, то состояние $i$ называется \emph{несущественным}.

	\item
	      Если $i\leftrightarrow j$, то
	      $
		      k_i = k_j.
	      $
\end{itemize}

\medskip

\textbf{Период состояния.}
Период состояния $i$ определяется как
$
	k_i=\gcd\{\,k:\; p_{ii}(k)>0\,\}.
$
