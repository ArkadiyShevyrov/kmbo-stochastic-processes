\subsection*{Вопрос 9}

\begin{condition}
	Марковский процесс с дискретным множеством состояний и непрерывным временем. Уравнения для стационарных вероятностей. Эргодичность.
\end{condition}

Случайный процесс $X_t$, $t\ge 0$, называется \emph{марковским}, если для любого
целого неотрицательного $m$, любых моментов времени
$
	0 \le s_1 < s_2 < \ldots < s_m \le s,\qquad t>0,
$
и любого набора состояний
$
	E_{i_1},E_{i_2},\ldots,E_{i_m},E_i,E_j
$
выполняется равенство
$
	P\!\left(
	X_{s+t}=E_j
	\mid
	X_{s_1}=E_{i_1},\ldots,X_{s_m}=E_{i_m},X_s=E_i
	\right)
	=
	P\!\left(X_{s+t}=E_j \mid X_s=E_i\right).
$

\medskip

\textbf{Система линейных алгебраических уравнений для стационарных вероятностей:}
$
	\begin{cases}
		\displaystyle
		\sum_j \lambda_{ji}\,z_i = 0, \qquad i=1,2,\ldots, \\
		\displaystyle
		\sum_j z_j = 1.
	\end{cases}
$

\medskip

Цепь Маркова называется \emph{неприводимой}, если
$
	S=S(i)\qquad \forall\, i\in S.
$

\medskip

\textbf{Период состояния.}
Период состояния $i$ определяется как
$
	k_i=\gcd\{\,k:\; p_{ii}(k)>0\,\}.
$

\medskip

Цепь Маркова называется \emph{апериодической}, если
$
	k_i=1 \qquad \forall\, i\in S.
$

\medskip

Цепь Маркова называется \emph{эргодической}, если она является
неприводимой и апериодической.

\medskip

Марковский процесс с непрерывным временем
$
	\bigl(\vec p(t),\; P(t)\bigr)_{t\ge 0}
$
является эргодическим.
