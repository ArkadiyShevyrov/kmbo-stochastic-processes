\subsection*{Вопрос 25}

\begin{condition}
	Спектральная плотность стационарного процесса, её свойства. Процесс «белый шум».
\end{condition}

\textbf{Стационарные процессы с непрерывным спектром}

Стационарный в широком смысле процесс $\xi_t$ называется процессом
с непрерывным спектром, если его спектральная функция $F_\xi(\omega)$
дифференцируема:
$
	F'_\xi(\omega)=S_\xi^*(\omega).
$

В этом случае
$
	K_\xi(\tau)=\int_{-\infty}^{+\infty} e^{i\omega\tau}\, S_\xi^*(\omega)\, d\omega.
$

Функция $S_\xi^*(\omega)$ является неотрицательной и называется
спектральной плотностью стационарного процесса.

\medskip

Функция $S_\xi^*(\omega)$ может быть найдена по формуле:
$
	S_\xi^*(\omega)=\frac{1}{2\pi}\int_{-\infty}^{+\infty}
	e^{-i\omega\tau}\, K_\xi(\tau)\, d\tau,
$
(формула обратного преобразования Фурье).

\medskip

Так как
$
	\overline{K_\xi(\tau)}=K_\xi(\tau)
	=\int_{-\infty}^{+\infty} e^{-i\omega\tau} S_\xi^*(\omega)\, d\omega
	=\int_{-\infty}^{+\infty} e^{i\omega\tau} S_\xi^*(-\omega)\, d\omega,
$
то
$
	S_\xi^*(\omega)=S_\xi^*(-\omega).
$

Следовательно,
\begin{align*}
	K_\xi(\tau)
	 & =\int_{-\infty}^{+\infty} e^{i\omega\tau} S_\xi^*(\omega)\, d\omega \\
	 & =\int_{0}^{+\infty} e^{i\omega\tau} S_\xi^*(\omega)\, d\omega
	+\int_{0}^{+\infty} e^{-i\omega\tau} S_\xi^*(\omega)\, d\omega         \\
	 & =\int_{0}^{+\infty} 2 S_\xi^*(\omega)\cos(\omega\tau)\, d\omega.
\end{align*}

Обозначая
$
	S_\xi(\omega)=2S_\xi^*(\omega),
$
получаем
$
	K_\xi(\tau)=\int_{0}^{+\infty} S_\xi(\omega)\cos(\omega\tau)\, d\omega,
$
причём
$
	S_\xi(\omega)=\frac{2}{\pi}\int_{0}^{+\infty}
	K_\xi(\tau)\cos(\omega\tau)\, d\tau,
$
и
$
	S_\xi(\omega)=2S_\xi^*(\omega)=2S_\xi^*(-\omega).
$

\medskip

\textbf{Белый шум.}

Процессом \emph{белого шума} с интенсивностью $G$ ($G=\mathrm{const}$)
называется стационарный процесс с спектральной плотностью
$
	S_\xi^*(\omega)=\frac{G}{2\pi}=\mathrm{const}.
$

Корреляционная функция процесса белого шума:
$
	K_\xi(\tau)=G\,\delta(\tau),
$
где $G$ — интенсивность белого шума,
$\delta(\tau)$ — дельта-функция Дирака.

Процесс белого шума имеет бесконечную дисперсию:
$
	D_\xi=G\,\delta(0)=+\infty.
$
