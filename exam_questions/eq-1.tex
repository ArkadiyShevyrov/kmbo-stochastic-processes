\subsection*{Вопрос 1}

\begin{condition}
	Траектории, сечения, конечномерные распределения случайных процессов. Классификация случайных процессов. Определение и свойства марковского процесса.
\end{condition}

\Definition{Случайный процесс}{
	семейство случайных велечин $\{X_t=X(t), t\in T\}$,
	принимающих значения в измеримом пространстве $(S,B)$,
	и определённых на одном вероятностном пространстве $(\Omega, A, P)$.
}

\Definition{Траектория случайного процесса}{
	отображение $t \to X_t(\omega)$,
	при фиксированном $\omega\in\Omega$.
}

\Definition{Сечение случайного процесса}{
	вектор $(X(t_1),X(t_2),...,X(t_n))$,
	для любого набора $t_1,t_2,...,t_n\in T$.
}

\Definition{Конечномерное распределение случайного процесса}{
	функция
	$F_{t_1,t_2,...,t_n}(x_1,x_2,...,x_n)=P(X(t_1)\leq x_1,...,X(t_n) \leq x_n)$.

	Где $n \in \mathbb{N}$,
	$t_1,t_2,...,t_n\in T$,
	$X(t)=(X(t_1),...X(t_n))$ - конечномерное сечение.
}

\textbf{Классификация случайных процессов}:
\begin{table}[ht]
	\centering
	\begin{tabularx}{\textwidth}{|c|>{\raggedright\arraybackslash}X|>{\raggedright\arraybackslash}X|}
		\hline
		\diagbox{$T$}{$S$}                                                            &
		Дискретное                                                                    &
		Непрерывное                                                                     \\
		\hline
		Дискретное                                                                    &
		Последовательность дискретных случайных величин                               &
		Последовательность непрерывных случайных величин                                \\
		\hline
		Непрерывное                                                                   &
		Случайный процесс с непрерывным временем и дискретным пространством состояний &
		Случайный процесс с непрерывным временем и непрерывным пространством состояний  \\
		\hline
	\end{tabularx}
\end{table}


\Definition{Марковский случайный процесс}{
	случайный процесс, для которого выполнено равенство условных вероятностей

	$
		P\left(X(t_n)\in B_n| X(t_1)\in B_1, ..., X(t_{n-1})\in B_{n-1}\right)
		=P(X(t_n)\in B_n | X(t_{n-1})\in B_{n-1})
	$,

	$\forall t_1<t_2<...<t_n\in T$, $B_i\in B$, $i=1,...,n$.
}

\Property{Однородность марковского случайного процесса}{
	Марковский процесс однороден,
	если переходные вероятности зависят только от разности времён:
	$P()$
}

\Property{Эргодичность марковского случайного процесса}{

}