\subsection*{Вопрос 13}

\begin{condition}
	Поток событий. Определение простейшего потока. Распределение времени между событиями в простейшем потоке, свойства этого распределения.
\end{condition}

\textbf{Поток однородных событий.}

\begin{enumerate}
	\item
	      Случайная последовательность
	      $
		      t_1 \le t_2 \le \ldots
	      $
	      — моменты наступления событий.

	\item
	      $
		      \tau_k = t_k - t_{k-1}, \qquad k\ge 1,\quad t_0=0,
	      $
	      — интервалы между событиями.

	\item
	      $
		      \{X(t),\ t\ge 0\}
	      $
	      — число событий на интервале $[0,t]$.
\end{enumerate}

\medskip

\textbf{Простейший (пуассоновский) поток.}

Поток $\{X(t)\}$ называется простейшим, если выполняются условия:
\begin{enumerate}
	\item
	      Поток $\{X(t)\}$ эквивалентен для любого $s>0$ потоку
	      $
		      \{X(t+s)-X(s)\}
	      $
	      (\emph{стационарность}).

	\item
	      $\{X(t)\}$ — процесс с независимыми приращениями
	      (\emph{отсутствие последействия}).

	\item
	      $
		      P\bigl(X(t)\ge 2 \mid X(t)\ge 1\bigr)\xrightarrow[t\to 0]{}0
	      $
	      (\emph{ординарность}).
\end{enumerate}

\medskip

Пусть
$
	\Pi=\{t_1,t_2,\ldots\}
$
— простейший поток заявок с интервалами
$
	\{\tau_k=t_k-t_{k-1},\ k\ge 1\}, \qquad t_0=0.
$
Обозначим функцию распределения
$
	F_{\tau_k}(t)=P(\tau_k\le t).
$
Тогда
$
	P(\tau_k>t)
	=
	P(t_k-t_{k-1}>t)
	=
	P\bigl(X(t_k,t_k+t]=0\bigr)
	=
	P(X(t)=0)
	=
	e^{-\lambda t},
$
следовательно,
$
	F_{\tau_k}(t)=1-e^{-\lambda t}.
$

Таким образом, все $\tau_k$ имеют одинаковое показательное распределение
с параметром $\lambda$.

\medskip

Свойства показательного распределения:
\begin{itemize}
	\item
	      $
		      F_{\tau}(s+t \mid \tau>s)=F_{\tau}(t),
	      $
	      не зависит от $s$.

	\item
	      Если $\{\tau_k,\ k=1,\ldots,n\}$ имеют показательные распределения
	      с параметрами $\lambda_1,\ldots,\lambda_n$, то
	      $
		      \tau=\min\{\tau_k,\ k=1,\ldots,n\}
	      $
	      имеет показательное распределение с параметром
	      $
		      \lambda=\lambda_1+\lambda_2+\ldots+\lambda_n.
	      $

	\item
	      Если $\{\tau_k,\ k=1,\ldots,n\}$ имеют показательные распределения
	      с параметрами $\lambda_1,\ldots,\lambda_n$,
	      $
		      \lambda=\lambda_1+\ldots+\lambda_n,
		      \qquad
		      \tau=\min\{\tau_k,\ k=1,\ldots,n\},
	      $
	      то
	      $
		      P(\tau=\tau_k \mid \tau=t)=\frac{\lambda_k}{\lambda}.
	      $
\end{itemize}
