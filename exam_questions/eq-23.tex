\subsection*{Вопрос 23}

\begin{condition}
	Стационарный процесс с дискретным спектром, спектральное разложение корреляционной функции на отрезке.
\end{condition}

Если спектральная функция стационарного процесса $\xi_t$ кусочно-постоянна,
то $\xi_t$ называется стационарным процессом с дискретным спектром.

Спектральное разложение корреляционной функции стационарного процесса
с дискретным спектром имеет вид
$
	K_\xi(\tau)=\sum_{k=0}^{\infty} D_k^*(\xi)\, e^{i\omega_k \tau}.
$

Спектральное разложение стационарного процесса $\xi_t$:
$
	\xi_t = m_\xi + \int_{-\infty}^{+\infty} e^{i\omega t}\,\Phi(d\omega),
$
где $\Phi(\Delta)$ называется спектральной стохастической мерой и удовлетворяет
свойствам:
\begin{enumerate}
	\item $M\Phi(\Delta)=0$ для любого отрезка $\Delta$;
	\item $M\bigl[\Phi(\Delta_1)\Phi(\Delta_2)\bigr]=0$ для любых
	      непересекающихся отрезков $\Delta_1$ и $\Delta_2$.
\end{enumerate}

Если $\xi_t$ — стационарный процесс, то имеет место спектральное разложение
его корреляционной функции
$
	K_\xi(\tau)=\int_{-\infty}^{+\infty} e^{i\omega\tau}\, dF_\xi(\omega).
$

$F_\xi(\omega)$ — спектральная функция стационарного процесса $\xi_t$.
Она связана со спектральной стохастической мерой $\Phi(\Delta)$
следующим соотношением:
$
	F(b)-F(a)=M\lvert\Phi(\Delta)\rvert^2,
$
где $\Delta=(a,b)$.
