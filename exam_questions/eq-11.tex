\subsection*{Вопрос 11}

\begin{condition}
	Распределение времени пребывания в одном состоянии в цепи Маркова с непрерывным временем. Марковское свойство показательного распределения.
\end{condition}

Пусть $T(i)$ — время пребывания марковского процесса $X(t)$
в состоянии $i$. Тогда
$
	P(T(i)>t)
	=
	\lim_{n\to\infty}\bigl[p_{ii}(t/n)\bigr]^n.
$

Но
$
	p_{ii}(t/n)=1+\lambda_{ii}\frac{t}{n}+o\!\left(\frac{t}{n}\right).
$
Положим
$
	\lambda_i=-\lambda_{ii}=\sum_{j\ne i}\lambda_{ij}.
$
Тогда
$
	\lim_{n\to\infty}\bigl[p_{ii}(t/n)\bigr]^n
	=
	\lim_{n\to\infty}
	\left(1-\lambda_i\frac{t}{n}+o\!\left(\frac{t}{n}\right)\right)^n
	=
	e^{-\lambda_i t}.
$

Следовательно,
$
	F_{T(i)}(t)=1-e^{-\lambda_i t},
$
то есть случайная величина $T(i)$ имеет показательное распределение
с параметром $\lambda_i$.
