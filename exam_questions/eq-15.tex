\subsection*{Вопрос 15}

\begin{condition}
	Комплексные случайные процессы, их характеристики. Примеры.
\end{condition}

\textbf{Математическое ожидание комплексного случайного процесса.}

Пусть
$
	X_t = X_t^{(1)} + i X_t^{(2)}.
$
Математическое ожидание комплексного случайного процесса $X_t$ определяется как
$
	m_X(t)=\mathbb{E}X_t
	=
	\mathbb{E}X_t^{(1)} + i\,\mathbb{E}X_t^{(2)},
	\qquad t\in T.
$

\medskip

\textbf{Дисперсия комплексного случайного процесса.}

Дисперсией комплексного случайного процесса $X_t$ называется
$
	D_X(t)=\mathbb{E}\,|X_t-m_X(t)|^2,
	\qquad t\in T.
$

Обозначая
$
	X_t^{0}=X_t-m_X(t)
$
(центрированный случайный процесс), получаем
$
	D_X(t)=\mathbb{E}|X_t^{0}|^2
	=
	\mathbb{E}|X_t|^2-|m_X(t)|^2.
$

Действительно,
$
	\begin{aligned}
		D_X(t)
		 & =
		\mathbb{E}\bigl[(X_t-m_X(t))(\overline{X_t-m_X(t)})\bigr] \\
		 & =
		\mathbb{E}|X_t|^2
		-
		m_X(t)\,\overline{m_X(t)}
		-
		\overline{m_X(t)}\,m_X(t)
		+
		|m_X(t)|^2                                                \\
		 & =
		\mathbb{E}|X_t|^2-|m_X(t)|^2.
	\end{aligned}
$

\medskip

\textbf{Корреляционная функция комплексного случайного процесса.}

Корреляционная функция комплексного случайного процесса $X_t$ определяется как
$
	K_X(t,s)
	=
	\mathbb{E}\bigl[(X_t-m_X(t))(\overline{X_s-m_X(s)})\bigr],
	\qquad t,s\in T.
$

Эквивалентно,
$
	K_X(t,s)
	=
	\mathbb{E}[X_t\overline{X_s}]
	-
	m_X(t)\,\overline{m_X(s)}.
$
