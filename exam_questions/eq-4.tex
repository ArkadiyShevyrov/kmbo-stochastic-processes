\subsection*{Вопрос 4}

\begin{condition}
	Стационарное распределение вероятностей конечной однородной цепи Маркова. Его вид для цепи Маркова с двумя состояниями.
\end{condition}

Последовательность случайных величин $\{X_k\}_{k=0}^{\infty}$ со значениями в
$S=\{E_1,E_2,\ldots\}$ называется цепью Маркова, если выполняется равенство
условных вероятностей
$
	P\!\left(X_{i_k}=E_{j_k}\mid X_{i_1}=E_{j_1},\ldots,X_{i_{k-1}}=E_{j_{k-1}}\right)
	=
	P\!\left(X_{i_k}=E_{j_k}\mid X_{i_{k-1}}=E_{j_{k-1}}\right)
$
для произвольных $i_1<i_2<\ldots<i_{k-1}<i_k$, $(k=3,4,\ldots)$ и любых
$E_{j_1},\ldots,E_{j_k}$.

Если вероятности
$
	p_{ij}=P(X_{k+1}=E_j\mid X_k=E_i)
$
не зависят от $k$, то цепь Маркова называется однородной. При этом $p_{ij}$
называются переходными вероятностями, а матрица
$
	P=(p_{ij})
$
называется матрицей вероятностей перехода за один шаг или переходной матрицей.

\textbf{Определение 1.}
Распределение $\vec p^{\,*}$ цепи Маркова называется \emph{стационарным},
если оно остается неизменным на каждом шаге.
Стационарное распределение $\vec p^{\,*}$ удовлетворяет соотношению
$
	\vec p^{\,*}=\vec p^{\,*}P.
$

\textbf{Определение 2.}
Если существует предел
$
	\lim_{n\to\infty}\vec p(n)=\vec p(\infty)
$
и
$
	\sum_i p_i(\infty)=1,
$
то распределение $\vec p(\infty)$ называется \emph{предельным}.

\medskip

Стационарное распределение $\vec p^{\,*}$ для двусостояльной цепи Маркова
с матрицей перехода
$
	P=
	\begin{pmatrix}
		1-a & a   \\
		b   & 1-b
	\end{pmatrix},
	\qquad a+b>0,
$
имеет вид
$
	\vec p^{\,*}=
	\left(
	\frac{b}{a+b},\,
	\frac{a}{a+b}
	\right),
$
и удовлетворяет условию
$
	\vec p^{\,*}P=\vec p^{\,*}.
$

Если $a=b=0$, то оба состояния являются поглощающими, и стационарных
распределений бесконечно много: любая выпуклая комбинация $(1,0)$ и $(0,1)$.
В этом вырожденном случае условие
$
	\vec p^{\,*}P=\vec p^{\,*}
$
выполняется для любого
$
	\vec p^{\,*}=(\alpha,1-\alpha),\qquad \alpha\in[0,1].
$
