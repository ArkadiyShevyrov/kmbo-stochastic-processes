\subsection*{Вопрос 8}

\begin{condition}
	Марковский процесс с дискретным множеством состояний и непрерывным временем. Уравнения Колмогорова для вероятностей состояний.
\end{condition}

Случайный процесс $X_t$, $t\ge 0$, называется \emph{марковским}, если для любого
целого неотрицательного $m$, любых моментов времени
$
	0 \le s_1 < s_2 < \ldots < s_m \le s,\qquad t>0,
$
и любого набора состояний
$
	E_{i_1},E_{i_2},\ldots,E_{i_m},E_i,E_j
$
выполняется равенство
$
	P\!\left(
	X_{s+t}=E_j
	\mid
	X_{s_1}=E_{i_1},\ldots,X_{s_m}=E_{i_m},X_s=E_i
	\right)
	=
	P\!\left(X_{s+t}=E_j \mid X_s=E_i\right).
$

\medskip

\textbf{Система дифференциальных уравнений Колмогорова.}

Пусть $p_i(t)=P(X_t=E_i)$. Тогда
$
	\frac{d p_i(t)}{dt}
	=
	\sum_j \lambda_{ji} p_j(t)
	=
	\lambda_{ii} p_i(t)
	+
	\sum_{j\ne i} \lambda_{ji} p_j(t)
	=
	\sum_{j\ne i} \lambda_{ji} p_j(t)
	-
	p_i(t)\sum_{j\ne i}\lambda_{ij},
	\qquad i=1,2,\ldots
$

При этом для всех $t$ выполняется условие нормировки
$
	\sum_j p_j(t)=1.
$

Дифференцируя это равенство по $t$ в точке $t=0$, получаем
$
	\sum_j \lambda_{ij}=0,
	\qquad
	\lambda_{ii}=-\sum_{j\ne i}\lambda_{ij}.
$

Вектор вероятностей
$
	\vec p(t)=(p_1(t),p_2(t),\ldots)
$
удовлетворяет матричному уравнению
$
	\vec p\,'(t)
	=
	\lim_{\tau\to 0}
	\frac{\vec p(t+\tau)-\vec p(t)}{\tau}
	=
	\lim_{\tau\to 0}
	\frac{\vec p(t)P(\tau)-\vec p(t)}{\tau}
	=
	\vec p(t)\,
	\lim_{\tau\to 0}
	\frac{P(\tau)-P(0)}{\tau}
	=
	\vec p(t)\Lambda.
$
