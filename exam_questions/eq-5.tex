\subsection*{Вопрос 5}

\begin{condition}
	Предельные вероятности. Эргодические цепи Маркова. Теорема Маркова.
\end{condition}

Если существует предел
$
	\lim_{m\to\infty}\vec p(m)
	=
	\lim_{m\to\infty}\vec p(0)\,P^{m}
	=
	\vec p(\infty),
$
то $\vec p(\infty)$ называется \emph{предельным распределением вероятностей}
цепи Маркова с начальным распределением $\vec p(0)$.

\medskip

Цепь Маркова с переходной матрицей $P=(p_{ij})$ называется
\emph{эргодической}, если выполняются условия:
\begin{enumerate}
	\item существует предел
	      $
		      \lim_{k\to\infty} p_{ij}(k)=q_{ij};
	      $
	\item величины $q_{ij}$ не зависят от $i$, то есть
	      $
		      q_{ij}=q_j
	      $
	      (в матрице $Q=(q_{ij})$ все строки одинаковы);
	\item
	      $
		      q_j>0 \quad \forall j.
	      $
\end{enumerate}

\medskip

Теорема (Маркова).
Если для конечной цепи Маркова с переходной матрицей $P=(p_{ij})$
существует такое $s$, что
$
	p_{ij}(s)>0
	\quad
	(\,P(s)=\{p_{ij}(s)\}=P^{\,s}\,),
	\qquad \forall\, i,j,
$
то цепь Маркова является эргодической.
