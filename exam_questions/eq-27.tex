\subsection*{Вопрос 27}

\begin{condition}
	Преобразование стационарного процесса с непрерывным спектром линейной стационарной динамической системой.
\end{condition}

\textbf{Стационарные процессы с непрерывным спектром}

Стационарный в широком смысле процесс $\xi_t$ называется процессом
с непрерывным спектром, если его спектральная функция $F_\xi(\omega)$
дифференцируема:
$
	F'_\xi(\omega)=S_\xi^*(\omega).
$

В этом случае
$
	K_\xi(\tau)=\int_{-\infty}^{+\infty} e^{i\omega\tau}\, S_\xi^*(\omega)\, d\omega.
$

Функция $S_\xi^*(\omega)$ является неотрицательной и называется
спектральной плотностью стационарного процесса.

\medskip

\textbf{Преобразования стационарных процессов}

Стационарный процесс $\eta_t$ является результатом преобразования
стационарного процесса $\xi_t$
стационарной линейной динамической системой, если
$
	\sum_{k=0}^{n} a_k \frac{d^k \xi_t}{dt^k}
	=
	\sum_{k=0}^{m} b_k \frac{d^k \eta_t}{dt^k},
$
где $a_k$, $b_k$ — постоянные коэффициенты.

При этом $\xi_t$ называется входным процессом,
а $\eta_t$ — выходным процессом.

Для математических ожиданий процессов получаем
$
	b_0 m_\eta = a_0 m_\xi,
	\qquad
	m_\eta=\frac{a_0}{b_0} m_\xi .
$

Обозначения:
$
	A(p)=\sum_{k=0}^{n} a_k p^k,
	\qquad
	B(p)=\sum_{k=0}^{m} b_k p^k .
$

Многочлены $A(p)$ и $B(p)$ являются характеристическими
для дифференциальных операторов
в левой и правой частях уравнения динамической системы.

\medskip

\textbf{Преобразование непрерывного спектра}

Если входной процесс $\xi_t$ имеет непрерывный спектр
и $S_\xi^*(\omega)$ — его спектральная плотность,
то $\eta_t$ также является стационарным процессом
с непрерывным спектром и спектральной плотностью
$
	S_\eta^*(\omega)=\lvert T(i\omega)\rvert^2 S_\xi^*(\omega).
$
