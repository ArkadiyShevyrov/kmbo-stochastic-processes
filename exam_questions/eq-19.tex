\subsection*{Вопрос 19}

\begin{condition}
	Дифференцируемость случайного процесса в среднем квадратическом. Математическое ожидание и корреляционная функция производной.
\end{condition}

\textbf{Дифференцируемость в среднем квадратическом.}

Случайный процесс $X_t$ называется \emph{дифференцируемым в среднем квадратическом}
на отрезке $[a,b]$, если для всех $t\in[a,b]$ существует предел
$
	\lim_{\Delta t\to 0}
	\frac{X_{t+\Delta t}-X_t}{\Delta t}
	=
	\frac{dX_t}{dt}
$
в среднем квадратическом.

Случайный процесс
$
	Y_t=\frac{dX_t}{dt}
$
называется \emph{производной в среднем квадратическом}
случайного процесса $X_t$.

\medskip

\textbf{Критерий дифференцируемости в среднем квадратическом.}

Случайный процесс $X_t$ дифференцируем в среднем квадратическом
на отрезке $[a,b]$ тогда и только тогда, когда выполняются условия:
\begin{enumerate}
	\item
	      функция математического ожидания $m_X(t)$ дифференцируема на $[a,b]$;
	\item
	      существует смешанная производная
	      $
		      \frac{\partial^2 K_X(t,s)}{\partial t\,\partial s}
	      $
	      при $t=s$ для всех $t\in[a,b]$.
\end{enumerate}

\medskip

При этом для процесса
$
	Y_t=\frac{dX_t}{dt}
$
выполняются соотношения:
$
	m_Y(t)=m_X'(t), \qquad \forall\,t\in[a,b],
$
$
	K_Y(t,s)
	=
	\frac{\partial^2 K_X(t,s)}{\partial t\,\partial s},
	\qquad \forall\, t,s\in(a,b).
$
