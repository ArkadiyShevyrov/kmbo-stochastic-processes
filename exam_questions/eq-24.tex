\subsection*{Вопрос 24}

\begin{condition}
	Стационарный процесс с непрерывным спектром, спектральное разложение корреляционной функции на оси.
\end{condition}

\textbf{Стационарные процессы с непрерывным спектром}

Стационарный в широком смысле процесс $\xi_t$ называется процессом
с непрерывным спектром, если его спектральная функция $F_\xi(\omega)$
дифференцируема:
$
	F'_\xi(\omega)=S_\xi^*(\omega).
$

В этом случае
$
	K_\xi(\tau)=\int_{-\infty}^{+\infty} e^{i\omega\tau}\, S_\xi^*(\omega)\, d\omega.
$

Функция $S_\xi^*(\omega)$ является неотрицательной и называется
спектральной плотностью стационарного процесса.

\medskip

\textbf{Спектральное разложение стационарного процесса} $\xi_t$:
$
	\xi_t = m_\xi + \int_{-\infty}^{+\infty} e^{i\omega t}\, \Phi(d\omega),
$
где $\Phi(\Delta)$ называется спектральной стохастической мерой и
удовлетворяет свойствам:
\begin{enumerate}
	\item $M\Phi(\Delta)=0$ для любого отрезка $\Delta$;
	\item $M\bigl[\Phi(\Delta_1)\Phi(\Delta_2)\bigr]=0$ для любых
	      непересекающихся отрезков $\Delta_1$ и $\Delta_2$.
\end{enumerate}

Если $\xi_t$ — стационарный процесс, то имеет место спектральное
разложение его корреляционной функции:
$
	K_\xi(\tau)=\int_{-\infty}^{+\infty} e^{i\omega\tau}\, dF_\xi(\omega).
$

$F_\xi(\omega)$ — спектральная функция стационарного процесса $\xi_t$.
Она связана со спектральной стохастической мерой $\Phi(\Delta)$
соотношением:
$
	F(b)-F(a)=M\lvert \Phi(\Delta)\rvert^2,
$
где $\Delta=(a,b)$.

\medskip

\textbf{Действительнозначный случай.}

В случае действительнозначных стационарных процессов спектральное
разложение корреляционной функции имеет вид:
$
	K_\xi(\tau)=\int_{0}^{+\infty} S_\xi(\omega)\cos(\omega\tau)\, d\omega,
$
при этом
$
	S_\xi(\omega)=\frac{2}{\pi}\int_{0}^{+\infty} K_\xi(\tau)\cos(\omega\tau)\, d\tau,
$
$
	S_\xi(\omega)=2S_\xi^*(\omega)=2S_\xi^*(-\omega).
$
