\subsection*{Вопрос 16}

\begin{condition}
	Корреляционная функция случайного процесса и её свойства. Примеры.
\end{condition}


Корреляционная функция комплексного случайного процесса $X_t$ определяется как
$
	K_X(t,s)
	=
	\mathbb{E}\bigl[(X_t-m_X(t))\,\overline{(X_s-m_X(s))}\bigr],
	\qquad t,s\in T.
$

Эквивалентно,
$
	K_X(t,s)
	=
	\mathbb{E}[X_t\overline{X_s}]
	-
	m_X(t)\,\overline{m_X(s)}.
$

\medskip

\textbf{Свойства корреляционной функции.}

\begin{enumerate}
	\item
	      $
		      K_X(t,s)=0
	      $
	      для некоррелированных значений процесса.

	\item
	      $
		      K_X(t,t)=D_X(t)\ge 0.
	      $

	\item
	      $
		      |K_X(t,s)|^2 \le D_X(t)\,D_X(s).
	      $

	\item
	      $
		      K_X(s,t)=\overline{K_X(t,s)}.
	      $

	\item
	      Пусть
	      $
		      Y_t=\varphi(t)\,X_t+d(t),
	      $
	      где $\varphi(t)$ и $d(t)$ — неслучайные функции.
	      Тогда
	      $
		      K_Y(t,s)=\varphi(t)\,\overline{\varphi(s)}\,K_X(t,s).
	      $

	\item
	      Для некоррелированных случайных процессов $X_t$ и $Y_t$
	      $
		      K_{X+Y}(t,s)=K_X(t,s)+K_Y(t,s).
	      $

	\item
	      Для любых $t_1,t_2,\ldots,t_n\in T$ и любых
	      $z_1,z_2,\ldots,z_n\in\mathbb{C}$
	      выполняется неравенство
	      $
		      \sum_{i=1}^{n}\sum_{j=1}^{n}
		      K_X(t_i,t_j)\,z_i\,\overline{z_j}\ge 0.
	      $
\end{enumerate}