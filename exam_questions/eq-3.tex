\subsection*{Вопрос 3}

\begin{condition}
	Конечные однородные цепи Маркова. Матрица переходных вероятностей за n шагов. Примеры.
\end{condition}

Если вероятности
$
	p_{ij} = P(X_{k+1}=E_j \mid X_k=E_i)
$
не зависят от $k$, то цепь Маркова называется \textit{однородной}.

Матрица
$
	P(m) = \bigl(p_{ij}(m)\bigr)
$
называется матрицей переходных вероятностей за $m$ шагов, где
$
	p_{ij}(m) = P(X_{k+m}=E_j \mid X_k=E_i).
$

При этом выполняются условия
$
	0 \le p_{ij}(m) \le 1, \qquad \sum_{j} p_{ij}(m) = 1,
$
то есть сумма элементов каждой строки равна единице.
