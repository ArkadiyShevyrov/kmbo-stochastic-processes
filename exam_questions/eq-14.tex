\subsection*{Вопрос 14}

\begin{condition}
	Процесс Пуассона. Составление уравнений Колмогорова и их решение.
\end{condition}

Для пуассоновского потока событий случайный процесс
$
	\{X(t),\ t\ge 0\}
$
является марковским и называется \emph{процессом Пуассона}.

\medskip

\textbf{Граф процесса Пуассона:}
$
	0 \xrightarrow{\lambda} 1 \xrightarrow{\lambda} 2 \xrightarrow{\lambda} \ldots
$

\medskip

\textbf{Система дифференциальных уравнений Колмогорова:}
$
	\begin{cases}
		p_0'(t) = -\lambda\,p_0(t), \\[6pt]
		p_k'(t) = \lambda\,p_{k-1}(t) - \lambda\,p_k(t), \qquad k\ge 1.
	\end{cases}
$

\medskip

Решение этой системы с начальными условиями
$
	p_0(0)=1,\qquad p_k(0)=0,\ k\ge 1,
$
имеет вид
$
	p_k(t)=P(X_t=k)=\frac{(\lambda t)^k}{k!}\,e^{-\lambda t},
	\qquad k\ge 0.
$

Следовательно, случайная величина $X_t$ имеет распределение Пуассона
с параметром $\lambda t$.
