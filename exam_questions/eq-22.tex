\subsection*{Вопрос 22}

\begin{condition}
	Стационарные процессы в широком и узком смыслах. Свойства корреляционной функции.
\end{condition}

Пусть $X_t=X(t)$, $t\in\mathbb{R}$ — случайный процесс с конечномерными
функциями распределения
$
	F_{t_1,\ldots,t_n}(x_1,\ldots,x_n)
	=
	P(X_{t_1}\le x_1,\ldots,X_{t_n}\le x_n).
$

Случайный процесс $X_t$, $t\in\mathbb{R}$, называется
\emph{стационарным в узком смысле}, если для любого $n$,
произвольных значений $s,t_1,\ldots,t_n\in\mathbb{R}$ и $x_1,\ldots,x_n$
выполняется равенство
$
	F_{t_1+s,\ldots,t_n+s}(x_1,\ldots,x_n)
	=
	F_{t_1,\ldots,t_n}(x_1,\ldots,x_n).
$

\medskip

Случайный процесс $X_t$, удовлетворяющий условию
$
	\mathbb{E}|X_t|^2<\infty \qquad \forall\,t\in\mathbb{R},
$
с математическим ожиданием
$
	m_X(t)=\mathbb{E}X_t
$
и корреляционной функцией $K_X(t,s)$ называется
\emph{стационарным в широком смысле}, если
$
	m_X(t)=\mathrm{const},
	\qquad
	K_X(t+u,s+u)=K_X(t,s)
	\quad \forall\,u\in\mathbb{R}.
$

Корреляционная функция стационарного в широком смысле процесса
зависит только от разности
$
	\tau=t-s
$
и может рассматриваться как функция одной переменной:
$
	K_X(\tau)=K_X(t-s)=K_X(t-s,0)=K_X(t,s).
$

Если стационарный в узком смысле процесс $X_t$
удовлетворяет условию $\mathbb{E}|X_t|^2<\infty$,
то он является и стационарным в широком смысле.

\medskip

\textbf{Свойства корреляционной функции стационарного в широком смысле процесса:}

\begin{enumerate}
	\item
	      $
		      K_X(0)=D_X=\mathrm{const}\ge 0.
	      $

	\item
	      $
		      |K_X(\tau)|\le K_X(0),
		      \qquad \forall\,\tau\in\mathbb{R}.
	      $

	\item
	      $
		      K_X(-\tau)=\overline{K_X(\tau)}.
	      $

	\item
	      Для любого $n$, произвольных $t_1,\ldots,t_n\in\mathbb{R}$ и
	      $z_1,\ldots,z_n\in\mathbb{C}$ выполняется неравенство
	      $
		      \sum_{i=1}^{n}\sum_{j=1}^{n}
		      K_X(t_i-t_j)\,z_i\,\overline{z_j}\ge 0.
	      $

	\item
	      Случайный процесс
	      $
		      Y_t=\frac{dX_t}{dt}
	      $
	      также является стационарным в широком смысле, причём
	      $
		      m_Y=0,
		      \qquad
		      K_Y(\tau)=-K_X''(\tau).
	      $
\end{enumerate}

\medskip

\textbf{Свойства корреляционной функции $\xi_t$:}

\begin{enumerate}
	\item
	      $
		      K_\xi(-\tau)=\overline{K_\xi(\tau)}.
	      $

	\item
	      $
		      K_\xi(0)=D_\xi=\mathrm{const}.
	      $

	\item
	      $
		      |K_\xi(\tau)|\le D_\xi=K_\xi(0).
	      $

	\item
	      Если
	      $
		      \eta_t=\alpha\,\xi_t+\beta,
	      $
	      то
	      $
		      K_\eta(\tau)=|\alpha|^2\,K_\xi(\tau).
	      $

	\item
	      Для любых $t_1,\ldots,t_n$ и $z_1,\ldots,z_n\in\mathbb{C}$:
	      $
		      \sum_{i=1}^{n}\sum_{j=1}^{n}
		      K_\xi(t_i-t_j)\,z_i\,\overline{z_j}\ge 0.
	      $

	\item
	      Если $\xi_t$ и $\eta_t$ некоррелированы, то
	      $
		      K_{\xi+\eta}(\tau)=K_\xi(\tau)+K_\eta(\tau).
	      $
\end{enumerate}
