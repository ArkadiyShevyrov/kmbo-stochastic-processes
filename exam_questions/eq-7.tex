\subsection*{Вопрос 7}

\begin{condition}
	Марковский процесс с дискретным множеством состояний и непрерывным временем. Матрица интенсивностей перехода, её свойства.
\end{condition}

Случайный процесс $X_t$, $t\ge 0$, называется \emph{марковским}, если для любого
целого неотрицательного $m$, любых моментов времени
$
	0 \le s_1 < s_2 < \ldots < s_m \le s,\qquad t>0,
$
и любого набора состояний
$
	E_{i_1},E_{i_2},\ldots,E_{i_m},E_i,E_j
$
выполняется равенство
$
	P\!\left(X_{s+t}=E_j \mid X_{s_1}=E_{i_1},\ldots,X_{s_m}=E_{i_m},X_s=E_i\right)
	=
	P\!\left(X_{s+t}=E_j \mid X_s=E_i\right).
$

\medskip

Матрица вероятностей перехода за время $t$ определяется как
$
	P(t)=\|p_{ij}(t)\|.
$

Предполагается, что переходные вероятности $p_{ij}(t)$ дифференцируемы в нуле.
При этом
$
	p'_{ij}(0)=\lambda_{ij}, \qquad i\ne j,
$
и для малых $t$ справедливы разложения
$
	p_{ij}(t)=\lambda_{ij}t+o(t), \qquad i\ne j,
$
$
	p_{ii}(t)=1+\lambda_{ii}t+o(t).
$

Матрица
$
	P'(0)=\Lambda=\|\lambda_{ij}\|
$
называется \emph{матрицей интенсивностей} (или плотностей вероятностей) перехода.

\medskip

\textbf{Свойства матрицы интенсивностей перехода:}
\begin{enumerate}
	\item
	      $
		      \lambda_{ij}\ge 0, \qquad i\ne j;
	      $
	\item
	      $
		      \lambda_{ii}\le 0;
	      $
	\item
	      $
		      \sum_j \lambda_{ij}=0, \qquad
		      \lambda_{ii}=-\sum_{j\ne i}\lambda_{ij}.
	      $
\end{enumerate}
