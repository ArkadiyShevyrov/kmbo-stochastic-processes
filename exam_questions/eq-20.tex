\subsection*{Вопрос 20}

\begin{condition}
	Интегрируемость случайного процесса на отрезке в среднем квадратическом. Математическое ожидание и корреляционная функция интеграла.
\end{condition}

\textbf{Интегрируемость в среднем квадратическом.}

Случайный процесс $X_t$ называется \emph{интегрируемым в среднем квадратическом}
на отрезке $[a,b]$, если существует предел
$
	\lim_{\max |t_i-t_{i-1}|\to 0}
	\sum_{i=1}^{n}(t_i-t_{i-1})\,X_{t_i}
	=
	\int_a^b X_t\,dt,
$
где
$
	a=t_0<t_1<t_2<\ldots<t_n=b.
$

Случайный процесс
$
	\int_a^b X_t\,dt
$
называется \emph{интегралом в среднем квадратическом}
случайного процесса $X_t$.

\medskip

Если случайный процесс $X_t$ интегрируем в среднем квадратическом
на $[a,b]$ и
$
	Y_t=\int_a^t X_\tau\,d\tau,
$
то выполняются соотношения:
\begin{enumerate}
	\item
	      Математическое ожидание:
	      $
		      m_Y(t)=\int_a^t m_X(\tau)\,d\tau,
		      \qquad \forall\, t\in[a,b].
	      $

	\item
	      Корреляционная функция:
	      $
		      K_Y(t,s)
		      =
		      \int_a^t d\tau
		      \int_a^s K_X(\tau,\sigma)\,d\sigma,
		      \qquad \forall\, t,s\in[a,b].
	      $
\end{enumerate}
