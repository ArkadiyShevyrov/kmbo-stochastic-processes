\subsection*{Вопрос 26}

\begin{condition}
	Преобразование стационарного процесса с дискретным спектром линейной стационарной динамической системой.
\end{condition}

\textbf{Стационарные процессы с дискретным спектром}

Если спектральная функция стационарного процесса $\xi_t$ кусочно-постоянна,
то $\xi_t$ называется стационарным процессом с дискретным спектром.

Спектральное разложение корреляционной функции стационарного процесса
с дискретным спектром имеет вид
$
	K_\xi(\tau)=\sum_{k=0}^{\infty} D_k^*(\xi)\, e^{i\omega_k \tau}.
$

\medskip

\textbf{Преобразования стационарных процессов}

Стационарный процесс $\eta_t$ является результатом преобразования
стационарного процесса $\xi_t$ стационарной линейной динамической системой,
если
$
	\sum_{k=0}^{n} a_k \frac{d^k \xi_t}{dt^k}
	=
	\sum_{k=0}^{m} b_k \frac{d^k \eta_t}{dt^k},
$
где $a_k$, $b_k$ — постоянные коэффициенты.

При этом $\xi_t$ называется входным процессом,
а $\eta_t$ — выходным процессом.

Для математических ожиданий процессов получаем
$
	b_0 m_\eta = a_0 m_\xi,
	\qquad
	m_\eta = \frac{a_0}{b_0} m_\xi .
$

Обозначения:
$
	A(p)=\sum_{k=0}^{n} a_k p^k,
	\qquad
	B(p)=\sum_{k=0}^{m} b_k p^k .
$

Многочлены $A(p)$ и $B(p)$ являются характеристическими
для дифференциальных операторов в левой и правой частях
уравнения динамической системы.

\medskip

\textbf{Преобразование дискретного спектра}

Если стационарный процесс $\xi_t$ имеет дискретный спектр
с корреляционной функцией
$
	K_\xi(\tau)=\sum_{k=-\infty}^{+\infty} D_k^*(\xi)\, e^{i\omega_k \tau},
$
то
$
	D_k^*(\eta)=\lvert T(i\omega_k)\rvert^2 D_k^*(\xi).
$

Корреляционная функция выходного стационарного процесса $\eta_t$
имеет вид
$
	K_\eta(\tau)
	=\sum_{k=-\infty}^{+\infty}
	\lvert T(i\omega_k)\rvert^2
	D_k^*(\xi)\, e^{i\omega_k \tau}.
$

Следовательно, процесс $\eta_t$ также имеет дискретный спектр.
