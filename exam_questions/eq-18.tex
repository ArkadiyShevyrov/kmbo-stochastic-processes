\subsection*{Вопрос 18}

\begin{condition}
	Непрерывность случайного процесса в среднем квадратическом. Связь между непрерывностью математического ожидания и корреляционной функции.
\end{condition}

\textbf{Непрерывность в среднем квадратическом.}

Случайный процесс $\{X_t,\ t\in[a,b]\}$ сходится к случайной величине $Y$
при $t\to s$ \emph{в среднем квадратическом}
$
	\left(\lim_{t\to s} X_t = Y\right),
$
если
$
	\lim_{t\to s}\mathbb{E}\bigl|X_t-Y\bigr|^2=0.
$

\medskip

Случайный процесс $\{X_t,\ t\in[a,b]\}$ называется
\emph{непрерывным в среднем квадратическом}
на отрезке $[a,b]$, если
$
	\lim_{t\to s} X_t = X_s
$
в среднем квадратическом для всех $s\in[a,b]$.

\medskip

\textbf{Критерий непрерывности в среднем квадратическом.}

Случайный процесс $X_t$ непрерывен в среднем квадратическом
на отрезке $[a,b]$ тогда и только тогда, когда выполняются условия:
\begin{enumerate}
	\item
	      функция математического ожидания $m_X(t)$ непрерывна на $[a,b]$;
	\item
	      корреляционная функция $K_X(t,s)$ непрерывна при $t=s$
	      для всех $t\in[a,b]$.
\end{enumerate}
